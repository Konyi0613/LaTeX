\documentclass[12pt,a4paper]{article}
\usepackage{fontspec}    % 加這個就可以設定字體
\usepackage{type1cm}	 % 設定fontsize用
\usepackage{titlesec}   % 設定section等的字體
\usepackage{titling}    % 加強 title 功能
\usepackage[top=2cm,margin=2.5cm,a4paper]{geometry}
\usepackage{setspace}
\usepackage[english]{babel}
\usepackage{amsmath}
\usepackage{tikz,lipsum,lmodern}
\usepackage[most]{tcolorbox}
\usepackage{subfigure}
\usepackage{svg}
\usepackage[svgnames]{xcolor}
\RequirePackage[framemethod=default]{mdframed}
\usepackage{mathdots}
\usepackage{wrapfig}
\usepackage{float}
\usepackage{pdfpages}
\usepackage{enumitem}
\usepackage{appendix}
\usepackage{graphicx}
\usepackage{comment}
\usepackage{tikz}
\usepackage{tikz-qtree}
\usepackage{witharrows}
\usepackage{moresize}
\usepackage{tcolorbox}
\tcbuselibrary{skins, breakable, theorems}
\usepackage{booktabs}
\usepackage{enumerate} % 加強版enumerate
\usepackage{fancyhdr} % 頁首頁尾
\usepackage{amsmath,amsthm,amssymb,amsfonts} % 引入 AMS 數學環境
\usepackage{bm} %加粗數學符號
\usepackage[dvipsnames]{} %顏色
\usepackage{yhmath}      % math symbol
\usepackage{graphicx}    % 圖形插入用
\usepackage{mathtools}
\usepackage{hyperref}
\hypersetup{
colorlinks=true,
linkcolor=black,
filecolor=blue,
urlcolor=teal,
citecolor=green
}
\usepackage{pgfplots}%製作表格
\pgfplotsset{width=9cm,compat=1.18}
\usepackage{caption}
\setlength{\headheight}{15pt}
\newcommand\tnr{\fontspec{Times New Roman}}%設定
\renewcommand{\abstractname}{\LARGE Abstract}

\DeclareMathOperator{\arccot}{arccot}
\DeclareMathOperator{\arcsec}{arcsec}
\DeclareMathOperator{\arccsc}{arccsc}


% new commands: %
\definecolor{amethyst}{rgb}{0.6, 0.4, 0.8}
\definecolor{aquamarine}{rgb}{0.5, 1.0, 0.83}
\newcommand{\np}[1]{\\[{#1}] \indent}
\newcommand{\red}[1]{\textcolor{red}{#1}}
\newcommand{\blue}[1]{\textcolor{blue}{#1}}
\newcommand{\amethyst}[1]{\textcolor{amethyst}{#1}}
\newcommand{\abs}[1]{\left|{#1}\right|}
\newcommand{\R}{\mathbb{R}}
\newcommand{\Z}{\mathbb{Z}}
\newcommand{\N}{\mathbb{N}}
\newcommand{\Q}{\mathbb{Q}}
\newcommand{\C}{\mathbb{C}}
\newcommand{\lr}[1]{\left( {#1} \right)}
\newcommand{\set}[1]{\{{#1}\}}
\newcommand{\transpose}[1]{{#1}^\mathrm{T}}
\newcommand{\adj}{\mathrm{adj}}
\newcommand{\degree}{^\circ}
\newcommand{\undereq}[1]{\underset{\text{#1}}{=}}
\newcommand{\dlim}[1]{\displaystyle\lim_{#1}}
\newcommand{\vertequal}{\vcenter{\hbox{\text{=}}}}
\newcommand{\dsum}[2]{\displaystyle\sum_{#1}^{#2}}
\newcommand{\dint}[2]{\displaystyle\int_{#1}^{#2}}
\newcommand{\Arc}[1]{\wideparen{{#1}}}
\newcommand{\Line}[1]{\overleftrightarrow{{#1}}}
\newcommand{\pll}{\kern 0.56em/\kern -0.8em /\kern 0.56em}
\newcommand{\Ray}[1]{\overrightarrow{{#1}}}
\newcommand{\Segment}[1]{\overline{{#1}}}
\newcommand{\ord}{\operatorname{ord}}
\newcommand{\verteq}{\rotatebox{90}{$\,=$}}
\newcommand{\equalto}[2]{\underset{\scriptstyle\overset{\mkern4mu\verteq}{#2}}{#1}}
\newcommand{\resetcounters}
    {
    \setcounter{section}{0}
    \setcounter{ans}{0}
    \setcounter{ax}{0}
    \setcounter{cl}{0}
    \setcounter{con}{0}
    \setcounter{clm}{0}
    \setcounter{df}{0}
    \setcounter{ex}{0}
    \setcounter{exs}{0}
    \setcounter{lm}{0}
    \setcounter{pr}{0}
    \setcounter{pp}{0}
    \setcounter{prop}{0}
    \setcounter{rem}{0}
    \setcounter{thm}{0}
    \setcounter{pf}{0}
    }
\renewcommand{\proofname}{\emph {Proof.}} %修改Proof 標頭
\newtheoremstyle{mystyle}
  {6pt}{15pt}
  {}
  {}
  {\bf}
  {.}
  {1em}
  {}
\theoremstyle{mystyle}
\numberwithin{figure}{subsection}
\newtheorem{ans}{Answer}[subsection] %答案
\newtheorem{ax}{Axiom}[subsection] %公理
\newtheorem{cor}{Corollary}[subsection] %推論
\newtheorem{con}{Conclusion}[subsection] %結論
\newtheorem{clm}{Claim}[subsection] %主張,聲稱
%\newtheorem{df}{Definition}[subsection] %定義
\newtheorem{exm}{Example}[subsection] %例題
\newtheorem{exs}{Exercise}[subsection] %習題
\newtheorem{lm}{Lemma}[subsection] %引理
\newtheorem{prm}{Problem}[subsection] %問題
\newtheorem{pp}{Property}[subsection] %性質
\newtheorem{prp}{Proposition}[subsection] %提議
\newtheorem{rem}{Remark}[subsection] %備註
%\newtheorem{thm}{Theorem}[subsection] %定理
\newtheorem{pf}{Proof}[subsection]
\newtheorem{sol}{Solution}[subsection]%解法
\newtcbtheorem[auto counter,number within=subsection]{thm}%
  {Theorem}{fonttitle=\bfseries\upshape, fontupper=\slshape,
     arc=0mm, colback=blue!5!white,colframe=blue!75!black}{theorem}
\newtcbtheorem[auto counter,number within=subsection]{df}%
  {Definition}{fonttitle=\bfseries\upshape, fontupper=\slshape,
     arc=0mm, colback=green!10!white, colframe=green!70!black}{definition}
\newmdenv[skipabove=7pt,
skipbelow=7pt,
rightline=false,
leftline=true,
topline=false,
bottomline=false,
linecolor=Orange,
backgroundcolor=Orange!10,
innerleftmargin=5pt,
innerrightmargin=5pt,
innertopmargin=2pt,
leftmargin=0cm,
rightmargin=0cm,
linewidth=2.5pt,
innerbottommargin=5pt]{exBox}    
\newenvironment{ex}{\begin{exBox}\begin{exm}}{\end{exm}\end{exBox}}
\newmdenv[skipabove=7pt,
skipbelow=7pt,
rightline=false,
leftline=true,
topline=false,
bottomline=false,
linecolor=purple,
backgroundcolor=purple!10,
innerleftmargin=5pt,
innerrightmargin=5pt,
innertopmargin=2pt,
leftmargin=0cm,
rightmargin=0cm,
linewidth=2.5pt,
innerbottommargin=5pt]{prBox}    
\newenvironment{pr}{\begin{prBox}\begin{prm}}{\end{prm}\end{prBox}}
\newmdenv[skipabove=7pt,
skipbelow=7pt,
rightline=false,
leftline=true,
topline=false,
bottomline=false,
linecolor=aquamarine,
backgroundcolor=aquamarine!10,
innerleftmargin=5pt,
innerrightmargin=5pt,
innertopmargin=2pt,
leftmargin=0cm,
rightmargin=0cm,
linewidth=2.5pt,
innerbottommargin=5pt]{prpBox}    
\newenvironment{prop}{\begin{prpBox}\begin{prp}}{\end{prp}\end{prpBox}}

\newmdenv[skipabove=7pt,
skipbelow=7pt,
rightline=false,
leftline=true,
topline=false,
bottomline=false,
linecolor=lime,
backgroundcolor=lime!10,
innerleftmargin=5pt,
innerrightmargin=5pt,
innertopmargin=2pt,
leftmargin=0cm,
rightmargin=0cm,
linewidth=2.5pt,
innerbottommargin=5pt]{corBox}    
\newenvironment{cl}{\begin{corBox}\begin{cor}}{\end{cor}\end{corBox}}



\title{Calculus Note}
\author{Kon Yi}
\begin{document}
\maketitle
\section{Differential Rules}
\subsection{Linear approximations}
We think that $y=f(a)+f'(a)(x-a)$ is a good approximation of $y=f(x)$ near $x=a$.
\begin{figure}[h]
    \centering
    \includesvg[width=\textwidth]{3.10.1.svg}
    \caption{function $f$}
    \label{fig:3.10.1}
\end{figure}
\begin{df}{Linear Approximation}{LA}
Let $L(x) := f(a) + f'(a)(x-a).$
\begin{itemize}
    \item $L(x)$ is called the \textcolor{blue}{\underline{linearization of $f$ at $a$}.}
    \item $f(x) \approx L(x)$ is called the \textcolor{blue}{\underline{linear approximation of $f$ at $a$}.}
\end{itemize}
\end{df}
\begin{ex}
$f(x) = \sqrt{x+3}$, find the linear approximation of $f$ at $x=1$.
\end{ex}
\begin{align*}
f'(x) = \dfrac{1}{2}\dfrac{1}{\sqrt{x+3}} &\Rightarrow f'(1) = \dfrac{1}{4} \\
&\Rightarrow L(x) = f(1) + f'(1)(x-1) = 2 + \dfrac{1}{4}(x-1)
\end{align*}
Approximate $\sqrt{3.98}$ and $\sqrt{4.05}$.
\begin{align*}
\sqrt{3.98} = f(1-0.02) \approx L(1-0.02) = 2+\dfrac{1}{4}(-0.02)= 1.995 \\
\sqrt{4.05} = f(1+0.05)\approx L(1+0.05)=2+\dfrac{1}{4}(0.05)=2.0125
\end{align*}
We denote $\Delta y := f(x)-f(a)$, then
\begin{align*}
    \Delta y &= f(x) - f(a) \approx f'(a)\underbrace{(x-a)}_{\Delta x} \\
    &\Rightarrow \dfrac{\Delta y}{\Delta x} \approx f'(a) = \dfrac{dy}{dx}
\end{align*}
Hence, the idea of linear approximation is to \textcolor{red}{use the slope of the tangent line to approximate the slopes of nearby secant line.(which is opposite to the definition of differentiation)}
\begin{df}{$dx$ and $dy$}{dx/dy}
    If we denote $dx := \Delta x$, define the differential of $y=f(x)$ at $a$ to be \[dy := f'(a) \cdot dx\]
\end{df}
Using this notation, the linear approximation become
\[
\Delta y \approx f'(a)\equalto{(x-a)}{\Delta x = dx}=dy.
\]
\begin{ex}
    The radius of a sphere is $21$cm(measured with a possible error at most $0.05$cm). What is the maximal error in computing the volume of the sphere?
\end{ex}
The linear approximation of the volume $V(r) = \dfrac{4}{3}\pi r^3$ at $21$ is
\begin{align*}
L(r)&=V(21)+4\pi r_0^2 \cdot (r-21). \\
\Rightarrow \Delta V &= V(r)-V(21) \approx 4\pi(21)^2 \cdot \underbrace{(r-21)}_{\le 0.05} \approx 277 cm^3
\end{align*}
Using the notation of differential,
\[
\Delta V \approx dv = V'(21) \cdot dr= 4\pi (21)^2 \cdot 0.05
\]
If we want the \underline{relative error} $\dfrac{\Delta V}{V}$ to be at most $3\%$, what is the relative error allowed in measuring the radius?
\[
\underbrace{\dfrac{\Delta V}{V}}_{\le 3\%} \approx \frac{dV}{V} = \frac{4\pi r^2 dr}{\frac{4}{3}\pi r^3}=3\underbrace{\frac{\Delta r}{r}}_{\le 1\%}
\]
\section{Application of differentiation}
\subsection{Maximum and minimum values}
\begin{df}{Absolute Extreme Value}{Absolute Extreme Value}
Let $f: U \longrightarrow \R.$
\begin{itemize}
    \item If $\exists c \in U$ such that $f(c) \ge f(x) \forall \textcolor{red}{x \in U}$, then $f(c)$ is called the \\ \textcolor{blue}{\textbf{absolute maximum value}} of $f$ on $U$.
    \item If $\exists c \in U$ such that $f(c) \le f(x) \forall \textcolor{red}{x \in U}$, then $f(c)$ is called the \\ \textcolor{blue}{\textbf{absolute minimum value}} of $f$ on $U$.
\end{itemize}
The set of absolute maximum and absolute minimum is called the \textbf{extreme value of $f$ on $U$}.
\begin{itemize}
    \item If there is $c$ such that $f(x) \ge f(x)$ for all \textcolor{red}{$x$ near $c$}, then $f(c)$ is called a \textcolor{red}{\textbf{local}} \textbf{maximum value} of $f$ on $U$.
    \item If there is $c$ such that $f(x) \le f(x)$ for all \textcolor{red}{$x$ near $c$}, then $f(c)$ is called a \textcolor{red}{\textbf{local}} \textbf{minimum value} of $f$ on $U$.
\end{itemize}
\end{df}
\begin{ex}
By the below figure we can see that \textcolor{red}{global max value is not attained!}
\end{ex}
\begin{figure}[h]
    \centering
    \includesvg[width=\linewidth]{4.1.1.svg}
    \caption{$f:[a,b) \longrightarrow \R$}
    \label{fig:4.1.1}
\end{figure}
\begin{ex}
$(1) \quad f(x) = x^2, \ (2) \quad g(x) = x^3$
\end{ex}
\begin{figure}[h]
    \centering
    \subfigure[$y=x^2$]{
    \begin{minipage}[h]{0.4\linewidth}
    \centering
    \includesvg[width=\linewidth, height = 3cm]{x^2.svg}
    %\caption{$y=x^2$}
    \end{minipage}%
    }
    \centering
    \subfigure[$y=x^3$]{
    \begin{minipage}[h]{0.4\linewidth}
    \centering
    \includesvg[width=\linewidth, height = 3cm]{y=x^3.svg}
    %\caption{$y=x^3$}
    \end{minipage}%
    }
\end{figure}
We can see that the global min value$=0$, \textcolor{red}{but global value is not attained by any $x \in \R.$}
\begin{thm}{Extreme Value Theorem}{EVT}
    If $f$ is continuous on $[a,b]$, then $f$ attains an global max value and an global min value on $[a,b]$.
\end{thm}
\begin{rem}
\textcolor{amethyst}{Being continuous on a close and bounded(compact) interval.}
\end{rem}
\begin{thm}{Fermat's theorem}{Fermat's Theorem}
    Suppose $f$ is differentiable at $c$ and $f$ has a local max./min. at $c$, then $f'(c) = 0.$
\end{thm}
\begin{rem}
\textcolor{amethyst}{Local extreme value have "horizontal tangent lines".}
\end{rem}
\begin{ex}
    $f(x)=x^3, \ x \in \R.$
\end{ex}
$f'(c) = 3c^2 = 0 \Longleftrightarrow c = 0.$ But $f(0)=0$ is neither a local max nor a local min. \\
So we know that \textcolor{red}{the converse of Fermat's Theorem does not hold!}
\begin{ex}
$f(x)=|x|, \ x\in \R.$
\end{ex}
$f(x) \ge f(0)=0 \ \forall x \in \R$, so $f$ attains a global min at $0$. But $f'(0)$ does not exist.
\begin{rem}
    \textcolor{amethyst}{differentiability is crucial.}
\end{rem}
\begin{ex}
    $f(x)=\dfrac{1}{x}, \ x \in \R_+$
\end{ex}
$f'(x) = -\dfrac{1}{x^2} \neq 0$ on $\R_+$. By Fermat's Theorem, $f$ does not attain any local extereme on $\R_+.$
\begin{pf}[Proof of Fermat's Theorem]
Let $f:U \longrightarrow \R$, $c \in U.$ Suppose $c$ is a local maximum, then $\exists \delta > 0$ such that if $x \in U, \ |x-c|<\delta$, then $f(x) \le f(c).$
\begin{description}
    \item[Case 1($x > c$)] For any $c<x<c+\delta$, we have
    \[
    \frac{f(x)-f(c)}{x-c} \le 0. \ \Rightarrow \lim_{x \to c^+}\frac{f(x)-f(c)}{x-c} \le 0
    \]
    \item[Case 2($x<c$)] For any $c-\delta<x<c$, we have
    \[
    \frac{f(x)-f(c)}{x-c} \ge 0. \Rightarrow \lim_{x \to c^-} \frac{f(x)-f(c)}{x-c} \ge 0.
    \]
\end{description}
Since $f$ is differentiable at $c$,
\[
\textcolor{red}{0 \ge} \lim_{x \to c^+}\frac{f(x)-f(c)}{x-c} = \lim_{x \to c}\frac{f(x)-f(c)}{x-c} =\lim_{x \to c^-}\frac{f(x)-f(c)}{x-c} \textcolor{red}{\ge 0}
\]
$\Rightarrow f'(c)=0.$ Similar argument works if $c$ is a local minimum.
\end{pf}
\begin{df}{Critical Number}{Critical Number}
    For $f: U \longrightarrow \R$, define the critical numbers:
    \[
    Crit(f)=\{c \in U: f'(c)=0\text{ or }f'(c) \text{ doesn't exist}\}
    \]
\end{df}
\begin{prop}
Steps to find global max/min of $f: [a,b] \underset{\text{conti.}}{\longrightarrow} \mathbb{R}$:
\begin{itemize}
    \item[1)] Find $Crit(f)$ in $(a,b).$
    \item[2)] Find $f(a)$ and $f(b).$
    \item[3)] 
    $\max\{f(x): x \in Crit(f) \cup \{a,b\}\} \text{ is the global max.}\\
    \vspace{2pt}
    \min\{f(x): x \in Crit(f) \cup \{a,b\}\} \text{ is the global min.}$
\end{itemize}
\end{prop}
\begin{ex}
    Find the global max and global min of $f:[-1,3] \longrightarrow \R$,
    \[
    f(x) = \begin{cases}
    -x, \hspace{70pt} x \in [-1,0). \\
    \vspace{2pt}
    \sqrt{4-(x-2)^2}, \quad x\in [0,3].
    \end{cases}
    \]
\end{ex}
\begin{itemize}
    \item[\textcolor{blue}{1)}] \textcolor{blue}{$Crit(f) = \{0,2\}$, $f(2)=2, \ f(0) = 0.$}
    \item[\textcolor{blue}{2)}]
    \textcolor{blue}{$f(-1)=1, \ f(3)=\sqrt{3}.$}
\end{itemize}
So $f$ attains its global max value $2$ at $x=2$, and $f$ attains its global min value $0$ at $x=0.$(You can see the picture of the function in next page.)
\begin{figure}[t]
    \centering
    \includesvg[width=\linewidth]{4.1.2.svg}
    \caption{function $f$ in \textbf{Example 2.1.6.}}
    \label{fig:4.1.2}
\end{figure}
\begin{ex}
    $f(x)=x^3-3x^2+1, \ x \in [\frac{-1}{2},4].$
\end{ex}
$f'(x)=3x^2-6x=3x(x-2)$
\begin{itemize}
    \item [1)] $Crit(f) = \{0,2\}, \ f(0)=1, \ f(2)=-3.$
    \item [2)] $f\left(-\dfrac{1}{2}\right) =\dfrac{1}{8}, \ f(4) = 17.$
\end{itemize}
$\Rightarrow$ global max $=17$ at $4$, global min $=0$ at $x=0.$
\newpage
\subsection{TA class week 6}
\begin{pr}
Find the absolute maximum and absolute minimum values of $f(x)=xe^{x-x^2}$ ont the interval $[-2,2].$
\end{pr}
\begin{sol}
We have
\[f'(x) = e^{x-x^2}+xe^{x-x^2}(1-2x) = e^{x-x^2}(-2x^2+x+1).\]
By this we know the critical points are $x \in \{\dfrac{-1 \pm \sqrt{9}}{-8},\ 1, \ -\dfrac{1}{2}\}$. Thus,
\[
\begin{cases}
    f(-2) = 2e^{-2} \red{<1} \\
    \vspace{3pt}
    f\left(-\dfrac{1}{2}\right)=-\dfrac{1}{2}e^{-\frac{3}{4}} \textcolor{red}{<-\dfrac{1}{2}\cdot\dfrac{1}{8}<\dfrac{-1}{16}} \\
    \vspace{3pt}
    f(1)=1 \\
    f(2) = -2e^{-6} \red{>-\dfrac{1}{32}}
\end{cases}
\]
$\Rightarrow \ x=1$ is absolute maximum, while $x=-2$ is absolute minimum.
\end{sol}
\begin{pr}
    Show for $x>0$ that
    \[
    x-\dfrac{x^2}{2}<\log(1+x)<x.
    \]
\end{pr}
\begin{sol}
For $x>0$, then since $f$ is differentiable on $(0,x)$ and thus continuous on $[0,x]$, so by MVT and consider $f(x)=\log(1+x)-(x-\dfrac{x^2}{2})$:
\[
\dfrac{f(x)-f(0)}{x-0}=f'(c)\text{, for some }c \in (0,x)
\]
\begin{description}
    \item[Claim 1]$f'(x)>0$, $\forall x > 0$ \\
    \textbf{Proof:}
    \[
    f'(x) = \dfrac{1}{1+x}-1+x=\dfrac{x^2}{1+x}>0
    \]
    So
    \[
    0 < f'(c) = \dfrac{f(x)-f(0)}{x-0}
    \]
    and by $f(0)=0$ and $x>0$ we can get $f(x)>0$, which is what we want.\\
    Now consider $g(x) = x-\log(1+x)$, similarly:
    \[
    \exists c' \in (0,x) \text{ such that }g'(c') = \dfrac{g(x)-g(0)}{x-0}
    \]
    and also we have:
    \item[Claim 2]$g'(x) > 0$, $\forall x > 0$ \\
    \textbf{Proof:}
    \[
    g'(x) = 1 - \dfrac{1}{1+x} = \dfrac{x}{1+x} >0
    \]
    and the rest of step is same as $f(x)$,
    and we're done.  
\end{description}
\end{sol}
\begin{pr}
Show for $x>0$ that $e^x \ge \dsum{k=0}{n} \dfrac{x^k}{k!}.$(Hint: induction)
\end{pr}
\begin{sol}
We first prove the base case:
\[
e^x \ge e^0 \ge 1 = \dfrac{x^0}{0!} 
\]
Now suppose for all $x>0$ we have $e^x \ge \dsum{k=0}{n}\dfrac{x^k}{k!}\text{, for some }n \in \mathbb{N} \text{ and } 0 \le n \le n'.$ Consider
\[
f(x) = e^x-\dsum{k=0}{n'+1}\dfrac{x^k}{k!}
\]
so by MVT and because
\[
\dfrac{d}{dx}\left(\dfrac{x^k}{k!}\right)=\dfrac{x^{k-1}}{(k-1)!}\ge 0
\]
so $\exists x' \in (0,x)$ such that
\[
f'(x') = \dfrac{f(x)-f(0)}{x-0} = e^{x'} - \dsum{k=0}{n'}\dfrac{x'^k}{k!} \ge 0 \Rightarrow f(x)>0 \Leftrightarrow e^{x} - \dsum{k=0}{n'+1}\dfrac{x^k}{k!} \ge 0.
\]
\end{sol}
\begin{pr}
Let $f(x)$ be a twice-differentiable one-to-one function. Let $g(x) = f^{-1}(x).$ Suppose that $f(2)=1$, $f'(2)=3$, $f''(2)=e.$ Find $g'(1), \ g''(1).$
\end{pr}
\begin{sol}By the definition of inverse function, we have
\par
\centering
$\begin{WithArrows}
\vspace{5pt}
&g(f(x)) = x \Arrow[jump=1, xoffset=-7.3cm, tikz={bend right, '}]{$\dfrac{d}{dx}$}\\
\vspace{5pt}
&g'(f(x))\cdot f'(x) = 1\Arrow[jump=1, xoffset=-7.3cm, tikz={bend right, '}]{$\dfrac{d}{dx}$}\\
&g''(f(x)) \cdot (f'(x))^2 + g'(f(x)) \cdot f''(x) = 0 \\
&\Rightarrow g'(1) \cdot 3=1 \Rightarrow g'(1)=\dfrac{1}{3} \\
&\Rightarrow g''(1) \cdot 9 +\dfrac{1}{3} \cdot e = 0\Rightarrow g''(1) = -\dfrac{e}{27}
\end{WithArrows}$
\end{sol}
\begin{pr}
Suppose $f(x)$ is a continuous function, and that $f(x)$ is differentiable on $(a,x_0) \cup (x_0,b)$. Suppose $f'(x) \rightarrow L$ as $x \to x_0$. Show that $f'(x_0)$ exists and is equal to $L$.
\end{pr}
\begin{sol}
%\begin{description}
   %\item The first step will prove the existence of $f'(x_0)$, while the second step will prove the value of $f'(x_0)$.
   %\item[Step 1:]
   Suppose $x\in (a, x_0) \Rightarrow \exists c \in (x,x_0)$ such that $f'(c) = \displaystyle\frac{f(x) - f(x_0)}{x - x_0}$(By Mean Value Theorem), and take $x \longrightarrow x_0^-$, and then we can obtain
   \[
   \lim_{x \to x_0^-}\frac{f(x) - f(x_0)}{x - x_0} = f'(c)
   \]
   Note that we can get $c \longrightarrow x_0$ since $c \in (x,x_0)$. \\
   Similarly, suppose $x' \in (x_0,b)$ and take $x' \longrightarrow x_0^+$, so by Mean Value Theorem $\exists c' \in (x_0,x')$ such that 
   $$f'(c') = \lim_{x' \to x_0^+} \frac{f(x') - f(x_0)}{x' - x_0}$$
   Note that we can also get $c' \longrightarrow x_0$ since $c' \in (x_0,x')$. \\
   And since
   \begin{align*}
       &\exists c: f'(c) = \lim_{x \to x_0^-}\frac{f(x) - f(x_0)}{x - x_0} = \lim_{h \to 0^-}\frac{f(x_0+h) - f(x_0)}{h} \\
       &\exists c': f'(c') = \lim_{x' \to x_0^+}\frac{f(x') - f(x_0)}{x' - x_0} = \lim_{h \to 0^+}\frac{f(x_0+h) - f(x_0)}{h}
   \end{align*}
   and because $\displaystyle\lim_{x \to x_0}f'(x) = L$. Therefore,
   \begin{align*}
       &L = \lim_{x \to x_0^-}f'(x) = f'(c) \\
       &L = \lim_{x' \to x_0^+}f'(x) = f'(c')
   \end{align*}
   which means
   \[
   \lim_{h \to 0^-}\frac{f(x_0+h) - f(x_0)}{h} = \lim_{h \to 0^+}\frac{f(x_0+h) - f(x_0)}{h} = L
   \]
   By this, we can get $f'(x_0) = L$.
   %\item[Step 2:]
%\end{description}
\end{sol}
\begin{pr}
Suppose $f(x)$ is differentiable on $\R$, $f(0)=0$, and $|f'(x)| \le |f(x)|$ for all $x.$ Show that $f(x)=0$ identically.
\end{pr}
\begin{sol}
Suppose $f(t)=0$ for some $t$, and define $S=\left[t-\dfrac{1}{2}, t+\dfrac{1}{2}\right]$, and by Extreme Value Theorem, we suppose $x=c$ has the absolute maximum in $S$ such that $|f(c)| > |f(x)|$, for all $x$ between $c$ and $t$. Now by MVT we suppose $\exists \ k$ which is between $c$ and $t$ and have
\[
f'(k) = \dfrac{f(c)-f(t)}{c-t}
\]
and we can have:
\begin{description}
    \item[Claim 1]  \red{$|f(k)| > |f(c)|$} \\
    \textbf{Proof} by $|c-t|<1$ and $f(t)=0$, we can get:
    \[
    |f(k)|=|f'(k)| = \left|\dfrac{f(c)-f(t)}{c-t}\right|=\left|\dfrac{f(c)}{c-t}\right|>|f(c)|
    \]
    \item[Claim 2] \red{$|f(k)| \le |f(c)|$} \\
    this is trivial because we suppose $|f(c)| > |f(x)|$ for all $x$ between $c$ and $t$
\end{description}
So by this we get a contradiction and hence know the maximum of $|f(x)|$ should be $0$, which means $f(x)=0$, and we are done.
\end{sol}
\subsection{The Mean Value Theorem}
The most basic version is Rolle's theorem:
\begin{thm}{Rolle's theorem}{Rolle's Theorem}
Suppose
\begin{itemize}
    \item [$(1)$] $f$ is continuous on $[a,b]$
    \item [$(2)$] $f$ is differentiable on $(a,b)$
    \item [$(3)$] $f(a)=f(b)$ 
\end{itemize}
Then $\exists \ c \in (a,b)$ such that $f'(c)=0.$ \\
\red{\textbf{Note: }$c$ is not necessarily unique.}
\end{thm}
\begin{pf}
We have $3$ cases:
    \begin{description}
    \item[Case 1] $f(x) \equiv k$, for all $x \in [a,b]$ \\
    $\Rightarrow f'(x) \equiv 0$, for all $x \in (a,b)$. ($c$ is any $x \in (a,b)$)
    \item[Case 2] $\exists x \in (a,b)$ such that $f(x)>f(a)$ \\
    \textbf{Claim.} $\exists \ c \in (a,b)$ such that $f(c)$ is the global max value of $f$ on $[a,b].$ \\
    \textbf{Proof of claim: }By Extreme Value Theorem, $f$ attains its global max value on $[a,b]$, say at $c \in [a,b].$ If $c=a$, then $f(c)=f(a)<f(x)$, which is a contradiction. Hence, $c \neq a.$ Similarly, $c \neq b$, since $f(a)=f(b).$ Therefore, $c \in (a,b).$ \\
    So by Fermat's theorem, $f'(c)=0$.
    \item[Case 3] $\exists x \in (a,b)$ such that $f(x)<f(a)$ \\
    Similarly as in Case 2, $\exists c \in (a,b)$ which is the global min of $f$ on $[a,b].$ By Fermat's theorem, $f'(c)=0$.
\end{description}
\end{pf}
\begin{ex}
Show that $x^3+x-1=0$ has exactly one root.
\end{ex}
$f(1)=1$, $f(-1)=-3$. By intermediate value theorem, $\exists x_0 \in (-1,1)$ such that $f(x_0)=0$.
Suppose $\exists x \in \R$, $x_1 > x_0$, such that $f(x_1)=0$. Then since $f$ is continuous on $[-1,x_1+1]$ and differentiable on $(-1,x_1+1)$, by Rolle's Theorem $\exists c \in (-1,x_1+1)$ such that $f'(c)=0.$ But $f'(c)=3c^2+1 \ge 1$, which is a contradiction.
\begin{thm}{Mean Value Theorem}{Mean Value Theorem}
Suppose
    \begin{itemize}
        \item[(1)] $f$ is continuous on $[a,b]$.
        \item[(2)] $f$ is differentiable on $(a,b)$
    \end{itemize}
Then $\exists \ c \in (a,b)$ such that
\[
f'(c) = \dfrac{f(b)-f(a)}{b-a}.
\]
\end{thm}
\begin{figure}[h]
    \centering
    \includesvg[width=\linewidth]{mvt.svg}
    \caption{Mean Value Theorem}
    \label{MVT}
\end{figure}
\begin{rem}
If $f(a)=f(b)$, then MVT reduces to Rolle's theorem.
\end{rem}
\begin{pf}
    Let $a=(a,f(a)), \ B = (b,f(b))$. Then
\[
\overleftrightarrow{AB}: \ y-f(a)=\dfrac{f(b)-f(a)}{b-a}(x-a)
\]
Let $h(x)=f(x)-\left(f(a)+\dfrac{f(b)-f(a)}{b-a}(x-a)\right).$ Then
\begin{align*}
&h(a)=f(a)-f(a)=0 \\
&h(b)=f(b)-f(a)-\dfrac{f(b)-f(a)}{b-a}(b-a)=0 \\
\Rightarrow &h(a)=h(b)
\end{align*}
Since $h$ is continuous on $[a,b]$($h$ is the sum of some continuous function) and differentiable on $(a,b)$($h$ is the sum of some differentiable function), by Rolles's theorem.
\[
\exists  \ c \in (a,b) \text{ such that } h'(c)=0.
\]
i.e.
\[
0 = h'(c)=f'(c)-\dfrac{f(b)-f(a)}{b-a} \Longrightarrow f'(c)= \dfrac{f(b)-f(a)}{b-a}
\]
\end{pf}
\begin{rem}
the Mean Value Theorem is the principle to measure the velocity in a distance interval.
\end{rem}
\begin{ex}
    Suppose that $f$ is differentiable on $\R$, $f(0)=-3$ and $f'(x) \le , \ \forall x \in \R.$ How large can $f(2)$ possibly be?
\end{ex}
By Mean Value Theorem, $\exists \ c \in (0,2)$ such that
\begin{align*}
&f'(c) = \dfrac{f(2)-f(0)}{2-0}. \\
&\Rightarrow f(2)-\underbrace{f(0)}_{=-3} = 2\underbrace{f'(c)}_{\le 5} \le 10 \\
&\Rightarrow f(2) \le 10 -3 = 7
\end{align*}
Now we think that instantaneous information(conditions on the derivative) gives global information(the function itself).
\begin{thm}{Constancy theorem}{Constancy Theorem}
Suppose $f$ is continuous on $[a,b]$, and $f'(x)=0, \ \forall x \in (a,b)$. Then $f$ isa constant, i.e. $f(x)=c, \ \forall x \in (a,b)$, for some $c \in \R.$
\end{thm}
\begin{pf}
Choose $x_1,x_2 \in (a,b)$ such that $x_1 < x_2.$ By MVT, $\exists \ d \in (x_1,x_2)$ such that $f'(d) = \dfrac{f(x_2)-f(x_1)}{x_2-x_1}$, by the assumption, we say $f'(d)=0 \ \Rightarrow f(x_2)=f(x_1).$ Since the choice of $x_1, x_2$ is arbitrary, $f(x)=c, \ \forall x \in (a,b).$
\end{pf}
\begin{cl}
Suppose $f,g$ are continuous on $[a,b]$ and differentiable on $(a,b)$, and $f'(x)=g'(x), \ \forall x \in (a,b).$ Then $f \equiv g+c$ for some constant $c \in \R$ on $(a,b).$
\end{cl}
\begin{pf}
Apply Constancy Theorem to $h(x)=f(x)-g(x).$
\end{pf}
\begin{ex}
Prove that:
\[
\text{arctan} x + \arccot x = \dfrac{\pi}{2}, \forall x \in \R.
\]
\end{ex}
let $f(x)= \text{arctan}x+\arccot x.$ Then $f(1)=\dfrac{\pi}{4}+\dfrac{\pi}{4}=\dfrac{\pi}{2}.$
\[
f'(x)=\dfrac{1}{1+x^2}+\dfrac{-1}{1+x^2}=0, \ \forall x \in \R.
\]
By Constancy Theorem, $f(x)=\dfrac{\pi}{2}, \ \forall x \in \R.$
\begin{thm}{Cauchy's MVT}{Cauchy's MVT}
Suppose $f$ and $g$ are
\begin{itemize}
    \item [(1)] continuous on $[a,b]$
    \item [(2)] differentiable on $(a,b)$
\end{itemize}
Then $\exists \ c \in (a,b)$ such that
\[
\big(f(b)-f(a)\big)g'(c)=\big(g(b)-g(a)\big)f'(c)
\]
\end{thm}
\begin{rem}
Take $g(x)=x$, then $g'(x)=1, \forall \ x \in (a,b)$
\[
\Rightarrow f(b)-f(a)=(b-a)f'(c)
\]
is the original MVT.
\end{rem}
\begin{pf}
We have 2 cases:
\begin{description}
    \item[Case 1] $g(a) = g(b)$ \\ 
    By Rolle's theorem, $\exists c \in (a,b)$ such that $g'(c)=0.$ This is as desired.$(\because 0=0)$
    \item[Case 2] $g(a) \neq g(b)$ \\
    Consider $$h(x)=f(x)-f(a)-\dfrac{f(b)-f(a)}{g(b)-g(a)}(g(x)-g(a)).$$ Then $h(a)=h(b)=0$, and then apply Rolle's theorem, and we are done.
\end{description}
\end{pf}
\begin{rem}
slope = $\dfrac{\Delta g}{\Delta f}=\dfrac{g(x+\Delta x)-g(x)}{f(x+\Delta x)-f(x)}=\dfrac{\dfrac{g(x+\Delta x)-g(x)}{\Delta x}}{\dfrac{f(x+\Delta x)-f(x)}{\Delta x}} \xrightarrow[\Delta x \to 0]{} \dfrac{g'(x)}{f'(x)}$
\end{rem}
\newpage
\subsection{L'Hospital's Rule}
\begin{thm}{L'Hospital's rule}{L'Hospital's rule}
    Suppose $f,g: \underbrace{I}_{\text{open}} \subseteq \R \longrightarrow \R$ are differentiable except possibly at $a \in I$. Then if $g'(x) \neq 0, \ \forall x \in I$, $\dlim{x \to a}\dfrac{f'(x)}{g'(x)}$ exists(the limit can be $\pm \infty$), and either
    \begin{itemize}
        \item [(1)] $\dlim{x \to a}f(x)=0=\dlim{x \to a}g(x)$
        \item[(2)] $\dlim{x \to a}f(x)=\pm\infty$, $\dlim{x \to a}g(x)=\pm\infty$
    \end{itemize}
    Then
    \[
    \dlim{x \to a}\dfrac{f(x)}{g(x)} = \dlim{x \to a}\dfrac{f'(x)}{g'(x)}
    \]
\end{thm}
\begin{ex}
$(1) \ \dlim{x \to 0}\dfrac{\sin x}{x}\quad (2) \ \dlim{x \to \infty}\dfrac{e^x}{x^n}, \ n \in \N$
\end{ex}
$(1)$
\[
\dlim{x \to 0}\dfrac{\sin x}{x} \underset{\text{L'Hospital}}{=}  \ \dlim{x \to 0} \dfrac{\cos x}{1} = \cos (0) = 1.
\]
$(2)$
\[
\dlim{x \to \infty}\dfrac{e^x}{x^n} \underset{\text{H}}{=} \ \dlim{x \to \infty}\dfrac{e^x}{nx^{n-1}} \underset{\text{H}}{=} \cdots \undereq{H}\dlim{x \to \infty}\dfrac{e^x}{n!}=+\infty.
\]
This tells us $e^x$ grows faster than polynomials of any order!
\begin{ex}
$\dlim{x \to \infty}\dfrac{\ln x}{x^{\frac{1}{n}}}$, $n \in \N$
\end{ex}
\[
    \dlim{x \to \infty}\dfrac{\ln x}{x^{\frac{1}{n}}} \underset{\text{H}}{=} \dlim{x \to \infty}\dfrac{\frac{1}{x}}{\frac{1}{n}x^{\frac{1}{n}-1}}=\dlim{x \to \infty}\dfrac{1}{\frac{1}{n}x^{\frac{1}{n}}}=0.
\]
This tells us $\ln x$ grows slower than $x^{\frac{1}{n}}, \ \forall n \in \N.$
\begin{ex}
$\dlim{x \to \pi^-}\dfrac{\sin x}{1-\cos x}$
\end{ex}
You should notice that $\dlim{x \to \pi^-}(1-\cos x)=2$, so you cannot use L'Hospital rule here.
\begin{ex}
$\dlim{x \to 0^+} x\ln x$
\end{ex}
You should notice that this is the type of $0 \cdot \infty$
\[
\dlim{x \to 0^+} x \ln x = \dlim{x \to 0^+} \dfrac{\ln x}{\frac{1}{x}} \undereq{H} \dlim{x \to 0^+}\dfrac{\frac{1}{x}}{-\frac{1}{x^2}}=\dlim{x \to 0^+}(-x)=0.
\]
 
\end{document}
