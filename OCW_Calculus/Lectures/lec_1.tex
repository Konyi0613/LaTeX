\chapter{Some Basics}
\section{Introduction}
\lecture{1}{8 July. 21:00}{Introduction}
\begin{definition}
  Suppose we have a function \(f(x)\), then the signed area between \(x=a\) and \(x=b\) is called the definite integral of the function \(f\) on the interval \([a,b]\).  
\end{definition}

\begin{notation}
  \(\int _a^b f(x) \, \mathrm{d}x\) is the definite integral of \(f\) on the interval \([a,b]\).   
\end{notation}

Let \(A(x)\) be the signed area of the \(f(x)\) on the interval \([a,x]\)  of the below figure,

\begin{figure}[H]
  \centering
  \incfig{signed_area}
  \caption{\(y=f(x)\) }
  \label{fig:signed_area}
\end{figure}
then we can draw the figure of \(A(x)\). 

\begin{figure}[H]
  \centering
  \incfig{Ax}
  \caption{A part of \(y=A(x)\).}
  \label{fig:Ax}
\end{figure}

Now we want to show that \(y=A^{\prime} (x)\) is the graph of \(y=f(x)\). Compute
\[
  A^{\prime} (x) = \frac{\mathrm{d}A}{\mathrm{d}x}(x) = \lim_{h \to 0} \frac{A(x+h)-A(x)}{h} =  \lim_{h \to 0} \frac{\text{signed area of } f(x) \text{ on } [x,x+h]}{h} \thickapprox f(x)
\]
since \(h \to 0\), so the area divided by \(h\) is approximately equal to \(f(x)\). 

Also, we know \(\int _a^b f(x) \, \mathrm{d}x = A(b)\), but we only know \(A(a)=0\) and \(A^{\prime} (x)=f(x)\).

Suppose that we have found a function \(g(x)\) such that \(g^{\prime} (x) = f(x)\). What is the relation between \(g\) and \(A\)?

Consider the function \(g(x)-A(x)\), we know its derivative is \(0\). Hence, we know \(g(x) - A(x) = \text{constant}\). Therefore, we have 
\[
  g(b)-g(a) = \left( A(b) + C \right) - \left( A(a) + C \right) = A(b) - A(a) = A(b).  
\] 
By this, we have \(\int _a^b f(x) \, \mathrm{d} x = A(b) = g(b) - g(a)\).

\begin{theorem}
  If \(g^{\prime}  = f\), then \(g(b) - g(a) = \int _a^b f(x) \, \mathrm{d} x \).    
\end{theorem}

\begin{remark}
  Such a function \(g\) is called a primitive (function) (原函數) of \(f\).  
\end{remark}

\begin{remark}
  We have to believe that if \(g^{\prime} (x) = 0\) for some function \(g\), then \(g\) is a constant function to finish the above proof.   
\end{remark}

Now we talk about continuity. If we give a function \(f\) on \([0,\frac{\pi}{2})\). We should notice that not every \(f\) has maximum or minimum value. For example, \(y = \tan (x)\) has no maximum. We say that a function \(f\)  defind on an interval \(I\) is continuous at \(a \in I\) if \(\lim_{x \to a} f(x) = f(a) \). We may think of that if \(f\) is a continuous function defined on a closed interval, then it must have maximum and minimum.

Now if we extend the concept of continuity to a parametrized curve \(f(t)=(x(t),y(t))\), then we say that the parametrized curve is continuous if both \(x(t)\) and \(y(t)\) are, and we may think that whether there is a continuous parametrized curve \(f\)  mapping \(\mathbb{R}\) to \(\mathbb{R} ^2\) such that \(f(\mathbb{R} ) = \mathbb{R} ^2\).  Or, if we think \(\mathbb{R} \) is too scared, then we can think that whether there is a continuous \(f\) mapping \([0,1]\) to a triangle on \(\mathbb{R} ^2\). This result is called Peano's curve.           
