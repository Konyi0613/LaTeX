\lecture{6}{18. July 17:00}{Open Set and Closed Set}
\begin{prev}
    \(S (\subseteq X)\) is called an open set of \(X\) (with respect to \(d\)) if \(\forall x_0 \in S, \ \exists r>0\) such that \(B_r(x_0) \subseteq X\).   
\end{prev}
\begin{figure}[H]
    \centering
    \incfig{openSet}
    \caption{open set}
    \label{fig:openSet}
\end{figure}
\begin{prev}
    \(F (\subseteq X)\) is a closed set of \(X\) if \(X \backslash F \subseteq X\) is open. 
\end{prev}

\begin{eg}
    A set is either an open set or a closed set?
\end{eg}
\begin{explanation}
    No. Consider \((1,3]\), and the metric is the absolute value. 
\end{explanation}
\begin{eg}
    A set can not be open and closed at the same time?
\end{eg}
\begin{explanation}
No. Consider \(\varnothing \). By definition, it is open. Also, the universal space is open, so \(\varnothing \) is closed. 
\end{explanation}

\begin{note}
    Closed ball is close.
\end{note}
\begin{explanation}
    
\end{explanation}

\begin{exercise}
    Let \((X,d)\) be a metric space. Show that
    \begin{enumerate}
        \item \(X\) and \(\varnothing \) are open. 
        \item \(O_1\) and \(O_2\) are open implies \(O_1 \cap O_2\) is open.
        \item Suppose we have \(O_\alpha \subseteq X\) for all \(\alpha  \in A\), and they are all open, then \(\bigcup_{\alpha \in A} O_\alpha  \) is open in \(X\).
        \item What if we change "open" to "closed" in the above 3 statements? Should we make some modification to the statements to make them true?         
    \end{enumerate} 
\end{exercise}

\begin{eg}
    Intersection of infinitely many open set is still open? 
\end{eg}
\begin{explanation}
    Consider 
    \[
        \bigcap_{i=1}^{\infty} \left( -\frac{1}{i}, \frac{1}{i} \right). 
    \]
\end{explanation}

\begin{exercise}
    Show that 
    \begin{align*}
        &\phantom{.=}\lim_{n \to \infty} a_n = L \\
        &\iff \lim_{n \to \infty} d(a_n, L) = 0 \\
        &\iff \forall \text{ open } U \subseteq  X \text{ such that } L \in U, \ \exists N \in \mathbb{N} \text{ such that }  n \ge N \implies a_n \in U
    \end{align*}
\end{exercise}

\begin{eg}
    If we have a metric space and a subspace of this space, can we restrict the metric on this subspace, and this space is still a metric space?
\end{eg}
\begin{explanation}
Yes. This does not violate the definition of metric space.
\end{explanation}

\begin{definition}
    \(S \subseteq X\) is bounded with respect to \(d\) if \(\exists r>0\) and \(x_0 \in X\) such that \(S \subseteq B_r(x_0)\).     
\end{definition}

\begin{theorem}[Bolzano-Weierstrass theorem]\label{thm: bolzano-weierstrass}
    Suppose we have a bounded sequence \(a_n \in \mathbb{R} ^m\), then \(\exists \) a subsequence \(a_{n(m)}\) such that \(a_{n(m)}\) is convergent.   
\end{theorem}
\begin{proof}
    We just talk about the case \(m=2\), and the higher case is similar. Choose \(M>0\) such that \(a_n \in [-M,M] \times [-M,M]\) for all \(n \in \mathbb{N} \). Suppose \(a_n \in [-M,M] \times [-M,M]\) is called \(Q\). Divide \(Q\) into \(4\) squares with equal size, and choose one, say \(Q_1\) such that \(\left\vert \left\{ n \mid a_n \in Q_1 \right\} = \infty   \right\vert \). Select \(n_1 \in \mathbb{N} \) such that \(a_{n_1} \in Q_1\). Repeat this step, that is, divide \(Q_1\) into \(4\) subparts, then says the one subpart with infinite many \(a_n\) in it is \(Q_2\) (\(Q_2\) must exists). Select \(n_2 \in \mathbb{N} \) such that \(a_{n_2} \in Q_2\) and \(n_2 > n_1\). Keep repeating this step, then by \autoref{thm: nested interval} we know
    \[
        \bigcap_{n=1}^{\infty} Q_n \neq \varnothing.
    \] 
    \begin{note}
        Just think of the nested intervals are in \(x\) and \(y\) directions.  
    \end{note}
    Actually, \(\bigcap_{n=1}^{\infty} Q_n = \left\{ a \right\} \) for some \(a \in \mathbb{R} ^2\), otherwise if there are two points in the intersection, then at some moment we will divide them into different subpart, which is a contradiction. It can been seen that \(\lim_{k \to \infty} a_{n_k} = a\).   
\end{proof}

\begin{exercise}
    Suppose \((X,d)\) is a metric space, \(F \subseteq X\). Show that 
    \begin{center}
        \(F\) is closed  \(\iff \) If \(a_n \in F\) and \(\lim_{n \to \infty} a_n = a \in X \), then \(a \in F\).    
    \end{center}
\end{exercise}

\begin{definition}[Open Cover]
    Suppose \((X,d)\) is a metric space, and \(S \subseteq X\). If we have \(O_\alpha \subseteq X\) for \(\alpha \in A\) and they are all open. We say that all \(O_\alpha \)s form an open cover of \(S\) if \(S \subseteq \bigcup_{\alpha \in A} O_\alpha\).       
\end{definition}

\begin{definition}[Compact Set]
    \(S\) is called a compact set if \(\forall \) open cover \(O_\alpha \)(\(\alpha \in A\)) of \(S\), \(\exists \alpha _1, \alpha _2, \dots , \alpha _m \in A\) such that
    \[
        S \subseteq \bigcup_{i=1}^{m} O_{\alpha _i}, \quad  \text{where} \bigcup_{i=1}^{m} O_{\alpha _i}\text{is a finite subcover. }
    \]
\end{definition}

\begin{eg}
    Is \((0,1)\) with normal metric a compact set? 
\end{eg}
\begin{explanation}
    No. Consider
    \[
        \bigcup_{n=1}^{\infty} \left( \frac{1}{n}, 2 \right),
    \] we cannot pick finite many interval to cover \((0,1)\). 
\end{explanation}

\begin{eg}
    Is \((1,\infty )\) a compact set? 
\end{eg}
\begin{explanation}
    No. Consider 
    \[
        \bigcup_{n=1}^{\infty} \left( \frac{1}{2}, n \right). 
    \]
\end{explanation}

\begin{theorem}[The Heine Borel theorem]\label{thm: Heine Borel}
    Let \(S \subseteq \mathbb{R} ^m\), then 
    \begin{center}
        \(S\) is compact \(\iff \) \(S\) is bounded and closed.   
    \end{center} 
\end{theorem}
\begin{proof}[Proof of \(\Leftarrow\)]
    Suppose that \(S\) is bounded and closed, and there is an open cover \(O_\alpha \) of \(S\), which admits no finite subcover. 
    
    First, choose a cube \(Q\) containing \(S\). Divide \(Q\) into \(4\) equal-sized cubes and select one of them, say \(Q_1\), such that \(Q_1 \cap S\) cannot be covered by finitely many \(O_\alpha \). Keep repeating this step and get \(Q_2, Q_3, \dots \). Note that we have 
    \[
        Q_1 \supseteq Q_2 \supseteq \dots .
    \]
    Hence,
    \[
        \bigcap_{n=1}^{\infty} Q_n = \left\{ a \right\} \quad \text{for some }a. 
    \]
    Choose \(s_1 \in Q_1 \cap S\), \(s_2 \in Q_2 \cap S\) and so on, then we know \(\lim_{n \to \infty} s_n = a\) (think of this is also nested intervals). However, the sequence \(s_i\) is in \(S\) and \(S\) is closed, so by the previous exercise we know \(a \in S\). Hence, \(\exists \alpha \) such that \(a \in O_\alpha \) since \(\bigcup_{i=1}^{\infty} O_i\) is a cover of \(S\). Since \(O_\alpha \) is open, so there exists an open ball \(B_r(a) \subseteq Q_\alpha \). However, \(a\) is in many subcubes, so \(\exists n \in \mathbb{N} \) such that \(Q_n \subseteq B_r(a) \subseteq O_\alpha \), and since \(O_n \cap S \subseteq Q_n\), so we know \(Q_n \cap S \subseteq O_\alpha \), which is a contradiction since we suppose that \(Q_i \cap S\) cannot be covered by finitely many \(O_\alpha \).
    \begin{note}
        Note that we need "bounded" and "closed" since we need to use \autoref{thm: nested interval}. 
    \end{note}          
\end{proof}

\begin{exercise}
    How to prove if \(S\) is compact, then \(S\) is bounded and close?  
\end{exercise}