\lecture{5}{16. July 2025}{Metric Space}
\section{Metric Space}
Recall that we use some concepts of distance, that is, the absolute value, which has the following properties:
\begin{enumerate}
    \item \(\vert x \vert \geq 0 \) for all \(x \in \mathbb{R} \) and the equal sign holds only when \(x=0\). 
    \item \(\vert x \vert = \vert -x \vert \) for all \(x \in \mathbb{R} \). 
    \item \(\vert x + y \vert  \le \vert x \vert + \vert y \vert \) for all \(x,y \in \mathbb{R} \).    
\end{enumerate}
In the definition of limit, we use absolute value, and we want to abstract the concept of absolute value. In \(\mathbb{R} ^m\), we know \(x=(x_1, x_2, \dots x_m)\) has \(\vert x \vert = \sqrt{\sum_{j=1}^m \vert x_j \vert^2  }  \), and \(\vert x \vert \) also has the properties above.  Hence, we want to introduce Distance function / metric space (賦距空間). 

Let \(X\) be a set. 
\begin{definition} \label{dfn: metric}
    A function \(X \times X \xrightarrow{d} \mathbb{R} \) is called a distance /metric (function) on \(X\) if 
    \begin{enumerate}
        \item \(\forall x, y \in X\) we have \(d(x,y) \ge 0\) and the equal sign holds only when \(x = y\). 
        \item \(\forall x,y \in X\), we have \(d(x,y) = d(y,x)\). 
        \item \(\forall x,y,z \in X\), we have \(d(x,z) \le d(x,y) + d(y,z)\).       
    \end{enumerate} 
\end{definition} 

\begin{eg}
    In \(\mathbb{R} ^m\), if \(x = (x_1, x_2, \dots , x_m)\) and \(y = (y_1, y_2, \dots , y_m)\), and define
    \[
        d_2(x,y) \coloneqq \sqrt{\vert x_1 - y_1 \vert^2 + \vert x_2 - y_2 \vert^2 + \dots + \vert x_m - y_m \vert^1   },
    \] then \(d_2\) is a metric on \(\mathbb{R} ^m\) by Cauchy inequality.  
\end{eg}

\begin{eg}
    Suppose we define 
    \[
        d_1(x,y) \coloneqq \vert x_1 - y_1 \vert + \vert x_2 - y_2 \vert + \dots + \vert x_n - y_n \vert   
    \] in \(\mathbb{R} ^n\), then \(d_1\) is a metric.  
\end{eg}

\begin{note}
    On a set we can define more than one way of defining metric. For example, \(d_1(x,y)\) and \(d_2(x,y)\).  
\end{note}

\begin{eg}
    \[
        d_\infty \coloneqq \max \left\{ \vert x_1 - y_1 \vert, \vert x_2 - y_2 \vert, \dots , \vert x_n - y_n \vert\right\} 
    \] is also a metric on \(\mathbb{R} ^n\). 
\end{eg}

\begin{eg}
    Suppose \(X\) is a set and \(x,y \in X\), then let 
    \[
        d(x,y) = \begin{dcases}
            0, &\text{ if } x = y ;\\
            1, &\text{ if } x \neq y .
        \end{dcases}
    \] This is called the discrete metric.
\end{eg}

\begin{eg}
    Suppose \(p\) is a prime number, \(x, y \in \mathbb{Q} \), then 
    \[
        \vert x \vert_{p\text{-adic}} \coloneqq p^{-m} \text{ if } x = \frac{a}{b} p^m \text{ with } a,b,m \in \mathbb{Z} \text{ and } \gcd(a,p) = \gcd(b,p) = 1
    \]  and 
    \[
        d_{p\text{-adic}}(x,y) \coloneqq \vert x-y \vert_{p\text{-adic}}, 
    \] which is also a metric.
\end{eg}

Actually now we have 
\begin{enumerate}
    \item \(\vert x+y \vert_{p\text{-adic}} \le \max \left\{ \vert x \vert_{p\text{-adic}}, \vert y \vert_{p\text{-adic}} \right\} \)  
    \item \(\vert x+y \vert_{p\text{-adic}} = \vert x \vert_{p\text{-adic}}  \) if \(\vert x \vert_{p\text{-adic}} > \vert y \vert_{p\text{-adic}} \).  
\end{enumerate}
\begin{intuition}
    All triangles in \(\mathbb{Q} \) are isosceles if we define the distance function as \(p\)-adic distance.   
\end{intuition}

\begin{exercise}
    Suppose \((X,d)\) is a metric space. Show that \(\forall x,y,z \in X\), 
    \[
        \left\vert d(x,y) - d(y,z) \right\vert \le d(x,y). 
    \]  
\end{exercise}

We may generalize the definitions about limits and convergence to metric spaces. 

\begin{definition}
    Let \((X,d)\) be a metric space, \(a_n\) (\(n \in \mathbb{N} \)) be a sequence in \(X\), and \(L\) is an element in \(X\), we say that \(\lim_{n \to \infty} a_n = L \) or \(a_n\) converges to \(L\) or \(L\) is the limit of \(a_n\) as \(n \to \infty \) if 
    \[
        \forall \varepsilon > 0, \ \exists N \in \mathbb{N}  \text{ such that } n \ge N \implies d(a_n, L) < \varepsilon .
    \]          
\end{definition}
\begin{note}
    Notice that 
    \[
        \forall \varepsilon > 0, \ \exists N \in \mathbb{N}  \text{ such that } n \ge N \implies d(a_n, L) < \varepsilon . 
    \]
    is equivalent to 
    \[
        \lim_{n \to \infty} d(a_n, L) = 0. 
    \]
    (\(d(a_n,L)\) is a real number.)
\end{note}
\begin{intuition}
    Given any open ball centered at \(L\), we can find \(N\) such that \(n \ge N\) implies \(a_n\) is in this open ball.   
\end{intuition}
\begin{exercise}
    Can we prove that the limit is unique in a metric space?
\end{exercise}
\begin{exercise}
    In \(\mathbb{R} ^m\), can we prove the basic limit properties hold when the metric is \(d_2\)? (The basic limit properties are something like \(\lim_{n \to \infty} a_n \pm b_n = \lim_{n \to \infty}  a_n \pm \lim_{n \to \infty} b_n \).)  
\end{exercise}

\begin{definition}
    Let \((X,d)\) is a metric space and \(S \subseteq X\). 
    \begin{enumerate}
        \item For \(r > 0\) and \(x_0 \in X\), we let
        \[
            B_r(x_0)=B(x_0,r)=B_{x_0}(r)\coloneqq \left\{ x \in X \mid d(x,x_0) < r \right\},
        \] an open ball. 
        \item \(S\) is an open set (of \((X,d)\)) if
        \[
            \forall x_0 \in S, \ \exists r>0 \text{ such that } B_r(x_0) \subseteq S. 
        \]  
    \end{enumerate}  
\end{definition}
\begin{definition}
   For \(r > 0\) and \(x_0 \in X\), we let
        \[
            \overline{B_r(x_0)} =\overline{B(x_0,r)} =\overline{B_{x_0}(r)} \coloneqq \left\{ x \in X \mid d(x,x_0) \le r \right\},
        \] a closed ball. 
\end{definition}
\begin{definition}
    Let \((X,d)\) be a metric space and \(S \subseteq X\). 
    \begin{enumerate}
        \item \(S\) is a \emph{closed set} (of \((X,d)\)) if \(X \setminus S\) is open; that is,
        \[
            S \text{ is closed } \iff X \setminus S \text{ is open}.
        \]
        \item Equivalently, \(S\) is closed if for every sequence \(\{x_n\} \subseteq S\) that converges to some \(x \in X\), we have \(x \in S\); that is,
        \[
            \forall \{x_n\} \subseteq S, \ x_n \to x \in X \Rightarrow x \in S.
        \]
    \end{enumerate}
\end{definition}

\begin{exercise}
    Suppose \((X,d)\) is a metric space, \(x_0 \in X\), and \(r>0\).  
    \begin{enumerate}
        \item Show that \(B_r(x_0)\) is open. 
        \item \(\left\{ x \in X \mid d(x, x_0) > r \right\} = X \backslash \overline{B_r(x_0)} \) is open. 
    \end{enumerate} 

\end{exercise}