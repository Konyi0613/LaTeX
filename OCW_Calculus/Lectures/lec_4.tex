\lecture{4}{11 July. 9:00}{}
We first finish the proof above.
\begin{exercise}
    Let \(S \subseteq \mathbb{R} \). Prove that if \(\left\vert s - s^{\prime}  \right\vert \le 3 \) for all \(s, s^{\prime} \in S\), then
    \begin{enumerate}
        \item \(S\) is bounded. 
        \item \(\sup S - \inf S \le 3\).  
    \end{enumerate}   
\end{exercise}

\begin{exercise}
    If we change the \(\le\) to \(<\) in the above exercise, is it still correct?  
\end{exercise}

\section{Series}
\begin{definition}
    Let \(a_n\) be a sequence in \(\mathbb{R} \). We say that the series \(\sum_{i=1}^{\infty}  \) converges to / has sum a real number \(S\) if \(\lim_{n \to \infty} s_n = S \), where \(s_n \coloneqq \sum_{i=1}^{n} a_i \), the \(n\)-th partial sum of \(\sum_{i}a_i \).     If such \(S\) exists (resp. does not exist), we say that the series \(\sum a_n \) is convergent (resp. divergent).  
\end{definition}

\begin{eg}
    Suppose \(a_n = \frac{1}{2^n}\), then \(s_n = 1 + \frac{1}{2} + \dots + \frac{1}{2^{n-1} }= \frac{1 - \left( \frac{1}{2} \right)^n }{1 - \frac{1}{2}}\), so \(\lim_{n \to \infty} s_n = \frac{1}{1 - \frac{1}{2}} = 2\).  
\end{eg}

\begin{note}
    However, given a sequence \(a_n\), it is not always possible to write down \(s_n\) explicitly. 
\end{note} 

\begin{eg}
    \(1 + \frac{1}{2} + \frac{1}{3} +\dots + \frac{1}{n}\)=?, \ \(1 - \frac{1}{2} + \frac{1}{3} - \frac{1}{4} + \dots + \frac{(-1)^{n-1} }{n}\) = ?  
\end{eg}

\begin{notation}
    For a series \(\sum_{n} a_n\) and \(l, m \in \mathbb{N} \), \(l < m\), we let
    \[
        s_{l,m} \coloneqq \sum_{i=l}^m a_i,
    \] the \((l,m)\)-tail. 
\end{notation}

\begin{exercise}
    \(\sum_{n} a_n = 0 \) implies \(\lim_{n \to \infty} a_n = 0 \).  
\end{exercise}
\begin{proof}
    We know \(a_n = s_n - s_{n-1}\) and also \(\lim_{n \to \infty} s_n - s_{n-1} = 0\) if \(\sum_{n} a_n \) is convergent. 
\end{proof}

\begin{prev}
 \(\sum_{n} a_n \) converges iff \(s_n\) converges iff \(s_n\) is Cauchy, i.e. \(\forall \varepsilon > 0\), \(\exists N \in \mathbb{N} \) such that \(n, m \ge N\) implies \(\left\vert s_n - s_m \right\vert \le \varepsilon  \).       
\end{prev}
Now if \(n > m\), then \(s_n - s_m = a_{m+1} + \dots + a_n\), so we can rewrite the stamtement as: \(\forall \varepsilon > 0\), \(\exists N \in \mathbb{N} \) such that \(k > N\) and \(l \ge 0\) implies \(\left\vert a_k + \dots + a_{k+l} \right\vert < \varepsilon  \). Hence, there is another equivalent condition of converging, which is: \(\forall \varepsilon > 0\), \(\exists N \in \mathbb{N} \) such that \(k > N\) and \(l \ge 0\) implies \(\vert a_k \vert + \dots + \left\vert a_{k+1} \right\vert < \varepsilon \). If we have this, we can deduce that \(\sum_{n} a_n \) is convergent since by triangle inequality this implies the above Cauchy condition. Also, notice that "\(\forall \varepsilon > 0\), \(\exists N \in \mathbb{N} \) such that \(k > N\) and \(l \ge 0\) implies \(\vert a_k \vert + \dots + \left\vert a_{k+1} \right\vert < \varepsilon \)" is equivalent to \(\sum_{n} \vert a_n \vert  \) is convergent.  Besides, notice that
\begin{center}
    \(\sum_{n} \vert a_n \vert  \) converges if and only if \(\exists M > 0 \forall n \in \mathbb{N} \) we have \(\vert a_1 \vert + \vert a_2 \vert + \dots + \vert a_n \vert \le M\),
\end{center}
since \(\left\{ s_n  \right\} \) is nondecreasing. 

This tells us 
\begin{corollary}
    If \(b_n\) is a positive sequence, then \(\sum_{n} b_n \) converges if and only if \(\exists M > 0\) such that \(\forall n \in \mathbb{N} \) we have \(b_1 + b_2 + \dots + b_n \le M\).   
\end{corollary}
and the important result is 
\begin{corollary}
    If \(\sum_{n} \vert a_n \vert\) converges, then \(\sum_{n} a_n \) converges.  
\end{corollary}

\begin{notation}
    If \(a_n \ge 0\) for all \(n\), then we write \(\sum_{n} a_n < \infty  \) (resp. \(\infty \)) to mean that \(\sum_{n} a_n \) converges (resp. diverges).  
\end{notation}

\begin{eg}
    Is \(\sum_{n=1}^{\infty} \frac{1}{n} \) convergent?
\end{eg}
\begin{explanation}
    Notice that
    \[
        s_4 = 1+ \frac{1}{2} + \frac{1}{3} + \frac{1}{4} > 1 + \frac{1}{2} + \frac{1}{4} + \frac{1}{4}.
    \] 
    Hence, we have \(\forall m \in \mathbb{N} \),
    \[
        s_{2^m} = 1+ \frac{1}{2} + \frac{1}{3} + \frac{1}{4} + \dots + \frac{1}{2^{m-1} + 1} + \dots + \frac{1}{2^{m-1} + 2^{m-1}} > 1 + \frac{m}{2},
    \] so \(s_n\) is unbounded, and thus \(\sum_{n=1}^{\infty} \frac{1}{n} = \infty  \).  
\end{explanation}

\begin{eg}
    Is \(\sum_{n=1}^{\infty} \frac{1}{n^2} \) convergent? 
\end{eg}
\begin{explanation}
    Since 
    \[
        \frac{1}{n^2} < \frac{1}{(n-1)n} = \frac{1}{n-1} - \frac{1}{n},
    \] so we know
    \[
        s_n = 1 + \frac{1}{2^2} + \dots + \frac{1}{n^2} < 1 + \left( \frac{1}{1} - \frac{1}{2} \right) + \left( \frac{1}{2} - \frac{1}{3} \right) + \dots + \left( \frac{1}{n-1} - \frac{1}{n} \right) = 2 - \frac{1}{n} < 2.   
    \]
    Hence, we know \(s_n\) is bounded above and nondecreasing, and thus \(s_n\) is convergent.  
\end{explanation}

\begin{eg}
    Is \(\sum_{n=1}^{\infty} \frac{\sin n}{n^k} \) convergent for some \(k \ge 2\) and \(k \in \mathbb{N} \)?   
\end{eg}

\begin{explanation}
    Since we know 
    \[
        \left\vert \frac{\sin 1}{1^k} \right\vert + \dots + \left\vert \frac{\sin n}{n^k} \right\vert < 1 + \dots + \frac{1}{n^k} < 1 + \dots + \frac{1}{n^2} < 2 \quad \forall n \in \mathbb{N}. 
    \]
    Hence, we know \(\sum_{n=1}^{\infty} \left\vert \frac{\sin n}{n^k} \right\vert < \infty \), and thus \(\sum_{n=1}^{\infty} \frac{\sin n}{n^k} \) converges. 
\end{explanation}

\begin{exercise}
    For \(a > 1\) and \(k \in \mathbb{N} \), show that \(\sum_{n=1}^{\infty} \frac{n^k}{a^n} < \infty  \).   
\end{exercise}

\begin{definition}
    Given a sequence \(a_n\), we say that
    \begin{enumerate}
        \item \(\sum_{n} a_n \) converges absolutely if \(\sum_{n} \vert a_n \vert < \infty \) (and thus \(\sum_{n} a_n \) converges). 
        \item \(\sum_{n} a_n \) converges conditionally if \(\sum_{n} \vert a_n \vert = \infty \) but \(\sum_{n} a_n \) converges.     
    \end{enumerate} 
\end{definition}

\subsection{Comparison Test} \label{subsec: ct}
\begin{theorem}[Comparison Test] \label{thm: Comparison Test}
    Given \(a_n, b_n\) are two non-negative sequences, then if "\(\exists C > 0\) and \(N \in \mathbb{N} \) such that \(n \ge N\) implies \(a_n \le C b_n\)", then "\(\sum_{n} b_n < \infty \) implies \(\sum_{n} a_n < \infty \)".   
\end{theorem}
\begin{proof}
    For \(n \ge N\), we have 
    \begin{align*}
        a_1 + \dots + a_n &= a_1 +\dots + a_N + a_{N+1} + \dots + a_n  \le a_1 + \dots + a_N + C \left( b_{N+1} + \dots + b_n  \right) \\
        &\le a_1 + \dots + a_N + CM \quad \text{for some }M>0. 
    \end{align*}
\end{proof}

\begin{claim}
    If \(a_n,b_n\) are two non-negative sequences, then \(\lim_{n \to \infty} \frac{a_n}{b_n} \) exists implies \(\exists C > 0\) and \(N \in \mathbb{N} \) such that \(n \ge N\) implies \(a_n \le C b_n\).   
\end{claim}

\begin{explanation}
    Let \(l \coloneqq \lim_{n \to \infty} \frac{a_n}{b_n} \) and \(\varepsilon = 1\), then \(\exists N \in \mathbb{N} \) such that \(n \ge N\) implies \(\left\vert \frac{a_n}{b_n} - l \right\vert \le 1 \). Hence,
    \[
        \frac{a_n}{b_n} \le l + 1 \iff a_n \le (l + 1) b_n.
    \]    
\end{explanation}

\begin{corollary}
    If \(\lim_{n \to \infty} \frac{a_n}{b_n} \) exists, then 
    \[
        \sum_{n} b_n < \infty \implies \sum_{n} a_n < \infty. 
    \] 
\end{corollary}

\begin{exercise}
    If \(\lim_{n \to \infty} u_n\) exists, where \(u_n = \sup \left\{ \frac{a_m}{b_m} \mid m \ge n \right\} \), then "\(\exists C > 0\) and \(N \in \mathbb{N} \) such that \(n \ge N\) implies \(a_n \le C b_n\)".  
\end{exercise}
\begin{exercise}[The ratio test]
    Let \(a_n\) be a non-negative sequence. Then show that 
    \[
        \lim_{n \to \infty} \frac{a_{n+1}}{a_n} < 1 \implies \exists N \in \mathbb{N} \text{ and } C < 1 \text{ such that } n \ge N \text{ implies } a_{n+1} \le C a_n,  
    \] and thus \(\sum_{n} a_n \) converges. Besides, prove that  
    \[
        \lim_{n \to \infty} \frac{a_{n+1}}{a_n} > 1 \implies \exists N \in \mathbb{N} \text{ and } C > 1 \text{ such that } n \ge N \text{ implies } a_{n+1} \ge C a_n,  
    \]and thus \(\sum_{n} a_n = \infty  \). However, if \(\lim_{n \to \infty} \frac{a_{n+1}}{a_n} = 1 \), then we cannot conclude anything.  
\end{exercise}
\begin{exercise}[The root test]
    If \(a_n\) is a non-negative sequence, then 
    \[
        \lim_{n \to \infty} \left( a_n \right)^{\frac{1}{n}} < 1 \implies \exists N \in \mathbb{N} \text{ and } C < 1 \text{ such that } n \ge N \text{ implies } a_{n} \le C^n,
    \] and thus \(\sum_{n} a_n \) converges. Besides, prove that 
    \[
        \lim_{n \to \infty} \left( a_n \right)^{\frac{1}{n}} > 1 \implies \exists N \in \mathbb{N} \text{ and } C > 1 \text{ such that } n \ge N \text{ implies } a_{n} \ge C^n,
    \] and thus \(\sum_{n} a_n = \infty  \). 
\end{exercise}

Now we have a question: Does \(\sum_{n=1}^{\infty} \frac{(-1)^{n-1}}{n} \) converges (conditionally)?
\newpage

\begin{definition}
    A series \(\sum_{n} a_n \) is an alternating series if \(\exists b_n > 0\) (\(n \in \mathbb{N} \)) such that
    \[
        a_n = (-1)^{n-1} b_n (n \in \mathbb{N} ).
    \]
\end{definition}

\begin{theorem}[Leibnize's criterion]
    If \(\sum_{n} a_n\) is an alternating series and \(b_n = \vert a_n \vert \) is decreasing and converging to \(0\) as \(n \to \infty \), then \(\sum_{n} a_n \) converges.     
\end{theorem}
\begin{proof}
    Suppose \(b_n = (-1)^{n-1} a_n \)
    \begin{align*}
        \left\vert a_k + \dots + a_{k+l} \right\vert  &= \left\vert  (-1)^{k-1} \left( b_k - b_{k+1} + \dots + (-1)^l b_{k+l} \right) \right\vert \\
        &= b_k - b_{k+1} + \dots + (-1)^l b_{k+l} \\ &= 
        \begin{dcases}
            b_k - (b_{k+1} - b_{k+2}) - \dots - (b_{k + l - 1} - b_{k+l}), &\text{ if } 2 \mid l  ;\\
            b_k - (b_{k+1} - b_{k+2}) - \dots - (b_{k+l-2} - b_{k+l-1}) - b_{k+l}, &\text{ otherwise}.
        \end{dcases} \\
        & \le b_k = \vert a_k \vert 
    \end{align*}
    Since \(\lim_{n \to \infty} b_n = 0 \), so \(\forall \varepsilon > 0 \ \exists N \in \mathbb{N} \) such that \(n \ge N \implies \vert b_n \vert < \varepsilon  \). Hence, if \(k \ge N\), we have 
    \[
        \left\vert a_k + \dots + a_{k+l}  \right\vert \le b_k \le \varepsilon \quad \forall l > 0. 
    \] 
\end{proof}

Given a sequence \(a_n\), we can separate all terms into two sequences. The first sequence is 
\[
    a_{n_1}, a_{n_2}, \dots,
\] while the second one is 
\[
    a_{n_1^{\prime} }, a_{n_2^{\prime} }, \dots 
\] with \(n_1 < n_2 < \dots \) and \(n_1^{\prime} < n_2^{\prime} < \dots \) and \(\left\{ n_1, n_2, \dots \right\} \cup \left\{ n_1^{\prime} , n_2^{\prime} , \dots  \right\}  = \mathbb{N} \) such that \(a_{n_j} \ge 0\) and \(a_{n_k^{\prime} } \le 0\) for all \(j, k \in \mathbb{N} \). Let \(p_j \coloneqq a_{n_j}\) and \(q_k \coloneqq -a_{n_k^{\prime} }\) to construct two non-negative sequences.  

\begin{exercise}
    Show that \(\sum_{n} \vert a_n \vert < \infty  \) iff \(\sum_{j} p_j \) nad \(\sum_{k} q_k \) are both convergent. Moreover, if any side holds, then
    \[
        \sum_{n} \vert a_n \vert = \sum_{j} p_j + \sum_{k} q_k \text{ and } \sum_{n} a_n = \sum_{j} p_j - \sum_{k} q_k.       
    \]   
\end{exercise}

\begin{exercise}
    Suppose \(\sum_{n} a_n \) and \(\sum_{n} b_n \) are both convergent, then 
\[
    \sum_{n} \left( a_n \pm b_n \right) = \sum_{n} a_n \pm \sum_{n} b_n.  
\]
\end{exercise}

\begin{exercise}
    Inserting \(0\)s to a series will not affect its convergence / divergence and its sum if the sum exists. 
\end{exercise}

\subsection{Rearrangement}

\begin{definition}
    We know a sequence is a map from \(\mathbb{N} \) to \(\mathbb{R} \), and a subsequence is a double-map frome \(\mathbb{N} \) to \(\mathbb{N} \) to \(\mathbb{R} \), where the first map from \(\mathbb{N} \) to \(\mathbb{N} \) is increasing. Moreover, a rearrangement is also a double-map from \(\mathbb{N} \) to \(\mathbb{N} \) to \(\mathbb{R} \), where the first map is a bijection.           
\end{definition}

Now we have a question. If \(\sum_{n} a_n \) converges, then 
\begin{enumerate}
    \item is \(\sum_{m} a_{n(m)} \) converges where \(a_{n(m)}\) is a rearrangement of \(a_n\)?
    \item \(\sum_{m} a_{n(m)} = \sum_{n} a_n \)?    
\end{enumerate} 

\begin{note}
    We will prove that if the series is absolutely convergent, then rearrangement does not affect the sum of the infinite series.
\end{note}

\begin{enumerate}
    \item If \(a_n \ge 0\) for \(n \in \mathbb{N} \), the answers are affirmative. 

    \item If \(\sum_{n} \vert a_n \vert < \infty  \) , then the answers are affirmative.
\end{enumerate}
In fact, 1. implies 2 (by the 4-th exercise in \autoref{subsec: ct}). (Why?)

Why does 1. hold? Actually \(\sum_{n} a_n = \sup \left\{ a_{n_1} + a_{n_2} + \dots + a_{n_k} \mid n_1 < \dots < n_k, \ k \in \mathbb{N} \right\}  \) (including the case \(\infty \)) (Why?), and hence 1. holds.

\begin{theorem}[Dirichlet's Rearrangement theorem (1829)]
    If \(\sum_{n} a_n \) converges absolutely, then for every rearrangement \(a_{n(m)}\) we have \(\sum_{m=1}^{\infty} a_{n(m)} = \sum_{n=1}^{\infty} a_n \).   
\end{theorem}

\begin{theorem}[Riemann's rearrangement theorem (1852)]
    If \(\sum_{n} a_n \) converges conditionally, then for every \(L \in \mathbb{R} \) there exists a rearrangement \(a_{n(m)}\) of \(a_n\) such that \(\sum_{m=1}^{\infty} a_{n(m)} = L\).     
\end{theorem}
\begin{proof}
    We will only use two known facts given by the conditional convergence of \(\sum_{n} a_n \): 
    \begin{enumerate}
        \item \(\sum_{j} p_j = \infty  \) and \(\sum_{k} q_k = \infty\). 
        \item \(\lim_{n \to \infty} p_n = \lim_{n \to \infty}q_n = 0 \). (Since \(p_n, q_n\) are subsequence of \(a_n\).)  
    \end{enumerate} 
    Our thought is to add up \(p_1 + p_2 + \dots + p_{m_1}\) so that 
    \[
        p_1 + p_2 + \dots + p_{m_1 - 1} < L < p_1 + p_2 + \dots + p_{m_1}.
    \] Now we start to minus \(q_1 + q_2 + \dots + q_{m_1^{\prime} }\) so that 
    \[
        \sum_{i=1}^{m_1} p_i - \sum_{i=1}^{m_1^{\prime}} q_i < L < \sum_{i=1}^{m_1} p_i - \sum_{i=1}^{m_1^{\prime} - 1} q_i,  
    \] and we continuew to add up some \(p_i\) to make the series bigger than \(L\), and jump back by minusing some \(q_i\). This is feasible since \(L\) is not a upper bound of \(\sum_{n} p_n \) and not a lower bound of \(\sum_{n} q_n \). Note that this method construct many partial sums of some rearrangement, say the partial sum of the rearrangement is \(s_n\), we want to show \(\lim_{n \to \infty} s_n = L \).

\begin{figure}[H]
    \centering
    \incfig{RiemannJump}
    \caption{Riemann jump}
    \label{fig:RiemannJump}
\end{figure}
    
    Since \(\exists\) natural numbers \(m_1 < m_2 < \dots \) and \(m_1^{\prime} < m_2^{\prime} < \dots  \) such that 
    \[
        \left\vert s_{m_{l-1}^{\prime} + m_l + k^{\prime} } - L \right\vert < p_{m_l} \quad \text{if } 0 \le k^{\prime}  < m_l^{\prime} - m_{l-1}^{\prime}. 
    \]

    Similarly,
    \[
        \left\vert s_{m_l + m_{l}^{\prime} + k} - L \right\vert < q_{m_l^{\prime} } \quad \text{if } 0 \le k < m_{l+1} - m_l 
    \] if we start jumping from \(q\). Since we know \(\lim_{n \to \infty} p_{m_l} = \lim_{n \to \infty} q_{m_l^{\prime} } = 0\), so \(\forall \varepsilon > 0, \ \exists N_0 \in \mathbb{N} \) such that \(l \ge N_0\) implies \(p_{m_l}\) and \(q_{m_l^{\prime} } < \varepsilon \). Let \(N = m_{N_0 - 1}^{\prime} + m_{N_0} \). Then \(n \ge N\) implies \(\left\vert s_n - L \right\vert < \varepsilon\).         
\end{proof}

\begin{remark}
    In 1827, Dirichlet made the following observation:
    \begin{align*}
        S &= 1 - \frac{1}{2} + \frac{1}{3} - \frac{1}{4} + \frac{1}{5} - \dots \\ 
        2S &= 2 - 1 + \frac{2}{3} - \frac{2}{4} - \frac{2}{5} + \frac{2}{6} - \dots \\
        &= 1 - \frac{1}{2} + \frac{1}{3} - \frac{1}{4} + \frac{1}{5} - \dots 
    \end{align*}
    by combining some terms.
\end{remark}

\subsubsection{Multiplying absolutely convergent series}
Let \(\sum_{n=0}^{\infty}  a_n \) and \(\sum_{n=0}^{\infty}  b_n \) both converge absolutely.

\begin{theorem}
    Let \(c_n = a_n b_0 + \dots + a_0 b_n = \sum_{\substack{i + k = n \\ i,k \ge 0}} a_i b_k \). Then \(\sum_{n} \vert c_n \vert < \infty\) and
\[
    \sum_{n=0}^{\infty} c_n = \left( \sum_{i=0}^{\infty} a_i  \right) \left( \sum_{k=0}^{\infty} b_k  \right).  
\]  
\end{theorem}
\begin{proof}
    First, for all \(n \in \mathbb{N} \) we know
\[
    \vert c_1 \vert + \vert c_2 \vert + \dots + \vert c_n \vert = \sum_{m=0}^n \left\vert \sum_{\substack{j + k = m \\ j,k \ge 0}} a_j b_k    \right\vert \le \sum_{m=0}^n \sum_{\substack{j + k = m \\ j,k \ge 0}} \vert a_j \vert \vert b_k \vert \le \left( \sum_{j=0}^n \vert a_j \vert   \right) \left( \sum_{k=0}^n \vert b_k \vert   \right) \le MN
\] for some \(M, N\). Hence, \(\sum_{n=0}^{\infty} c_n \) is absolutely convergent.

Let \(A_n \coloneqq a_0 + \dots + a_n\), \(B_n \coloneqq b_0 + \dots + b_n\), \(C_n \coloneqq c_0 + \dots + c_n\). Hence, we want to show
\[
    \lim_{n \to \infty} C_n = \lim_{n \to \infty} A_n \cdot \lim_{n \to \infty} B_n. 
\]  
\begin{claim}
    \(\lim_{n \to \infty} A_n B_n - C_n = 0 \). 
\end{claim}  
\begin{explanation}
First, we know 
\begin{align*}
        0 \le \left\vert A_n B_n - C_n \right\vert  =  \sum_{\substack{j + k > n \\ 0 \le j,k \le n}} \left\vert a_j b_k \right\vert &\le \left( \sum_{j = \lfloor \frac{n}{2}  \rfloor}^n \vert a_j \vert  \right) \left( \sum_{k = 0}^n \vert b_k \vert  \right) + \left( \sum_{j = 0}^n \vert a_j \vert  \right) \left( \sum_{k = \lfloor \frac{n}{2} \rfloor}^n \vert b_k \vert  \right) \\ &\to 0 * M + N * 0
\end{align*}
when \(n \to \infty \). The last part is derived by convergence implies Cauchy, and the former part can be derived by drawing grids in a square representing the choice of \(a_j\) and \(b_k\).
\end{explanation}
\end{proof}

\begin{theorem}
    If \(\sum_{n=1}^{\infty}  a_n \) and \(\sum_{n=1}^{\infty}  b_n \) converges absolutely and \(\mathbb{N} \to \mathbb{N} \times \mathbb{N} \) is a bijection such that \(n \mapsto \left( j(n), k(n) \right) \), and \(c_n \coloneqq a_{j(n)} b_{k(n)}\) for all \(n \in \mathbb{N} \), then \(\sum_{n} \vert c_n \vert < \infty \) and 
    \[
        \sum_{n} c_n = \left( \sum_{n} a_n  \right) \left( \sum_{n} b_n  \right).   
    \]      
\end{theorem}
\begin{proof}
    \(\forall n \in \mathbb{N} \), let \(l = \max \left\{ j(1), j(2), \dots , j(n), k(1), k(2), \dots , k(n)\right\} \). Then
    \begin{align*}
                \vert c_1 \vert + \vert c_2 \vert + \dots + \vert c_n \vert &= \vert a_{j(1)} b_{k(1)} \vert + \vert a_{j(2)} b_{k(2)} \vert + \dots + \vert a_{j(n)} b_{k(n)} \vert \\ &\le \left( \sum_{j=1}^l \vert a_j \vert   \right) \left( \sum_{k=1}^l \vert b_k \vert \right) \le M \times N. 
    \end{align*}
    Hence, \(\sum_{n} \vert c_n \vert\) converges absolutely. Note that since \(\sum_{n} \vert c_n \vert\) converges absolutely, so we can change the order of every term if we want. That is, we can use other bijection to get same value of sum.
    
    Let \(A_n = a_1 + \dots + a_n\), \(B_n = b_1 + \dots + b_n\), and \(C_n = c_1 + \dots + c_n\), then we know 
    \begin{align*}
        A_n B_n &= \left( a_1 + \dots + a_n \right) \left( b_1 + \dots + b_n \right) = \sum_{1 \le j,k \le n} a_j b_k.  
    \end{align*} 
    Replace the bijection \(\mathbb{N} \to \mathbb{N} \times \mathbb{N} \) by: 
\begin{figure}[H]
    \centering
    \incfig{bijcn}
    \caption{New bijection}
    \label{fig:bijcn}
\end{figure}
then we know 
\[
    \sum_{1 \le j,k \le n} a_j b_k = C_{n^2}. 
\]
Hence, we know 
\[
    \lim_{n \to \infty} A_n B_n = \lim_{n \to \infty} C_{n^2} = \lim_{n \to \infty} C_n.   
\]
\end{proof}



