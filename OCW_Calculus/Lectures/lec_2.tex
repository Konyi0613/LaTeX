\section{Upper Bound and Lower Bound}
\lecture{2}{9 July. 08:00}{}
\begin{definition}
	Let \(S \subseteq \mathbb{R} \) and \(r \in \mathbb{R} \). We say that
	\begin{itemize}
		\item \(r\) is a upper (resp. lower) bound of \(S\) if \(\forall s \in S\), \(r \ge\) (resp. \(\le\) ) \(s\). 
		\item \(r\) is the greatest/largest/highest (resp. least/smallest/lowest) element of \(S\) if \(r\) is a upper (resps lower) bound of \(S\) and \(r \in S\). 
		\begin{notation}
			\(r = \max S\) (resp. \(\min S\) ). 
		\end{notation}
		\item   \(r\) is the least upper (greatest lower) bound of \(S\) if \(r = \min \left\{ u \in \mathbb{R} \mid u \text{ is a upper bound of } S \right\} \) (resp. \(r = \max \left\{ l \in \mathbb{R} \mid l \text{ is a lower bound of } S \right\} \) ).
		\begin{notation}
			\(r = \sup S\) (resp. \(\inf S\).) 
		\end{notation}          
	\end{itemize}  
\end{definition}

\begin{remark}
	Every \(r \in \mathbb{R} \) is a upper bound of \(\varnothing \) .
\end{remark}

\begin{note}
\begin{itemize}
	\item 	We write \(\sup S = \infty \) (resp. \(\inf S = \infty \) ) if and only if \(S\) has no upper (resp. lower) bound. If this is the case, we say \(\sup S\) (resp. \(\inf S\) ) doesn't exist.  
	\item \(S\) is bounded above (resp. below) iff \(S\) has a upper (resp. lower) bound.  
\end{itemize}  
\end{note}

\begin{definition}[Dedekind cut]
	Let \(A, B \subseteq \mathbb{R} \). We say that \((A,B)\) is a Dedekind cut (of \(\mathbb{R} \) ) if 
	\begin{enumerate}
		\item \(A \neq \varnothing \neq B\) 
		\item \(A \cup  B = \mathbb{R} \)
		\item \(\forall a \in A, \ b \in B\) we have \(a < b\).
	\end{enumerate}   
	We usually call \(A\) (resp. \(B\)) the lower (resp. upper) part of \((A,B)\).    
\end{definition}

\begin{theorem}\label{thm: Dedekind's gapless}
	From now on (until Professor Chi say stop), we assume that \(\mathbb{R} \) has the following property (Dedekind's gapless property): If \((A,B)\) is a D-cut of \(\mathbb{R} \), then exactly one of the following happens: 
	\begin{enumerate}
		\item \(\max A\) exists but \(\min B\) doesn't;
		\item \(\min B\) exists but \(\max A\) doesn't.    
	\end{enumerate}  
	We call the \(\max A\) in 1. (resp. \(\min B\) in 2. ) the cutting of \((A,B)\).  
\end{theorem}

\begin{exercise}
	We may define Dedekind cuts of \(\mathbb{Q}, \ \mathbb{Z}  \) similarly. Does the Dedekind gapless property hold for \(\mathbb{Q} \) or \(\mathbb{Z} \) ? 

	Hint: Consider \(B = \left\{ x \in \mathbb{Q} \mid x > 0, x^2 > 2 \right\} \). We are allowed to use the fact that there does not exist a rational number \(x\) such that \(x^2=2\).
\end{exercise}
\begin{proof}
We first show that for \(\mathbb{Q} \) this is incorrect. Consider \(B = \left\{ x \in \mathbb{Q} \mid x>0, x^2 > 2 \right\} \), and let \(A = Q \backslash B\). First, notice that \(\min B\) does not exist since if \(b = \min B\), then \(b^2 > 2\). Now we want to construct a \(b - \varepsilon \) such that \((b - \varepsilon )^2 > 2\) with \(\varepsilon \in \mathbb{Q} \) and \(b - \varepsilon > 0\), which is equivalent to 
\[
	b^2 - 2\varepsilon b + \varepsilon ^2 > 2 = b^2 - \delta \iff \delta >  2\varepsilon b - \varepsilon ^2 = \varepsilon (2b - \varepsilon ).
\]         
Hence, we can pick some \(\varepsilon \in \mathbb{Q} \) such that \(\frac{\delta}{2b} > \varepsilon \), then we have 
\[
	\delta > 2b \varepsilon > 2b \varepsilon - \varepsilon ^2,
\]  then we know \(\min B\) does not exist. Notice that by Archimedean Property we must can pick such \(\varepsilon \). Also, by same method, we can show that \(\max A\) does not exist. 

As for \(\mathbb{Z} \), we can let \(A = \left\{ z \in \mathbb{Z} \mid z \le 1 \right\} \) and \(B = \left\{ z \in \mathbb{Z} \mid z > 1 \right\} \), then it can be easily seen that \(\min B\) and \(\max A\) simultaneously exist.    
\end{proof}   

\begin{theorem}[Weierstrass]\label{thm: Weierstrass}
	Let \(\varnothing \neq S \subseteq \mathbb{R} \). If \(S\) has a upper bound, then \(\sup S\) exists.   
\end{theorem}
\begin{proof}
	Let \(B\) be the set of every upper bound of \(S\) and suppose \(A\coloneqq \mathbb{R} \backslash B\). We need to show that \(\min B\) exists. We first show that \((A,B)\) is a Dedekind cut of \(\mathbb{R} \).
	
	Since we know \(S\) is not empty, so we know \(\forall s \in S\), \(s-1\) is not an upper bound, and thus \(s-1 \in A\), which means \(A\) is not empty, and we know \(B\) is not empty by the description of the problem. 
	
	Also, it is trivial that \(A \cup B = \mathbb{R} \). 
	
	For \(a \in A\) and \(b \in B\), we need to show that \(a < b\). Were this false, \(a \ge b\), and hence \(a\) is a upper bound of \(S\) since \(b\) is a upper bound, so \(a \in B\). However, \(A \cap B = \varnothing \). Hence, we have shown that \((A,B)\) is a Dedekind cut of \(\mathbb{R} \). 
	
	Now we want to use Dedekind's gapless property to say that we must have \(\min B\) exists. Were this false, we know \(\max A\) exists, denoted by \(a_0\). Note that \(a_0 \in A\), so \(a_0 \notin B\), and thus \(a_0\) is not an upper bound of \(S\). Hence, there exists \(s_0 \in S\) such that \(s_0 > a_0\), but this implies \(s_0 \notin A\), so \(s_0 \in B\), and thus \(s_0\) is a upper bound of \(S\). Now choose some \(x\) such that \(a_0 < x < s_0\), so \(x \in B\) and thus \(x\) is an upper bound of \(S\), but we have \(x < s_0 \in S\), which is a contradiction.         
\end{proof}

\begin{theorem}[the Archimedean Property]\label{thm: Archimedean}
	Prove the following statement: \(\forall r \in \mathbb{R} \), then \(r > 0\) implies that \(\exists n \in \mathbb{N} \) such that \(\frac{1}{n} < r\).    

	Hint: Rephrase this statement in a way linking it to the upper bound of the set \(S = \mathbb{N} \subseteq \mathbb{R}\).
\end{theorem}
\begin{proof}
	It is equivalent to show that \(\forall \frac{1}{r} \in \mathbb{R} ^+\), \(\exists n \in \mathbb{N} \) such that \(n > \frac{1}{r}\). If not, says \(\frac{1}{r^{\prime} } \in \mathbb{R} ^+\) and for all \(n \in \mathbb{N} \) we have \(\frac{1}{r^{\prime} } > n\). Hence, \(\frac{1}{r^{\prime} }\) is an upper bound of \(\mathbb{N} \). Hence, by Weierstrass we know \(\sup N\) exists. Now we will show that in fact \(\sup \mathbb{N} \) does not exist. Suppose \(\alpha = \sup \mathbb{N} \), then \(\exists m \in \mathbb{N} \) such that \(m > \alpha - 1\), otherwise \(\alpha - 1 < \alpha = \sup \mathbb{N}  \) is also an upper bound of \(\mathbb{N} \)\, which is a contradiction. Thus, we know \(m + 1 > \alpha \) and \(m + 1 \in \mathbb{N} \), but this implies \(\alpha \) is not an upper bound of \(\mathbb{N} \), which is a contradiction. Hence, \(\sup \mathbb{N} \) does not exist.                    
\end{proof}
\section{Sequence}
\begin{definition*}
	We can define the limit of the sequence as follow.
\begin{definition}
		Let \(a_n (n \in \mathbb{N} )\) or \(\left\{ a_n \right\}_{n=1}^{\infty}  \) be a sequence in \(\mathbb{R} \) and \(L \in \mathbb{R} \). We say that \(a_n\) converges to \(L\) (as \(n \to \infty \)) if \(\forall \varepsilon > 0, \ \exists N \in \mathbb{N} \) such that \(n \ge N\) implies \(\left\vert a_n - L \right\vert < \varepsilon  \). 
\end{definition}    

\begin{notation}
	\(\lim_{n \to \infty} a_n = L \). 
\end{notation}

\begin{note}
	If such \(L\) exists, we call it the limit of \(a_n\) and we call \(\left\{ a_n \right\} \) a convergent sequence, otherwise we call it a divergent sequence.   
\end{note}

\begin{definition}
	\(\lim_{n \to \infty} a_n = \infty (-\infty )  \) means \(\forall M \in \mathbb{R} , \ \exists N \in \mathbb{N} \) such that \(n > N\) implies \(a_n \ge (\le) M\).  
\end{definition}
\end{definition*}

\begin{exercise}
\begin{enumerate}	
	\item 	\(\lim_{n \to \infty} a_n = L \) and \(\lim_{n \to \infty} a_n = M\) implies \(L=M\).  
	\item \(\left\{ a_n \right\} \) is convergent implies \(\left\{ a_n \right\} \) is bounded.  
	\item Suppose we have \(\left\{ a_n \right\}, \ \left\{ b_n \right\}  \), and \(a_n \le b_n\) for all \(n \in \mathbb{N} \), and \(\lim_{n \to \infty} a_n = L \), \(\lim_{n \to \infty} b_n = M \), prove that \(L \le M\). What if \(\le\) is replaced by \(<\), is this statement still correct?  
\end{enumerate} 
\end{exercise}
\begin{proof}[proof of 1.]
WLOG, suppose \(L > M\), then we know \(\forall \frac{\varepsilon}{2} > 0\), \(\exists N_1, N_2 \in \mathbb{N} \) s.t. 
\begin{align*}
	n \geq  N_1 &\implies \left\vert a_n - L \right\vert < \frac{\varepsilon}{2} \\
	n \geq  N_2 &\implies \left\vert a_n - M \right\vert < \frac{\varepsilon}{2}.
\end{align*}   
Hence, we know \(n > \max \left\{ N_1, N_2 \right\} \) implies 
\[
	\left\vert L - M \right\vert = \left\vert \left( a_n - L \right) - \left( a_n - M \right)  \right\vert < \left\vert a_n - L \right\vert + \left\vert a_n - M \right\vert < \varepsilon.   
\]
If \(\left\vert L - M \right\vert = \delta >0 \), then we can pick some \(\varepsilon ^{\prime} < \delta \) and then we will get a contradiction. Hence, \(\delta = 0\), which means \(L=M\).    
\end{proof}
\begin{proof}[proof of 2.]
	First, \(\left\{ a_n \right\} \) converges implies \(\forall \varepsilon > 0\), \(\exists N \in \mathbb{N} \) s.t. \(n \ge N\) implies \(\left\vert a_n - L \right\vert < \varepsilon \) for some \(L\). Say, for \(\varepsilon = \varepsilon _1\), the corresponding \(N\) is equal to \(N_1\). Hence, for \(n \ge N_1\), we have \(a_n < L + \varepsilon _1\). Hence, we know \(\max \left\{ a_1, a_2, \dots , a_{N_1 - 1}, L + \varepsilon _1 \right\} \) is an upper bound of \(\left\{ a_n \right\} \). Similarly, we can know \(\min \left\{ a_1, a_2, \dots , a_{N_1 - 1}, L - \varepsilon _1 \right\} \) is a lower bound of \(\left\{ a_n \right\} \).              
\end{proof}
\begin{proof}[proof of 3.]
	First, we know \(\forall \varepsilon > 0\), \(\exists N_1, N_2 \in \mathbb{N} \) such that   
	\begin{align*}
		n \geq N_1 &\implies \left\vert a_n - L \right\vert < \varepsilon \\
		n \geq N_2 &\implies \left\vert b_n - M \right\vert < \varepsilon. 
	\end{align*}
	Now if \(L > M\), say \(L = M + \delta \), where \(\delta > 0\), then we can pick \(\varepsilon < \frac{\delta}{2}\) so that for \(n \ge \max \left\{ N_1, N_2 \right\} \) we have 
	\begin{align*}
		L - \varepsilon &< a_n < L + \varepsilon \\
		M - \varepsilon &< b_n < M + \varepsilon. 
	\end{align*}    
	Also, we know
	\[
		b_n < M + \varepsilon < M + \frac{\delta}{2} = L - \frac{\delta}{2} < L - \varepsilon < a_n,
	\]
	which is a contradiction, so \(L \le M\). 
\end{proof}
\begin{remark}
	Changing or removing finitely many terms in \(a_n\) does not affect \(\left\{ a_n \right\} \) is convergent or not (and its limit).   
\end{remark}

\begin{proposition}
	If \(\lim_{n \to \infty} a_n = L\) and \(\lim_{n \to \infty} b_n = M \), then
	\begin{enumerate}
		\item \(\lim_{n \to \infty} (a_n \pm b_n) = L \pm M\). 
		\item \(\lim_{n \to \infty} a_n \cdot b_n = LM \). 
		\item If \(M \neq 0\), then \(b_n \neq 0\) for all but finitely many \(n\), and \(\lim_{n \to \infty} \frac{a_n}{b_n} = \frac{L}{M} \).       
	\end{enumerate}  
\end{proposition}
\begin{proof}[proof of 1.]
	Consider \(\left\vert (a_n \pm b_n) - (L \pm M) \right\vert \). We can see that 
	\begin{align*}
		\left\vert (a_n \pm b_n) - (L \pm M) \right\vert &= \left\vert (a_n - L) \pm (b_n - M) \right\vert \le \left\vert a_n - L \right\vert + \left\vert b_n - M \right\vert   .
	\end{align*}
	Also, we know \(\forall \varepsilon > 0, \ \exists N_1, N_2 \in \mathbb{N} \) such that \(n \ge N_1 \implies \vert a_n - L \vert < \varepsilon  \) and \(n \ge N_2 \implies \vert b_n - M \vert < \varepsilon  \). Let \(N = \max \left\{ N_1, N_2 \right\} \), then \(n \ge N\) implies 
	\[
		\left\vert (a_n \pm b_n) - (L \pm M) \right\vert \le \left\vert a_n - L \right\vert + \left\vert b_n - L \right\vert < \varepsilon + \varepsilon  = 2\varepsilon .   
	\]     
	Hence, we can pick \(\varepsilon_1 = \frac{\varepsilon}{2} = \varepsilon _2\) so that \(n \ge N_1^{\prime} \implies \vert a_n - L \vert < \varepsilon _1 = \frac{\varepsilon}{2} \) and \(n \ge N_2^{\prime} \implies \vert b_n - M \vert < \varepsilon _2 = \frac{\varepsilon}{2} \), and thus we know \(n \ge \max \left\{ N_1^{\prime} , N_2^{\prime}  \right\} \) implies that 
		\[
		\left\vert (a_n \pm b_n) - (L \pm M) \right\vert \le \left\vert a_n - L \right\vert + \left\vert b_n - L \right\vert < \frac{\varepsilon}{2} + \frac{\varepsilon}{2}  = \varepsilon .   
	\]      
\end{proof}
\begin{proof}[proof of 2.]
	Consider \(\left\vert a_n b_n - LM \right\vert \). Notice that 
	\[
		\left\vert a_n b_n - LM \right\vert = \left\vert a_n b_n - L b_n + Lb_n - LM \right\vert \le \vert a_n - L \vert \vert b_n \vert + \vert L \vert \vert b_n - M \vert.      
	\] 
	If we can choose \(C>0\) such that \(\vert b_n \vert \le C \) for all \(n \in \mathbb{N} \) and \(\vert L \vert \le C \), then  we have 
	\[
		\vert a_n - L \vert \vert b_n \vert + \vert L \vert \vert b_n - M \vert \le C \vert a_n - L \vert + C \vert b_n - M \vert.      
	\]    Hence, we want to find some \(N_1, N_2\) such that \(\forall \varepsilon > 0\), we have \(n \ge \max \left\{ N_1, N_2 \right\} \) implies \(\vert a_n - L \vert \le \frac{\varepsilon}{2C} \) and \(\vert b_n - M \vert \le \frac{\varepsilon}{2C} \), and then we're done. 
	
	Now since \(\left\{ b_n \right\} \) is convergent, so it is bounded and thus we can pick such \(C\) by a previous exercise. 
\end{proof}
\begin{proof}[proof of 3.]
We only need to prove \(\lim_{n \to \infty} \frac{1}{b_n} = \frac{1}{M}\), and removing the terms being zeros will not affect the convergence of this sequence.
\end{proof}

\begin{note}
	What if \(L,M= \pm \infty\)? 
\end{note}

\begin{exercise}
	Suppose \(a_n = \frac{1 - \left( \frac{1}{2} \right)^n }{1 - \frac{1}{2}}\), then \(\lim_{n \to \infty} a_n = \frac{1}{1 - \frac{1}{2}} \).
\end{exercise}
\begin{proof}
	We want to find an \(N \in \mathbb{N} \) so that for all \(n \ge N\) we have
	\[
		\left\vert \frac{1 - \left( \frac{1}{2} \right)^n }{1 - \frac{1}{2}} - \frac{1}{1 - \frac{1}{2}} \right\vert < \varepsilon. 
	\]  
	By simplifying this, we have 
	\[
		\frac{1}{\varepsilon } < 2^{n-1},
	\]
	so we just need to find an \(n \in \mathbb{N} \) such that \(n > 1 - \frac{\ln  \varepsilon }{\ln 2}\), then we are done.  
\end{proof}





