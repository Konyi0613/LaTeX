%%---------------------------------------------------------------
%% AMS-LaTeX Paper
%%---------------------------------------------------------------
\documentclass[12pt]{amsart}
\usepackage{latexsym, amsmath, amscd, amssymb, amsthm}
\usepackage{amssymb,amsmath,amsthm,graphicx,dsfont}
\usepackage{CJK}
\usepackage{hyperref}
 \usepackage{color}
%\usepackage{showkeys}
%%--LAYOUT-------------------------------------------------------
\renewcommand{\baselinestretch}{1}
\renewcommand{\arraystretch}{1.1}
\usepackage{calc}
\setlength\hoffset{0pt}
\setlength\marginparwidth{0pt}
\setlength\marginparsep{0pt}
\setlength\oddsidemargin{0.2in}
\setlength\evensidemargin{\oddsidemargin}
\setlength\textwidth{\paperwidth - 2\oddsidemargin - 2in}
\setlength\voffset{0pt}
\setlength\topmargin{0pt}
\setlength\headheight{1em}
\setlength\headsep{1em}
\setlength\footskip{2em}
\setlength\textheight{\paperheight - 2in - \voffset - \topmargin -
  \headheight - \headsep - \footskip}
%%--MATH---------------------------------------------------------
\DeclareMathOperator{\Ker}{Ker}
%%--OTHER ENVIRONMENTS--------------------------------------
\newtheorem{lemma}{Lemma}[section]
\newtheorem{theorem}[lemma]{Theorem}
\newtheorem{conjecture}[lemma]{Conjecture}
\newtheorem{corollary}[lemma]{Corollary}
\newtheorem{proposition}[lemma]{Proposition}
\newtheorem{problem}[lemma]{Problem}
\theoremstyle{definition}
\newtheorem{definition}[lemma]{Definition}
\newtheorem{notation}[lemma]{Notation}
\newtheorem{example}[lemma]{Example}
\newtheorem{principle}[lemma]{Principle}
\newtheorem{remark}[lemma]{Remark}
\newtheorem{axiom}[lemma]{Axiom}
\newtheorem*{acknowledgements}{Acknowledgments}
\theoremstyle{remark}
\renewcommand{\theequation}%
{\arabic{section}.\arabic{lemma}.\arabic{equation}}
\renewcommand{\labelenumi}{\textbf{(\alph{enumi})}}
%%----------------------------------------------------------


\begin{document}
\begin{CJK}{UTF8}{bkai}

{\centerline{\bf Math 2213 Introduction to Analysis  }}

{\centerline{\bf Homework 2  Due   September 17 (Thursday), 2025}}

{\centerline{\bf Please submit your homework online in PDF format.}}


 
\begin{enumerate}

\item[(1)]  (11 pts) 
If $(X,d)$ is a metric space, define
\[
d'(x,y) = \frac{d(x,y)}{1 + d(x,y)}.
\]
Prove that $d'$ is also a metric on $X$. 

\noindent
Note that $0 \leq d'(x,y) < 1$ for all $x,y \in X$.

\vfill
\bigskip


\item[(2)]  (12 pts) [exercise 1.2.4]
Let $(X,d)$ be a metric space, $x_0$ be a point in $X$, and $r>0$. Let $B$ be the open ball
\[
B := B(x_0,r) = \{ x \in X : d(x,x_0) < r \},
\]
and let $C$ be the closed ball
\[
C := \{ x \in X : d(x,x_0) \leq r \}.
\]

\begin{enumerate}
  \item[(a)] Show that $\overline{B} \subseteq C$.
  \item[(b)] Give an example of a metric space $(X,d)$, a point $x_0$, and a radius $r>0$ such that $\overline{B} \neq C$.
\end{enumerate}


	
\vfill
\bigskip

\item[(3)]  (21 pts) Two metrics $d_1$ and $d_2$ on a set $X$ are said to be \emph{Lipschitz equivalent} if there exist constants $C_1>0$ and $C_2>0$ such that
\[
C_1 d_2(x,y) \leq d_1(x,y) \leq C_2 d_2(x,y) \quad \text{for all } x,y \in X.
\]
Let $E \subset X$.
\begin{enumerate}
  \item[(a)] Prove that $E$ is open in $(X,d_1)$ if and only if $E$ is open in $(X,d_2)$.
  \item[(b)] Prove that $E$ is closed in $(X,d_1)$ if and only if $E$ is closed in $(X,d_2)$.
  \item[(c)] Two metrics $d_1$ and $d_2$ on a set $X$ are said to be \emph{topologically equivalent} if they induce the same topology on $X$. That is, a set $U \subset X$ is open in $(X,d_1)$ if and only if it is open in $(X,d_2)$. Give examples of topologically equivalent metrics that are not Lipschitz equivalent.
\end{enumerate}

\vfill
\bigskip

\item[(4)]  (15 pts) Let $\mathcal{M}_n = M_n(\mathbb{R})$ denote the set of all $n \times n$ real matrices. Define a function on $\mathcal{M}_n \times \mathcal{M}_n$ by
\[
\rho(A,B) = \operatorname{rank}(A-B).
\]
Then $\rho$ is a metric on $\mathcal{M}_n$ and it is topologically equivalent to the discrete metric on $\mathcal{M}_n$.



\vfill
\bigskip

\item[(5)]  (20 pts)  Let $E$ be a subset of a metric space $(X,d)$. Prove the following:
\begin{enumerate}
  \item[(a)] The boundary of $E$ is a closed set.  
  \item[(b)]  $\partial E= \overline{E} \cap    \overline{X \setminus E}$ 
    \item[(c)]  If $E$ is clopen (closed and open), what is $\partial E$?
  \item[(d)] Give an example of $S \subset \mathbb{R}$ such that $\partial(\partial S) \neq \varnothing$, and infer that ``the boundary of the boundary $\partial \circ \partial$ is not always zero.''
\end{enumerate}

\bigskip

\bigskip

\item[(6)] (21pts)  In a metric space $(X,d)$, if subsets satisfy $A \subseteq S \subseteq \overline{A}$, where $\overline{A}$ is the closure of $A$, then $A$ is said to be \emph{dense} in $S$. 
For example, the set $\mathbb{Q}$ of rational numbers is dense in $\mathbb{R}$. 

\begin{enumerate}
  \item[(a)] If $A$ is dense in $S$ and $S$ is dense in $T$, prove that $A$ is dense in $T$.
  
  \item[(b)]  If $A$ is dense in $S$ and if $B$ is open in $S$, prove that
\[
B \subseteq \overline{A \cap B}.
\]
  \item[(c)]  If each of $A$ and $B$ is dense in $S$ and if $B$ is open in $S$, prove that
\[
A \cap B \quad \text{is dense in } S.
\]
\end{enumerate}


\vfill
\bigskip





\end{enumerate}



\end{CJK}

\end{document}    