\begin{problem}[16pts]
  \vphantom{text}
  \begin{enumerate}
  \item[(a)] Let
\[
X := \left\{ (a_n)_{n=0}^\infty : \sum_{n=0}^\infty |a_n| < \infty \right\}
\]
be the space of absolutely convergent sequences. Define the $\ell^1$ and $\ell^\infty$ metrics on this space by
\[
d_{\ell^1}\big((a_n)_{n=0}^\infty,(b_n)_{n=0}^\infty\big)
:= \sum_{n=0}^\infty |a_n - b_n|,
\]
\[
d_{\ell^\infty}\big((a_n)_{n=0}^\infty,(b_n)_{n=0}^\infty\big)
:= \sup_{n\in\mathbb{N}} |a_n - b_n|.
\]

Show that these are both metrics on $X$, but show that there exist sequences 
\[
x^{(1)}, x^{(2)}, \dots
\]
of elements of $X$ (i.e.\ sequences of sequences) which are convergent with respect to the $d_{\ell^\infty}$ metric but not with respect to the $d_{\ell^1}$ metric. Conversely, show that any sequence which converges in the $d_{\ell^1}$ metric automatically converges in the $d_{\ell^\infty}$ metric.


 \item[(b)] Let $(X,d_{\ell^1})$ be the metric space from part (a).  
For each natural number $n$, let $e^{(n)} = (e^{(n)}_j)_{j=0}^\infty$ be the sequence in $X$ such that  
\[
e^{(n)}_j := 
\begin{cases}
1, & \text{if } n=j,\\
0, & \text{if } n\neq j.
\end{cases}
\]

Show that the set
\[
\{ e^{(n)} : n \in \mathbb{N} \}
\]
is a closed and bounded subset of $X$, but is not compact.  

(This is despite the fact that $(X,d_{\ell^1})$ is even a complete metric space---a fact which we will not prove here.  
The problem is not that $X$ is incomplete, but rather that it is ``infinite-dimensional,'' in a sense that we will not discuss here.)

 
  \end{enumerate}
\end{problem}
\begin{proof}[(a)]
  We first show that \(d_{\ell ^1}\) is a metric: 
  \begin{itemize}
    \item For any \((a_n)_{n=0}^{\infty} \in X\), we have 
    \[d_{\ell ^1}\left( (a_n)_{n=0}^{\infty} , (a_n)_{n=0}^{\infty}  \right) = \sum_{n=0}^{\infty} \vert a_n - a_n \vert = 0.   \] 
    \item For any distinct \((a_n)_{n=0}^{\infty}, (b_n)_{n=0}^{\infty} \in X\), we have 
    \[
      d_{\ell ^1} \left( (a_n)_{n=0}^{\infty}, (b_n)_{n=0}^{\infty} \right) = \sum_{n=0}^{\infty} \left\vert a_n - b_n \right\vert > 0.   
    \]
    \item For any \((a_n)_{n=0}^{\infty}, (b_n)_{n=0}^{\infty} \in X\), we have 
    \[
      d_{\ell ^1} \left( (a_n)_{n=0}^{\infty}, (b_n)_{n=0}^{\infty} \right) = d_{\ell ^1} \left( (b_n)_{n=0}^{\infty}, (a_n)_{n=0}^{\infty} \right).  
    \]
    \item For any \((a_n)_{n=0}^{\infty}, (b_n)_{n=0}^{\infty}, (c_n)_{n=0}^{\infty} \in X\), we have 
    \begin{align*} 
      d_{\ell ^1} \left( (a_n)_{n=0}^{\infty} , (c_n)_{n=0}^{\infty}  \right) &= \sum_{n=0}^{\infty} \left\vert a_n - c_n \right\vert \le \sum_{n=0}^{\infty} \left\vert a_n - b_n \right\vert + \left\vert b_n - c_n \right\vert \\
      &= d_{\ell ^1} \left( (a_n)_{n=0}^{\infty} , (b_n)_{n=0}^{\infty}  \right) + d_{\ell ^1} \left( (b_n)_{n=0}^{\infty} , (c_n)_{n=0}^{\infty}  \right).     
    \end{align*} 
  \end{itemize} 
  We then show that \(d_{l^{\infty} }\) is also a metric: 
  \begin{itemize}
    \item For any \((a_n)_{n=0}^{\infty} \in X\), we have 
    \[
      d_{\ell ^{\infty} } \left( (a_n)_{n=0}^{\infty} , (a_n)_{n=0}^{\infty}  \right) = \sup _{n \in \mathbb{N} } \vert a_n - a_n \vert = 0. 
    \]
    \item For any distinct \((a_n)_{n=0}^{\infty}, (b_n)_{n=0}^{\infty} \in X\), we have 
    \[
      d_{\ell ^{\infty} } \left( (a_n)_{n=0}^{\infty} , (b_n)_{n=0}^{\infty}  \right) = \sup _{n \in \mathbb{N} } \vert a_n - b_n  \vert > 0.  
    \]
    \item For any \((a_n)_{n=0}^{\infty}, (b_n)_{n=0}^{\infty} \in X\), we have 
    \[
      d_{\ell ^{\infty} } \left( (a_n)_{n=0}^{\infty} , (b_n)_{n=0}^{\infty}  \right) = d_{\ell ^{\infty} } \left( (b_n)_{n=0}^{\infty} , (a_n)_{n=0}^{\infty}  \right).  
    \]
    \item For any \((a_n)_{n=0}^{\infty}, (b_n)_{n=0}^{\infty}, (c_n)_{n=0}^{\infty} \in X\), we have 
   \begin{align*}
    d_{\ell ^{\infty} } \left( (a_n)_{n=0}^{\infty} , (c_n)_{n=0}^{\infty}  \right) &= \sup _{n \in \mathbb{N} } \vert a_n - c_n \vert \le \sup _{n \in \mathbb{N} } \vert a_n - b_n \vert + \vert b_n - c_n \vert \\
    &\le \sup _{n \in \mathbb{N} } \vert a_n - b_n \vert + \sup _{n \in \mathbb{N} } \vert b_n - c_n \vert \\
    &= d_{\ell ^{\infty} } \left( (a_n)_{n=0}^{\infty} , (b_n)_{n=0}^{\infty}  \right) + d_{\ell ^{\infty} } \left( (b_n)_{n=0}^{\infty} , (c_n)_{n=0}^{\infty}  \right).  
   \end{align*}
  \end{itemize} 
  Now we show that there exists a sequence of \(X\), say \(\left( x^{(n)} \right)_{n=1}^{\infty}  \) s.t. \(\left( x^{(n)} \right)_{n=1}^{\infty}  \) converges with respect to \(d_{\ell ^{\infty} }\) but not to \(d_{\ell ^1}\). Now we let \(\left( x^{(n)} \right)_{n=1}^{\infty}  \) to be
  \[
    x^{(k)}_n = \begin{dcases}
      \frac{1}{k}, &\text{ if } 0 \le n \le k - 1;\\
      0 , &\text{ if } n \ge k .
    \end{dcases}
  \]    
  We first show that \(\left( x_n \right)_{n=1}^{\infty}  \) converges with respect to \(d_{\ell ^{\infty} }\). Note that 
  \[
    d_{\ell ^{\infty} } \left( x^{(p)}, (0) \right) = \left\vert \frac{1}{p} - 0 \right\vert = \frac{1}{p} 
  \] where \((0)\) is the sequence with all entries \(0\). Hence, for every \(\varepsilon > 0\), then there exists \(N > 0\) s.t. \(\frac{1}{N} < \varepsilon \), and thus for all \(p \ge N\), we have 
  \[
    d_{\ell ^{\infty} } \left( x^{(p)}, 0 \right) = \frac{1}{p} \le \frac{1}{N} < \varepsilon. 
  \] Now we show that \(\left( x^{(n)} \right)_{n=1}^{\infty}  \) does not converge with respect to \(d_{\ell ^1}\). Suppose for contradiction, \(\left( x^{(n)} \right)_{n=1}^{\infty}\) converges with respect to \(d_{\ell ^1}\), then \(\left( x^{(n)} \right)_{n=1}^{\infty}\) is a Cauchy sequence in \((X, d_{\ell ^1})\) since every convergent sequence is a Cauchy sequence. Now if \(\left( x^{(n)} \right)_{n=1}^{\infty}\) is a Cauchy sequence, then for all \(\varepsilon > 0\), there exists \(N > 0\) s.t. \(p, q \ge N\) implies \(d_{\ell ^1} \left( x^{(p)}, x^{(q)} \right) < \varepsilon  \). Now if we pick some \(\varepsilon < 1\), and the corresponding \(N\) is \(N_{\varepsilon } \), and let \(q = N_{\varepsilon } \), then we know for all \(p > 2N_{\varepsilon } > N_{\varepsilon } \), we must have 
  \begin{align*}
    1 &> \varepsilon > d_{\ell ^1} \left( x^{(p)}, x^{(N_{\varepsilon } )} \right) \\
    &= \sum_{n=0}^{\infty} \left\vert x_n^{(p)} - x_n^{(N_{\varepsilon } )} \right\vert = \sum_{n=0}^{p} \left\vert x_n^{(p)} - x_n^{(N_{\varepsilon })} \right\vert \\
    &= \sum_{n=0}^{N_{\varepsilon } } \left\vert \frac{1}{p} - \frac{1}{N_{\varepsilon }} \right\vert + \sum_{n=N_{\varepsilon } + 1}^{p} \left\vert \frac{1}{p} - 0 \right\vert \\
    &= N_{\varepsilon } \left( \frac{1}{p} - \frac{1}{N_{\varepsilon }} \right) + \frac{p - N_{\varepsilon } }{p} = 2 - \frac{2N_{\varepsilon } }{p} > 1,  
  \end{align*}  
  which is a contradiction. Hence, \(\left( x_n \right)_{n=1}^{\infty}  \) cannot be Cauchy with respect to \(d_{\ell ^1}\), and thus it does not converge with respect to \(d_{\ell ^1}\). 
  
  Now we show that any sequence converges in the \(d_{\ell ^1}\) metric automatically converges in the \(d_{\ell ^{\infty} }\) metirc. If \(\left( x_n \right)_{n=1}^{\infty}  \) converges to \(y\), then for all \(\varepsilon > 0\), there exists \(N > 0\) s.t. \(k \ge N\) implies 
  \[
    \sum_{n=0}^{\infty} \left\vert x_n^{(k)} - y_n \right\vert < \varepsilon ,  
  \] and thus for all \(k \ge N\), we have \(\sup _{n \in \mathbb{N} } \left\vert x_n^{(k)} - y_n \right\vert < \varepsilon \). Hence, \(\left( x^{(n)} \right)_{n=1}^{\infty}  \) also converges to \(y\) in the \(d_{\ell ^{\infty} }\) metric.  
\end{proof}
\begin{proof}[(b)]
  We first show that \(\left\{ e^{(n)} \right\}_{n=1}^{\infty}  \) is closed. Suppose \(\left\{ e^{(n_j)} \right\}_{j=1}^{\infty} \subseteq \left\{ e^{(n)} \right\}_{n=1}^{\infty} \) converges to some \(y \in X\), then for all \(\varepsilon > 0\), there exists \(N > 0\) s.t. \(k \ge N\) implies \(\sum_{n=0}^{\infty} \left\vert e^{(n_k)}_n - y_n \right\vert < \varepsilon   \). Then we do case analysis: 
  \begin{itemize}
    \item Case 1: \(\left\{ n_k \right\}_{k=1}^{\infty}  \) has no constant tail, that is, there does not exists \(N^{\prime}  > 0\) s.t. \(k \ge N^{\prime} \) implies \(n_k = n_{N^{\prime} }\). If we pick some \(k^{\prime} > k \ge N\) with \(n_k \neq n_{k^{\prime} }\) (we can do this since the sequence has no constant tail), then we will have 
    \begin{align*}
      d_{\ell ^1} \left( e^{n_k}, y \right) &= \sum_{n=0}^{\infty} \left\vert e_n^{(n_k)} - y_n \right\vert = \left\vert 1 - y_{n_k} \right\vert + \sum_{n \neq n_k} \left\vert y_n \right\vert < \varepsilon. 
    \end{align*}
    Hence, we must have \(y_{n_k} = 1\) and \(y_n = 0\) for all \(n \neq n_k\), otherwise the above equation cannot holds for all \(\varepsilon > 0\). However, if we write down the same equation but replace \(n_k\) with \(n_{k^{\prime} }\), that is,
    \[
      d_{\ell ^1} \left( e^{n_{k^{\prime}} }, y \right) = \sum_{n=0}^{\infty} \left\vert e_n^{(n_{k^{\prime} })} - y_n \right\vert = \left\vert 1 - y_{n_{k^{\prime} }} \right\vert + \sum_{n \neq n_{k^{\prime} }} \left\vert y_n \right\vert < \varepsilon, 
    \] then we have \(y_{n_{k^{\prime} }} = 1\) and \(y_{n} = 0\) for all \(n \neq n_{k^{\prime} }\), but this means \(y_{n_k} = 1\) and \(y_{n_k} = 0\) since \(n_k \neq n_{k^{\prime} }\), so this is a contradiction, and so it is impossible that \(\left\{ n_k \right\}_{k=1}^{\infty}  \) has no constant tail if \(\left\{ e^{(n_j)} \right\}_{j=1}^{\infty}  \) converges.     
    \item Case 2: \(\left\{ n_k \right\}_{k=1}^{\infty}  \) has constant tail i.e. there exists \(N^{\prime} >0\) s.t. \(k \ge N^{\prime} \) implies \(n_k = n_{N^{\prime} }\). If so, we will show that \(\left\{ e^{(n_k)} \right\}_{k=1}^{\infty}  \) converges to \(y \in X\) s.t. 
    \[
      y_n = \begin{dcases}
        1, &\text{ if } n = n_{N^{\prime} } ;\\
        0, &\text{ if } n \neq n_{N^{\prime} }.
      \end{dcases}
    \]
    Here we know for all \(\varepsilon > 0\), if \(k \ge N^{\prime} \), then 
    \[
      d_{\ell ^1} \left( e^{(n_k)}, y \right) = \sum_{n=0}^{\infty} \left\vert e_n^{(n_k)} - y_n \right\vert = \left\vert 1 - y_{n_k} \right\vert + \sum_{n \neq n_k}\left\vert y_n \right\vert = 0 + 0 = 0 < \varepsilon   
    \] since for all \(k \ge N^{\prime} \) we have \(n_k = n_{N^{\prime} }\). Now since the limit of a sequence is unique, so \(\left\{ e^{(n_k)} \right\}_{k=1}^{\infty}  \) converges to this \(y\) and does not converge to any other \(y^{\prime} \in X\). Note that \(y \in \left\{ e^{(n)} \right\}_{n=1}^{\infty}  \), so \(\left\{ e^{(n_k)} \right\}_{k=1}^{\infty}  \) converges in \(\left\{ e^{(n)} \right\}_{n=1}^{\infty}  \).      
  \end{itemize} 
  Since we have discussed all cases, so we know if \(\left\{ e^{(n_k)} \right\}_{k=1}^{\infty}  \) converges, then it must converges in \(\left\{ e^{(n)} \right\}_{n=1}^{\infty}  \), which means \(\left\{ e^{(n)} \right\}_{n=1}^{\infty}  \) is closed. 
  
  Now we show that \(\left\{ e^{(n)} \right\}_{n=1}^{\infty}  \) is bounded. Note that 
  \[
    e^{(n)} \in B_{\left( X, d_{\ell ^1} \right) } \left( (0), 1.1 \right) \quad \forall n \in \mathbb{N}  
  \] since 
  \[
    d_{\ell ^1} \left( e^{(n)}, (0) \right) = 1 < 1.1 \quad \forall n \in \mathbb{N} .
  \]
  Hence, we have \(\left\{ e^{(n)} \right\}_{n=1}^{\infty} \subseteq B_{\left( X, d_{\ell ^1} \right) } \left( (0), 1.1 \right) \), and thus \(\left\{ e^{(n)} \right\}_{n=1}^{\infty}  \) is bounded. 
  
  Now we show that \(\left\{ e^{(n)} \right\}_{n=1}^{\infty}  \) is not compact. Consider \(\left\{ e^{(n)} \right\}_{n=1}^{\infty}  \) itself, which is a subsequence of \(\left\{ e^{(n)} \right\}_{n=1}^{\infty}  \). Since it corresponds to the Case 1 above, so it does not converges in \((X, d_{\ell ^1})\), and thus there is a subsequence of \(\left\{ e^{(n)} \right\}_{n=1}^{\infty}  \) that does not converge, and thus \(\left\{ e^{(n)} \right\}_{n=1}^{\infty}  \) is not compact.    
\end{proof}
%────────────────────────────────────────────────────────────────────────────────────────────────────────────────────────────────────────────────────

\begin{problem}[24pts]
  A metric space $(X,d)$ is called \emph{totally bounded} if for every $\varepsilon > 0$, there exists a natural number $n$ and a finite number of balls
\[
B(x^{(1)},\varepsilon), \; B(x^{(2)},\varepsilon), \; \dots, \; B(x^{(n)},\varepsilon)
\]
which cover $X$ (i.e.\ $X = \bigcup_{i=1}^n B(x^{(i)},\varepsilon)$).

\begin{enumerate}
\item[(a)] Show that every totally bounded space is bounded.

\item[(b)] Show the following stronger version of Proposition~1.5.5: if $(X,d)$ is compact, then it is complete and totally bounded.  
\emph{Hint:} if $X$ is not totally bounded, then there is some $\varepsilon > 0$ such that $X$ cannot be covered by finitely many $\varepsilon$-balls.  
Then use Exercise~8.5.20  (on page 182 of Analysis I)to find an infinite sequence of balls $B(x^{(n)},\varepsilon/2)$ which are disjoint from each other. Use this to construct a sequence which has no convergent subsequence.

\item[(c)] Conversely, show that if $X$ is complete and totally bounded, then $X$ is compact.  
\emph{Hint:} if $(x^{(n)})_{n=1}^\infty$ is a sequence in $X$, use the total boundedness hypothesis to recursively construct a sequence of subsequences $(x^{(n;j)})_{n=1}^\infty$ of $(x^{(n)})_{n=1}^\infty$ for each positive integer $j$, such that for each $j$ the elements of the sequence $(x^{(n;j)})_{n=1}^\infty$ are contained in a single ball of radius $1/j$.  
Also ensure that each sequence $(x^{(n;j+1)})_{n=1}^\infty$ is a subsequence of the previous one $(x^{(n;j)})_{n=1}^\infty$.  
Then show that the ``diagonal'' sequence $(x^{(n;n)})_{n=1}^\infty$ is a Cauchy sequence, and then use the completeness hypothesis.
\end{enumerate}
\end{problem}
%────────────────────────────────────────────────────────────────────────────────────────────────────────────────────────────────────────────────────

\begin{problem}[16pts]
  \vphantom{text}
  \begin{enumerate}
  \item[(a)]  A metric space $(X,d)$ is compact if and only if every sequence in $X$ has at least one limit point in $X$.

  
    \item[(b)] 
Let $(X,d)$ have the property that every open cover of $X$ has a finite subcover.  
Show that $X$ is compact.  

\emph{Hint:} If $X$ is not compact, then by part (a) there is a sequence $(x^{(n)})_{n=1}^\infty$ with no limit points.  
Then for every $x \in X$ there exists a ball $B(x,\varepsilon)$ containing $x$ which contains at most finitely many elements of this sequence.  
Now use the hypothesis.
  \end{enumerate}
\end{problem}
\begin{proof}[(a)]
  \vphantom{text}
  \begin{itemize}
    \item [\((\implies )\)] Suppose \((X, d)\) is compact, then for all sequence \(\left\{ a_n \right\}_{n=1}^{\infty} \subseteq X \), we know there exists a subsequence \(\left\{ a_{n_k} \right\}_{k=1}^{\infty}  \) converges to some \(L \in X\). Now we claim that \(L\) is a limit point of \(\left\{ a_n \right\}_{n=1}^{\infty}  \). For all \(\varepsilon > 0\), we know there exists \(N_{\varepsilon } > 0\) s.t. \(k \ge N_{\varepsilon }\) implies \(d \left( a_{n_k}, L \right) < \varepsilon  \). Hence, given any \(\varepsilon > 0\) and \(N > 0\), we know there exists \(k \ge \max \left\{ N_{\varepsilon }, N  \right\} \) s.t. \(d \left( a_{n_k}, L \right) < \varepsilon  \). Note that \(n_k \ge k\)  since \(\left\{ a_{n_k} \right\}_{k=1}^{\infty}  \) is a subsequence of \(\left\{ a_n \right\}_{n=1}^{\infty}  \) and thus \(1 \le n_1 \le n_2 < \dots \). By this, we know \(n_k \ge k \ge N\), and \(d \left( a_{n_k}, L \right) < \varepsilon  \), which means \(L\) is a limit point of \(\left\{ a_n \right\}_{n=1}^{\infty}  \). Thus, every sequence in \(X\) has at least one limit point in \(X\).                      
    \item [\((\impliedby )\)] If every sequence in \(X\) has at least one limit point in \(X\), then consider a sequence \(\left\{ a_n \right\}_{n=1}^{\infty}  \), and suppose \(L\) is a limit point of \(\left\{ a_n \right\}_{n=1}^{\infty}  \). Then for all \(\varepsilon > 0\) and \(N_{\varepsilon } > 0 \), we know there exists \(n_{\varepsilon } > N_{\varepsilon }\) s.t. \(d\left( a_{n_{\varepsilon }}, L \right) < \varepsilon \). Now we construct a subsequence \(\left\{ a_{n_p} \right\}_{p=1}^{\infty}  \) s.t. \(d\left( a_{n_p}, L \right) < \frac{1}{p} \). First, we pick \(\varepsilon = \frac{1}{1}\) and \(N_{1 } = 1 \), then there is a \(n_1 > N_{1} \) s.t. \(d\left( a_{n_1}, L \right) < \frac{1}{1} \). Then this is the \(a_{n_1} \) we want. Next, we pick \(\varepsilon = \frac{1}{2}\) and \(N_2 = n_1 + 1\), then there is a \(n_2 > N_2 > n_1\) s.t. \(d \left( a_{n_2}, L \right) < \frac{1}{2} \). By repeating this step, we can construct \(\left\{ a_{n_p} \right\}_{p=1}^{\infty}  \). Note that \(1 \le n_1 < n_2 < \dots \), so \(\left\{ a_{n_p} \right\}_{p=1}^{\infty}  \) is a subsequence of \(\left\{ a_{n} \right\}_{n=1}^{\infty}  \) and \(d\left( a_{n_p}, L \right) < \frac{1}{p}\) for all \(p \ge 1\). Now we claim that \(\left\{ a_{n_p} \right\}_{p=1}^{\infty}  \) converges to \(L\). For all \(\varepsilon > 0\), we can pick some \(N > 0\) s.t. \(\frac{1}{N} < \varepsilon \), then for all \(k \ge N\), we have 
    \[
      d \left( a_{n_k}, L \right) < \frac{1}{k} < \frac{1}{N} < \varepsilon.
    \]
    Hence, we know \(\left\{ a_{n_p} \right\}_{p=1}^{\infty}  \) converges to \(L\) and thus \((X, d)\) is compact.   
  \end{itemize}
\end{proof}
\begin{proof}[(b)]
  If \((X, d)\) is not compact, then \(\exists \left( x^{(n)} \right)_{n=1}^{\infty}  \) which has no limit point by (a). Thus, for all \(L \in X\) and for all \(\varepsilon > 0\), we know there exists some \(N > 0\) s.t. \(n \ge N\) implies \(d \left( x^{(n)}, L \right) \ge \varepsilon  \). Now for all \(x \in X\), we can pick some \(\varepsilon_x > 0\) so that \(X \subseteq \bigcup_{x \in X} B(x, \varepsilon_x) \), and by the hypothesis given in the problem, we know \(X \subseteq \bigcup_{i=1}^{n} B(x_i, \varepsilon _{x_i})\) for some \(x_i\)'s in \(X\). Now since for every \(1 \le j \le n\), there exists \(N_j > 0\) s.t. \(n \ge N_j\) implies \(d \left( x^{(n)}, x_j \right) \ge \varepsilon_{x_j} \), so \(B(x_j, \varepsilon _{x_j})\) contains at most \(N_j - 1\) points of \(\left( x^{(n)} \right)_{n=1}^{\infty}  \) . Hence, \(\bigcup_{i=1}^{n} B(x_i, \varepsilon _{x_{i} }) \) contains finitely many points of \(\left( x^{(n)} \right)_{n=1}^{\infty}  \). However,
  \[
    \left( x^{(n)} \right)_{n=1}^{\infty} \subseteq X \subseteq \bigcup_{i=1}^{n} B(x_i, \varepsilon _{x_i}),  
  \] so this is a contradiction. Hence, \((X, d)\) is compact.                      
\end{proof}
%────────────────────────────────────────────────────────────────────────────────────────────────────────────────────────────────────────────────────

\begin{problem}[10pts]
  Let $(X,d)$ be a compact metric space. Suppose that $(K_\alpha)_{\alpha \in I}$ is a collection of closed sets in $X$ with the property that any finite subcollection of these sets necessarily has non-empty intersection, thus
\[
\bigcap_{\alpha \in F} K_\alpha \neq \varnothing \quad \text{for all finite } F \subseteq I.
\]
(This property is known as the \emph{finite intersection property}.)  

Show that the entire collection has non-empty intersection, thus
\[
\bigcap_{\alpha \in I} K_\alpha \neq \varnothing.
\]

Show by counterexample that this statement fails if $X$ is not compact.
 
\end{problem}
%────────────────────────────────────────────────────────────────────────────────────────────────────────────────────────────────────────────────────

\begin{problem}[24pts]
  \vphantom{text}
  \begin{enumerate}

  \item[(a)] 
Let $(X,d)$ be a metric space, and let $(E,d|_{E \times E})$ be a subspace of $(X,d)$.  
Let $\iota_{E \to X} : E \to X$ be the inclusion map, defined by setting 
\[
\iota_{E \to X}(x) := x \quad \text{for all } x \in E.
\]  
Show that $\iota_{E \to X}$ is continuous.

 \item[(b)] Let $f : X \to Y$ be a function from one metric space $(X,d_X)$ to another $(Y,d_Y)$.  
Let $E$ be a subset of $X$ (which we give the induced metric $d_X|_{E \times E}$), and let $f|_E : E \to Y$ be the restriction of $f$ to $E$, thus
\[
f|_E(x) := f(x) \quad \text{when } x \in E.
\]  

If $x_0 \in E$ and $f$ is continuous at $x_0$, show that $f|_E$ is also continuous at $x_0$.  
(Is the converse of this statement true? Explain.)  

Conclude that if $f$ is continuous, then $f|_E$ is continuous.  
Thus restriction of the domain of a function does not destroy continuity.  

\emph{Hint: use part (a).}
 
 \item[(c)] 
Let $f : X \to Y$ be a function from one metric space $(X,d_X)$ to another $(Y,d_Y)$.  
Suppose that the image $f(X)$ of $X$ is contained in some subset $E \subseteq Y$ of $Y$.  
Let $g : X \to E$ be the function which is the same as $f$ but with the codomain restricted from $Y$ to $E$, thus $g(x) = f(x)$ for all $x \in X$.  

\medskip
\textbf{Note on codomain:}  
The \emph{codomain} of a function is the declared target set of the function, in contrast to the \emph{image} (or range), which is the set of values the function actually takes.  
So while $f$ is originally defined with codomain $Y$, its values all lie in the smaller set $E \subseteq Y$.  
Therefore, one can equivalently regard $f$ as a function $g : X \to E$.  
The metric on $E$ is the one \emph{induced from $Y$}, i.e.\ $d_Y|_{E \times E}$.

\medskip
Show that for any $x_0 \in X$, $f$ is continuous at $x_0$ if and only if $g$ is continuous at $x_0$.  
Conclude that $f$ is continuous if and only if $g$ is continuous.  

(Thus the notion of continuity is not affected if one restricts the codomain of the function.)

\end{enumerate}
\end{problem}
%────────────────────────────────────────────────────────────────────────────────────────────────────────────────────────────────────────────────────

\begin{problem}[20pts]
  Let $(X,d_X)$ and $(Y,d_Y)$ be metric spaces and $f:X \mapsto Y$ is a function from $X$ to $Y$.
\begin{enumerate}

  \item[(a)] 
  Prove that $f$ is continuous on $X$ if, and only if,
\[
f(\overline{A}) \subseteq \overline{f(A)}
\]
for every subset $A$ of $X$.

  \item[(b)]  Prove that $f$ is continuous on $X$ if, and only if, $f$ is continuous on every compact subset of $X$.  

\textit{Hint:} If $x_n \to p$ in $X$, the set $\{p, x_1, x_2, \dots \}$ is compact.

 \end{enumerate}
\end{problem}