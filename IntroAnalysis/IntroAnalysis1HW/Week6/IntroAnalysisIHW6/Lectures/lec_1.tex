 \begin{problem}[20pts]
    \vphantom{text}
    \begin{definition}[Totally ordered set]\label{def:total-order}
A \emph{totally ordered set} (or \emph{linearly ordered set}) is a pair $(X, \leq)$ consisting of a nonempty set $X$ together with a binary relation $\leq$ on $X$ satisfying the following properties:
\begin{enumerate}
    \item \textbf{Reflexivity:} For all $x \in X$, $x \leq x$.
    \item \textbf{Antisymmetry:} For all $x, y \in X$, if $x \leq y$ and $y \leq x$, then $x = y$.
    \item \textbf{Transitivity:} For all $x, y, z \in X$, if $x \leq y$ and $y \leq z$, then $x \leq z$.
    \item \textbf{Totality (or Comparability):} For all $x, y \in X$, either $x \leq y$ or $y \leq x$.
\end{enumerate}
A relation $\leq$ satisfying only (1)--(3) is called a \emph{partial order}.  
If, in addition, (4) holds, the order is said to be \emph{total}, meaning that any two elements of $X$ can be compared.
\end{definition}


\begin{definition}[Hausdorff space]
A topological space $(X,\mathcal{F})$ is called a \emph{Hausdorff space} (or $T_2$ space) if for every pair of distinct points $x,y \in X$ there exist neighborhoods 
$U,V \in \mathcal{F}$ such that
\[
x \in U, \quad y \in V, \quad \text{and } U \cap V = \varnothing.
\]
\end{definition}
  \begin{enumerate}
  \item[(a)] Given any totally ordered set $X$ with order relation $\le$, declare a set $V \subseteq X$ to be open if for every $x \in V$ there exists a set $I$, which is an interval  
$\{y \in X : a < y < b\}$ for some $a, b \in X$, or  
$\{y \in X : a < y\}$ for some $a \in X$, or  
$\{y \in X : y < b\}$ for some $b \in X$, or the whole space $X$, which contains $x$ and is contained in $V$.  
Let $\mathcal{F}$ be the set of all open subsets of $X$.  
Show that $(X, \mathcal{F})$ is a topology (this is the \emph{order topology} on the totally ordered set $(X, \le)$ which is Hausdorff in the sense of  Definition 2.5.4-2 or the definition above).  


\medskip
\item[(b)] 
Show that on the real line $\mathbb{R}$ (with the standard ordering $\le$), the order topology matches the standard topology (i.e., the topology arising from the standard metric).  

 
 \medskip
\item[(c)] If instead one defines $V$ to be open if the extended real line $\mathbb{R} \cup \{\pm \infty\}$ has an open set with boundary $\{\pm \infty\}$, then $(X, \mathcal{F})$ is a sequence of numbers in $\mathbb{R}$ (and hence in $\mathbb{R}$), show that $x_n \to +\infty$ if and only if $\inf_{n \geq N} x_n \to +\infty$, and $x_n \to -\infty$ if and only if $\sup_{n \geq N} x_n \to -\infty$.

  \end{enumerate}
\end{problem}
\begin{proof}[(a)]
  First note that \(\varnothing , X \subseteq \mathcal{F} \), which is trivial by the definition of \(\mathcal{F} \). Next, we give a claim:
  \begin{claim} \label{claim: 1a1}
    If \(V_1, V_2 \in \mathcal{F} \), then \(V_1 \cap V_2 \in \mathcal{F} \).  
  \end{claim}
  \begin{explanation}
    For all \(x \in V_1 \cap V_2\), there exists \(I_1, I_2\) s.t. \(x \in I_1 \subseteq V_1\) and \(x \in I_2 \subseteq V_2\) and \(I_1 = (a_1, b_1)\) and \(I_2 = (a_2, b_2)\) (\(a_1, b_1, a_2, b_2\) may be \(\pm \infty \) or some element in \(X\), please see following remark).  
    \begin{remark} \label{rmk: 1a1}
      To be convenient, if \(I_1\) or \(I_2\) is \(\left\{ y \in X: a < y < b \right\} \), then we use \((a, b)\) to denote them, and if it is \(\left\{ y \in X: a < y \right\} \), then we use \((a, \infty )\) to denote them, and if it is \(\left\{ y \in X: y < b \right\} \), then we use \((-\infty , b)\) to denote them. Also, if it is the whole \(X\), then we use \((-\infty , \infty )\) to denote. Also, we suppose \(-\infty < c\) and \(\infty > c\) for all \(c \in X\). This is notation may be not formal, but it is useful.       
    \end{remark}
    Now we can pick \(I_3 = I_1 \cap I_2 = \left( \max \left\{ a_1, a_2 \right\}, \min \left\{ b_1, b_2 \right\}   \right) \) (\(\min , \max \) is similarly defined as when \(\le\) is defined in \(\mathbb{R} \).) Hence, we know \(x \in I_3\) and \(I_3 \subseteq V_1 \cap V_2\), so \(V_1 \cap V_2 \subseteq \mathcal{F} \).  

      Note that \(I_3\) is well-defined since \(x \in I_1 \cap I_2\), so \(I_3\) is not empty, and it will not happen that \(\min \left\{ b_1, b_2 \right\} \le \max \left\{ a_1, a_2 \right\}  \).    
  \end{explanation}
  Now if we have \(V_1, V_2, \dots , V_n \in \mathcal{F} \), then by \autoref{claim: 1a1}, we know \(V_1 \cap V_2 \cap \mathcal{F} \), and applying \autoref{claim: 1a1} again, then we know \(V_1 \cap V_2 \cap V_3 \in \mathcal{F} \), then repeating this we have 
  \[
    \bigcap_{i=1}^n V_i \in \mathcal{F} . 
  \]   
  
  Now if we have \(\left\{ V_\alpha  \right\}_{\alpha \in A} \), then for all \(x \in \bigcup_{\alpha \in A} V_\alpha  \), we can pick some \(\alpha_0 \in A\) s.t. \(x \in V_{\alpha _0}\) , and we know there exists \(I_{\alpha _0}\) s.t. \(x \in I_{\alpha _0} \subseteq V_{\alpha _0} \subseteq \bigcup_{\alpha \in A} V_\alpha  \) and \(I_{\alpha _0}\) is an interval, so we know \(\bigcup_{\alpha \in A} V_\alpha \in \mathcal{F}  \). 
  
  By above arguments, we know \((X, \mathcal{F} )\) is a topology. 
\end{proof}
\begin{proof}[(b)]
  Suppose \(\mathcal{F} ^{\prime} \) is the order topology on \(\mathbb{R} \) and \(\mathcal{F} \) is the standard topology on \(\mathbb{R}  \), then if \(V \in \mathcal{F} ^{\prime} \), then for all \(x \in V\), we know there exists intevral \(I\) s.t. \(x \in I \subseteq V\), then similarly we use the notation in \autoref{rmk: 1a1}, which means \(I = (a, b)\), and this time, if \(I \neq X\), then we know 
  \[
    x \in B_{\mathbb{R} }(x, \min \left\{ x - a, b - x \right\} ),
  \] where we define \(\infty - x\) is still \(\infty \) and \(x - \infty \) is \(-\infty \) and \(-\infty - x\) is \(-\infty \) and \(x - (-\infty )\) is \(\infty \), then since \(I \neq X\), so we know \(\min \left\{ x - a, b - x \right\}\) must be some \(r \in \mathbb{R} \), and thus \(B_{\mathbb{R} } (x, \min \left\{ x-a, b-x \right\} )\) is well-defined. In this case, \(V \in \mathcal{F} \). If \(I = X = \mathbb{R} \), then \(\mathbb{R} = I \subseteq V \subseteq \mathbb{R}  \), so \(V = \mathbb{R} \) and thus \(V \in \mathcal{F} \). Thus, we know \(\mathcal{F} ^{\prime} \subseteq \mathcal{F} \).      
  
  Now if \(V \in \mathcal{F} \), then for all \(x \in V\), we know there exists \(r_x > 0\) s.t. \(B_{\mathbb{R} }(x, r_x) \subseteq V\), so 
  \[
    x \in (x - r_x, x + r_x) \subseteq V,
  \] and this means \(V \in \mathcal{F} ^{\prime} \) by definition. Thus, \(\mathcal{F} \subseteq \mathcal{F} ^{\prime} \). 
  
  Thus, we can conclude \(\mathcal{F} = \mathcal{F} ^{\prime} \). 
\end{proof}
\begin{proof}[(c)]
  We first show the \(x_n \to +\infty \) if and only if \(\inf _{n \ge N} x_n \to +\infty \) part: 
  \begin{itemize}
    \item [\((\implies )\)] Now if \(x_n \to +\infty \), then for all \((a, +\infty )\), there exists \(N > 0\) s.t. \(n \ge N\) implies \(x_n \in (a, +\infty )\). Hence, we know \(a < x_n\) for all \(n \ge N\) and thus \(a \le \inf _{n \ge N} x_n\), so we know \[\inf _{n \ge N}x_n \in (a - 1, +\infty ),\] which means \(\inf _{n \ge N} x_n \to +\infty \).          
    \item [\((\impliedby )\)] Now if \(\inf _{n \ge N} x_n \to +\infty \), then for all \((a, +\infty )\), we know there exists \(N_1 > 0\) s.t. \(N \ge N_1\) implies 
    \[
      \inf _{n \ge N} x_n \in (a, +\infty ),
    \] so for all \(n \ge N_1\), we have \(x_n \in (a, +\infty )\), which means \(x_n \to +\infty \).   
  \end{itemize}
  Next, we show that \(x_n \to -\infty \) if and only if \(\sup _{n \ge N} \to -\infty \):
  
\begin{itemize}
  \item [\((\implies)\)] 
  Suppose \(x_n \to -\infty\).  
  Then for all intervals \((-\infty, a)\), there exists \(N > 0\) such that \(n \ge N\) implies \(x_n \in (-\infty, a)\).  
  Hence, we know \(x_n < a\) for all \(n \ge N\), and thus \(\sup_{n \ge N} x_n \le a\).  
  Therefore, 
  \[
    \sup_{n \ge N} x_n \in (-\infty, a + 1),
  \]
  which means \(\sup_{n \ge N} x_n \to -\infty\).
  
  \item [\((\impliedby)\)] 
  Now suppose \(\sup_{n \ge N} x_n \to -\infty\).  
  Then for all intervals \((-\infty, a)\), there exists \(N_1 > 0\) such that \(N \ge N_1\) implies
  \[
    \sup_{n \ge N} x_n \in (-\infty, a).
  \]
  Hence, for all \(n \ge N_1\), we have \(x_n \in (-\infty, a)\), which means \(x_n \to -\infty\).
\end{itemize}  
\end{proof}
%────────────────────────────────────────────────────────────────────────────────────────────────────────────────────────────────────────────────────
\begin{problem}[15pts]
    \vphantom{text}
    \begin{definition}[Metrizable space]\label{def:metrizable}
A topological space $(X,\mathcal{F})$ is said to be \emph{metrizable} if there exists a metric $d : X \times X \to [0,\infty)$ such that the topology $\mathcal{F}$ coincides with the topology $\mathcal{F}_d$ induced by $d$.  
That is,
\[
\mathcal{F} = \mathcal{F}_d := \{\, U \subseteq X : \forall x \in U, \exists\, \varepsilon > 0 \text{ such that } B_d(x,\varepsilon) \subseteq U \,\},
\]
where $B_d(x,\varepsilon) := \{\, y \in X : d(x,y) < \varepsilon \,\}$ denotes the open ball centered at $x$ with radius $\varepsilon$.

\medskip
If no such metric $d$ exists, then $(X,\mathcal{F})$ is said to be \emph{not metrizable}.  
In other words, its topology cannot arise from any metric on $X$.
\end{definition}

\begin{enumerate}
\item[(a)]  Let $X$ be an uncountable set, and let $\mathcal{F}$ be the collection of all subsets $E$ in $X$ which are either empty or cofinite (which means that $X \setminus E$ is finite).  
Show that $(X, \mathcal{F})$ is a topology (this is called the \emph{cofinite topology} on $X$) which is not Hausdorff  and is compact.  

\item[(b)] 
Show that if $\{V_i : i \in I\}$ is any countable collection of open sets containing $x$, then $\bigcap_i V_i \neq \varnothing$.  
Use this to show that the cofinite topology cannot be derived from any metric (i.e., $(X, \mathcal{F})$ is not metrizable).  
(Hint: what is the set $\bigcap_{n=1}^\infty B(x, 1/n)$ equal to in a metric space?)

\end{enumerate}
\end{problem}

\begin{proof}[(a)]
\vphantom{text}
    \begin{claim}
        $(X, \mathcal{F})$ is topology.
    \end{claim}
    \begin{explanation}
        \vphantom{text}
        \begin{itemize}
            \item It is obvious that $\varnothing \in \mathcal{F}$ and $X = X \setminus \varnothing \in \mathcal{F}$
            \item If $u_1, u_2, ... u_n \in \mathcal{F}$, $\bigcap_{i=1}^{n}u_i = X \setminus \bigcup_{i=1}^{n} (X \setminus u_i)$, since for all $i \in [n]$, $(X \setminus u_i)$ is finite, $\bigcup_{i=1}^{n} (X \setminus u_i)$ is also finite, so $\bigcap_{i=1}^{n}u_i = X \setminus \bigcup_{i=1}^{n} (X \setminus u_i) \in \mathcal{F}$.
            \item Given $\{u_i\}_{i \in I} \in F$, $X \setminus \bigcup_{i \in I}u_i = \bigcap_{i \in I} (X \setminus u_i)$, since for all $i \in I$, $(X \setminus u_i)$ is finite, so $\bigcap_{i \in I} (X \setminus u_i)$ is also finite, and hence $\bigcup_{i \in I}u_i \in \mathcal{F}$.
        \end{itemize}
    \end{explanation}
    Then we prove that $(X, \mathcal{F})$ is not Hausdorff and it is compact.
    \begin{itemize}
        \item not Hausdorff: Proof by contradiction. \\
        Suppose it is, then given $x,y \in X, \exists U,V \in \mathcal{F}$ such that $x \in U, y \in V$ and $U \cap V = \varnothing$. \\
        For such $U$ and $V$, $U \neq \varnothing$ and $V \neq \varnothing$. \\
        Since $U, V \in \mathcal{F}$, $X \setminus U$ and $X \setminus V$ are finite, and hence $(X \setminus U) \cup (X \setminus V)$ is also finite. Since $U \cap V = \varnothing$, $X \setminus (U \cap V) = (X \setminus U) \cup (X \setminus V) = X$. However, we know $(X \setminus U) \cup (X \setminus V)$ is finite but $X$ is uncountable so it have infinitely many elements. So we get a contradiction.
        \item Compact:\\
        Given $\{U_{\alpha} : \alpha \in A\} \subseteq \mathcal{F}$ with $X \subseteq \bigcup_{\alpha \in A} U_{\alpha}$, we need to construct a finite set $F \subseteq A$ such that $X \subseteq \bigcup_{\alpha \in F} U_{\alpha}$. \\
        We can do the following construction: \\
        We first find the non-empty member in $\{U_{\alpha}\}_A$, let it $U_1$ and $X \setminus U_1$ is finite set since $U_1 \in \mathcal{F}$, then $\forall x \in X \setminus U_1$, we pick one member $U_x$ in $\{U_{\alpha}\}_A$ such that $x \in U_x$ (since $\{U_{\alpha}\}_A$ is cover so $U_x$ must exist), and collect those $U_x$ to get $V = \{\text{collection of those }U_x\}$, and $V$ is finite large, then we can pick $F = \{U_1\} \cup V$ and it satisfy $X \subseteq \bigcup_{\alpha \in F} U_{\alpha}$, and hence we done.
    \end{itemize}
\end{proof}
\begin{proof}[(b)]
    \vphantom{text} \\
    $\{V_i : i\in I\}$ is countable collection of open sets containing $x$, so $x \in \bigcap_{i \in I} V_i$.
    And we know $X \setminus \bigcap_{i \in I} V_i = \bigcup_{i \in I}(X \setminus V_i)$. \\
    Since we know $V_i$ is cofinite, so $(X \setminus V_i)$ is finite, and $\bigcup_{i \in I}(X \setminus V_i)$ is the union of countable finite set, so $\bigcup_{i \in I}(X \setminus V_i)$ is also countable.
    So we know that $\bigcap_{i \in I} V_i$ is the complement of countable set, so it is uncountable, this means $\{x\} \subsetneq \bigcap_{i \in I} V_i$. \\
    If the cofinite topology are metrizable for some $d$, then for each $x$, $\{B(x, \tfrac{1}{n})\}_{n=1}^{\infty}$ is a countable set of open sets all containing $x$, and from previous proof, we know $\bigcap_{n=1}^{\infty}B(x, \tfrac{1}{n})$ should be uncountable. However, in metric space, $\bigcap_{n=1}^{\infty}B(x, \tfrac{1}{n}) = \{x\}$, so we get a contradiction, and hence the cofinite topology are not metrizable for any $d$.
\end{proof}
%────────────────────────────────────────────────────────────────────────────────────────────────────────────────────────────────────────────────────
\begin{problem}[15pts]
    Let $(X, \mathcal{F})$ be a compact topological space.  
Assume that this space is first countable, which means that for every $x \in X$ there exist countable collections of open sets $V_1, V_2, \ldots$ of neighborhoods of $x$, such that every neighborhood of $x$ contains one of the $V_n$.  
Show that every sequence in $X$ has a convergent subsequence
 (see Exercise 1.5.11).
\end{problem}

\begin{proof}
  Suppose\(\left\{ x_n \right\}_{n=1}^{\infty}\) is a sequence in \(X\), then we can define
  \[
    C_m = \overline{\left\{ x_n : n \ge m \right\} },
  \]  
  and we give some claims.
  \begin{claim} \label{clm: Cm closed}
    \(C_m\) is closed for all \(m \in \mathbb{N} \). 
  \end{claim}
  \begin{explanation}
    Since 
    \[
      C_m = \overline{\left\{ x_n: n \ge m \right\} } = X \setminus \mathrm{Ext}\left( \left\{ x_n: n \ge m \right\}  \right) = X \setminus \mathrm{Int}\left( X \setminus \left\{ x_n: n \ge m \right\}  \right),   
    \] and 
    \[
      \mathrm{Int}\left( X \setminus \left\{ x_n: n \ge m \right\}  \right) = \bigcup_{x \in \mathrm{Int}(X \setminus \left\{ x_n: n \ge m \right\} ) } V_x,  
    \] where \(V_x\) is a neighborhood of \(x\) s.t. \(V_x \subseteq X \setminus \left\{ x_n: n \ge m \right\} \). Thus, \(\mathrm{Int}\left( X \setminus \left\{ x_n: n \ge m \right\}  \right)  \) is open since it is the union of a collection of open sets, and thus \(C_m\) is closed since it is the complement of an open set.    
  \end{explanation}
  \begin{claim} \label{clm: cap Ci neq ems}
    \(\bigcap_{i=1}^{\infty} C_i \neq \varnothing \). 
  \end{claim}
  \begin{explanation}
    Suppose by contradiction, \(\bigcap_{i=1}^{\infty} C_i = \varnothing  \), then 
    \[
      X = X \setminus \bigcap_{i=1}^{\infty} C_i = \bigcup_{i=1}^{\infty} (X \setminus C_i),
    \] and note that \(X \setminus C_i\) is open for all \(i \in \mathbb{N} \) since \(C_i\) is closed for all \(i \in \mathbb{N} \) by \autoref{clm: Cm closed}. Now since \(X\) is compact, and thus we know there exists \(\alpha _1, \alpha _2, \dots , \alpha _n \in \mathbb{N} \) s.t.
    \[
      X \subseteq \bigcup_{i=1}^{n} \left( X \setminus C_{\alpha _i} \right) = X \setminus \left( \bigcap_{i=1}^{n} C_{\alpha _i} \right) .  
    \]  
    Thus, \(\bigcap_{i=1}^{n} C_{\alpha _i} = \varnothing  \). However, note that \(C_{i+1} \subseteq C_i \) for all \(i \in \mathbb{N} \), which is trivial by the definition of \(C_i\). Hence, we know \(\varnothing = \bigcap_{i=1}^{n} C_{\alpha _i} = C_{\alpha _n} \). However,
    \[
      x_{\alpha _n} \in \left\{ x_k: k \ge \alpha _n \right\} \subseteq C_{\alpha _n},
    \] so \(C_{\alpha _n}\) is non-empty, which is a contradiction. 
  \end{explanation}
  Hence, by \autoref{clm: cap Ci neq ems}, we can pick some \(x^{\prime} \in \bigcap_{i=1}^{\infty} C_i \). Now suppose \(A = \left\{ x_n: n \ge \mathbb{N}  \right\} \), then we have two cases: 
  \begin{itemize}
    \item Case 1: \(A\) is a finite set. Then by pigeonhole principle, we know there exists \(e \in A\) s.t. \(e\) appears in \(\left\{ x_n \right\}_{n=1}^{\infty}  \) for infinitely many times. Hence, we can pick a subsequence of \(\left\{ x_n \right\}_{n=1}^{\infty}  \), say \(\left\{ x_{n_i} \right\}_{i=1}^{\infty}  \), and \(x_{n_i} = e\) for all \(i \in \mathbb{N} \). By this pick, we know \(\left\{ x_{n_i} \right\}_{i=1}^{\infty}  \) converges to \(e\), and we're done.          
    \item Case 2: \(A\) is an infinite set. We first give a claim:
    \begin{claim} \label{clm: infinitely many terms in V}
      For any neighborhood of \(x^{\prime} \), say \(V_{x^{\prime} }\), there are infinitely many \(x_n\)'s are contained in \(V_{x^{\prime} }\) i.e. there exists \(\left\{ k_i \right\}_{i=1}^{\infty} \subseteq \mathbb{N}  \) with \(k_i < k_{i+1}\) for all \(i \in \mathbb{N} \) s.t. \(x_{k_i} \in V_{x^{\prime} }\) for all \(i \in \mathbb{N} \).        
    \end{claim}
    \begin{explanation}
      Suppose by contradiction, only for all \(p \in \left\{ p_i \right\}_{i=1}^{n}  \) we have \(x_p \in V_{x^{\prime} }\) and we have \(p_i < p_{i+1}\) for all \(1 \le i \le n\), then we have 
      \[
        V_{x^{\prime} } \cap \left\{ x_k: k \ge p_n + 1 \right\} = \varnothing. 
      \] 
      However, \(x^{\prime} \in \bigcap_{i=1}^{\infty} C_i \subseteq C_{p_n + 1} \), so we know \(V_{x^{\prime} } \cap \left\{ x_k: k \ge p_n + 1 \right\} \neq \varnothing \) by the definition of \(C_{p_n + 1}\). Hence, we have a contradiction, and we're done.      
    \end{explanation}
    Now since \(X\) is first countable, so there exists a countable collection of open sets \(V_1, V_2, \dots \) of neighborhoods of \(x^{\prime} \) s.t. every neighborhood of \(x^{\prime} \) contains one of \(V_n\). Suppose \(U_n \coloneqq \bigcap_{i=1}^{n} V_i \) for all \(n \in \mathbb{N} \), then note that \(\left\{ U_n \right\}_{n=1}^{\infty}  \) is a collection of neighborhood of \(x^{\prime} \), so by \autoref{clm: infinitely many terms in V}, so we know there are infinitely many terms of \(\left\{ x_n \right\}_{n=1}^{\infty}  \) are contained in \(U_k\) for all \(k \in \mathbb{N} \). Now we can construct a convergent subsequence \(\left\{ x_{n_i} \right\}_{i=1}^{\infty}  \) of \(\left\{ x_n \right\}_{n=1}^{\infty}  \) by the following method:
    \begin{itemize}
      \item [1.] Choose \(x_{n_1} \in U_1\) for some \(n_1 \in \mathbb{N} \).  
      \item [2.] Choose some \(n_2  > n_1\) s.t. \(x_{n_2} \in U_2\).   
      \item [3.] Keep choosing \(n_i > n_{i-1}\) s.t. \(x_{n_i} \in U_i\) for all \(i \in \mathbb{N} \) and \(i > 2\).  
    \end{itemize} 
    We know this method is well-defined since for any \(i \in \mathbb{N} \) and \(i \ge 2\), there must exists \(n_i > n_{i-1}\) s.t. \(x_{n_i} \in U_i\), otherwise \(U_i\) contains only finitely many terms of \(\left\{ x_n \right\}_{n=1}^{\infty}  \), which is a contradiction. By this method, we can show that \(\left\{ x_{n_i} \right\}_{i=1}^{\infty}  \) converges to \(x^{\prime} \). For all neighborhood of \(x^{\prime}\), say \(W\), then by first countable property, we know there exists \(i \in \mathbb{N} \) s.t. \(V_i \subseteq W\), so we know \(U_i \subseteq V_i \subseteq W\). Hence, for all \(k \ge i\), we know 
    \[
      x_{n_k} \in U_k \subseteq U_i \subseteq V_i \subseteq W,
    \] and we're done.
  \end{itemize} 
\end{proof}
%────────────────────────────────────────────────────────────────────────────────────────────────────────────────────────────────────────────────────
\begin{problem}[15pts]
    Let $(X,\mathcal{F})$ be a compact topological space and $(Y,\mathcal{G})$ be a Hausdorff topological space. 
If $f:X\to Y$ is continuous, then $f$ is a \emph{closed map}; i.e., for every closed
subset $F\subseteq X$, the image $f(F)$ is closed in $Y$.
\end{problem}
\begin{proof}
  \begin{claim} \label{claim: 41}
    For all closed \(F \subseteq X\), \(F\) is compact.  
  \end{claim}
  \begin{explanation}
    Suppose \(\left\{ V_\alpha  \right\}_{\alpha \in A} \) is an open cover of \(F\), and since \(X \setminus F\) is open, so for all \(x \in X \setminus F\), there exists a neighborhood of \(x\), \(U_x\) s.t. \(U_x \subseteq X \setminus F\). Thus, we know 
    \[
      X \setminus F = \bigcup_{x \in X \setminus F} U_x,
    \] and thus 
    \[
      X = F \cup (X \setminus F) = \left( \bigcup_{\alpha \in A} V_\alpha   \right) \cup \left( \bigcup_{x \in X \setminus F} U_x  \right), 
    \] so \(\left\{ V_\alpha  \right\}_{\alpha \in A} \cup \left\{ U_x \right\}_{x \in X \setminus F}  \) is an open cover of \(X\). Now since \(X\) is compact, so there exists \(\alpha _1, \alpha _2, \dots , \alpha _n \in A\) and \(x_1, x_2, \dots , x_m \in X \setminus F\) s.t. 
    \[
      X \subseteq \left( \bigcup_{i=1}^{n} V_{\alpha _i}  \right) \cup \left( \bigcup_{i=1}^{m} U_{x_i} \right).  
    \]  
    Now since \(\bigcup_{i=1}^{m} U_{x_i} \subseteq X \setminus F \), so we know \(F \subseteq \bigcup_{i=1}^{n} V_{\alpha _i} \), which shows \(F\) is compact.   
  \end{explanation}
  Hence, for all \(F \subseteq X\), since \(F\) is compact by \autoref{claim: 41} and \(f\) is continuous, so \(f(F)\) is comapct in \(Y\). Now we show that \(f(F)\) is closed in \(Y\). Thus, we want to show \(Y \setminus f(F)\) is open. Now suppose \(y \in Y \setminus f(F)\), then for all \(x \in f(F)\), there exists a neighborhood of \(x\), \(V_x \in \mathcal{G} \), and a neighborhood of \(y\), \(U_x \in \mathcal{G} \) s.t. \(U_x \cap V_x = \varnothing \) since \((Y, \mathcal{G} )\) is Hausdorff. Note that \(U_x \subseteq Y \setminus V_x\) for all \(x \in X\). Also, we know 
  \[
    f(F) \subseteq  \bigcup_{x \in f(F)} V_x, 
  \] and since \(f(F)\) is compact, so there exists \(x_1, x_2, \dots , x_n \in f(F)\) s.t. 
  \[
    f(F) \subseteq \bigcup_{i=1}^{n} V_{x_i}. 
  \] Hence, we have 
  \[
    \bigcap_{i=1}^{n} (Y \setminus V_{x_i}) = Y \setminus \bigcup_{i=1}^{n} V_{x_i} \subseteq Y \setminus f(F),  
  \] and since \(U_x \subseteq Y \setminus V_x\) for all \(x \in f(F)\), so we know 
  \[
    \bigcap_{i=1}^{n} U_{x_i} \subseteq \bigcap_{i=1}^{n} \left( Y \setminus V_{x_i} \right) \subseteq Y \setminus f(F), 
  \] and by the definition of topology, we know \(\bigcap _{i = 1}^n U_{x_i} \in \mathcal{G} \) and it is a neighborhood of \(y\), so \(Y \setminus f(F)\) is open, and we're done.   

\end{proof}
%────────────────────────────────────────────────────────────────────────────────────────────────────────────────────────────────────────────────────
\begin{problem}[20pts]
    Let $\{f_n\}$ be a sequence of continuous functions real-valued defined on a compact metric space $S$ and assume that $\{f_n\}$ converges pointwise on $S$ to a limit function $f$.  
Prove that $f_n \to f$ uniformly on $S$ if, and only if, the following two conditions hold:

\begin{enumerate}
  \item[(i)] The limit function $f$ is continuous on $S$.
  \item[(ii)] For every $\varepsilon > 0$, there exist $m > 0$ and $\delta > 0$ such that $n > m$ and 
  \[
  |f_k(x) - f(x)| < \delta \implies |f_{k+n}(x) - f(x)| < \varepsilon
  \]
  for all $x \in S$ and all $k = 1, 2, \dots$.
\end{enumerate}

\noindent\textbf{Hint.} To prove the sufficiency of (i) and (ii), show that for each $x_0 \in S$ there is a neighborhood $B(x_0, R)$ and an integer $k$ (depending on $x_0$) such that
\[
|f_k(x) - f(x)| < \delta \quad \text{if } x \in B(x_0,R).
\]
By compactness, a finite set of integers, say $A = \{k_1, \dots, k_r\}$, has the property that for each $x \in S$, some $k \in A$ satisfies $|f_k(x) - f(x)| < \delta$.  
Uniform convergence is an easy consequence of this fact.
\end{problem}

\begin{proof}
    \vphantom{text}
    \begin{itemize}
        \item [\((\implies )\)]
            \begin{itemize}
                \item (i). Since $f_n$ is continuous on $x \in X$ and $f_n \to f$ uniformly, so $f$ is continuous on $x \in X$.
                \item (ii). Since $f$ is uniformly continuous,
                \[
                \forall \varepsilon > 0, \exists N > 0 \text{ such that } \forall x\in X,\forall n \geq N \implies |f_n(x) - f(x)| < \varepsilon 
                \]
                Then $\forall \varepsilon > 0$, we can take $m = N$, $\delta = 114514$. $\forall k, k+ n > k+m > m = N$. \\
                So $\forall f_{k+n}, |f_{k+n}(x) - f(x)| < \varepsilon$ for all $x \in S$. Note that $\delta$ is actually not important here.
            \end{itemize}
        \item [\((\impliedby )\)]
            $\forall \varepsilon >0, \exists m_{\varepsilon} > 0, \delta_{\varepsilon} > 0$ such that $n > m_{\varepsilon}$,
            \[
                |f_k(x)-f(x)| < \delta_{\varepsilon} \implies |f_{k+n}(x)-f(x)| < \varepsilon, \forall x \in S, k = 1,2,3,...
            \]
            $\forall x_0 \in S$, since $f_n(x_0) \to f(x_0)$ pointwise, $\exists k_{x_0}$ such that $|f_{k_{x_0}}(x_0) - f(x_0)| < \delta_{\varepsilon}$.
            Since $f_{k_{x_0}}$ and $f$ are both continuous, so $f_{k_{x_0}} - f$ is also continuous on $x_0$, and hence $\exists R_{x_0} > 0$ such that if $B(x_0, x) < R_{x_0}$, then $|f_{k_{x_0}}(x) - f(x)| < \delta_{\varepsilon}$. \\
            $\forall x_0$, we can get such ball, then we collect those balls to get $I = \{B(x_0, R_{x_0}) : x_0 \in S\}$, by compactness of $S$, exist a finite subset $F = \{x_1, x_2, ... ,x_r\}$ such that $\left\{B(x_i, R_{x_i})\right\}_{i=1}^{r} \subseteq I$ and $ S \subseteq \bigcup_{i=1}^{r} B(x_i, R_{x_i})$. \\
            For each $x_i$, $\exists k_{x_i}$ such that $|f_{k_{x_i}}(x) - f(x)| < \delta_{\varepsilon}$. Let $A = \{k_{x_1}, k_{x_2}, ..., k_{x_r}\}$, and define $N = \max{A} + m_{\varepsilon} + 1$.
            \begin{claim}
                $\forall t \geq N$, $|f_t(x)-f(x)| < \varepsilon$ $\forall x \in S$.
            \end{claim}
            \begin{explanation}
                $\forall x \in S$, $x \in B(x_j, R_{x_j})$ for some $j \in [r]$.\\
                So $\forall x \in S, |f_{k_{x_j}}(x) - f(x)| < \delta_{\varepsilon}$ for some $j$, by (ii) we know $\forall n > m_{\varepsilon}, |f_{k_{x_j}+n}(x) - f(x)| < \varepsilon$. \\
                Since 
                \[t \geq N = \max{A} + m_{\varepsilon} + 1 > k_{x_j} + m_\varepsilon , \forall j \in [r],  t - k_{x_j} > m_\varepsilon
                \]
                , and hence $|f_t(x)-f(x)| < \varepsilon, \forall t\geq N$.
            \end{explanation}
            Since $\forall t \geq N$, $|f_t(x)-f(x)| < \varepsilon$ $\forall x \in S$, so $f_n$ is uniformly converge to $f$ on $S$ by definition.
            
    \end{itemize}
\end{proof}
%────────────────────────────────────────────────────────────────────────────────────────────────────────────────────────────────────────────────────
\begin{problem}[15pts]
    The purpose of this exercise is to demonstrate a concrete relationship between continuity and pointwise convergence, and between uniform continuity and uniform convergence. 

Let $f:\mathbb{R} \to \mathbb{R}$ be a function. For any $a \in \mathbb{R}$, let $f_a : \mathbb{R} \to \mathbb{R}$ be the shifted function defined by
\[
f_a(x) := f(x - a).
\]

\begin{enumerate}
  \item[(a)] Show that $f$ is continuous if and only if, whenever $(a_n)_{n=0}^\infty$ is a sequence of real numbers which converges to zero, the shifted functions $f_{a_n}$ converge pointwise to $f$.

  \item[(b)] Show that $f$ is uniformly continuous if and only if, whenever $(a_n)_{n=0}^\infty$ is a sequence of real numbers which converges to zero, the shifted functions $f_{a_n}$ converge uniformly to $f$.
\end{enumerate}

\end{problem}
\begin{proof}[proof of (a)]
  \vphantom{text}
  \begin{itemize}
    \item [\((\implies )\)] If \(f\) is continuous and suppose \(\left( a_n \right)_{n=0}^{\infty}  \) is a sequence of real numbers which converges to \(0\), then given any \(x \in \mathbb{R} \) and \(\varepsilon > 0\), we know there exists \(\delta > 0\) s.t. \(\vert a_n \vert = \vert (x - a_n) - x \vert < \delta \) implies 
    \[
      \vert f(x - a_n) - f(x) \vert < \varepsilon,
    \] and since \(\left( a_n \right)_{n=0}^{\infty}  \) converges to \(0\), so there exists \(N > 0\) s.t. \(n \ge N\) implies \(\vert a_n \vert < \delta \). Thus, for all \(x \in \mathbb{R} \) and \(\varepsilon > 0\), there exists \(N > 0\) s.t. \(n \ge N\) implies 
    \[
      \left\vert f_{a_n}(x) - f(x) \right\vert  = \vert f(x - a_n) - f(x) \vert < \varepsilon,
    \] which means \(f_{a_n}\) converge pointwise to \(f\).  
    \item [\((\impliedby )\)] Now if we have a sequence in \(\mathbb{R} \), \(\left\{ b_n \right\}_{n=0}^{\infty}  \), converges to \(b \in \mathbb{R} \), then we know \(\left\{ c_n = b - b_n \right\}_{n=0}^{\infty}  \) is a sequence converges to \(0\), so \(f_{c_n}\) converge pointwise to \(f\). This means for all \(x \in \mathbb{R} \) and for all \(\varepsilon > 0\), there exists \(N > 0\) s.t. \(n \ge N\) implies 
    \[
      \vert f(x - b + b_n) - f(x) \vert = \vert f(x - c_n) - f(x) \vert = \vert f_{c_n}(x) - f(x) \vert < \varepsilon,
    \]
    so if we pick \(x = b \in \mathbb{R} \), we know for all \(\varepsilon > 0\), there exists \(N > 0\) s.t. \(n \ge N\) implies 
    \[
      \vert f(b_n) - f(b) \vert < \varepsilon ,
    \] which means \(\lim_{n \to \infty} f(b_n) = f(b) \), so \(f\) is continuous.  
  \end{itemize}
\end{proof}
\begin{proof}[proof of (b)]
  \vphantom{text}
  \begin{itemize}
    \item [\((\implies )\)] If \(f\) is uniformly continuous and \(\left( a_n \right)_{n=0}^{\infty} \to 0 \), then for all \(\varepsilon > 0\), we know there exists \(\delta > 0\) s.t. if \(\vert a_n \vert = \vert (x - a_n) - x \vert < \delta  \), then \(\vert f(x - a_n) - f(x) \vert < \varepsilon  \) for all \(x \in \mathbb{R} \), and since \(\left( a_n \right)_{n=0}^{\infty}  \) converges to \(0\), so there exists \(N > 0\) s.t. \(n \ge N\) implies \(\vert a_n \vert < \delta  \). Thus, for all \(\varepsilon > 0\), there exists \(N > 0\) s.t. \(n 
    \ge N\) implies 
    \[
      \vert f_{a_n}(x) - f(x) \vert = \vert f(x - a_n) - f(x) \vert < \varepsilon \text{ for all } x \in \mathbb{R}, 
    \] so \(f_{a_n}\) converges uniformly to \(f\).                 
    \item [\((\impliedby )\)] We want to show that for all \(\varepsilon > 0\), there exists \(\delta > 0\) s.t. \(\vert x - y \vert < \delta \) implies \(\vert f(x) - f(y) \vert < \varepsilon \). Now suppose by contradiction, there exists \(\varepsilon _1 > 0\) s.t. for all \(\delta > 0\), there exists \(x, y \in \mathbb{R} \) s.t. \(\vert x - y \vert < \delta  \) and \(\vert f(x) - f(y) \vert \ge \varepsilon _1\). Then for all \(\delta = \frac{1}{k}\) with \(k \in \mathbb{N} \), we can pick \(x_k, y_k\) s.t. \(\vert x_k - y_k \vert < \frac{1}{k}\) and \(\vert f(x_k) - f(y_k) \vert \ge \varepsilon _1 \). Hence, we know \(\left\{ c_n = y_n - x_n \right\}_{n=0}^{\infty}  \) is a sequence in \(\mathbb{R} \) which converges to \(0\). Thus, \(f_{c_n}\) converges uniformly to \(f\). Thus, there exists \(N > 0\) s.t. \(n \ge N\) implies 
    \[
      \left\vert f(x - y_n + x_n) - f(x) \right\vert  = \left\vert f(x - c_n) - f(x) \right\vert  = \left\vert f_{c_n}(x) - f(x) \right\vert < \varepsilon_1  
    \] for all \(x \in \mathbb{R} \), so if we pick \(x = y_N\), then for \(n = N\) we know 
    \[
      \left\vert f(x_N) - f(y_N) \right\vert < \varepsilon_1
    \] but this is impossible, so this is a contradiction. Hence, \(f\) is uniformly continuous. 
  \end{itemize}
\end{proof}
%────────────────────────────────────────────────────────────────────────────────────────────────────────────────────────────────────────────────────
