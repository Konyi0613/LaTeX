\begin{problem}[20pts]
    \vphantom{text}
    \begin{definition}[Totally ordered set]\label{def:total-order}
A \emph{totally ordered set} (or \emph{linearly ordered set}) is a pair $(X, \leq)$ consisting of a nonempty set $X$ together with a binary relation $\leq$ on $X$ satisfying the following properties:
\begin{enumerate}
    \item \textbf{Reflexivity:} For all $x \in X$, $x \leq x$.
    \item \textbf{Antisymmetry:} For all $x, y \in X$, if $x \leq y$ and $y \leq x$, then $x = y$.
    \item \textbf{Transitivity:} For all $x, y, z \in X$, if $x \leq y$ and $y \leq z$, then $x \leq z$.
    \item \textbf{Totality (or Comparability):} For all $x, y \in X$, either $x \leq y$ or $y \leq x$.
\end{enumerate}
A relation $\leq$ satisfying only (1)--(3) is called a \emph{partial order}.  
If, in addition, (4) holds, the order is said to be \emph{total}, meaning that any two elements of $X$ can be compared.
\end{definition}


\begin{definition}[Hausdorff space]
A topological space $(X,\mathcal{F})$ is called a \emph{Hausdorff space} (or $T_2$ space) if for every pair of distinct points $x,y \in X$ there exist neighborhoods 
$U,V \in \mathcal{F}$ such that
\[
x \in U, \quad y \in V, \quad \text{and } U \cap V = \varnothing.
\]
\end{definition}
  \begin{enumerate}
  \item[(a)] Given any totally ordered set $X$ with order relation $\le$, declare a set $V \subseteq X$ to be open if for every $x \in V$ there exists a set $I$, which is an interval  
$\{y \in X : a < y < b\}$ for some $a, b \in X$, or  
$\{y \in X : a < y\}$ for some $a \in X$, or  
$\{y \in X : y < b\}$ for some $b \in X$, or the whole space $X$, which contains $x$ and is contained in $V$.  
Let $\mathcal{F}$ be the set of all open subsets of $X$.  
Show that $(X, \mathcal{F})$ is a topology (this is the \emph{order topology} on the totally ordered set $(X, \le)$ which is Hausdorff in the sense of  Definition 2.5.4-2 or the definition above).  


\medskip
\item[(b)] 
Show that on the real line $\mathbb{R}$ (with the standard ordering $\le$), the order topology matches the standard topology (i.e., the topology arising from the standard metric).  

 
 \medskip
\item[(c)] If instead one defines $V$ to be open if the extended real line $\mathbb{R} \cup \{\pm \infty\}$ has an open set with boundary $\{\pm \infty\}$, then $(X, \mathcal{F})$ is a sequence of numbers in $\mathbb{R}$ (and hence in $\mathbb{R}$), show that $x_n \to +\infty$ if and only if $\inf_{n \geq N} x_n \to +\infty$, and $x_n \to -\infty$ if and only if $\sup_{n \geq N} x_n \to -\infty$.

  \end{enumerate}
\end{problem}
%────────────────────────────────────────────────────────────────────────────────────────────────────────────────────────────────────────────────────
\begin{problem}[15pts]
    \vphantom{text}
    \begin{definition}[Metrizable space]\label{def:metrizable}
A topological space $(X,\mathcal{F})$ is said to be \emph{metrizable} if there exists a metric $d : X \times X \to [0,\infty)$ such that the topology $\mathcal{F}$ coincides with the topology $\mathcal{F}_d$ induced by $d$.  
That is,
\[
\mathcal{F} = \mathcal{F}_d := \{\, U \subseteq X : \forall x \in U, \exists\, \varepsilon > 0 \text{ such that } B_d(x,\varepsilon) \subseteq U \,\},
\]
where $B_d(x,\varepsilon) := \{\, y \in X : d(x,y) < \varepsilon \,\}$ denotes the open ball centered at $x$ with radius $\varepsilon$.

\medskip
If no such metric $d$ exists, then $(X,\mathcal{F})$ is said to be \emph{not metrizable}.  
In other words, its topology cannot arise from any metric on $X$.
\end{definition}

\begin{enumerate}
\item[(a)]  Let $X$ be an uncountable set, and let $\mathcal{F}$ be the collection of all subsets $E$ in $X$ which are either empty or cofinite (which means that $X \setminus E$ is finite).  
Show that $(X, \mathcal{F})$ is a topology (this is called the \emph{cofinite topology} on $X$) which is not Hausdorff  and is compact.  

\item[(b)] 
Show that if $\{V_i : i \in I\}$ is any countable collection of open sets containing $x$, then $\bigcap_i V_i \neq \varnothing$.  
Use this to show that the cofinite topology cannot be derived from any metric (i.e., $(X, \mathcal{F})$ is not metrizable).  
(Hint: what is the set $\bigcap_{n=1}^\infty B(x, 1/n)$ equal to in a metric space?)

\end{enumerate}
\end{problem}
%────────────────────────────────────────────────────────────────────────────────────────────────────────────────────────────────────────────────────
\begin{problem}[15pts]
    Let $(X, \mathcal{F})$ be a compact topological space.  
Assume that this space is first countable, which means that for every $x \in X$ there exist countable collections of open sets $V_1, V_2, \ldots$ of neighborhoods of $x$, such that every neighborhood of $x$ contains one of the $V_n$.  
Show that every sequence in $X$ has a convergent subsequence
 (see Exercise 1.5.11).
\end{problem}
%────────────────────────────────────────────────────────────────────────────────────────────────────────────────────────────────────────────────────
\begin{problem}[15pts]
    Let $(X,\mathcal{F})$ be a compact topological space and $(Y,\mathcal{G})$ be a Hausdorff topological space. 
If $f:X\to Y$ is continuous, then $f$ is a \emph{closed map}; i.e., for every closed
subset $F\subseteq X$, the image $f(F)$ is closed in $Y$.
\end{problem}
%────────────────────────────────────────────────────────────────────────────────────────────────────────────────────────────────────────────────────
\begin{problem}[20pts]
    Let $\{f_n\}$ be a sequence of continuous functions real-valued defined on a compact metric space $S$ and assume that $\{f_n\}$ converges pointwise on $S$ to a limit function $f$.  
Prove that $f_n \to f$ uniformly on $S$ if, and only if, the following two conditions hold:

\begin{enumerate}
  \item[(i)] The limit function $f$ is continuous on $S$.
  \item[(ii)] For every $\varepsilon > 0$, there exist $m > 0$ and $\delta > 0$ such that $n > m$ and 
  \[
  |f_k(x) - f(x)| < \delta \implies |f_{k+n}(x) - f(x)| < \varepsilon
  \]
  for all $x \in S$ and all $k = 1, 2, \dots$.
\end{enumerate}

\noindent\textbf{Hint.} To prove the sufficiency of (i) and (ii), show that for each $x_0 \in S$ there is a neighborhood $B(x_0, R)$ and an integer $k$ (depending on $x_0$) such that
\[
|f_k(x) - f(x)| < \delta \quad \text{if } x \in B(x_0,R).
\]
By compactness, a finite set of integers, say $A = \{k_1, \dots, k_r\}$, has the property that for each $x \in S$, some $k \in A$ satisfies $|f_k(x) - f(x)| < \delta$.  
Uniform convergence is an easy consequence of this fact.
\end{problem}
%────────────────────────────────────────────────────────────────────────────────────────────────────────────────────────────────────────────────────
\begin{problem}[15pts]
    The purpose of this exercise is to demonstrate a concrete relationship between continuity and pointwise convergence, and between uniform continuity and uniform convergence. 

Let $f:\mathbb{R} \to \mathbb{R}$ be a function. For any $a \in \mathbb{R}$, let $f_a : \mathbb{R} \to \mathbb{R}$ be the shifted function defined by
\[
f_a(x) := f(x - a).
\]

\begin{enumerate}
  \item[(a)] Show that $f$ is continuous if and only if, whenever $(a_n)_{n=0}^\infty$ is a sequence of real numbers which converges to zero, the shifted functions $f_{a_n}$ converge pointwise to $f$.

  \item[(b)] Show that $f$ is uniformly continuous if and only if, whenever $(a_n)_{n=0}^\infty$ is a sequence of real numbers which converges to zero, the shifted functions $f_{a_n}$ converge uniformly to $f$.
\end{enumerate}

\end{problem}
%────────────────────────────────────────────────────────────────────────────────────────────────────────────────────────────────────────────────────
