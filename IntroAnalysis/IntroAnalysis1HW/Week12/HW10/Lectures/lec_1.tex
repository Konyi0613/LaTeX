\begin{problem}[15pts \textbf{Exercise 4.7.8}]
    Let $\tan : (-\pi/2,\pi/2) \to \mathbb{R}$ be the tangent function 
$\tan(x) := \sin(x)/\cos(x)$.  
Show that $\tan$ is differentiable and monotone increasing, with  
\[
\frac{d}{dx}\tan(x) = 1 + \tan(x)^2,
\]
and that $\lim_{x\to \pi/2} \tan(x) = +\infty$ and $\lim_{x\to -\pi/2} \tan(x) = -\infty$.  
Conclude that $\tan$ is in fact a bijection from $(-\pi/2,\pi/2) \to \mathbb{R}$, and thus has an inverse function 
\[
\tan^{-1} : \mathbb{R} \to (-\pi/2,\pi/2)
\]
(this function is called the \emph{arctangent function}).  
Show that $\tan^{-1}$ is differentiable and 
\[
\frac{d}{dx} \tan^{-1}(x) = \frac{1}{1+x^2}.
\]
\end{problem}
\begin{proof}
    Since \(\sin (x)\) and \(\cos (x)\) are both differentiable on \(\left( - \frac{\pi}{2}, \frac{\pi}{2} \right) \), and \(\cos (x) \neq 0\) on \(\left( -\frac{\pi}{2}, \frac{\pi}{2} \right) \), so \(\tan (x)\) is differentiable and 
    \begin{align*}
         \frac{\mathrm{d}}{\mathrm{d}x} \tan (x) &= \frac{\mathrm{d}}{\mathrm{d}x} \frac{\sin (x)}{\cos (x)} = \frac{(\sin (x))^{\prime} \cos (x) - \sin (x) \left( \cos (x) \right)^{\prime}  }{\cos ^2(x)} 
         \\ &= \frac{\cos ^2(x) + \sin ^2(x)}{\cos ^2(x)} = 1 + \frac{\sin ^2(x)}{\cos ^2(x)} = 1 + \tan^2(x).
    \end{align*} 
    Hence, 
    \[
        \frac{\mathrm{d}}{\mathrm{d}x} \tan (x) = 1 + \tan ^2(x) \ge 1 > 0 
    \]    
    for all \(x \in \left( -\frac{\pi}{2}, \frac{\pi}{2} \right) \) and thus \(\tan (x)\) is monotone increasing. Note that 
    \[
        \lim_{x \to \frac{\pi}{2}} \tan (x) = \lim_{x \to \left( \frac{\pi}{2} \right)^- } \tan (x) = \lim_{x \to \left( \frac{\pi}{2} \right)^- } \frac{\sin (x)}{\cos (x)},
    \]  
    and \(\sin (x) \to 1\) and \(\cos (x) \to 0^+\) as \(x \to \left( \frac{\pi}{2} \right)^- \), so 
    \[
        \lim_{x \to \left( \frac{\pi}{2} \right)^- } \frac{\sin (x)}{\cos (x)} = +\infty . 
    \]  
    Similarly, since \(\sin (x) \to -1\) and \(\cos (x) \to 0^+\) as \(x \to \left( -\frac{\pi}{2} \right)^+ \), so 
    \[
        \lim_{x \to -\frac{\pi}{2}} \tan (x) = \lim_{x \to \left( -\frac{\pi}{2} \right)^+ } \tan (x) = \lim_{x \to \left( -\frac{\pi}{2} \right)^+ } \frac{\sin (x)}{\cos (x)} = -\infty.   
    \]   
    Hence, we have shown that 
    \[
        \lim_{x \to \frac{\pi}{2}} \tan (x) = +\infty , \quad \lim_{x \to \left( - \frac{\pi}{2} \right) } \tan (x) = -\infty. 
    \]
    By this, we know \(\tan (x)\) is a bijection from \(\left( -\frac{\pi}{2}, \frac{\pi}{2} \right) \to \mathbb{R}  \), and thus the inverse function \(\tan ^{-1}(x)\) is well-defined. Also, since 
    \[
        \tan \left( \tan ^{-1}(x) \right) = x, 
    \]   
    so by differentiating on both sides, we have 
    \begin{align*}
        1 &= (x)^{\prime} = \left( \tan \left( \tan ^{-1}(x) \right)  \right)^{\prime} = \tan ^{\prime} \left( \tan ^{-1}(x) \right) \cdot \left( \tan ^{-1}(x) \right)^{\prime}  \\
        &= \left( 1 + \left( \tan \left( \tan ^{-1}(x) \right)  \right)^2  \right) \cdot \left( \tan ^{-1} (x) \right)^{\prime} = (1 + x^2) \cdot \left( \frac{\mathrm{d}}{\mathrm{d}x} \tan ^{-1}(x)  \right).   
    \end{align*}
    Hence, 
    \[
        \frac{\mathrm{d}}{\mathrm{d}x} \tan ^{-1} (x) = \frac{1}{1 + x^2}. 
    \]
\end{proof}

\begin{problem}[15pts \textbf{Exercise 4.7.9} ]
    Recall the arctangent function $\tan^{-1}$ from Exercise 4.7.8.  
By modifying the proof of Theorem 4.5.6(e), establish the identity
\[
\tan^{-1}(x) = \sum_{n=0}^{\infty} \frac{(-1)^n x^{2n+1}}{2n+1}
\]
for all $x \in (-1,1)$.  
Using Abel's theorem (Theorem 4.3.1) to extend this identity to the case $x=1$, conclude in particular the identity
\[
\pi = 4 - \frac{4}{3} + \frac{4}{5} - \frac{4}{7} + \cdots 
= 4 \sum_{n=0}^{\infty} \frac{(-1)^n}{2n+1}.
\]

(Note that the series converges by the alternating series test, Proposition 7.2.11.)  
Conclude in particular that $4 - \tfrac{4}{3} < \pi < 4$.  
(One can of course compute $\pi = 3.1415926\ldots$ to much higher accuracy, though if one wishes to do so it is advisable to use a different formula than the one above, which converges very slowly.)
\end{problem}

\begin{problem}[30pts \textbf{Exercise 4.7.10}]
    Let $f : \mathbb{R} \to \mathbb{R}$ be the function
\[
f(x) := \sum_{n=1}^{\infty} 4^{-n} \cos(32^n \pi x).
\]

\begin{enumerate}
\item[(a)] Show that this series is uniformly convergent, and that $f$ is continuous.

\item[(b)] Show that for every integer $j$ and every integer $m \ge 1$, we have
\[
\left| f\!\left( \frac{j+1}{32^m} \right) - 
       f\!\left( \frac{j}{32^m} \right) \right| 
\ge 4^{-m}.
\]


\textit{Hint: use the identity}
\[
\sum_{n=1}^{\infty} a_n 
= \left( \sum_{n=1}^{m-1} a_n \right)
+ a_m 
+ \sum_{n=m+1}^{\infty} a_n
\]
\textit{for certain sequences } $a_n$.  
Also, use the fact that the cosine function is periodic with period $2\pi$, as well as the geometric series formula  
$\sum_{n=0}^{\infty} r^n = \frac{1}{1-r}$ for any $|r|<1$.  
Finally, you will need the inequality $|\cos(x)-\cos(y)| \le |x-y|$ for any real numbers $x$ and $y$; this can be proven by using the mean value theorem.

\item[(c)] Using (b), show that for every real number $x_0$, the function $f$ is not differentiable at $x_0$.  
(Hint: for every $x_0$ and every $m \ge 1$, there exists an integer $j$ such that 
$j \le 32^m x_0 \le j+1$, thanks to Exercise 5.4.3.)

\item[(d)] Explain briefly why the result in (c) does not contradict Corollary 3.7.3.
\end{enumerate}
\end{problem}

\begin{problem}[20pts]
    \begin{enumerate}

\item[(a)]  Prove that \[
(\cos\theta + i\sin\theta)^{n}
= \cos(n\theta) + i\sin(n\theta)
\] or all integers $n$ and all real $\theta$.
This is the classical \emph{DeMoivre’s theorem}.

\item[(b)]
By equating imaginary parts in DeMoivre's formula, prove that
\[
\sin n\theta
= \sin^n \theta \left\{
\binom{n}{1} \cot^{\,n-1}\theta
- \binom{n}{3} \cot^{\,n-3}\theta
+ \binom{n}{5} \cot^{\,n-5}\theta
-\ \cdots
\right\}.
\]

\item[(c)]
If $0<\theta<\pi/2$, prove that
\[
\sin(2m+1)\theta
= \sin^{\,2m+1}\!\theta \; P_m(\cot^2\theta)
\]
where $P_m$ is the polynomial of degree $m$ given by
\[
P_m(x)
= \binom{2m+1}{1} x^m
- \binom{2m+1}{3} x^{m-1}
+ \binom{2m+1}{5} x^{m-2}
- \cdots .
\]

Use this to show that $P_m$ has zeros at the $m$ distinct points
\[
x_k = \cot^2\!\left(\frac{\pi k}{2m+1}\right),
\qquad k = 1,2,\dots,m.
\]

\item[(d)]
Show that the sum of the zeros of $P_m$ is given by
\[
\sum_{k=1}^m \cot^2\!\left(\frac{\pi k}{2m+1}\right)
= \frac{m(2m-1)}{3}.
\]
\end{enumerate}
\end{problem}

\begin{problem}[20pts]
    This exercise outlines a simple proof of the formula $\zeta(2)=\sum_{n=1}^{\infty}\frac{1}{n^2}=\pi^2/6$.
Start with the inequality
\[
\sin x < x < \tan x, \qquad 0<x<\frac{\pi}{2},
\]
take reciprocals, and square each member to obtain
\[
\cot^2 x < \frac{1}{x^2} < 1 + \cot^2 x.
\]
Now put $x = \dfrac{k\pi}{2m+1}$, where $k$ and $m$ are integers with $1 \le k \le m$,
and sum on $k$ to obtain
\[
\sum_{k=1}^m \cot^2\!\left( \frac{k\pi}{2m+1} \right)
< \frac{(2m+1)^2}{\pi^2} \sum_{k=1}^m \frac{1}{k^2}
< m + \sum_{k=1}^m \cot^2\!\left( \frac{k\pi}{2m+1} \right).
\]
\medskip
Use the formula in problem 4(d) to deduce the inequality
\[
\frac{m(2m-1)\pi^2}{3(2m+1)^2}
< \sum_{k=1}^m \frac{1}{k^2}
< \frac{2m(m+1)\pi^2}{3(2m+1)^2}.
\]
Now let $m\to\infty$ to obtain $\zeta(2) = \pi^2/6$.
\end{problem}
\begin{proof}
\begin{claim}
    \(\sin x < x < \tan x\) for \(0 < x < \frac{\pi}{2}\). 
\end{claim}
\begin{explanation}
    If \(0 < x < \frac{\pi}{2}\), then 
    \[
        \sin x = \int _0^x \cos t \, \mathrm{d} t < \int _0^x 1 \, \mathrm{d} t = x  
    \]
    and 
    \[
        \tan (x) = \int _0^x \sec ^2(t) \, \mathrm{d} t = \int _0^x \frac{1}{\cos ^2(t)} \, \mathrm{d} t > \int_0^x 1 \, \mathrm{d} t = x,   
    \]
    so we have 
    \[
        \sin x < x < \tan x.
    \]
\end{explanation}
Take reciprocals, and square each member, we have 
\[
    \cot ^2 x < \frac{1}{x^2} < 1 + \cot ^2x \text{ for all } 0 < x < \frac{\pi}{2}.
\]
Now put \(x = \frac{k \pi }{2m + 1}\), where \(k\) and \(m\) are integers with \(1 \le k \le m\), then we know for all such \(x\) we have \(0 < x < \frac{\pi}{2}\), so we can sum up all \(k\) to obtain 
\[
    \sum_{k=1}^m \cot ^2 \left( \frac{k \pi }{2m + 1} \right) < \frac{(2m + 1)^2}{\pi ^2} \sum_{k=1}^m \frac{1}{k^2} < m + \sum_{k=1}^m \cot ^2 \left( \frac{k \pi }{2m + 1} \right),     
\]      
and by 4(d) we know 
\[
    \sum_{k=1}^m \cot ^2 \left( \frac{\pi k}{2m + 1} \right) = \frac{m(2m - 1)}{3},  
\]
so apply this formula we have 
\[
    \frac{m(2m - 1)\pi ^2}{3(2m + 1)^2} < \sum_{k=1}^m \frac{1}{k^2} < \frac{2m(m+1) \pi ^2}{3(2m+1)^2}, 
\]
and since 
\[
    \lim_{m \to \infty} \frac{m(2m - 1)\pi ^2}{3(2m + 1)^2} = \lim_{m \to \infty} \frac{(4m - 1)\pi ^2}{12(2m + 1)} = \lim_{m \to \infty} \frac{4 \pi ^2}{24} = \frac{\pi ^2}{6}  
\]
and 
\[
    \lim_{m \to \infty} \frac{2m(m+1)\pi ^2}{3(2m + 1)^2} = \lim_{m \to \infty} \frac{(4m + 2)\pi ^2}{12(2m + 1)} = \lim_{m \to \infty} \frac{4 \pi ^2}{24} = \frac{\pi ^2}{6}   
\]
by L'Hôpital's rule, so by Squeeze theorem we know 
\[
    \zeta (2) = \lim_{m \to \infty} \sum_{k=1}^m \frac{1}{k^2} = \frac{\pi ^2}{6}. 
\]
\end{proof}