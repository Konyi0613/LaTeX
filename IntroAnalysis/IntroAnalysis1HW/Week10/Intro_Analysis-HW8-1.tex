\documentclass[addpoints]{exam}
% leqno 將方程式編號放在左側

%\printanswers

\usepackage[top=2.5cm,bottom=2cm,left=2.5cm,right=2.5cm,headsep=10pt,a4paper]{geometry} % 引用頁面幾何套件,設定頁面邊界

\usepackage{siunitx} % 角度
% \input{NewCommands}

\usepackage{amsmath,amssymb,amsthm} % 引用常見的數學符號與設定


\theoremstyle{definition}
\newtheorem*{definition*}{Definition}
\newtheorem*{theorem}{Theorem}
\newtheorem*{remark}{Remark}
\newtheorem*{claim}{Claim}
\newtheorem*{proposition}{Proposition
}
\usepackage{lastpage} % 取得最後頁碼




\usepackage{graphicx} % 引用圖檔內嵌入套件
\graphicspath{{Pictures/}} % 設定圖檔目錄
\usepackage{xcolor} %引用顏色套件
\usepackage{enumerate}
\usepackage{float}
\usepackage{hyperref}
\usepackage{mdwlist} % use suspend & resume to successive enumerate

\usepackage{CJKutf8} % 設定中文

\def \lflr{\left\lfloor} %自定義 floor function
\def \rflr{\right\rfloor} %自定義

% question separation
\renewcommand{\questionshook}{%
  \setlength{\itemsep}{0.5cm}%
}

\allowdisplaybreaks %equation allowed to next page

\firstpageheader{\today}
                {}
                {}
\footer{}{\sf Page \thepage~of~\pageref{LastPage}}{}

\begin{document}

\begin{CJK*}{UTF8}{gbsn}
% Title
\begin{center}
   {\Large{\bf Introduction to Mathematical Analysis\\
    Homework 8  Due  November  14  (Friday), 2025\\
    Please submit your homework online in PDF format.
  %  Homework X Brief Solution
  }} \\ 
    \large
    %\text{Due date: 3/9/2023}
\end{center}

\noindent\rule{16.2cm}{0.4pt}


\begin{questions}


% ============ Question x ============
\question (25 pts)
Give examples of a formal power series
\[
\sum_{n=0}^{\infty} c_n x^n
\]
centered at \(0\) with radius of convergence \(1\), which
\begin{enumerate}
\item[(a)] diverges at both \(x=1\) and \(x=-1\);
\item[(b)] diverges at \(x=1\) but converges at \(x=-1\);
\item[(c)] converges at \(x=1\) but diverges at \(x=-1\);
\item[(d)] converges at both \(x=1\) and \(x=-1\);
\item[(e)] converges pointwise on \((-1,1)\), but does not converge uniformly on \((-1,1)\).
\end{enumerate}

\begin{solution}




\end{solution}



\question (25 pts)
\noindent\textbf{Exercise 4.2.7.}  
Let \(m \ge 0\) be a positive integer, and let \(0  < r\) be real numbers.  
Prove the identity
\[
\frac{r}{\,r-x\,} = \sum_{n=0}^{\infty} x^n r^{-n}
\]
for all \(x \in (-r, r)\).  

Using Proposition 4.2.6, conclude the identity
\[
\frac{r}{(r-x)^{\,m+1}}
= \sum_{n=m}^{\infty} \frac{n!}{m!(n-m)!}\, x^{\,n-m} r^{-n}
\]
for all integers \(m \ge 0\) and all \(x \in (-r, r)\).  
Also explain why the series on the right-hand side is absolutely convergent.
  
  \begin{solution}




\end{solution}


  
  \question (25 pts)
  Let \(E\) be a subset of \(\mathbb{R}\), let \(a\) be an interior point of \(E\), and let \(f:E\to\mathbb{R}\) be a function which is real analytic at \(a\) and has a power series expansion
\[
f(x)=\sum_{n=0}^{\infty} c_n (x-a)^n
\]
at \(a\) which converges on the interval \((a-r,\, a+r)\). Let \((b-s,\, b+s)\) be any subinterval of \((a-r,\, a+r)\) for some \(s>0\).

\begin{enumerate}
\item[(a)] Prove that \(|a-b| \le r-s\), so in particular \(|a-b| < r\).

\item[(b)] Show that for every \(0<\varepsilon<r\), there exists a \(C>0\) such that \(|c_n| \le C(r-\varepsilon)^{-n}\) for all integers \(n\ge 0\).  
\emph{(Hint: what do we know about the radius of convergence of the series \(\sum_{n=0}^{\infty} c_n(x-a)^n\)?)}

\item[(c)] Show that the numbers \(d_0,d_1,\ldots\), given by the formula
\[
d_m := \sum_{n=m}^{\infty} \frac{n!}{m!(n-m)!}(b-a)^{\,n-m} c_n \qquad \text{for all integers } m\ge 0,
\]
are well-defined, in the sense that the above series is absolutely convergent.  
\emph{(Hint: use (b) and the comparison test, Corollary 7.3.2, followed by Exercise 4.2.7.)}

\item[(d)] Show that for every \(0<\varepsilon<s\) there exists a \(C>0\) such that
\[
|d_m| \le C(s-\varepsilon)^{-m}
\]
for all integers \(m\ge 0\).  
\emph{(Hint: use the comparison test, and Exercise 4.2.7.)}



\item[(e)] Show that the power series \(\sum_{m=0}^{\infty} d_m (x-b)^m\) is absolutely convergent for \(x \in (b-s,\, b+s)\) and converges to \(f(x)\).  
(You may need Fubini’s theorem for infinite series, Theorem 8.2.2 of \emph{Analysis I}, as well as Exercise 4.2.5. One may also need to use a variant of the \(d_m\) in which the \(c_n\) are replaced by \(|c_n|\).)

Note. You can use Exercise 4.2.5. Let \(a, b\) be real numbers, and let \(n \ge 0\) be an integer. Prove the identity
\[
(x-a)^n = \sum_{m=0}^{n} \frac{n!}{m!(n-m)!}(b-a)^{\,n-m}(x-b)^m
\]
for any real number \(x\).

\item[(f)] Conclude that \(f\) is real analytic at \(b\), and thus analytic at every point in \((a-r,\, a+r)\).

  \end{enumerate}
  
\begin{solution}




\end{solution}


  \question (25 pts)
  \begin{enumerate}
  
  \item[(a)] If each \(a_n \ge 0\) and if \(\sum a_n\) diverges, show that \(\sum a_n x^n \to +\infty\) as \(x \to 1^{-}\).  
(Assume \(\sum a_n x^n\) converges for \(|x|<1\).)

\item[(b)] If each \(a_n \ge 0\) and if \(\lim_{x \to 1^{-}} \sum a_n x^n\) exists and equals \(A\), prove that \(\sum a_n\) converges and has sum \(A\).  


  \end{enumerate}
  
  
\begin{solution}




\end{solution}

\end{questions}

 You can do the following problems to practice. You don’t have to submit the following problems.

\begin{questions}

\question 
Let the power series
$$
f(x)=\sum_{n=0}^{\infty} a_n x^n
$$
converge for $-1<x<1$. For each $n$, define the partial sum
$$
s_n = \sum_{k=0}^{n} a_k,
\qquad
\sigma_n = \sum_{k=0}^{n} k|a_k|.
$$
Suppose that $\displaystyle \lim_{x\to 1^-} f(x) = S$ and $\displaystyle \lim_{n\to\infty} n a_n = 0$.

In this problem, you will show that the series $\sum_{n=0}^{\infty} a_n$ converges and that its sum is $S$.

\begin{enumerate}
\item[(a)] \textbf{Preliminary Identity.}  
Show that for any $x \in (0,1)$,
$$
s_n - f(x) = \sum_{k=0}^{n} a_k (1-x^k) - \sum_{k=n+1}^{\infty} a_k x^k.
$$

\medskip

\item[(b)] \textbf{Bounding the First Sum.}  
Show that for all $m\ge 1$ and $x\in(0,1)$,
$$
1 + x + \cdots + x^{m-1} \le \frac{1}{1-x},
$$
and deduce that
$$
|1 - x^k| = (1-x)(1 + x + \cdots + x^{k-1}) \le k(1-x).
$$

\medskip

\item[(c)] \textbf{Application of the Bound.}  
Use part (b) to prove that for $x \in (0,1)$,
$$
\left| \sum_{k=0}^{n} a_k (1-x^k) \right| \le (1-x)\sigma_n.
$$

\medskip

\item[(d)] \textbf{Estimate of the Tail.}  
Use the assumption that $\lim_{n\to \infty} n|a_n| = 0$ to show that for any $\varepsilon>0$, there exists $N$ such that for all $n\ge N$,
$$
n|a_n| < \frac{\varepsilon}{3}.
$$
Then prove that for such $n$ and all $x \in (0,1)$,
$$
\left| \sum_{k=n+1}^{\infty} a_k x^k \right| \le \frac{\varepsilon}{3(1-x)}.
$$

\medskip

\item[(e)] \textbf{Putting the Estimates Together.}  
Combine parts (a)--(d) to show that for all $n \ge N$ and $x \in (0,1)$,
$$
|s_n - S|
\le |f(x) - S| + (1-x)\sigma_n + \frac{\varepsilon}{3(1-x)}.
$$

\medskip

\item[(f)] \textbf{Strategic Choice of $x$.}  
Let $x = x_n = 1 - \frac{1}{n}$.  
Use part (e) to show that when $n$ is sufficiently large,
$$
|s_n - S| < \varepsilon.
$$
Conclude that $s_n \to S$, and therefore
$$
\sum_{n=0}^{\infty} a_n = S.
$$
\end{enumerate}

 
\begin{solution}




\end{solution}

 \question   \begin{enumerate}
\item[(1)] Let $\sum_{n=0}^{\infty} a_n x^n$ be a power series with radius of convergence $R>0$. Show that the radius of convergence of $\sum_{n=0}^{\infty} \frac{a_n}{n!} x^n$ is $+\infty$.

\item[(2)] Suppose that the power series $\sum_{n=0}^{\infty} \frac{a_n}{n!} x^n$ has radius of convergence $R < +\infty$. What can we say about the radius of convergence of $\sum_{n=0}^{\infty}a_n x^n$?
\end{enumerate}
 
\begin{solution}




\end{solution}


 \question Let $(a_n)_{n \ge 1}$ be a sequence of nonzero real numbers such that
\[
\frac{|a_{n+2}|}{|a_n|} \xrightarrow[n\to\infty]{} 2.
\]
Show that the radius of convergence of the power series $\sum_{n=0}^{\infty}  a_n x^n$ is $\frac{1}{\sqrt{2}}$.

 
\begin{solution}




\end{solution}



\end{questions}

% \appendix
% \section{}
\end{CJK*}
\end{document}

\question Let $\sum_{n=0}^{\infty}  a_n x^n$ be a power series with radius of convergence $R$. Let $S_n = \sum_{k=0}^n a_k$ be the partial sums of $\sum a_n$. Denote the radius of convergence of $\sum_{n=0}^{\infty}  S_n x^n$ by $r$.

\begin{enumerate}
\item[(1)] Show that $r \le R$.

\item[(2)] Show that $\min\{1, R\} \le r$. \emph{Hint: The power series 
$\sum_{n=0}^{\infty} S_nx^n$
 can be seen as the Cauchy product between $\sum_{n=0}^{\infty}  a_n x^n$ and a 
 specific power series that you need to choose,).}
