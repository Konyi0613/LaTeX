%%---------------------------------------------------------------
%% AMS-LaTeX Paper
%%---------------------------------------------------------------
\documentclass[12pt]{amsart}
\usepackage{latexsym, amsmath, amscd, amssymb, amsthm}
\usepackage{amssymb,amsmath,amsthm,graphicx,dsfont}
\usepackage{CJK}
\usepackage{hyperref}
 \usepackage{color}
%\usepackage{showkeys}
%%--LAYOUT-------------------------------------------------------
\renewcommand{\baselinestretch}{1}
\renewcommand{\arraystretch}{1.1}
\usepackage{calc}
\setlength\hoffset{0pt}
\setlength\marginparwidth{0pt}
\setlength\marginparsep{0pt}
\setlength\oddsidemargin{0.2in}
\setlength\evensidemargin{\oddsidemargin}
\setlength\textwidth{\paperwidth - 2\oddsidemargin - 2in}
\setlength\voffset{0pt}
\setlength\topmargin{0pt}
\setlength\headheight{1em}
\setlength\headsep{1em}
\setlength\footskip{2em}
\setlength\textheight{\paperheight - 2in - \voffset - \topmargin -
  \headheight - \headsep - \footskip}
%%--MATH---------------------------------------------------------
\DeclareMathOperator{\Ker}{Ker}
%%--OTHER ENVIRONMENTS--------------------------------------
\newtheorem{lemma}{Lemma}[section]
\newtheorem{theorem}[lemma]{Theorem}
\newtheorem{conjecture}[lemma]{Conjecture}
\newtheorem{corollary}[lemma]{Corollary}
\newtheorem{proposition}[lemma]{Proposition}
\newtheorem{problem}[lemma]{Problem}
\theoremstyle{definition}
\newtheorem{definition}[lemma]{Definition}
\newtheorem{notation}[lemma]{Notation}
\newtheorem{example}[lemma]{Example}
\newtheorem{principle}[lemma]{Principle}
\newtheorem{remark}[lemma]{Remark}
\newtheorem{axiom}[lemma]{Axiom}
\newtheorem*{acknowledgements}{Acknowledgments}
\theoremstyle{remark}
\renewcommand{\theequation}%
{\arabic{section}.\arabic{lemma}.\arabic{equation}}
\renewcommand{\labelenumi}{\textbf{(\alph{enumi})}}
%%----------------------------------------------------------


\begin{document}
\begin{CJK}{UTF8}{bkai}

{\centerline{\bf Math 2213 Introduction to Analysis  }}

{\centerline{\bf Homework 1  Due   September 10 (Thursday), 2025}}

{\centerline{\bf Please submit your homework online in PDF format.}}


 
\begin{enumerate}

\item[(1)] (10 pts) Dyadic density via the Archimedean property]
\label{ex:dyadic-density}
Let $a<b$ be real numbers. Prove that there exists a \emph{dyadic rational}
\[
q=\frac{k}{2^n}\in\mathbb{Q}\qquad(k \in \mathbb{Z},n\in\mathbb{N})
\]
such that $a<q<b$.
Further show that there are \emph{infinitely many} such dyadic rationals in $(a,b)$.
\vfill
\bigskip



\item[(2)] \textbf{A tour of the $p$-adic world.}

The field $\mathbb{Q}$ inherits the Euclidean metric from $\mathbb{R}$, but it also carries 
a very different metric: the \emph{$p$-adic metric}. 

\medskip

Given a prime number $p$ and an integer $n$, the $p$-adic norm of $n$ is defined as
\[
|n|_p = \frac{1}{p^k},
\]
where $p^k$ is the largest power of $p$ dividing $n$.  
(We define $|0|_p := 0$.)  
The more factors of $p$ appear in $n$, the smaller the $p$-adic norm becomes. 

\medskip

For a rational number $x=\tfrac{a}{b}$ with $a,b\in\mathbb{Z}$, we may factor $x$ as
\[
x = p^k \cdot \frac{r}{s},
\]
where $k \in \mathbb{Z}$ and $p$ divides neither $r$ nor $s$. We then define
\[
|x|_p = p^{-k}.
\]

\medskip

The $p$-adic metric on $\mathbb{Q}$ is given by
\[
d_p(x,y) := |x-y|_p.
\]

\begin{enumerate}

\item[\textbf{(a)}] To compute the $5$-adic norm $|x|_5$ of a rational number $x$, 
we examine how many factors of $5$ occur in $x$ (in either numerator or denominator).

\begin{itemize}
\item If $x=5^k \cdot \tfrac{a}{b}$ with $a,b$ not divisible by $5$ and $k\in \mathbb{Z}$, 
then the $5$-adic norm is
\[
|x|_5 = 5^{-k}.
\]

\item \textbf{Examples.}
\begin{enumerate}
\item $30 = 2 \cdot 3 \cdot 5$. There is exactly one factor of $5$, so
\[
|30|_5 = 5^{-1} = \tfrac{1}{5}.
\]

\item $32 = 2^5$. There is no factor of $5$, so
\[
|32|_5 = 5^{0} = 1.
\]

\item Compute $\left|\tfrac{1}{250}\right|_5$.

\[
250 = 2 \cdot 5^3.
\]

So
\[
\frac{1}{250} = \frac{1}{2 \cdot 5^3} 
= 5^{-3} \cdot \frac{1}{2},
\]
where $\tfrac{1}{2}$ has no factor of $5$ in numerator or denominator.

Therefore,
\[
\left|\tfrac{1}{250}\right|_5 
= 5^{-(-3)} 
= 5^3 
= 125.
\]

Hence,
\[
\boxed{\;\;\left|\tfrac{1}{250}\right|_5 = 125.\;\;}
\]
\end{enumerate}
\end{itemize}

\medskip

Now practice computing the following $5$-adic norms:
 (6 pts) 
\begin{enumerate}
\item $|75|_5$
\item $\left|\tfrac{10}{9}\right|_5$
\item $\left|-\tfrac{20}{375}\right|_5$
\end{enumerate}

\medskip

\item[\textbf{(b)}]  (9 pts)  Further properties of the $5$-adic norm.
\begin{enumerate}
\item Determine all rational numbers $x$ satisfying $|x|_5\le 1$. 
\item Which rational numbers $x$ satisfy $|x|_5=1$?
\item What is $\lim_{n \to \infty} 5^n$ in $(\mathbb{Q}, d_5)$ (the $5$-adic metric)? \\
\emph{Hint:} Compute $d_5(5^n,0)$.
\end{enumerate}

\medskip

\item[\textbf{(c)}] (15 pts) \textbf{Non-Archimedean absolute value and metric.}  
Prove that $|\cdot|_p$ satisfies
\[
|xy|_p=|x|_p|y|_p,\qquad |x+y|_p\le \max\{|x|_p,|y|_p\},
\]
and show that $d_p$ is a metric on $\mathbb{Q}$.

\end{enumerate}
\vfill
\bigskip

\item[(3)] (exercise 1.1.3)  (20 pts) 
Let $X$ be a set, and let $d : X \times X \to [0,\infty)$ be a function. 

\begin{enumerate}
\item[(a)] Give an example of a pair $(X,d)$ which obeys axioms (bcd) of Definition~1.1.2, but not (a). 
\hfill (Hint: modify the discrete metric.)
\item[(b)] Give an example of a pair $(X,d)$ which obeys axioms (acd) of Definition~1.1.2, but not (b).
\item[(c)] Give an example of a pair $(X,d)$ which obeys axioms (abd) of Definition~1.1.2, but not (c).
\item[(d)] Give an example of a pair $(X,d)$ which obeys axioms (abc) of Definition~1.1.2, but not (d). 
\hfill (Hint: try examples where $X$ is a finite set.)
\end{enumerate}

\vfill
\bigskip

\item[(4)]  (20 pts)  Let $x=(x_1,\dots,x_n)$ and $y=(y_1,\dots,y_n)$ be vectors in $\mathbb{R}^n$.

\vfill
\bigskip

\begin{enumerate}
\item[(a)] The $\ell^1$ metric is defined by
\[
d_1(x,y) := \sum_{i=1}^n |x_i - y_i|.
\]
Show that $d_1$ is a metric on $\mathbb{R}^n$


\item[(b)] The $\ell^\infty$ metric is defined by
\[
d_\infty(x,y) := \max_{1 \leq i \leq n} |x_i - y_i|.
\]
Show that $d_{\infty}$ is a metric on $\mathbb{R}^n$

\end{enumerate}


\item[(5)]  (10 pts)  A \emph{vector space} $V$ over $\mathbb{R}$ s a set 
equipped with two operations:
\begin{enumerate}
  \item \textbf{Vector addition:} $+: V \times V \to V$, written $(u,v) \mapsto u+v$.
  \item \textbf{Scalar multiplication:} $\cdot : \mathbb{R} \times V \to V$, written $(\alpha,v) \mapsto \alpha v$,
\end{enumerate}
such that the following properties hold for all $u,v,w \in V$ and $\alpha,\beta \in \mathbb{R}$:
\begin{enumerate}
  \item[(VS1)] $(u+v)+w = u+(v+w)$ \hfill (associativity of addition)
  \item[(VS2)] $u+v = v+u$ \hfill (commutativity of addition)
  \item[(VS3)] There exists $0 \in V$ such that $u+0=u$ \hfill (additive identity)
  \item[(VS4)] For each $u \in V$, there exists $-u \in V$ such that $u+(-u)=0$ \hfill (additive inverse)
  \item[(VS5)] $\alpha(u+v) = \alpha u + \alpha v$ \hfill (distributivity I)
  \item[(VS6)] $(\alpha+\beta)u = \alpha u + \beta u$ \hfill (distributivity II)
  \item[(VS7)] $(\alpha\beta)u = \alpha(\beta u)$ \hfill (compatibility of scalar multiplication)
  \item[(VS8)] $1 \cdot u = u$ \hfill (identity element of scalar multiplication)
\end{enumerate}

A function $\|\cdot\| : V \to [0,\infty)$ is called a \emph{norm} on $V$ if, 
for all $u,v \in V$ and $\alpha \in \mathbb{R}$, the following properties hold:
\begin{enumerate}
  \item[(N1)] $\|v\| \geq 0$, and $\|v\| = 0$ if and only if $v=0$. \hfill (positivity)
  \item[(N2)] $\|\alpha v\| = |\alpha| \cdot \|v\|$. \hfill (homogeneity)
  \item[(N3)] $\|u+v\| \leq \|u\| + \|v\|$. \hfill (triangle inequality)
\end{enumerate}

Given a norm $\|\cdot\|$ on $V$, define $d : V \times V \to [0,\infty)$ by
\[
d(u,v) = \|u-v\|.
\]


 Prove that $d$ is a \emph{metric} on $V$, that is, for all $x,y,z \in V$ show that:
  \begin{enumerate}
    \item $d(x,y) \geq 0$ and $d(x,y)=0$ if and only if $x=y$.
    \item $d(x,y) = d(y,x)$.
    \item $d(x,z) \leq d(x,y) + d(y,z)$.
  \end{enumerate}
(Thus we conclude that every normed vector space $(V,\|\cdot\|)$ is also a metric space with metric $d(u,v)=\|u-v\|$.
)

\vfill
\bigskip

\item[(6)]  (10 pts)  
Let $S$ be a bounded nonempty set of real numbers, and let $a$ and $b$
be fixed  nonzero real numbers. Define $T=\{as+b| s\in S\}$ Find formulas
for
$\sup T$ and $\inf T$ in terms of $\sup S$ and $\inf S$. Prove your
formulas.



\end{enumerate}



\end{CJK}

\end{document}    