%────────────────────────────────────────────────────────────────────────────────────────────────────────────────────────────────────────────────────

\begin{problem}[10pts]
Dyadic density via the Archimedean property. Let \(a < b\)  be real numbers. Prove that there exists a dyadic rational
\[
  q = \frac{k}{2^n} \in \mathbb{Q} \quad (k \in \mathbb{Z}, n \in \mathbb{N} )
\]
such that \(a < q < b\). Further show that there are infinitely many such dyadic
rationals in \((a,b)\).
\end{problem}
\begin{proof}
We first need to show a lemma first:
\begin{lemma}\label{lm: lm in 1}
  For any real numbers \(a, b\) such that \(a<b\) , there exists \(n \in \mathbb{N} \) such that \(2^n a > b\).   
\end{lemma}
\begin{explanation}
By Archimedean Property, we know there exists \(q \in \mathbb{N} \) such that \(qa > b\), so if we pick \(n = q+2\), then we have 
\[
  2^n = 2^{q+2} > q+2 > q, 
\] 
so we have \(2^n a > qa > b\), and we're done.  
\end{explanation}
Now using \autoref{lm: lm in 1} , we can get there exists some \(n \in \mathbb{N} \) such that \(2^n(b-a)>1\), so if we let \(k = \lfloor 2^n a \rfloor + 1\), then we have 
\[
  2^n a < \lfloor 2^n a \rfloor + 1 = k \le 2^n a + 1 < 2^n b.
\]   
Hence, 
\[
  a < \frac{k}{2^n} < b
\]
here. Note that \(n \in \mathbb{N} \) and \(k \in \mathbb{Z} \), so we can pick \(q = \frac{k}{2^n}\).

Next we'll show that there are infinitely many such dyadic rationals in \((a,b)\). Actually we can just repeat the step above but let \(a\) be \(q^{(0)}\) that \(q^{(0)}\) is the \(q\) we found above and then we know there exists another dyadic rationals \(q^{(1)}\)  in \((q^{(0)}, b)\), and then doing again this step we know there exists another dyadic rationals \(q^{(2)}\) in \((q^{(1)}, b)\). and so on. Then, since \(q^{(i)} \neq q^{(j)}\) if \(i \neq j\), so we know 
\[
  a < q^{(0)} < q^{(1)} < q^{(2)} < \dots < b,
\]  which means there are infinitely many such dyadic rationals in \((a,b)\). 
\end{proof}

%────────────────────────────────────────────────────────────────────────────────────────────────────────────────────────────────────────────────────

\begin{problem}[\textbf{A tour of the $p$-adic world.}]

The field $\mathbb{Q}$ inherits the Euclidean metric from $\mathbb{R}$, but it also carries 
a very different metric: the \emph{$p$-adic metric}. 

\medskip

Given a prime number $p$ and an integer $n$, the $p$-adic norm of $n$ is defined as
\[
|n|_p = \frac{1}{p^k},
\]
where $p^k$ is the largest power of $p$ dividing $n$.  
(We define $|0|_p := 0$.)  
The more factors of $p$ appear in $n$, the smaller the $p$-adic norm becomes. 

\medskip

For a rational number $x=\tfrac{a}{b}$ with $a,b\in\mathbb{Z}$, we may factor $x$ as
\[
x = p^k \cdot \frac{r}{s},
\]
where $k \in \mathbb{Z}$ and $p$ divides neither $r$ nor $s$. We then define
\[
|x|_p = p^{-k}.
\]

\medskip

The $p$-adic metric on $\mathbb{Q}$ is given by
\[
d_p(x,y) := |x-y|_p.
\]

\begin{enumerate}

\item[\textbf{(a)}] To compute the $5$-adic norm $|x|_5$ of a rational number $x$, 
we examine how many factors of $5$ occur in $x$ (in either numerator or denominator).

\begin{itemize}
\item If $x=5^k \cdot \tfrac{a}{b}$ with $a,b$ not divisible by $5$ and $k\in \mathbb{Z}$, 
then the $5$-adic norm is
\[
|x|_5 = 5^{-k}.
\]

\item \textbf{Examples.}
\begin{enumerate}
\item $30 = 2 \cdot 3 \cdot 5$. There is exactly one factor of $5$, so
\[
|30|_5 = 5^{-1} = \tfrac{1}{5}.
\]

\item $32 = 2^5$. There is no factor of $5$, so
\[
|32|_5 = 5^{0} = 1.
\]

\item Compute $\left|\tfrac{1}{250}\right|_5$.

\[
250 = 2 \cdot 5^3.
\]

So
\[
\frac{1}{250} = \frac{1}{2 \cdot 5^3} 
= 5^{-3} \cdot \frac{1}{2},
\]
where $\tfrac{1}{2}$ has no factor of $5$ in numerator or denominator.

Therefore,
\[
\left|\tfrac{1}{250}\right|_5 
= 5^{-(-3)} 
= 5^3 
= 125.
\]

Hence,
\[
\boxed{\;\;\left|\tfrac{1}{250}\right|_5 = 125.\;\;}
\]
\end{enumerate}
\end{itemize}

\medskip

Now practice computing the following $5$-adic norms:
 (6 pts) 
\begin{enumerate}
\item $|75|_5$
\item $\left|\tfrac{10}{9}\right|_5$
\item $\left|-\tfrac{20}{375}\right|_5$
\end{enumerate}

\medskip

\item[\textbf{(b)}]  (9 pts)  Further properties of the $5$-adic norm.
\begin{enumerate}
\item Determine all rational numbers $x$ satisfying $|x|_5\le 1$. 
\item Which rational numbers $x$ satisfy $|x|_5=1$?
\item What is $\lim_{n \to \infty} 5^n$ in $(\mathbb{Q}, d_5)$ (the $5$-adic metric)? \\
\emph{Hint:} Compute $d_5(5^n,0)$.
\end{enumerate}

\medskip

\item[\textbf{(c)}] (15 pts) \textbf{Non-Archimedean absolute value and metric.}  
Prove that $|\cdot|_p$ satisfies
\[
|xy|_p=|x|_p|y|_p,\qquad |x+y|_p\le \max\{|x|_p,|y|_p\},
\]
and show that $d_p$ is a metric on $\mathbb{Q}$.

\end{enumerate}
\end{problem}
\begin{proof}
  \vphantom{text}
  \begin{itemize}
    \item [(a)] 
    \vphantom{text}
    \begin{itemize}
      \item [(a)] First note that \(75 = 5^2 \cdot 3\), so \(\vert 75 \vert_5 = 5^{-2} = \frac{1}{25}\).   
      \item [(b)] First note that \(\frac{10}{9} = 5 \cdot \frac{2}{9}\), so \(\left\vert \frac{10}{9} \right\vert _5 = 5^{-1} = \frac{1}{5} \).  
      \item [(c)] First note that \(-\frac{4 \cdot 5}{5^3 \cdot 3} = 5^{-2} \cdot \frac{-4}{3}\), so \(\left\vert - \frac{20}{375} \right\vert _ 5 = 5^{-(-2)} = 25 \).  
    \end{itemize}
    \item [(b)] \vphantom{text}
    \begin{itemize}
      \item [(a)]Suppose \(x = 5^k \cdot \frac{r}{s}\) where \(k,r,s \in \mathbb{Z} \) and \(5\) divides neither \(r\) nor \(s\), then we know \(\vert x \vert_5 = 5^{-k}\), and we want \(5^{-k} \le 1\), which means \(k \ge 0\). Hence,
      \[
        \left\{ \text{all rational numbers } x \text{ satisfying } \vert x \vert_5 \le 1  \right\} = \left\{ 5^k \cdot \frac{r}{s} \mid k,r,s \in \mathbb{Z} \text{ and } k \ge 0 \text{ and } 5 \nmid rs \right\}.  
      \]
      \item [(b)] 
      \[
       \left\{ \text{all rational numbers } x \text{ satisfying } \vert x \vert_5 = 1  \right\} = \left\{\frac{r}{s} \mid r,s \in \mathbb{Z} \text{ and } 5 \nmid rs \right\} 
      \]
      \item [(c)] First notice that \(d_5 \left( 5^n, 0 \right) = \vert 5^n - 0 \vert_5 = 5^{-n}  \). Also, we know 
      \[
        0 = \lim_{n \to \infty} 5^{-n} = \lim_{n \to \infty} d_5(5^n, 0), 
      \] so we know \(\lim_{n \to \infty} 5^n = 0 \) in \((\mathbb{Q} , d_5)\) . 
    \end{itemize}
    \item [(c)] Now suppose \(x = p^{k_1} \frac{r_1}{s_1}\) and \(y = p^{k_2} \frac{r_2}{s_2}\), where \(p \nmid r_1 s_1 r_2 s_2\). Hence, \(xy = p^{k_1 + k_2} \frac{r_1 r_2}{s_1 s_2}\), and thus 
    \[
      \vert xy \vert_p = p^{-(k_1 + k_2)}. 
    \]   
    Also, we know 
    \[
      \vert x \vert_p = p^{-k_1} \quad \vert y \vert_p = p^{-k_2}, 
    \]
    so 
    \[
      \vert xy \vert_p = p^{-(k_1 + k_2)} = p^{-k_1}p^{-k_2} = \vert x \vert_p \vert y \vert_p.  
    \]
    Now without lose of genrality, suppose \(k_1 \ge k_2\), then we know 
    \[
      x+y = p^{k_2} \left( \frac{p^{k_1 - k_2}r_1 s_2 + r_2 s_1}{s_1 s_2} \right),
    \] and thus 
    \[
      \vert x+y \vert_p \le p^{-k_2} = \vert y \vert_p = \max \left\{ \vert x \vert_p, \vert y \vert_p   \right\}.   
    \]
    \begin{note}
      When \(k_1 = k_2\), it may happen that \(\vert x+y \vert_p < \max \left\{ \vert x \vert_p , \vert y \vert_p   \right\}  \).
    \end{note}
    And the case that \(k_2 \ge k_1\) is similar. 
  \end{itemize}
\end{proof}

%────────────────────────────────────────────────────────────────────────────────────────────────────────────────────────────────────────────────────
\begin{problem}[exercise 1.1.3 (20 pts)] 
Let $X$ be a set, and let $d : X \times X \to [0,\infty)$ be a function. 

\begin{enumerate}
\item[(a)] Give an example of a pair $(X,d)$ which obeys axioms (bcd) of Definition~1.1.2, but not (a). 
\hfill (Hint: modify the discrete metric.)
\item[(b)] Give an example of a pair $(X,d)$ which obeys axioms (acd) of Definition~1.1.2, but not (b).
\item[(c)] Give an example of a pair $(X,d)$ which obeys axioms (abd) of Definition~1.1.2, but not (c).
\item[(d)] Give an example of a pair $(X,d)$ which obeys axioms (abc) of Definition~1.1.2, but not (d). 
\hfill (Hint: try examples where $X$ is a finite set.)
\end{enumerate}
\end{problem}

\begin{problem}[20 pts]  
Let $x=(x_1,\dots,x_n)$ and $y=(y_1,\dots,y_n)$ be vectors in $\mathbb{R}^n$.

\vfill
\bigskip

\begin{enumerate}
\item[(a)] The $\ell^1$ metric is defined by
\[
d_1(x,y) := \sum_{i=1}^n |x_i - y_i|.
\]
Show that $d_1$ is a metric on $\mathbb{R}^n$


\item[(b)] The $\ell^\infty$ metric is defined by
\[
d_\infty(x,y) := \max_{1 \leq i \leq n} |x_i - y_i|.
\]
Show that $d_{\infty}$ is a metric on $\mathbb{R}^n$

\end{enumerate}
\end{problem}

%────────────────────────────────────────────────────────────────────────────────────────────────────────────────────────────────────────────────────

\begin{problem}[10 pts]  
  A \emph{vector space} $V$ over $\mathbb{R}$ s a set 
equipped with two operations:
\begin{enumerate}
  \item \textbf{Vector addition:} $+: V \times V \to V$, written $(u,v) \mapsto u+v$.
  \item \textbf{Scalar multiplication:} $\cdot : \mathbb{R} \times V \to V$, written $(\alpha,v) \mapsto \alpha v$,
\end{enumerate}
such that the following properties hold for all $u,v,w \in V$ and $\alpha,\beta \in \mathbb{R}$:
\begin{enumerate}
  \item[(VS1)] $(u+v)+w = u+(v+w)$ \hfill (associativity of addition)
  \item[(VS2)] $u+v = v+u$ \hfill (commutativity of addition)
  \item[(VS3)] There exists $0 \in V$ such that $u+0=u$ \hfill (additive identity)
  \item[(VS4)] For each $u \in V$, there exists $-u \in V$ such that $u+(-u)=0$ \hfill (additive inverse)
  \item[(VS5)] $\alpha(u+v) = \alpha u + \alpha v$ \hfill (distributivity I)
  \item[(VS6)] $(\alpha+\beta)u = \alpha u + \beta u$ \hfill (distributivity II)
  \item[(VS7)] $(\alpha\beta)u = \alpha(\beta u)$ \hfill (compatibility of scalar multiplication)
  \item[(VS8)] $1 \cdot u = u$ \hfill (identity element of scalar multiplication)
\end{enumerate}

A function $\|\cdot\| : V \to [0,\infty)$ is called a \emph{norm} on $V$ if, 
for all $u,v \in V$ and $\alpha \in \mathbb{R}$, the following properties hold:
\begin{enumerate}
  \item[(N1)] $\|v\| \geq 0$, and $\|v\| = 0$ if and only if $v=0$. \hfill (positivity)
  \item[(N2)] $\|\alpha v\| = |\alpha| \cdot \|v\|$. \hfill (homogeneity)
  \item[(N3)] $\|u+v\| \leq \|u\| + \|v\|$. \hfill (triangle inequality)
\end{enumerate}

Given a norm $\|\cdot\|$ on $V$, define $d : V \times V \to [0,\infty)$ by
\[
d(u,v) = \|u-v\|.
\]


 Prove that $d$ is a \emph{metric} on $V$, that is, for all $x,y,z \in V$ show that:
  \begin{enumerate}
    \item $d(x,y) \geq 0$ and $d(x,y)=0$ if and only if $x=y$.
    \item $d(x,y) = d(y,x)$.
    \item $d(x,z) \leq d(x,y) + d(y,z)$.
  \end{enumerate}
(Thus we conclude that every normed vector space $(V,\|\cdot\|)$ is also a metric space with metric $d(u,v)=\|u-v\|$.
)
\end{problem}
%────────────────────────────────────────────────────────────────────────────────────────────────────────────────────────────────────────────────────

\begin{problem}[10 pts]  
Let $S$ be a bounded nonempty set of real numbers, and let $a$ and $b$
be fixed  nonzero real numbers. Define $T=\{as+b| s\in S\}$ Find formulas
for
$\sup T$ and $\inf T$ in terms of $\sup S$ and $\inf S$. Prove your
formulas.
\end{problem}
\begin{proof}
\begin{claim}
  \(\sup T = a \sup S + b\). 
\end{claim}
\begin{explanation}
First notice that for all \(t \in T\), we can write \(t = as + b\) for some \(s \in S\). Hence, 
\[
  t = a s + b \le  a \sup S + b,
\]   
which means \(a \sup S + b\) is an upper bound of \(T\). Now if \(a \sup S + b \neq \sup T\), then there exists some \(t^{\prime} \in T\) such that \(t^{\prime} > a \sup S + b\), and we can write \(t^{\prime} = as^{\prime} +b\) for some \(s^{\prime} \in S\), so we have 
\[
  a s^{\prime} + b = t^{\prime} > a \sup S + b \iff s^{\prime}  > \sup S,
\] which is a contradiction, so \(\sup T = a \sup S + b\).        
\end{explanation}
\begin{claim}
  \(\inf T = a \inf S + b\) 
\end{claim}
\begin{explanation}
First notice that for all \(t \in T\), we can write \(t = as + b\) for some \(s \in S\). Hence, 
\[
  t = a s + b \ge  a \inf S + b,
\]   
which means \(a \inf S + b\) is a lower bound of \(T\). Now if \(a \inf S + b \neq \inf T\), then there exists some \(t^{\prime} \in T\) such that \(t^{\prime} < a \inf S + b\), and we can write \(t^{\prime} = as^{\prime} +b\) for some \(s^{\prime} \in S\), so we have 
\[
  a s^{\prime} + b = t^{\prime} < a \inf S + b \iff s^{\prime}  < \inf S,
\] which is a contradiction, so \(\inf T = a \inf S + b\).   
\end{explanation}
\end{proof}