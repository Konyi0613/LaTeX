\begin{problem}[15pts]
\vphantom{text}
      \begin{enumerate}
  \item[(a)] Let $(X,d_{\mathrm{disc}})$ be a metric space with the discrete metric.  
Let $E$ be a subset of $X$ which contains at least two elements.  
Show that $E$ is disconnected.

\medskip
\item[(b)] 
Let $f : X \to Y$ be a function from a connected metric space $(X,d)$ to a metric space $(Y,d_{\mathrm{disc}})$ with the discrete metric.  
Show that $f$ is continuous if and only if it is constant.  
\emph{(Hint: use part (a))}
 
  \end{enumerate}
\end{problem}

\begin{proof}[(a)]
    We know $E$ has at least two elements, and suppose $e_1, e_2 \in E$.
    
    Then we let $V = \{e_1\}, e_2 \in W = E \setminus \{e_1\}$. It is clear that $V, W$ are non-empty.
    
    And $\forall x \in V$, we can always find $B_X(x, 0.48763) = \{x\} \subseteq V$, so $V$ is open.
    
    Similarly, $\forall x \in W$, we can always find $B_X(x, 0.48763) = \{x\} \subseteq W$, so $W$ is open. 
    
    And clearly, $V \cap W = \varnothing$ and $V \cup W = E$, so $E$ is disconnected.
\end{proof}

\begin{proof}[(b)]
    \vphantom{text}
    \begin{itemize}
        \item [\((\implies )\)] Since we know $X$ is connected and $f$ is continuous, $f(X)$ is also connected.
        
        From (a) we know if $f(X)$ have more than two elements, then $f(X)$ is disconnected, and hence $f(X)$ only has single element or it is an empty set (which is not well-defined).
        
        Since $f(X)$ only have single element, $f$ is constant function.
        
        \item [\((\impliedby )\)] Let $f(x) = c$. The only possible subsets $V \subseteq Y$ are $\{c\}$ or the empty set.  

        If $V = \varnothing$, then $f^{-1}(V) = \varnothing$, and both $V$ and $f^{-1}(V)$ are open (and also closed).  
        
        If $V = \{c\}$, then $V$ is an open set since $Y$ is equipped with the discrete metric. Moreover, $f^{-1}(V) = X$, and the entire metric space $X$ is also open.  
        
        Therefore, $f$ satisfies the condition that whenever $V$ is open in $Y$, its preimage $f^{-1}(V)$ is open in $X$. Hence, $f$ is continuous.
    \end{itemize}
\end{proof}
%────────────────────────────────────────────────────────────────────────────────────────────────────────────────────────────────────────────────────

\begin{problem}[15pts]
    Let $(X,d)$ be a metric space, and let $(E_\alpha)_{\alpha\in I}$ be a collection of connected sets in $X$ with $I$ non-empty.  
Suppose also that $\bigcap_{\alpha\in I} E_\alpha$ is non-empty.  
Show that $\bigcup_{\alpha\in I} E_\alpha$ is connected.
\end{problem}
\begin{proof}
    Suppose by contradiction, \(\bigcup_{\alpha \in I} E_\alpha  \) is disconnected, then there exists non-empty \(V, W\) open in \(\bigcup_{\alpha \in I} E_\alpha  \) s.t. \(V \cup W = \bigcup_{\alpha \in I} E_\alpha  \) and \(V \cap W = \varnothing \). Hence, we know 
    \[
        \begin{dcases}
            V = O_1 \cap \left( \bigcup_{\alpha \in I} E_\alpha   \right) = \bigcup_{\alpha \in I} \left( O_1 \cap E_\alpha  \right) \\
            W = O_2 \cap \left( \bigcup_{\alpha \in I} E_\alpha   \right) = \bigcup_{\alpha \in I} \left( O_2 \cap E_\alpha  \right)  
        \end{dcases},
    \] where \(O_1, O_2\) are open in \(X\). Since \(I\) is non-empty, so we can suppose \(i \in I\) s.t. \(O_1 \cap E_i\) and \(O_2 \cap E_i\) are both open in \(E_i\). We first claim that there exists \(i, j \in I\) s.t. \(O_1 \cap E_i\) and \(O_2 \cap E_j\) are non-empty. Suppose by contradiction, \(O_1 \cap E_i = O_2 \cap E_j = \varnothing \) for all \(i, j \in I\), then \(V = O_1 \cap \bigcup_{\alpha \in I} E_\alpha  = \varnothing  \) and \(W = O_2 \cap \bigcup_{\alpha \in I} E_\alpha = \varnothing  \), which are contradictions. Now we claim that there exists \(i \in I\) s.t. \(O_1 \cap E_i \neq \varnothing \) and \(O_2 \cap E_i \neq \varnothing \).
    However, \(\exists p \in \bigcap_{\alpha \in I} E_\alpha \) since \(\bigcap_{\alpha \in I} E_\alpha\) is non-empty, so either \(p \in V\) or \(p \in W\), so either \(p \in O_1\) or \(p \in O_2\). (If \(p \in O_1 \cap O_2\), then \(p \in O_1 \cap O_2 \cap \bigcup_{\alpha \in I} E_\alpha  = V \cap W = \varnothing \).) WLOG, suppose \(p \in O_1\), then \(O_1 \cap E_k \neq \varnothing \) for all \(k \in I\), and since we know there exists \(j \in I\) s.t. \(O_2 \cap E_j \neq \varnothing \), so we know there exists \(i \in I\) s.t. \(O_1 \cap E_i\) and \(O_2 \cap E_i\) are both non-empty.           
    Now since 
    \[
        \bigcup_{\alpha \in I} E_\alpha = V \cup W = \left( O_1 \cap \bigcup_{\alpha \in I} E_\alpha \right) \cup \left( O_2 \cap \bigcup_{\alpha \in I} E_\alpha \right) = (O_1 \cup O_2) \cap \bigcup_{\alpha \in I} E_\alpha.  
    \]  Hence, we have \(\bigcup_{\alpha \in I} E_\alpha \subseteq O_1 \cup O_2\), which gives \(E_i \subseteq O_1 \cup O_2\). By this, we have
    \[
        (O_1 \cap E_i) \cup (O_2 \cap E_i) = (O_1 \cup O_2) \cap E_i = E_i.
    \]
    Now since 
    \[
        \varnothing = V \cap W = \left( O_1 \cap \bigcup_{\alpha \in I} E_\alpha \right) \cap \left( O_2 \cap \bigcup_{\alpha \in I} E_\alpha \right) = O_1 \cap O_2 \cap \bigcup_{\alpha \in I} E_\alpha, 
    \] so we have 
    \[
        \left( O_1 \cap E_i \right) \cap (O_2 \cap E_i)  = \left( O_1 \cap O_2 \right) \cap E_i = \varnothing. 
    \] However, we have shown that for \(A = O_1 \cap E_i\) and \(B = O_2 \cap E_i\), \(A, B\) are open in \(E_i\) and \(A \cup B = E_i\) and \(A \cap B = \varnothing \), which means \(E_i\) is disconnected, and this is a contradiction, so \(\bigcup_{\alpha \in I} E_\alpha\) must be connected.       
\end{proof}
%────────────────────────────────────────────────────────────────────────────────────────────────────────────────────────────────────────────────────

\begin{problem}[20pts]
    Let $(X,d)$ be a metric space, and let $E$ be a subset of $X$.  
We say that $E$ is \emph{path-connected} iff, for every $x,y \in E$, there exists a continuous function  
\[
\gamma : [0,1] \to E
\]
from the unit interval $[0,1]$ to $E$ such that $\gamma(0)=x$ and $\gamma(1)=y$.  
Show that every non-empty path-connected set is connected.  
(The converse is false, but is a bit tricky to show and will not be detailed here.)
\end{problem}

\begin{proof}
    Proof by contradiction. \\
    Assume $E$ is disconnected, then there exists non-empty $U, V$ such that $U \cup V = E$ and $U \cap V = \varnothing$.
    % \vphantom{text}
    \begin{itemize}
        \item Case 1: $x,y \in U$ or $x,y \in V$: \\
        Without loss of generality, suppose $x, y \in U$.\\
        Let $A = \gamma^{-1}(U)$ and $B = \gamma^{-1}(V)$, then $0 = \gamma^{-1}(x) \in A$, and $1 = \gamma^{-1}(y) \in A$, so we know that $A$ is non-empty. And we know that $\gamma^{-1}(U \cup V) = A \cup B = [0,1]$, $\gamma^{-1}(U \cap V) = A \cap B = \varnothing$. Since we know that $\gamma$ is a continuous function and $U, V$ are open set, it follows that $A, B$ are also the open set. By the theorem we have shown in class, $[0,1]$ is an interval, so $[0,1]$ is connected set. This force that $B$ is an empty set, and lead to $V$ is empty set, and hence we get a contradiction. 
        % We know that $\operatorname{Im}{\gamma} \cap A \neq \varnothing$
        % By the theorem we have shown in class, $[0,1] \subseteq A$. Since $\forall z \in E, \gamma^{-1}(z) \in [0,1] \subseteq A$, and hence $z \in U$. However, this will lead $V$ be an empty set and lead to a contradiction.
        \item Case 2: $x\in U, y \in V$ or $x \in V, y \in U$: \\
        Without loss of generality, suppose $x\in U, y \in V$. \\
        Let $A = \gamma^{-1}(U)$, $B = \gamma^{-1}(V)$, then $0 = \gamma^{-1}(x) \in A$, $1 = \gamma^{-1}(y) \in B$, and $\gamma^{-1}(U \cup V) = A \cup B = [0,1]$, $\gamma^{-1}(U \cap V) = A \cap B = \varnothing$ Since we know that $\gamma$ is a continuous function and $U, V$ are open set, it follows that $A, B$ are also the open set. And furthermore, $A, B$ are non-empty, $A \cup B = [0,1]$ and $A \cap B = \varnothing$. By the theorem we have shown in class, $[0,1]$ is an interval, so $[0,1]$ is connected set. This mean such $A, B$ is impossible to exist, and hence we get a contradiction.
    \end{itemize}
\end{proof}
%────────────────────────────────────────────────────────────────────────────────────────────────────────────────────────────────────────────────────

\begin{problem}[15pts]
    Let $(X,d)$ be a metric space, and let $E$ be a subset of $X$.  
Show that if $E$ is connected, then the closure $\overline{E}$ of $E$ is also connected.  
Is the converse true?
\end{problem}
\begin{proof}
    If \(E\) is connected, and suppose by contradiction, \(\overline{E} \) is disconnected, then \(\overline{E} = V \cup W\) and \(V \cap W = \varnothing \) for some non-empty \(V, W\) open in \(\overline{E} \). Hence, we can write \(V = O_1 \cap \overline{E} \) and \(W = O_2 \cap \overline{E} \), where \(O_1, O_2\) are open in \(X\). Now suppose \(A = O_1 \cap E\) and \(B = O_2 \cap E\), then we know \(A, B\) are open in \(E\). 
    \begin{claim}
        \(E \subseteq O_1 \cup O_2\). 
    \end{claim}              
    \begin{explanation}
        If \(\exists x \in E\) but \(x \notin O_1 \cup O_2\), then \(x \notin V\) and \(x \notin W\) since 
        \[
            \begin{dcases}
                V = O_1 \cap \overline{E} \\
                W = O_2 \cap \overline{E} .       
            \end{dcases}
        \]    
        Thus, \(x \notin V \cup W = X\), but \(x \in E \subseteq X\), so it is a contradiction, and thus \(E \subseteq O_1 \cup O_2\).   
    \end{explanation}
    Now by this claim, we know
    \[
        A \cup B = \left( O_1 \cap E \right) \cup \left( O_2 \cap E \right) = \left( O_1 \cup O_2 \right) \cap E = E.   
    \]
    Also, 
    \begin{align*}
         A \cap B &= \left( O_1 \cap E \right) \cap \left( O_2 \cap E \right) = O_1 \cap O_2 \cap E \\ & \subseteq V \cap W = \left( O_1 \cap \overline{E}  \right) \cap \left( O_2 \cap \overline{E}  \right)  = \left( O_1 \cap \left( E \cup \partial E \right)   \right) \cap \left( O_2 \cap \left( E \cup \partial E \right)  \right),  
    \end{align*}
    and since \(V \cap W = \varnothing \), so \(A \cap B = \varnothing \). Now we show that \(A, B\) are non-empty. If \(O_1 \cap E = \varnothing \), then since \(V\) is non-empty and 
    \[
        V = O_1 \cap \overline{E} = O_1 \cap \left( E \cup \partial E \right) = (O_1 \cap E) \cup \left( O_1 \cap \partial E \right),    
    \] so \(O_1 \cap \partial E \neq \varnothing \), say \(x \in O_1 \cap \partial E\). Then since \(O_1\) is open in \(X\), so there exists \(r > 0\) s.t. \(B_X(x, r) \subseteq O_1\), and since \(x \in \partial E\), so \(B_X(x, r) \cap E \neq \varnothing \), say \(y \in B_X(x, r) \cap E\). Thus, we have
    \[
        y \in B_X(x, r) \cap E \subseteq O_1 \cap E,
    \] but this means \(O_1 \cap E\) is non-empty, which is a contradiction. Hence, \(O_1 \cap E\) is non-empty, and we can use similar method to prove \(O_2 \cap E\) is non-empty. Now since 
    \[
        \begin{dcases}
            A, B \neq \varnothing \\
            A \cup B = E \\
            A \cap B = \varnothing \\
            A, B \text{ are open in } E,
        \end{dcases}
    \]   
    so \(E\) is connected, which is a contradiction, so \(\overline{E} \) must be connected. 
    
    Now we show that the converse may not be true. Suppose \(X = \mathbb{R} \) and \(\overline{E} = [1, 2] \), then we know \(\overline{E} \) is connected since in \(\mathbb{R} \) connected space is equivalent to an interval. However, we know \(E\) may be \((1, 1.5) \cup (1.5, 2)\), and this is disconnected in \(\mathbb{R} \), so this is a counterexample.        
    
\end{proof}
%────────────────────────────────────────────────────────────────────────────────────────────────────────────────────────────────────────────────────

\begin{problem}[20pts]
    Let $(X,d)$ be a metric space. Let us define a relation $x \sim y$ on $X$ by declaring  
$x \sim y$ iff there exists a connected subset of $X$ which contains both $x$ and $y$.  
Show that this is an equivalence relation (i.e., it obeys the reflexive, symmetric, and transitive axioms).  
Also, show that the equivalence classes of this relation (i.e., the sets of the form  
\[
\{ y \in X : y \sim x \} \quad \text{for some } x \in X
\]  
are all closed and connected.  
\emph{(Hint: use Problem 2 and Problem 4)}  
These sets are known as the \emph{connected components} of $X$.
You can read a note about equivalence relation in the file at NTU cool.
\end{problem}
\begin{proof}
    We first show that \(\sim \) is an equivalence relation. 
    \begin{itemize}
        \item reflexive: Note that for all \(x \in X\), \(\left\{ x \right\} \) is connected since it cannot be cut into two non-empty part, so \(x \sim x\). 
        \item symmetry: This is trivial since if there exists a connected subset of \(X\) which contains both \(x\) and \(y\), then this connected subset of \(X\) contains \(y\) and \(x\). 
        \item transitive: If \(x \sim y\) and \(y \sim z\), then we know there exists connected \(E_1, E_2\) s.t. \(E_1\) contains \(x, y\) and \(E_2\) contains \(y, z\). Since \(y \in E_1 \cap E_2\), so \(E_1 \cap E_2\) is non-empty, and thus by Problem 2 we know \(E_1 \cup E_2\) is connected, and since \(x, z \in E_1 \cup E_2\), so \(x \sim z\).                      
    \end{itemize} 
    Now if we fix \(x \in X\), and say \([x] = \left\{ y \in X: y \sim x \right\} \), then now we show that \([x]\) is closed and connected. We first show that \([x]\) is connected. If not, then there exists non-empty \(V, W\) s.t. \([x] = V \cup W\) and \(V \cap W = \varnothing \) and \(V, W\) are open in \([x]\). If \(x \in V\), then \(x \notin W\) and since \(W\) is non-empty, we know there exists \(z \neq x\) s.t. \(z \in W\). Since \(z \in W \subseteq [x]\), so there exists connected \(E_z\) s.t. \(x, z \in E_z\). 
    \begin{claim}
        \(E_z \subseteq [x]\). 
    \end{claim}              
    \begin{explanation}
        For all \(p \in E_z\), since we know \(x \in E_z\), so \(E_z\) is a connected set containing \(p\) and \(x\), so \(p \sim x\), which means \(p \in [x]\). Hence, \(E_z \subseteq [x]\).         
    \end{explanation}    
    Now we know 
    \[
        E_z = E_z \cap [x] = E_z \cap (V \cup W) = (E_z \cap V) \cup (E_z \cap W),
    \] and since \(x \in E_z \cap V\) and \(z \in E_z \cap W\), so \(E_z \cap V\) and \(E_z \cap W\) are non-empty. Also, since \(E_z \subseteq [x]\) and \(V, W\) are open in \([x]\), so \(E_z \cap V\) and \(E_z \cap W\) are open in \(E_z\). Besides, 
    \[
        (E_z \cap V) \cap (E_z \cap W) = E_z \cap (V \cap W) = E_z \cap \varnothing = \varnothing,
    \] so we know \(E_z\) is disconnected since 
    \[
        \begin{dcases}
            (E_z \cap V), (E_z \cap W) \neq \varnothing \\
            E_z = (E_z \cap V ) \cup (E_z \cup W) \\
            (E_z \cap V) \cap (E_z \cap W) = \varnothing \\
            \left( E_z \cap V \right), \left( E_z \cap W \right) \text{ are open in } E_z.  
        \end{dcases}
    \] 
    However, \(E_z\) is connected, so this is a contradiction. Hence, \([x]\) is connected. 
    
    Now we show that \([x]\) is closed. Since \([x]\) is connected and by Problem 4, we know \(\overline{[x]} \) is connected. Note that \(x \in \overline{[x]} \) since \(x \in [x]\) and thus for all \(r > 0\) we have \(x \in B_X(x, r) \cap [x]\). However, this means for all \(y \in \overline{[x]} \), we have \(x, y \in \overline{[x]} \) and \(\overline{[x]} \) is connected, so \(y \sim x\), which means \(y \in [x]\). Hence, \(\overline{[x]} \subseteq [x]\), and thus \([x]\) is closed.             
\end{proof}
%────────────────────────────────────────────────────────────────────────────────────────────────────────────────────────────────────────────────────
\begin{problem}[15pts]
    Let $f : S \to T$ be a function from a metric space $S$ to another metric space $T$.  
Assume $f$ is uniformly continuous on a subset $A$ of $S$ and that $T$ is complete.  
Prove that there is a unique extension of $f$ to $\overline{A}$ which is uniformly continuous on $\overline{A}$.
\end{problem}

\begin{proof}
    First, we define $\overline{f} : \overline{A} \to T$ as follows:
    \[
    \overline{f}(x) = \lim_{n \to \infty} f(a^{(n)})
    %\begin{cases}
    %f(x), & \text{if } x \in A, \\[6pt]
    %\lim_{n \to \infty} f(a^{(n)}), & \text{if } x \in % \overline{A}\setminus A,
    % \end{cases}
    \]
    where $(a^{(n)})_{n=1}^\infty \subseteq A$ is a sequence converging to $x$ in $\overline{A}$. \\
    Since $(a^{(n)})$ converges in $\overline{A}$, it is also a Cauchy sequence. 
    Because $f$ is uniformly continuous, $(f(a^{(n)}))$ is a Cauchy sequence in $T$. 
    By the completeness of $T$, the sequence $(f(a^{(n)}))$ converges and thus defines a limit. \\[6pt]
    We will now show that the extension is well-defined, unique, and uniformly continuous.


    \begin{itemize}
        \item Well-defined: \\
        First, we show that the value of $\overline{f}$ is independent of the choosing of convergent sequence.
        Suppose $(a^{(n)})_{n=1}^\infty, (b^{(n)})_{n=1}^\infty$ are two sequences in $A$, and both converge to $x \in \overline{A}$.
        Then
        \[
        \begin{aligned}
        &\forall \varepsilon_1 > 0, \ \exists N_1 \ \text{such that } 
           \forall n_1 \ge N_1, \ d_S(a^{(n_1)}, x) \le \varepsilon_1, \\[6pt]
        &\forall \varepsilon_2 > 0, \ \exists N_2 \ \text{such that } 
           \forall n_2 \ge N_2, \ d_S(b^{(n_2)}, x) \le \varepsilon_2.
        \end{aligned}
        \]
        Since $f$ is uniformly continuous, $\forall \varepsilon > 0$, we can pick some $\delta > 0$ such that $d_T(f(u), f(v)) < \varepsilon$ whenever $d_S(u, v) < \delta$ for all $u, v \in A$. \\
        Given that $\delta > 0, \text{we can pick } N = \max\{N_1', N_2'\} \ \text{such that}$
        \[
        \begin{aligned}
        \forall n_1 \ge N_1',\ & d_S\!\bigl(a^{(n_1)}, x\bigr) \le \tfrac{\delta}{2}, \\[6pt]
        \forall n_2 \ge N_2',\ & d_S\!\bigl(b^{(n_2)}, x\bigr) \le \tfrac{\delta}{2}.
        \end{aligned}
        \]
        And by triangular inequality,
        \[
        \forall n \ge N, d_S(a^{(n)}, b^{(n)}) \le d_S(a^{(n)}, x) + d_S(b^{(n)}, x) \le \delta
        \]
        This shows that indeed $d_T(f(a^{(n)}), f(b^{(n)})) < \varepsilon$ for all $n \ge N$, and hence $\lim_{n \to \infty} f(a^{(n)}) =  \lim_{n \to \infty} f(b^{(n)})$. Thus $\overline{f}$ is well-defined.

        And clearly, for those $x \in A$, $\overline{f}(x) = f(x)$ since we can choose $(a^{(n)}) = x$ for all $n$, so $\overline{f}|_A = f$, and hence $\overline{f}$ is indeed an extension of $f$.

        \item Uniqueness: \\
        Suppose there is another uniformly continuous $\overline{f'} : \overline{A} \to T$ such that it is also an extension. Since uniformly continuous imply the function is continuous. $\forall x \in \overline{A}$, we can take $(a^{(n)})_{n=1}^\infty \subseteq A \to x$, and 
        \[
        \overline{f'}(x) = \lim_{n \to \infty} \overline{f'}(a^{(n)}) = \lim_{n \to \infty} f(a^{(n)}) = \overline{f}(x)
        \]
        And we can find that $\overline{f} = \overline{f'}$, and hence the extension is unique.

        \item Uniformly continuity: \\
        We want to show that
        \[
        \forall \varepsilon > 0, \exists \delta > 0 \text{ such that } d_T(\overline{f}(x), \overline{f}(y)) < \varepsilon \text{ if } d_S(x, y) < \delta 
        \]
        And we know $f$ is uniformly continuous on $A$,
        \[
        \forall \varepsilon' > 0, \exists \delta' > 0 \text{ such that } d_T({f}(x), {f}(x')) < \varepsilon' \text{ if } d_S(x, x') < \delta' 
        \]
        \begin{claim}
            If we choose $\delta = \tfrac{\delta'}{3}, d_T(\overline{f}(x), \overline{f}(x')) < \varepsilon'$
        \end{claim}
        \begin{explanation}
            Suppose $x, y \in \overline{A}$ with $d_S(x,y) < \tfrac{\delta'}{3}$. \\
            We can choose sequence $(a^{(n)})_{n=1}^\infty$ and $(b^{(n)})_{n=1}^\infty$ in $A$ such that $(a^{(n)})_{n=1}^\infty \to x$ and $(b^{(n)})_{n=1}^\infty \to y$. \\
            Since $(a^{(n)})$ and $(b^{(n)})$ are converge, we may choose large enough $N$ such that
            \[
                \forall n \ge N, d_S(a^{(n)}, x) < \tfrac{\delta'}{3} \text{ and } d_S(b^{(n)}, y) < \tfrac{\delta'}{3}
            \]
            By triangular inequality,
            \[
                d_S(a^{(n)}, b^{(n)}) \le d_S(a^{(n)},x) + d_S(x,y) + d_S(y,b^{(n)}) \le \delta'
            \]
            And by the uniformly continuity of $f$,
            \[
            \forall n \ge N, d_T(f(a^{(n)}), f(b^{(n)})) < \varepsilon' \text{ since } \forall n \ge N, d_S(a^{(n)}, b^{(n)}) < \delta'
            \]
            So $d_T(\lim_{n \to \infty}f(a^{(n)}), \lim_{n \to \infty}f(b^{(n)})) < \varepsilon'$, and this means that $d_T(\overline{f}(x), \overline{f}(y)) < \varepsilon'$.
        \end{explanation}

        Since $\varepsilon'$ can be arbitrary, so $\forall \varepsilon = \varepsilon' > 0$ we can find such $\delta = \tfrac{\delta'}{3} > 0$ such that $d_T(\overline{f}(x), \overline{f}(y)) < \varepsilon$ whenever $d_S(x,y) < \delta$. Hence $\overline{f}$ is uniformly continuous on $\overline{A}$.
        
    \end{itemize}
\end{proof}