\begin{problem}[15pts]
Let $\sum_{n=0}^{\infty}  a_n x^n$ be a power series with radius of convergence $R$. Let $S_n = \sum_{k=0}^n a_k$ be the partial sums of $\sum a_n$. Denote the radius of convergence of $\sum_{n=0}^{\infty}  S_n x^n$ by $r$.

\begin{enumerate}
\item[(1)] Show that $r \le R$.

\item[(2)] Show that $\min\{1, R\} \le r$. \emph{Hint: The power series 
$\sum_{n=0}^{\infty} S_nx^n$
 can be seen as the Cauchy product between $\sum_{n=0}^{\infty}  a_n x^n$ and a 
 specific power series that you need to choose,).}
\end{enumerate}
\end{problem}

\begin{problem}[30pts]
    For each real $t$, define 
\[
f_t(x) = 
\begin{cases}
\dfrac{x e^{x t}}{e^x - 1}, & x \in \mathbb{R},\ x \ne 0,\\[6pt]
1, & x = 0.
\end{cases}
\]

\begin{itemize}
\item[(a)] Show that there exists $\delta>0$ such that $f_t$ admits a power
series expansion in $x$ for all $|x|<\delta$.

\medskip
\noindent\textit{Hint.}
Write
\[
f_t(x) = e^{xt} g(x)
\]
Where
\[
g(x) =
\begin{cases}
\dfrac{x}{e^x - 1}, & x \neq 0,\\[6pt]
1, & x = 0.
\end{cases}
\]
Both $e^{xt}$ and $g(x)$ are analytic near $0$.
Also $g(x)=\frac{1}{h(x)}$ where $h(x)=\dfrac{e^x - 1}{x}$ for $x \neq 0$ and we can express
it as an power series in $x$.
Then may use the fact that if $h$ is analytic on $\mathbb{R}$
and $h(0)\neq 0$, then $1/h$ is analytic on a smaller interval
$(-\delta,\delta)$.

\item[(b)] Define $P_0(t), P_1(t), P_2(t), \ldots$ by the equation
\[
f_t(x) = \sum_{n=0}^{\infty} P_n(t)\, \frac{x^n}{n!}, \qquad x \in (-\delta,\delta),
\]
and use the identity
\[
\sum_{n=0}^{\infty} P_n(t)\, \frac{x^n}{n!}
= e^{t x} \sum_{n=0}^{\infty} P_n(0)\, \frac{x^n}{n!}
\]
to prove that
\[
P_n(t) = \sum_{k=0}^{n} \binom{n}{k} P_k(0)\, t^{\,n-k}.
\]
(Hint:  $f_t(x)=e^{tx}f_0(x)$ and $f_0(x)=g(x)$.)
This shows that each function $P_n$ is a polynomial.  
These are the \emph{Bernoulli polynomials}.  
The numbers $B_n := P_n(0)$ ($n=0,1,2,\ldots$) are called the \emph{Bernoulli numbers}.  
Derive the following further properties:

\item[(c)] $B_0 = 1,\qquad B_1 = -\tfrac{1}{2},\qquad 
\sum_{k=0}^{n-1} \binom{n}{k} B_k = 0,\ \text{if } n=2,3,\ldots$

\item[(d)] $P_n'(t) = n\, P_{n-1}(t)$, \quad if $n=1,2,\ldots$

\item[(e)] $P_n(t+1) - P_n(t) = n\, t^{n-1}$, \quad if $n=1,2,\ldots$

\item[(f)] $P_n(1-t) = (-1)^n P_n(t)$

\item[(g)] $B_{2n+1} = 0$, \quad if $n=1,2,\ldots$

\item[(h)] 
\[
1^n + 2^n + \cdots + (k-1)^n = \frac{P_{n+1}(k) - P_{n+1}(0)}{n+1},
\qquad (n = 2,3,\ldots).
\]

\end{itemize}
\end{problem}

\begin{problem}[15pts \textbf{Exercise 4.2.7}]
    Show that for every integer $n \ge 3$, we have
$$
0 < \frac{1}{(n+1)!} + \frac{1}{(n+2)!} + \cdots < \frac{1}{n!}.
$$

\noindent
\textit{(Hint: first show that $(n+k)! > 2^k n!$ for all $k = 1, 2, 3, \ldots$.)}
Conclude that $n! e$ is not an integer for every $n \ge 3$.  
Deduce from this that $e$ is irrational.  
\textit{(Hint: prove by contradiction.)} 
\end{problem}
\begin{proof}
    For \(n \ge 3\), we know 
    \[
        (n+k)! = (n+k)(n+k-1)\dots (n+1) n! > 2^k
    \] for all \(k = 1,2,3,\dots \) since \(n + i > 2\) for all \(1 \le i \le k\). Hence, 
    \[
        \sum_{k=1}^{\infty} \frac{1}{(n+k)!} < \sum_{k=1}^{\infty} \frac{1}{2^k n!} = \frac{1}{n!} \sum_{k=1}^{\infty} \frac{1}{2^k} = \frac{1}{n!} \left( \frac{\frac{1}{2}}{1 - \frac{1}{2}} \right) = \frac{1}{n!},    
    \] and since \(\frac{1}{(n+k)!} > 0\) for all \(k = 1,2,3,\dots \), so 
    \[
        0 < \sum_{k=1}^{\infty} \frac{1}{(n+k)!} < \frac{1}{n!}. 
    \]  
    Thus, we know 
    \begin{align*}
        n! e &= n! \left( 1 + \frac{1}{1!} + \frac{1}{2!} + \dots  \right) = n! \left( 1 + \frac{1}{1!} + \dots + \frac{1}{n!} + \sum_{k=1}^{\infty} \frac{1}{(n+k)!}  \right) \\
        &< n!  \left( 1 + \frac{1}{1!} + \dots + \frac{1}{n!} + \frac{1}{n!} \right) = n! \sum_{m=0}^n \frac{1}{m!} + 1.  
    \end{align*}
    Also, we know 
    \begin{align*}
        n! e = n! \sum_{m=0}^{\infty} \frac{1}{m!} > n! \sum_{m=0}^n \frac{1}{m!},  
    \end{align*}
    so 
    \[
        n! \sum_{m=0}^n \frac{1}{m!} < n! e < n! \sum_{m=0}^n \frac{1}{m!} + 1,  
    \] which means \(n! e\) is not an integer. Now if \(e\) is rational, then \(e = \frac{q}{p}\) for some \(q \in \mathbb{Z} \) and \(p \in \mathbb{N} \), so \(n! e = \frac{n! q}{p}\), and if we pick \(n = \max \left\{ 3, p \right\} \), then we know \(\frac{n! q}{p}\) is an integer since \(p \mid n!\), but \(\frac{n! q}{p} = n! e\) is not an integer, so it is a contradiction, and thus \(e\) is irrational.            
\end{proof}

\begin{problem}[10pts \textbf{Exercise 4.5.6}]
Prove that the natural logarithm function $\ln x$ is real analytic on $(0,+\infty)$.  
  Hint: For any $a>0$, consider the change of variable $y=x-a$.
\end{problem}

\begin{problem}[10pts \textbf{Exercise 4.5.7}]
Let $f : (0,\infty) \to \mathbb{R}$ be a positive, real analytic function such that 
$f'(x) = f(x)$ for all $x \in \mathbb{R}$.  
Show that $f(x) = C e^x$ for some positive constant $C$; justify your reasoning.  
\textit{(Hint: there are basically three different proofs available. One proof uses the logarithm function, another proof uses the function $e^{-x}$, and a third proof uses power series. Of course, you only need to supply one proof.)}
\end{problem}

\begin{problem}[10pts \textbf{Exercise 4.5.8}]
    Let $m > 0$ be an integer.  
Prove
$$
\lim_{x \to +\infty} \frac{e^x}{x^m} = +\infty.
$$ without using the L'Hopital's rule.

\noindent
\textit{(Hint: $e^x \ge \sum_{k=0}^{m+1} \frac{x^k}{k!}$ for $x>0$.)}
\end{problem}
\begin{proof}
    Note that for \(x > 0\), 
    \[
        e^x = \sum_{k=0}^{\infty} \frac{x^k}{k!} > \sum_{k=0}^{m+1} \frac{x^k}{k!}.  
    \] 
    Thus, 
    \[
        \frac{e^x}{x^m} \ge \left( \sum_{k=0}^m \frac{1}{x^{m-k} k!}  \right) + \frac{x}{(m+1)!} > \frac{x}{(m+1)!}.
    \]
    Since \(\lim_{x \to +\infty} \frac{x}{(m+1)!} = +\infty  \), so \(\lim_{m \to +\infty} \frac{e^x}{x^m} = +\infty  \).  

\end{proof}

\begin{problem}[10pts \textbf{Exercise 4.5.9}]
Let $P(x)$ be a polynomial, and let $c>0$.  
Show that there exists a real number $N > 0$ such that $e^{cx} > |P(x)|$ for all $x > N$; 
thus an exponentially growing function, no matter how small the growth rate $c$, 
will eventually overtake any given polynomial $P(x)$, no matter how large.  
\textit{(Hint: use Exercise 4.5.8.)}
\end{problem}
\begin{proof}
    If \(P(x)\) is a constant polynomial, then there exists \(N > 0\) s.t. \(x > N\) implies \(e^{cx} > \vert P(x) \vert \) since \(e^{cx}\) is strictly increasing and has no upper bound. Now suppose \(P(x) = a_m x^m + \dots + a_1 x + a_0\) for some \(m \ge 1\), then by Problem 6 we know for all \(m \in \mathbb{N} \) we have  
    \[
        \lim_{cx \to \infty} \frac{e^{cx}}{c^m x^m} = \infty, 
    \] so for all \(i = 1,2, \dots , m\), there exists \(N_i > 0\) s.t. \(cx \ge N_i\) implies 
    \[
        \frac{e^{cx}}{c^i x^i} > (m+1) \left\vert \frac{a_i}{c^i} \right\vert \iff e^{cx} > (m+1) \vert a_i \vert x^i.
    \]       
    Also, we know there exists \(N_0 > 0\) s.t. \(x > N_0 \) implies \(e^{cx} > (m + 1) \vert a_0 \vert \) since \(e^{cx}\) is strictly increasing and has no upper bound.  
    Hence, we know \(x > \max \left\{ N_0, \left\lceil \frac{N_1}{c} \right\rceil, \left\lceil \frac{N_2}{c} \right\rceil, \dots , \left\lceil \frac{N_m}{c} \right\rceil    \right\} \) implies 
    \[
        (m+1) \cdot e^{cx} > \sum_{i=0}^m (m+1) \vert a_i \vert x^i = (m+1) \left\vert P(x) \right\vert,
    \] which means 
    \[
        e^{cx} > \vert P(x) \vert, 
    \] so we're done.
\end{proof}