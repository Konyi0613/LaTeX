\documentclass[addpoints]{exam}
% leqno 將方程式編號放在左側

%\printanswers

\usepackage[top=2.5cm,bottom=2cm,left=2.5cm,right=2.5cm,headsep=10pt,a4paper]{geometry} % 引用頁面幾何套件,設定頁面邊界

\usepackage{siunitx} % 角度
% \input{NewCommands}

\usepackage{amsmath,amssymb,amsthm} % 引用常見的數學符號與設定


\theoremstyle{definition}
\newtheorem*{definition*}{Definition}
\newtheorem*{theorem}{Theorem}
\newtheorem*{remark}{Remark}
\newtheorem*{claim}{Claim}
\newtheorem*{proposition}{Proposition
}
\usepackage{lastpage} % 取得最後頁碼




\usepackage{graphicx} % 引用圖檔內嵌入套件
\graphicspath{{Pictures/}} % 設定圖檔目錄
\usepackage{xcolor} %引用顏色套件
\usepackage{enumerate}
\usepackage{float}
\usepackage{hyperref}
\usepackage{mdwlist} % use suspend & resume to successive enumerate

\usepackage{CJKutf8} % 設定中文

\def \lflr{\left\lfloor} %自定義 floor function
\def \rflr{\right\rfloor} %自定義

% question separation
\renewcommand{\questionshook}{%
  \setlength{\itemsep}{0.5cm}%
}

\allowdisplaybreaks %equation allowed to next page

\firstpageheader{}
                {}
                {}
\footer{}{\sf Page \thepage~of~\pageref{LastPage}}{}

\begin{document}

\begin{CJK*}{UTF8}{gbsn}
% Title
\begin{center}
   {\Large{\bf Introduction to Mathematical Analysis\\
    Homework 9 Due  November  21  (Friday), 2025\\
    Please submit your homework online in PDF format.
  %  Homework X Brief Solution
  }} \\ 
    \large
    %\text{Due date: 3/9/2023}
\end{center}

\noindent\rule{16.2cm}{0.4pt}


\begin{questions}


% ============ Question x ============
\question (15 pts) Let $\sum_{n=0}^{\infty}  a_n x^n$ be a power series with radius of convergence $R$. Let $S_n = \sum_{k=0}^n a_k$ be the partial sums of $\sum a_n$. Denote the radius of convergence of $\sum_{n=0}^{\infty}  S_n x^n$ by $r$.

\begin{enumerate}
\item[(1)] Show that $r \le R$.

\item[(2)] Show that $\min\{1, R\} \le r$. \emph{Hint: The power series 
$\sum_{n=0}^{\infty} S_nx^n$
 can be seen as the Cauchy product between $\sum_{n=0}^{\infty}  a_n x^n$ and a 
 specific power series that you need to choose,).}
\end{enumerate}

\begin{solution}




\end{solution}

\question (30 pts)
For each real $t$, define 
\[
f_t(x) = 
\begin{cases}
\dfrac{x e^{x t}}{e^x - 1}, & x \in \mathbb{R},\ x \ne 0,\\[6pt]
1, & x = 0.
\end{cases}
\]

\begin{itemize}
\item[(a)] Show that there exists $\delta>0$ such that $f_t$ admits a power
series expansion in $x$ for all $|x|<\delta$.

\medskip
\noindent\textit{Hint.}
Write
\[
f_t(x) = e^{xt} g(x)
\]
Where
\[
g(x) =
\begin{cases}
\dfrac{x}{e^x - 1}, & x \neq 0,\\[6pt]
1, & x = 0.
\end{cases}
\]
Both $e^{xt}$ and $g(x)$ are analytic near $0$.
Also $g(x)=\frac{1}{h(x)}$ where $h(x)=\dfrac{e^x - 1}{x}$ for $x \neq 0$ and we can express
it as an power series in $x$.
Then may use the fact that if $h$ is analytic on $\mathbb{R}$
and $h(0)\neq 0$, then $1/h$ is analytic on a smaller interval
$(-\delta,\delta)$.

\item[(b)] Define $P_0(t), P_1(t), P_2(t), \ldots$ by the equation
\[
f_t(x) = \sum_{n=0}^{\infty} P_n(t)\, \frac{x^n}{n!}, \qquad x \in (-\delta,\delta),
\]
and use the identity
\[
\sum_{n=0}^{\infty} P_n(t)\, \frac{x^n}{n!}
= e^{t x} \sum_{n=0}^{\infty} P_n(0)\, \frac{x^n}{n!}
\]
to prove that
\[
P_n(t) = \sum_{k=0}^{n} \binom{n}{k} P_k(0)\, t^{\,n-k}.
\]
(Hint:  $f_t(x)=e^{tx}f_0(x)$ and $f_0(x)=g(x)$.)
This shows that each function $P_n$ is a polynomial.  
These are the \emph{Bernoulli polynomials}.  
The numbers $B_n := P_n(0)$ ($n=0,1,2,\ldots$) are called the \emph{Bernoulli numbers}.  
Derive the following further properties:

\item[(c)] $B_0 = 1,\qquad B_1 = -\tfrac{1}{2},\qquad 
\sum_{k=0}^{n-1} \binom{n}{k} B_k = 0,\ \text{if } n=2,3,\ldots$

\item[(d)] $P_n'(t) = n\, P_{n-1}(t)$, \quad if $n=1,2,\ldots$

\item[(e)] $P_n(t+1) - P_n(t) = n\, t^{n-1}$, \quad if $n=1,2,\ldots$

\item[(f)] $P_n(1-t) = (-1)^n P_n(t)$

\item[(g)] $B_{2n+1} = 0$, \quad if $n=1,2,\ldots$

\item[(h)] 
\[
1^n + 2^n + \cdots + (k-1)^n = \frac{P_{n+1}(k) - P_{n+1}(0)}{n+1},
\qquad (n = 2,3,\ldots).
\]

\end{itemize}

\begin{solution}




\end{solution}



\question (15 pts)
\noindent\textbf{Exercise 4.2.7.}  
Show that for every integer $n \ge 3$, we have
$$
0 < \frac{1}{(n+1)!} + \frac{1}{(n+2)!} + \cdots < \frac{1}{n!}.
$$

\noindent
\textit{(Hint: first show that $(n+k)! > 2^k n!$ for all $k = 1, 2, 3, \ldots$.)}
Conclude that $n! e$ is not an integer for every $n \ge 3$.  
Deduce from this that $e$ is irrational.  
\textit{(Hint: prove by contradiction.)}  
  \begin{solution}




\end{solution}


  
  

  \question (10 pts)
  \noindent\textbf{Exercise 4.5.6}
  Prove that the natural logarithm function $\ln x$ is real analytic on $(0,+\infty)$.  
  Hint: For any $a>0$, consider the change of variable $y=x-a$.
  
\begin{solution}




\end{solution}



  \question (10 pts)
  
\noindent\textbf{Exercise 4.5.7}
Let $f : (0,\infty) \to \mathbb{R}$ be a positive, real analytic function such that 
$f'(x) = f(x)$ for all $x \in \mathbb{R}$.  
Show that $f(x) = C e^x$ for some positive constant $C$; justify your reasoning.  
\textit{(Hint: there are basically three different proofs available. One proof uses the logarithm function, another proof uses the function $e^{-x}$, and a third proof uses power series. Of course, you only need to supply one proof.)}

\begin{solution}




\end{solution}

 \question (10 pts)
 \noindent\textbf{Exercise 4.5.8}
Let $m > 0$ be an integer.  
Prove
$$
\lim_{x \to +\infty} \frac{e^x}{x^m} = +\infty.
$$ without using the L'Hopital's rule.

\noindent
\textit{(Hint: $e^x \ge \sum_{k=0}^{m+1} \frac{x^k}{k!}$ for $x>0$.)}

\begin{solution}




\end{solution}

 \question  (10 pts) \noindent\textbf{Exercise 4.5.9}
Let $P(x)$ be a polynomial, and let $c>0$.  
Show that there exists a real number $N > 0$ such that $e^x > |P(x)|$ for all $x > N$; 
thus an exponentially growing function, no matter how small the growth rate $c$, 
will eventually overtake any given polynomial $P(x)$, no matter how large.  
\textit{(Hint: use Exercise 4.5.8.)}

\begin{solution}




\end{solution}




\end{questions}

\bigskip

 You can do the following problems to practice. You don’t have to submit the following problems.

\begin{questions}

\question 
\noindent\textbf{Exercise 4.5.4}
Let $f : \mathbb{R} \to \mathbb{R}$ be the function defined by setting 
$f(x) := \exp(-1/x)$ when $x > 0$, and $f(x) := 0$ when $x \le 0$.  
Prove that $f$ is infinitely differentiable, and $f^{(k)}(0) = 0$ for every integer 
$k \ge 0$, but that $f$ is \emph{not} real analytic at $0$.

 
\begin{solution}




\end{solution}

 \question  In class, we proved that the function $f(x)=a^x$ is continuous on $\mathbb{Q}$ for $a>1$.  
Let $n \in \mathbb{N}$. Prove that $f$ is uniformly continuous on the rational interval
$$
[-n,n] \cap \mathbb{Q}.
$$ 
\noindent\textbf{Remark.}  
If this is true, then $f(x)=a^x$ admits a unique continuous extension to all real numbers 
$x \in [-n,n]$. 
\begin{solution}




\end{solution}


 \question Define the sequence
$$
\forall n \ge 1, \qquad S_n = \sum_{k=1}^n \ln k.
$$

\begin{enumerate}
\item[(1)]
Show that for every $k \ge 2$, we have
$$
\int_{k-1}^{k} \ln t \, dt \;\le\; \ln k \;\le\; \int_{k}^{k+1} \ln t \, dt.
$$
Deduce that
$$
S_n = n \ln n - n + o(n).
$$

\item[(2)]
By considering the sequence $(A_n)_{n\ge 1}$, defined by
$$
\forall n \ge 1, \qquad A_n = S_n - n \ln n + n,
$$
show that $A_n - A_{n-1} \sim \dfrac{1}{2n}$ and deduce that
$$
A_n \sim \frac12 \ln n.
$$

\item[(3)]
Let
$$
D_n := S_n - n \ln n + n - \frac12 \ln n \qquad \text{for } n \ge 1.
$$
Show that
$$
D_n - D_{n-1} \sim -\,\frac{1}{12 n^2}.
$$

\item[(4)]
Show that $D_n$ converges to some $D_\infty$ when $n \to \infty$.  
Deduce that there exists some constant $C > 0$ such that
$$
n! \sim C \left( \frac{n}{e} \right)^n \sqrt{n}.
$$

\item[(5)]
Using the expression of $I_{2n}= \int_{0}^{\pi/2} \sin^{2n} x \, dx
=\frac{\pi}{2} \cdot \frac{(2n)!}{2^{2n}(n!)^2} = \sqrt{\frac{\pi}{4n}}\,(1 + o(1))$ (proved in the following), show that
$$
C = \sqrt{2\pi}.
$$

\item[(6)]
Show that
$$
n! \sim \sqrt{2\pi n}\left(\frac{n}{e}\right)^n 
\left( 1 + \frac{1}{12n} + o\!\left(\frac{1}{n}\right) \right).
$$

\end{enumerate}
 
\begin{solution}




\end{solution}

\question Let $\mathcal{P}$ be the set of all the primes.  
In this exercise, we will prove that $\displaystyle \sum_{p \in \mathcal{P}} \frac{1}{p}$ is divergent.

\begin{enumerate}
\item[(1)]  
Show that for $s > 1$, we have
$$
- \sum_{p \in \mathcal{P}} \log\!\left(1 - \frac{1}{p^{\,s}}\right)
= \log \zeta(s).
$$

\item[(2)]
Deduce that there exists $M > 0$ such that for any $s > 1$, we have
$$
\left|\, \sum_{p\in\mathcal{P}} \frac{1}{p^{\,s}} - \log \zeta(s) \,\right| < M.
$$

\item[(3)]
Show that as $s \to 1^{+}$, we have $\zeta(s) \to +\infty$.

\item[(4)]
Conclude that  
$$
\sum_{p \in \mathcal{P}} \frac{1}{p}
$$
is divergent.

\end{enumerate}

\begin{solution}




\end{solution}


\begin{theorem}[Wallis Integrals --- Factorial Version]
For each integer $n \ge 0$, define
\[
I_n := \int_{0}^{\pi/2} \sin^n x \, dx.
\]
Then:

\begin{itemize}
\item[(a)]
\[
I_0 = \frac{\pi}{2}, \qquad I_1 = 1.
\]

\item[(b)]
For all $n \ge 2$,
\[
n I_n = (n-1) I_{n-2}.
\]

\item[(c)]
For each $m\in\mathbb{N}$,
\[
I_{2m-1} 
= \frac{2^{\,2m-1}(m-1)!\,m!}{(2m)!},
\qquad
I_{2m}
= \frac{\pi}{2}\cdot \frac{(2m)!}{2^{\,2m}(m!)^2}.
\]

\item[(d)]
For all $n\ge1$,
\[
I_n I_{n-1} = \frac{\pi}{2n}.
\]

\item[(e)]
As $n\to\infty$,
\[
I_n = \sqrt{\frac{\pi}{2n}}\,(1+o(1)).
\]

\item[(f)]
In particular,
\[
I_{2n}
= \frac{\pi}{2}\cdot \frac{(2n)!}{2^{\,2n}(n!)^2}.
\]
\end{itemize}
\end{theorem}

\begin{proof}

\textbf{(a) Direct computation.}
We directly compute:
\[
I_0 = \int_0^{\pi/2} 1\,dx = \frac{\pi}{2},
\qquad
I_1 = \int_0^{\pi/2} \sin x \, dx = [-\cos x]_{0}^{\pi/2} = 1.
\]


\medskip
\noindent
\textbf{(b) Recurrence formula.}

For $n\ge2$,
\[
I_n = \int_0^{\pi/2}\sin^{n-1}x\sin x\,dx.
\]
Using $u=\sin^{n-1}x$ and $dv=\sin x\,dx$ gives
\[
n I_n = (n-1) I_{n-2}.
\]

\medskip
\noindent
\textbf{(c) Explicit formulas.}

Iterating the recurrence gives:

\emph{Odd case:} $n=2m-1$,
\[
I_{2m-1}
= \frac{2m-2}{2m-1}\cdot\frac{2m-4}{2m-3}\cdots\frac{2}{3}\cdot I_1
= \frac{2^{\,2m-1}(m-1)!\,m!}{(2m)!}.
\]

\emph{Even case:} $n=2m$,
\[
I_{2m}== \frac{2m-1}{2m}\cdot\frac{2m-3}{2m-2}\cdots\frac{1}{2}\cdot I_0
= \frac{\pi}{2}\cdot\frac{(2m)!}{2^{\,2m}(m!)^2}.
\]

\medskip
\noindent
\textbf{(d) Product formula.}

For $n=2m$,
\[
I_{2m}I_{2m-1}
= \left(\frac{\pi}{2}\cdot\frac{(2m)!}{2^{2m}(m!)^2}\right)
  \left(\frac{2^{2m-1}(m-1)!\,m!}{(2m)!}\right)
= \frac{\pi}{2(2m)}
= \frac{\pi}{2n}.
\]
For $n=2m+1$, a similar calculation yields \[
I_{2m+1}I_{2m}
= \frac{\pi}{2(2m+1)}
= \frac{\pi}{2n}.
\] 
Thus,
\[
I_n I_{n-1} = \frac{\pi}{2n}.
\]

\medskip
\noindent
\textbf{(e) Asymptotics.}

The product identity gives
\[
I_{n+1} I_n = \frac{\pi}{2(n+1)},
\qquad
I_n I_{n-1} = \frac{\pi}{2n}.
\]
Since $I_{n+1} \le I_n \le I_{n-1}$,
\[
\frac{\pi}{2(n+1)} \le I_n^2 \le \frac{\pi}{2n}.
\]
Multiplying by $\frac{2n}{\pi}$:
\[
\frac{n}{n+1} \le \frac{2n}{\pi}I_n^2 \le 1.
\]
Taking square roots:
\[
\sqrt{\frac{n}{n+1}}
\le \sqrt{\frac{2n}{\pi}}\,I_n \le 1.
\]
Thus,
\[
I_n \sim \sqrt{\frac{\pi}{2n}}.
\]

\medskip
\noindent
\textbf{(f) Even case formula.}

Directly from part (c),
\[
I_{2n}
= \frac{\pi}{2}\cdot\frac{(2n)!}{2^{\,2n}(n!)^2}.
\]

\end{proof}
\end{questions}

% \appendix
% \section{}
\end{CJK*}
\end{document}

