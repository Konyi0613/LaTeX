\begin{problem}
    Let $(x^{(n)})_{n=m}^\infty$ be a sequence of points in a metric space $(X,d)$, and let $L\in X$. Show that if $L$ is a limit point of the sequence $(x^{(n)})_{n=m}^\infty$, then $L$ is an adherent point of the set
\[
S = \{ x^{(n)} : n\ge m \}.
\]
Is the converse true?
\end{problem}

\begin{problem}
    The following construction generalizes the construction of the reals from the rationals in Chapter~5, allowing one to view any metric space as a subspace of a complete metric space. In what follows we let $(X,d)$ be a metric space.
\begin{enumerate}
  \item[(a)] Given any Cauchy sequence $(x_n)_{n=1}^\infty$ in $X$, we introduce the \emph{formal limit} 
  \[
  \operatorname{LIM}_{n\to\infty} x_n.
  \]
  We say that two formal limits $\operatorname{LIM}_{n\to\infty} x_n$ and $\operatorname{LIM}_{n\to\infty} y_n$ are equal if 
  \[
  \lim_{n\to\infty} d(x_n,y_n) = 0.
  \]
  Show that this equality relation obeys the reflexive, symmetry, and transitive axioms, i.e.\ that it is an equivalence relation.

  \item[(b)] Let $\overline{X}$ be the space of all formal limits of Cauchy sequences in $X$, modulo the above equivalence relation. Define a metric $d_{\overline{X}}:\overline{X}\times\overline{X}\to [0,\infty)$ by
  \[
  d_{\overline{X}}\!\left(\operatorname{LIM}_{n\to\infty}x_n, \operatorname{LIM}_{n\to\infty} y_n\right) := \lim_{n\to\infty} d(x_n,y_n).
  \]
  Show that this function is well-defined (the limit exists and does not depend on the choice of representatives) and that it satisfies the axioms of a metric. Thus $(\overline{X},d_{\overline{X}})$ is a metric space.

  \item[(c)] Show that the metric space $(\overline{X},d_{\overline{X}})$ is complete.

  \item[(d)] We identify an element $x\in X$ with the corresponding constant Cauchy sequence $(x,x,x,\dots)$, i.e.\ with the formal limit $\operatorname{LIM}_{n\to\infty} x$. Show that this is legitimate: for $x,y\in X$, 
  \[
  x=y \quad \Longleftrightarrow \quad \operatorname{LIM}_{n\to\infty} x = \operatorname{LIM}_{n\to\infty} y.
  \]
  With this identification, show that 
  \[
  d(x,y) = d_{\overline{X}}(x,y),
  \]
  and thus $(X,d)$ can be thought of as a subspace of $(\overline{X},d_{\overline{X}})$.

  \item[(e)] Show that the closure of $X$ in $\overline{X}$ is $\overline{X}$ itself. (This explains the choice of notation.)

  \item[(f)] Finally, show that the formal limit agrees with the actual limit: if $(x_n)_{n=1}^\infty$ is a Cauchy sequence in $X$ that converges in $X$, then
  \[
  \lim_{n\to\infty} x_n = \operatorname{LIM}_{n\to\infty} x_n \quad \text{in } \overline{X}.
  \]
\end{enumerate}
\end{problem}
\begin{proof}[a]
  We verify the following properties:
    \begin{itemize}
        \item Reflexive: $\operatorname{LIM}_{n\to\infty} x_n$ and $\operatorname{LIM}_{n\to\infty} x_n$ are equal since $d$ is metric, so $\forall n, d(x_n, x_n) = 0$.
        \item Symmetry: If $\operatorname{LIM}_{n\to\infty} x_n$ and $\operatorname{LIM}_{n\to\infty} y_n$ are equal, this mean $\lim_{n\to\infty} d(x_n,y_n) = 0$. And since $d$ is metric, so $\lim_{n\to\infty} d(y_n,x_n) = 0$, hence $\operatorname{LIM}_{n\to\infty} y_n$ and $\operatorname{LIM}_{n\to\infty} x_n$ are equal.
        \item Transitive: If $\operatorname{LIM}_{n\to\infty} x_n$ and $\operatorname{LIM}_{n\to\infty} y_n$ are equal and $\operatorname{LIM}_{n\to\infty} y_n$ and $\operatorname{LIM}_{n\to\infty} z_n$ are equal, then we have \(\lim_{n \to \infty} d(x_n, y_n) = \lim_{n \to \infty}  d(y_n, z_n) = 0  \). By definition, there exists \(N_1, N_2 > 0\) s.t. for all \(n \ge N_1\), we have \(d(x_n, y_n) < \frac{\varepsilon}{2}\) and for all \(n \ge N_2\) we have \(d(y_n, z_n) < \frac{\varepsilon}{2}\). Thus, for all \(n \ge \max \left\{ N_1, N_2 \right\} \), we have
        \[
            d(x_n, z_n) \le d(x_n, y_n) + d(y_n, z_n) < \frac{\varepsilon}{2} + \frac{\varepsilon}{2} = \varepsilon ,
        \] which means \(\lim_{n \to \infty} d(x_n, z_n) = 0 \), and thus \(\mathrm{LIM}_{n \to \infty} x_n = \mathrm{LIM}_{n \to \infty } z_n  \).  
    \end{itemize}
\end{proof}
\begin{proof}[b]
  We first show that the limit exists. Note that \(\lim_{n \to \infty} d(x_n, y_n) \in \mathbb{R}_{\ge 0} \) for all Cauchy sequence \(\left\{ x_n \right\}_{n=1}^{\infty} , \left\{ y_n \right\}_{n=1}^{\infty}   \) in \(X\). We already know \((\mathbb{R}, \vert \cdot \vert ) \) is complete, so we know \((\mathbb{R} _{\ge 0}, \vert \cdot \vert )\) is also complete as it is a closed subspace of \((\mathbb{R}, \vert \cdot \vert ) \). Now we define \(u_n \coloneqq d(x_n, y_n)\) for all \(n \ge 1\), then we want to show that \(\left\{ u_n \right\}_{n=1}^{\infty}  \) is Cauchy in \(\mathbb{R}_{\ge 0}\) , and then we can conclude that \(\left\{ u_n \right\}_{n=1}^{\infty}  \) converges in \(\mathbb{R} _{\ge 0}\), and thus \(\lim_{n \to \infty} d(x_n, y_n) \) exists.
 \begin{claim} \label{clm: reverse triangle}
  Suppose \((X, d)\) is a metric space, then for all \(a, b, c, d \in X\) we have 
  \[
    \left\vert d(a, b) - d(c, d) \right\vert \le d(a, c) + d(b, d)
  \]  
 \end{claim}
 \begin{explanation}
    Since
    \[
      \begin{dcases}
        d(a, b) \le d(a, c) + d(c, b) \le d(a, c) + d(c, d) + d(d, b) \\
        d(c, d) \le d(c, a) + d(a, d) \le d(c, a) + d(a, b) + d(b, d),
      \end{dcases}
    \] so we have 
    \[
      \begin{dcases}
        d(a, b) - d(c, d) \ge d(a, c) + d(d, b) \\
        -d(c, a) - d(b, d) \le d(a, b) - d(c, d),
      \end{dcases}
    \] so we can conbine these two equations and get the result.
 \end{explanation}
  By \autoref{clm: reverse triangle}, we know for all \(p, q \ge 1\), we have 
  \[
    \left\vert u_p - u_q \right\vert = \left\vert d(x_p, y_p) - d(x_q, y_q) \right\vert \le d(x_p, x_q) + d(y_p, y_q).
  \]  
  Now since \(\left\{ x_n \right\}_{n=1}^{\infty}  \) and \(\left\{ y_n \right\}_{n=1}^{\infty }  \) are Cauchy, so for every \(\varepsilon > 0\), there exists \(N_1, N_2 > 0\) s.t. 
  \[
    \begin{dcases}
      d(x_p, x_q) < \frac{\varepsilon}{2} \quad \forall p, q \ge N_1 \\
      d(y_p, y_q) < \frac{\varepsilon}{2} \quad \forall p, q \ge N_2.
    \end{dcases}
  \]   
  Thus, for all \(p, q \ge \max \left\{ N_1, N_2 \right\} \), we know
  \[
    \vert u_p - u_q \vert \le d(x_p, x_q) + d(y_p, y_q) \le \frac{\varepsilon}{2} + \frac{\varepsilon}{2} = \varepsilon.
  \]
  Hence, we know \(\left\{ u_n \right\}_{n=1}^{\infty}  \) is Cauchy in \(\mathbb{R}_{\ge 0}, \vert \cdot \vert \).  

  Now we show that \(d_{\overline{X} }\) is well-defined. In other words, if \(\mathrm{LIM}_{n \to \infty } x_n = \mathrm{LIM}_{n \to \infty } z_n \), then we want to show
  \[
    d_{\overline{X} }\left( \mathrm{LIM}_{n \to \infty }x_n, \mathrm{LIM}_{n \to \infty }y_n   \right) = d_{\overline{X} } \left( \mathrm{LIM}_{n \to \infty }z_n, \mathrm{LIM}_{n \to \infty } y_n  \right) \quad \forall \text{ Cauchy } \left\{ y_n \right\}_{n=1}^{\infty} \text{ in } (X, d).   
  \]  Equivalently, we want to show \(\lim_{n \to \infty} d(x_n, y_n) = \lim_{n \to \infty} d(z_n, y_n) \). Note that we have 
  \[
     \lim_{n \to \infty} d(x_n, z_n) = 0 \text{ and } d(x_n, y_n) \le d(x_n, z_n) + d(z_n, y_n),
  \] so we know
  \[
    \lim_{n \to \infty} d(x_n, y_n) \le \lim_{n \to \infty} d(x_n, z_n) + \lim_{n \to \infty} d(z_n, y_n) = \lim_{n \to \infty} d(z_n, y_n).  
  \]
  Also, we have \(d(z_n, y_n) \le d(z_n, x_n) + d(x_n, y_n)\), so we know 
  \[
  \lim_{n \to \infty} d(z_n, y_n) \le \lim_{n \to \infty} d(z_n, x_n) + \lim_{n \to \infty} d(x_n, y_n) = \lim_{n \to \infty} d(x_n, y_n),
  \] and thus we can conclude that \(\lim_{n \to \infty} d(x_n, y_n) = \lim_{n \to \infty} d(z_n, y_n)  \). 
  
  Finally, we want to show that \(\left( \overline{X}, d_{\overline{X} }  \right) \) is a metric space. 
  \begin{itemize}
    \item \(\forall \text{ Cauchy } \left\{ x_n \right\}_{n=1}^{\infty} \in X, \ d_{\overline{X} } (\mathrm{LIM}_{n \to \infty } x_n, \mathrm{LIM}_{n\to \infty } x_n  ) = \lim_{n \to \infty} d(x_n, x_n) = 0  \). 
    \item \(\forall \text{ Cauchy } \left\{ x_n \right\}_{n=1}^{\infty} , \left\{ y_n \right\}_{n=1}^{\infty} \in X  \), 
    \begin{align*}
      d_{\overline{X} }(\mathrm{LIM}_{n \to \infty }x_n, \mathrm{LIM}_{n \to \infty }y_n  ) = \lim_{n \to \infty} d(x_n, y_n)  &= \lim_{n \to \infty} d(y_n, x_n) \\ &= d_{\overline{X}} (\mathrm{LIM}_{n \to \infty } y_n, \mathrm{LIM}_{n \to \infty } x_n  )   
    \end{align*}
    \item \(\forall \text{ Cauchy } \left\{ x_n \right\}_{n=1}^{\infty} , \left\{ y_n \right\}_{n=1}^{\infty}, \left\{ z_n \right\}_{n=1}^{\infty}   \in X  \), \begin{align*}
      d_{\overline{X} }(\mathrm{LIM}_{n \to \infty }x_n, \mathrm{LIM}_{n \to \infty }z_n  ) &= \lim_{n \to \infty} d(x_n, z_n) \\
      &\le \lim_{n \to \infty} (d(x_n, y_n) + d(y_n, z_n)) = \lim_{n \to \infty} d(x_n, y_n) + \lim_{n \to \infty} d(y_n, z_n) \\
      &= d_{\overline{X} } (\mathrm{LIM}_{n \to \infty }x_n, \mathrm{LIM}_{n \to \infty }y_n ) + d_{\overline{X} } (\mathrm{LIM}_{n \to \infty }y_n, \mathrm{LIM}_{n \to \infty }z_n ). 
    \end{align*}
  \end{itemize}
  Hence, we know \(\left( \overline{X}, d_{\overline{X} }  \right) \) is a metric space. 
\end{proof}
\begin{proof}[c]
  We want to show that for all \(\left\{ u_n \right\}_{n=1}^{\infty}  \subseteq \overline{X}  \), there exists \(\left\{ z_n \right\}_{n=1}^{\infty} \subseteq X \) s.t. \(\lim_{n \to \infty} u_n = \mathrm{LIM}_{n \to \infty } z_n \). Since \(\left\{ u_n \right\}_{n=1}^{\infty } \) is a sequence of formal limit of Cauchy sequences in \(X\), so we can define \(u_k = \mathrm{LIM}_{n \to \infty } x_n^{(k)} \) for all \(k \ge 1\). Now we construct \(\left\{ z_n \right\}_{n=1}^{\infty}  \). Since we know for all \(k \ge 1\), \(\left\{ x^{(k)}_n \right\}_{n=1}^{\infty}  \) is a Cauchy sequence in \(X\), so for all \(k \ge 1\), there exists \(N_k > 0\) s.t. \(n \ge N_k\) implies \(d\left( x_n^{(k)}, x_{N_k}^{(k)} \right) < \frac{1}{k} \). Now we let \(z_k = x_{N_k}^{(k)}\) for all \(k \ge 1\). 
  \begin{claim} \label{clm: zk Cauchy}
    \(\left\{ z_k \right\}_{k=1}^{\infty}  \) is a Cauchy sequence in \(X\).  
  \end{claim} 
  \begin{explanation}
    For all \(\varepsilon > 0\), we know there exists \(K \ge 0\) s.t. \(\frac{1}{K} < \frac{\varepsilon}{3}\). Also, since \(\left\{ u_n \right\}_{n=1}^{\infty}  \) is Cauchy, so there exists \(N > 0\) s.t. \(i, j \ge N\) implies \(d_{\overline{X} }(u_i, u_j) < \frac{\varepsilon}{3} \), which can be writen as \(\lim_{n \to \infty} d\left( x_n^{(i)}, x_n^{(j)} \right) < \frac{\varepsilon}{3}  \). To be more precise, there exists \(N > 0\) and \(N^{\prime}  > 0\) s.t. if \(i, j \ge N\) and \(n \ge N^{\prime} \), then \(d\left( x_n^{(i)}, x_n^{(j)} \right) < \frac{\varepsilon}{3} \). Now for all \(p, q \ge \max \left\{ N, K \right\} \) and \(n \ge \max \left\{ N_p, N_q, N^{\prime}  \right\} \), we have 
    \begin{align*}
       d(z_p, z_q ) &= d\left( x_{N_p}^{(p)}, x_{N_q}^{(q)} \right) \le d\left( x_{N_p}^{(p)}, x_n^{(p)} \right) + d\left( x_n^{(p)}, x_{N_q}^{(q)} \right) \\
       &\le d\left( x_{N_p}^{(p)}, x_n^{(p)} \right) + d\left( x_n^{(p)}, x_n^{(q)} \right) + d\left( x_n^{(q)}, x_{N_q}^{(q)} \right) \\
       &< \frac{1}{p} + \varepsilon + \frac{1}{q} < \frac{1}{K} + \frac{\varepsilon}{3} + \frac{1}{K} < \frac{\varepsilon}{3} + \frac{\varepsilon}{3} + \frac{\varepsilon}{3} = \varepsilon.  
    \end{align*}    
    Hence, we know \(\left\{ z_k \right\}_{k=1}^{\infty}  \) is Cauchy.           
  \end{explanation}

  \begin{claim} \label{clm: cauchy converge to z}
    \(\lim_{n \to \infty} u_n = \mathrm{LIM}_{n \to \infty } z_n\). 
  \end{claim}
  \begin{explanation}
    Suppose \(L = \mathrm{LIM}_{n \to \infty } z_n \). For all \(\varepsilon > 0\), we want to show there exists \(N > 0\) s.t. \(m \ge N\) implies \(d_{\overline{X} }\left( u_m, L \right) < \varepsilon  \), which is equivalent to \(\lim_{n \to \infty} d\left( x_n^{(m)}, z_n \right) < \varepsilon   \). To be more precise, we want to show there exists \(N \ge 0\) and \(N^{\prime} > 0\) s.t. if \(m \ge N\) and \(n \ge N^{\prime} \), then \(d\left( x_n^{(m)}, z_n \right) < \varepsilon  \). Note that \(d\left( x_n^{(m)}, z_n \right) \le d\left( x_n^{(m)}, z_m \right) + d(z_m, z_n)  \). Suppose \(K > 0\) has \(\frac{1}{K} < \frac{\varepsilon}{2}\), we know such \(K\) exists. Also, since \(\left\{ z_n \right\}_{n=1}^{\infty}  \) is Cauchy, so we know there exists \(N_1^{\prime} > 0\) s.t. for all \(p, q \ge N_1^{\prime} \), we have \(d\left( z_p, z_q \right) < \frac{\varepsilon}{2} \). Hence, if we pick \(m \ge \max \left\{ K, N_1^{\prime}  \right\} \) and \(n \ge \max \left\{ N_m, N_1^{\prime}  \right\} \), then 
    \begin{align*}
      d\left( x_n^{(m)}, z_n \right) &\le d\left( x_n^{(m)}, z_m \right) + d(z_m, z_n) < \frac{1}{m} + \frac{\varepsilon}{2} \\
      &\le \frac{1}{K} + \frac{\varepsilon}{2} < \frac{\varepsilon}{2} + \frac{\varepsilon}{2} = \varepsilon ,
    \end{align*}
    and we're done.           
  \end{explanation}
  By \autoref{clm: zk Cauchy} and \autoref{clm: cauchy converge to z}, we know every Cauchy sequence in \(\overline{X} \) converges to a formal limit of a Cauchy sequence of \(X\), which means it converges in \(\overline{X} \), and thus \(\left( \overline{X}, d_{\overline{X} }  \right) \) is complete.
\end{proof}

\begin{proof}[d]
  We first show that \(x = y \iff \mathrm{LIM}_{n \to \infty } x = \mathrm{LIM}_{n \to \infty } y  \). If \(x = y\), then we know 
  \[
    \lim_{n \to \infty} d(x, y) = \lim_{n \to \infty} d(x, x) = 0,  
  \]  which means \(\mathrm{LIM}_{n \to  \infty } x = \mathrm{LIM}_{n \to \infty }y  \). Now we prove the converse, if \(\mathrm{LIM}_{n \to \infty } x = \mathrm{LIM}_{n \to \infty } y  \), then we know \(\lim_{n \to \infty} d(x, y) = d(x, y) = 0 \), so \(x = y\). 
  
  Now we show that \(d(x, y) - d_{\overline{X} }(x, y)\). Note that 
  \[
    d_{\overline{X} }(x, y) = \lim_{n \to \infty} d(x, y) = d(x, y), 
  \] so this is true.
\end{proof}

\begin{proof}[e]
  Since we know \(\mathrm{cl}_{\overline{X}} (X) \subseteq \overline{X} \), we only need to show \(\overline{X} \subseteq \mathrm{cl}_{\overline{X} }(X)  \). Suppose \(x \in \overline{X} \), then \(x = \mathrm{LIM}_{n \to \infty } x_n\) where \(\left\{ x_n \right\}_{n=1}^{\infty}  \) is a Cauchy sequence in \(X\). Now we want to show that \(x \in \mathrm{cl}_{\overline{X} } (X) \), which is equivalent to show for all \(\varepsilon > 0\), there exists \(y \in X\) s.t. \(y \in B_{\overline{X} }(x, \varepsilon )\). If such \(y\) exists, then \(d_{\overline{X} }(x, y)< \varepsilon \), which means \(\lim_{n \to \infty} d(x_n, y) < \varepsilon  \). However, \(\left\{ x_n \right\}_{n=1}^{\infty}  \) is a Cauchy sequence, so there exists \(N > 0\) s.t. \(i, j \ge N\) implies \(d(x_i, x_j) < \frac{\varepsilon}{2}\). Thus, we can pick \(y = x_N\), and then we have for all \(n \ge N\), \(d(x_n, y) < \frac{\varepsilon}{2} < \varepsilon \) Hence, we have \(\lim_{n \to \infty} d(x_n, y) < \varepsilon \), and we're done.              
\end{proof}
\begin{proof}[f]
  Since \(\left\{ x_n \right\}_{n=1}^{\infty}  \) can be seen as a sequence of elements in \(\overline{X} \), and notice that \(\left\{ x_n \right\}_{n=1}^{\infty}  \) is still Cauchy in \(\overline{X} \) since for all \(\varepsilon > 0\), we know there exists \(N > 0\) s.t. \(p, q \ge N\) implies \(d(x_p, x_q) < \varepsilon  \), so under same circumstances, we know 
  \[
    d_{\overline{X} }(x_p, x_q) = \lim_{n \to \infty} d(x_p, x_q) < \varepsilon , 
  \] and we're done. Now since we have proved \(\overline{X} \) is complete in (c), so we know there exists \(L \in \overline{X} \) s.t. \(\lim_{n \to \infty} x_n = L \). Also, since \(L \in \overline{X} \), so \(L = \mathrm{LIM}_{n \to \infty }a_n \) for some Cauchy sequence \(\left\{ a_n \right\}_{n=1}^{\infty}  \) in \(X\). Now we want to show \(\mathrm{LIM}_{n \to \infty } a_n = \mathrm{LIM}_{n \to \infty } x_n  \). Hence, we want to show \(\lim_{n \to \infty} d(a_n, x_n) = 0 \), which is equivalent to prove \(\forall \varepsilon > 0\), \(\exists N > 0\) s.t. \(n \ge N\) implies \(d(a_n, x_n) < \varepsilon \). 
  \begin{itemize}
    \item Notice that since \(\lim_{n \to \infty} x_n = L \in \overline{X} \), so \(\forall \varepsilon >0\), \(\exists N_1 > 0\) s.t. \(p \ge N_1\) implies \(d_{\overline{X} }(x_p, L) < \frac{\varepsilon}{2}\), and thus \(\lim_{n \to \infty} d(x_p, a_n) < \frac{\varepsilon}{2} \). Hence, there exists \(M > 0\) s.t. if \(p \ge N_1\) and \(n \ge M\), then \(d(x_p, a_n) < \frac{\varepsilon}{2}\). 
    \item Also, since \(\left\{ x_n \right\}_{n=1}^{\infty}  \) is Cauchy in \(X\), so there exists \(N_2 > 0\) s.t. \(p, q \ge N_2\) implies \(d(x_p, x_q) < \frac{\varepsilon}{2}\).             
  \end{itemize}            
  Use the above two properties, we know for all \(n \ge \max \left\{ M, N_2 \right\} \) we can choose \(s \ge \max \left\{ N_1, N_2 \right\} \) so that 
  \[
    d(a_n, x_n) \le d(a_n, x_s) + d(x_s, x_n) < \frac{\varepsilon}{2} + \frac{\varepsilon}{2} = \varepsilon,
  \] and we're done.
\end{proof}
\begin{problem}
    In the following, all the sets are subsets of a metric space $(X,d)$.

 \begin{enumerate}
  \item[(a)] If $\overline{A}\cap\overline{B}=\varnothing$, then 
  \[
  \partial(A\cup B) = \partial A \cup \partial B.
  \]

  \item[(b)] For a finite family $\{A_i\}_{i=1}^n\subseteq X$, show that
  \[
  \operatorname{int}\!\Bigl(\bigcap_{i=1}^n A_i\Bigr)
  \;=\;
  \bigcap_{i=1}^n \operatorname{int}(A_i).
  \]

  \item[(c)] For an arbitrary (possibly infinite) family $\{A_\alpha\}_{\alpha\in F}\subseteq X$, prove that
  \[
  \operatorname{int}\!\Bigl(\bigcap_{\alpha\in F} A_\alpha\Bigr)
  \;\subseteq\;
  \bigcap_{\alpha\in F}\operatorname{int}(A_\alpha).
  \]

  \item[(d)] Give an example where the inclusion in part \textup{(c)} is strict (i.e., equality fails).

  \item[(e)] For any family $\{A_\alpha\}_{\alpha\in F}\subseteq M$, prove that
  \[
  \bigcup_{\alpha\in F}\operatorname{int}(A_\alpha)
  \;\subseteq\;
  \operatorname{int}\!\Bigl(\bigcup_{\alpha\in F} A_\alpha\Bigr).
  \]

  \item[(f)] Give an example of a finite collection $F$ in which equality does not hold in part \textup{(e)}.
\end{enumerate}

\end{problem}

\begin{problem}
    Let $(X, d)$ be a metric space and $Y \subset X$ be an open subset. For any subset $A \subset Y$, show
that $A$ is open in $Y$ if and only if it is open in $X$.
\end{problem}

\begin{problem}
    On the space $(0,1]$, we may consider the topology induced by the metric space $(\mathbb{R},d)$ defined by
$d(x,y)=|x-y|$ . Alternatively, we may also define a distance $d'$ on $(0,1]$, given by
\[
d'(x,y) = \left| \frac{1}{x} - \frac{1}{y} \right|, \qquad \forall x,y \in (0,1].
\]

\begin{enumerate}
 \item[(a)] Show that $d'$ is a metric on $(0,1]$
 \item[(b)] Let $x \in (0,1]$ and $\varepsilon>0$. Let $B = B_{d}(x,\varepsilon)=\{y | |y-x| < \varepsilon \} \cap (0,1]$  be the open ball centered at $x$ of radius $\varepsilon$ for the metric $d$ in $(0,1]$.  
  Show that for any $y \in B$, we may find $\varepsilon'>0$ such that
  \[
  B_{d'}(y,\varepsilon') \subseteq B = B_{d}(x,\varepsilon).
  \]

 \item[(c)]Show that an open ball in $((0,1],d')$ is also an open ball in $((0,1],d)$.

 \item[(d)] Conclude that the metric spaces $((0,1],d)$ and $((0,1],d')$ are topologically equivalent, that is, a set $A$ is open in one space if and only if it is also open in the other one.

 \item[(e)] Is $((0,1],d')$ a complete metric space? How about $((0,1],d)$?
\end{enumerate}
\end{problem}

\begin{problem}
    \begin{enumerate}

  \item[(a)] 
  We say that a family of closed balls 
\[
\bigl(\overline{B}(x_n,r_n)\bigr)_{n\ge 1}
\]
is a \emph{decreasing sequence of closed balls} if 
the nesting condition
\[
\overline{B}(x_{n+1},r_{n+1}) \;\subseteq\; \overline{B}(x_n,r_n)
\quad\text{for all } n\in\mathbb{N}
\]
is satisfied. Give an example of a decreasing sequence of closed balls in a complete metric space with empty intersection. 

  \item[(b)]  We say that a family of closed balls 
\[
\bigl(\overline{B}(x_n,r_n)\bigr)_{n\ge 1}
\]
is a \emph{decreasing sequence of closed balls with radii tending to zero} if 
\[
r_n \;\to\; 0 \quad\text{as } n\to\infty,
\]
and the nesting condition
\[
\overline{B}(x_{n+1},r_{n+1}) \;\subseteq\; \overline{B}(x_n,r_n)
\quad\text{for all } n\in\mathbb{N}
\]
is satisfied.
  Show that a metric space $(M,d)$ is complete if and only if every decreasing sequence of closed balls with radii going to zero has a nonempty intersection. \end{enumerate}
\end{problem}
