\begin{problem}
    Let $(x^{(n)})_{n=m}^\infty$ be a sequence of points in a metric space $(X,d)$, and let $L\in X$. Show that if $L$ is a limit point of the sequence $(x^{(n)})_{n=m}^\infty$, then $L$ is an adherent point of the set
\[
S = \{ x^{(n)} : n\ge m \}.
\]
Is the converse true?
\end{problem}

\begin{proof}
    Suppose $L$ is a limit point of the sequence. By definition,
    \[
    \forall N \geq m, \forall \varepsilon > 0,\ \exists n \geq N \text{ such that } d(x^{(n)}, L) \leq \varepsilon.
    \]
    This implies that for every $\varepsilon > 0$, there exists $n \geq m$ with
    \[
    x^{(n)} \in B(L,\varepsilon) \cap S \neq \varnothing.
    \]
    Hence,
    \[
    \forall \varepsilon > 0,\ B(L,\varepsilon)\cap S \neq \varnothing 
    \quad \Rightarrow \quad 
    L \text{ is an adherent point of } S.
    \]

    Now, we check the converse. The converse statement is \textbf{NOT} true.  
    Consider $X = \mathbb{R}$ with the standard metric, and let $m=1$. Define
    \[
    x^{(n)} = \frac{1}{n}, \quad \text{so } S = \{1, \tfrac{1}{2}, \tfrac{1}{3}, \dots \}.
    \]
    It is clear that $1$ is an adherent point of $S$, since for every $\varepsilon > 0$,
    \[
    1 \in (1-\varepsilon, 1+\varepsilon), 
    \quad \Rightarrow \quad 
    B(1,\varepsilon)\cap S \neq \varnothing.
    \]
    However, $1$ is not a limit point of the sequence. Indeed, if $N \geq 2$, then for all $n \geq N$,
    \[
    d(x^{(n)},1) = \bigg| \frac{1}{n} - 1 \bigg| \geq \frac{1}{2}.
    \]
    So if we take $\varepsilon = 0.48763$ and $N = 2$, there is no $n \geq N$ such that $d(x^{(n)},1)\leq \varepsilon$.
    Therefore, $1$ is not a limit point of $(x^{(n)})_{n=1}^\infty$, 
    even though it is an adherent point of $S$.
\end{proof}

\begin{problem}
    The following construction generalizes the construction of the reals from the rationals in Chapter~5, allowing one to view any metric space as a subspace of a complete metric space. In what follows we let $(X,d)$ be a metric space.
\begin{enumerate}
  \item[(a)] Given any Cauchy sequence $(x_n)_{n=1}^\infty$ in $X$, we introduce the \emph{formal limit} 
  \[
  \operatorname{LIM}_{n\to\infty} x_n.
  \]
  We say that two formal limits $\operatorname{LIM}_{n\to\infty} x_n$ and $\operatorname{LIM}_{n\to\infty} y_n$ are equal if 
  \[
  \lim_{n\to\infty} d(x_n,y_n) = 0.
  \]
  Show that this equality relation obeys the reflexive, symmetry, and transitive axioms, i.e.\ that it is an equivalence relation.

  \item[(b)] Let $\overline{X}$ be the space of all formal limits of Cauchy sequences in $X$, modulo the above equivalence relation. Define a metric $d_{\overline{X}}:\overline{X}\times\overline{X}\to [0,\infty)$ by
  \[
  d_{\overline{X}}\!\left(\operatorname{LIM}_{n\to\infty}x_n, \operatorname{LIM}_{n\to\infty} y_n\right) := \lim_{n\to\infty} d(x_n,y_n).
  \]
  Show that this function is well-defined (the limit exists and does not depend on the choice of representatives) and that it satisfies the axioms of a metric. Thus $(\overline{X},d_{\overline{X}})$ is a metric space.

  \item[(c)] Show that the metric space $(\overline{X},d_{\overline{X}})$ is complete.

  \item[(d)] We identify an element $x\in X$ with the corresponding constant Cauchy sequence $(x,x,x,\dots)$, i.e.\ with the formal limit $\operatorname{LIM}_{n\to\infty} x$. Show that this is legitimate: for $x,y\in X$, 
  \[
  x=y \quad \Longleftrightarrow \quad \operatorname{LIM}_{n\to\infty} x = \operatorname{LIM}_{n\to\infty} y.
  \]
  With this identification, show that 
  \[
  d(x,y) = d_{\overline{X}}(x,y),
  \]
  and thus $(X,d)$ can be thought of as a subspace of $(\overline{X},d_{\overline{X}})$.

  \item[(e)] Show that the closure of $X$ in $\overline{X}$ is $\overline{X}$ itself. (This explains the choice of notation.)

  \item[(f)] Finally, show that the formal limit agrees with the actual limit: if $(x_n)_{n=1}^\infty$ is a Cauchy sequence in $X$ that converges in $X$, then
  \[
  \lim_{n\to\infty} x_n = \operatorname{LIM}_{n\to\infty} x_n \quad \text{in } \overline{X}.
  \]
\end{enumerate}
\end{problem}

\begin{proof}[a] %In 2.a, I change the transitive one so that we dont need squeeze lemma
  We verify the following properties:
    \begin{itemize}
        \item Reflexive: $\operatorname{LIM}_{n\to\infty} x_n$ and $\operatorname{LIM}_{n\to\infty} x_n$ are equal since $d$ is metric, so $\forall n, d(x_n, x_n) = 0$.
        \item Symmetry: If $\operatorname{LIM}_{n\to\infty} x_n$ and $\operatorname{LIM}_{n\to\infty} y_n$ are equal, this mean $\lim_{n\to\infty} d(x_n,y_n) = 0$. And since $d$ is metric, so $\lim_{n\to\infty} d(y_n,x_n) = 0$, hence $\operatorname{LIM}_{n\to\infty} y_n$ and $\operatorname{LIM}_{n\to\infty} x_n$ are equal.
        \item Transitive: If $\operatorname{LIM}_{n\to\infty} x_n$ and $\operatorname{LIM}_{n\to\infty} y_n$ are equal and $\operatorname{LIM}_{n\to\infty} y_n$ and $\operatorname{LIM}_{n\to\infty} z_n$ are equal, then we have \(\lim_{n \to \infty} d(x_n, y_n) = \lim_{n \to \infty}  d(y_n, z_n) = 0  \). By definition, there exists \(N_1, N_2 > 0\) s.t. for all \(n \ge N_1\), we have \(d(x_n, y_n) < \frac{\varepsilon}{2}\) and for all \(n \ge N_2\) we have \(d(y_n, z_n) < \frac{\varepsilon}{2}\). Thus, for all \(n \ge \max \left\{ N_1, N_2 \right\} \), we have
        \[
            d(x_n, z_n) \le d(x_n, y_n) + d(y_n, z_n) < \frac{\varepsilon}{2} + \frac{\varepsilon}{2} = \varepsilon ,
        \] which means \(\lim_{n \to \infty} d(x_n, z_n) = 0 \), and thus \(\mathrm{LIM}_{n \to \infty} x_n = \mathrm{LIM}_{n \to \infty } z_n  \).  
    \end{itemize}
\end{proof}

\begin{proof}[b]
  We first show that the limit exists. Note that \(\lim_{n \to \infty} d(x_n, y_n) \in \mathbb{R}_{\ge 0} \) for all Cauchy sequence \(\left\{ x_n \right\}_{n=1}^{\infty} , \left\{ y_n \right\}_{n=1}^{\infty}   \) in \(X\). We already know \((\mathbb{R}, \vert \cdot \vert ) \) is complete, so we know \((\mathbb{R} _{\ge 0}, \vert \cdot \vert )\) is also complete as it is a closed subspace of \((\mathbb{R}, \vert \cdot \vert ) \). Now we define \(u_n \coloneqq d(x_n, y_n)\) for all \(n \ge 1\), then we want to show that \(\left\{ u_n \right\}_{n=1}^{\infty}  \) is Cauchy in \(\mathbb{R}_{\ge 0}\) , and then we can conclude that \(\left\{ u_n \right\}_{n=1}^{\infty}  \) converges in \(\mathbb{R} _{\ge 0}\), and thus \(\lim_{n \to \infty} d(x_n, y_n) \) exists.
 \begin{claim} \label{clm: reverse triangle}
  Suppose \((X, d)\) is a metric space, then for all \(a, b, c, d \in X\) we have 
  \[
    \left\vert d(a, b) - d(c, d) \right\vert \le d(a, c) + d(b, d)
  \]  
 \end{claim}
 \begin{explanation}
    Since
    \[
      \begin{dcases}
        d(a, b) \le d(a, c) + d(c, b) \le d(a, c) + d(c, d) + d(d, b) \\
        d(c, d) \le d(c, a) + d(a, d) \le d(c, a) + d(a, b) + d(b, d),
      \end{dcases}
    \] so we have 
    \[
      \begin{dcases}
        d(a, b) - d(c, d) \le d(a, c) + d(d, b) \\
        -d(c, a) - d(b, d) \le d(a, b) - d(c, d),
      \end{dcases}
    \] so we can conbine these two equations and get the result.
 \end{explanation}
  By \autoref{clm: reverse triangle}, we know for all \(p, q \ge 1\), we have 
  \[
    \left\vert u_p - u_q \right\vert = \left\vert d(x_p, y_p) - d(x_q, y_q) \right\vert \le d(x_p, x_q) + d(y_p, y_q).
  \]  
  Now since \(\left\{ x_n \right\}_{n=1}^{\infty}  \) and \(\left\{ y_n \right\}_{n=1}^{\infty }  \) are Cauchy, so for every \(\varepsilon > 0\), there exists \(N_1, N_2 > 0\) s.t. 
  \[
    \begin{dcases}
      d(x_p, x_q) < \frac{\varepsilon}{2} \quad \forall p, q \ge N_1 \\
      d(y_p, y_q) < \frac{\varepsilon}{2} \quad \forall p, q \ge N_2.
    \end{dcases}
  \]   
  Thus, for all \(p, q \ge \max \left\{ N_1, N_2 \right\} \), we know
  \[
    \vert u_p - u_q \vert \le d(x_p, x_q) + d(y_p, y_q) \le \frac{\varepsilon}{2} + \frac{\varepsilon}{2} = \varepsilon.
  \]
  Hence, we know \(\left\{ u_n \right\}_{n=1}^{\infty}  \) is Cauchy in \(\mathbb{R}_{\ge 0}, \vert \cdot \vert \).  

  Now we show that \(d_{\overline{X} }\) is well-defined. In other words, if \(\mathrm{LIM}_{n \to \infty } x_n = \mathrm{LIM}_{n \to \infty } z_n \), then we want to show
  \[
    d_{\overline{X} }\left( \mathrm{LIM}_{n \to \infty }x_n, \mathrm{LIM}_{n \to \infty }y_n   \right) = d_{\overline{X} } \left( \mathrm{LIM}_{n \to \infty }z_n, \mathrm{LIM}_{n \to \infty } y_n  \right) \quad \forall \text{ Cauchy } \left\{ y_n \right\}_{n=1}^{\infty} \text{ in } (X, d).   
  \]  Equivalently, we want to show \(\lim_{n \to \infty} d(x_n, y_n) = \lim_{n \to \infty} d(z_n, y_n) \). Note that we have 
  \[
     \lim_{n \to \infty} d(x_n, z_n) = 0 \text{ and } d(x_n, y_n) \le d(x_n, z_n) + d(z_n, y_n),
  \] so we know
  \[
    \lim_{n \to \infty} d(x_n, y_n) \le \lim_{n \to \infty} d(x_n, z_n) + \lim_{n \to \infty} d(z_n, y_n) = \lim_{n \to \infty} d(z_n, y_n).  
  \]
  Also, we have \(d(z_n, y_n) \le d(z_n, x_n) + d(x_n, y_n)\), so we know 
  \[
  \lim_{n \to \infty} d(z_n, y_n) \le \lim_{n \to \infty} d(z_n, x_n) + \lim_{n \to \infty} d(x_n, y_n) = \lim_{n \to \infty} d(x_n, y_n),
  \] and thus we can conclude that \(\lim_{n \to \infty} d(x_n, y_n) = \lim_{n \to \infty} d(z_n, y_n)  \). 
  
  Finally, we want to show that \(\left( \overline{X}, d_{\overline{X} }  \right) \) is a metric space. 
  \begin{itemize}
    \item \(\forall \text{ Cauchy } \left\{x_n\right\}_{n=1}^{\infty}, \left\{y_n\right\}_{n=1}^{\infty}\), we have 
    \[
        d_{\overline{X}}(\mathrm{LIM}_{n \to \infty}, x_n, \mathrm{LIM}_{n \to \infty}, y_n) = \lim_{n \to \infty} d(x_n, y_n) \ge 0
    \] since \(d\) is a metric.
    \item \(\forall \text{ Cauchy } \left\{ x_n \right\}_{n=1}^{\infty} \in X, \ d_{\overline{X} } (\mathrm{LIM}_{n \to \infty } x_n, \mathrm{LIM}_{n\to \infty } x_n  ) = \lim_{n \to \infty} d(x_n, x_n) = 0  \). 
    \item \(\forall \text{ Cauchy } \left\{ x_n \right\}_{n=1}^{\infty} , \left\{ y_n \right\}_{n=1}^{\infty} \in X  \), 
    \begin{align*}
      d_{\overline{X} }(\mathrm{LIM}_{n \to \infty }x_n, \mathrm{LIM}_{n \to \infty }y_n  ) = \lim_{n \to \infty} d(x_n, y_n)  &= \lim_{n \to \infty} d(y_n, x_n) \\ &= d_{\overline{X}} (\mathrm{LIM}_{n \to \infty } y_n, \mathrm{LIM}_{n \to \infty } x_n  )   
    \end{align*}
    \item \(\forall \text{ Cauchy } \left\{ x_n \right\}_{n=1}^{\infty} , \left\{ y_n \right\}_{n=1}^{\infty}, \left\{ z_n \right\}_{n=1}^{\infty}   \in X  \), \begin{align*}
      d_{\overline{X} }(\mathrm{LIM}_{n \to \infty }x_n, \mathrm{LIM}_{n \to \infty }z_n  ) &= \lim_{n \to \infty} d(x_n, z_n) \\
      &\le \lim_{n \to \infty} (d(x_n, y_n) + d(y_n, z_n)) = \lim_{n \to \infty} d(x_n, y_n) + \lim_{n \to \infty} d(y_n, z_n) \\
      &= d_{\overline{X} } (\mathrm{LIM}_{n \to \infty }x_n, \mathrm{LIM}_{n \to \infty }y_n ) + d_{\overline{X} } (\mathrm{LIM}_{n \to \infty }y_n, \mathrm{LIM}_{n \to \infty }z_n ). 
    \end{align*}
  \end{itemize}
  Hence, we know \(\left( \overline{X}, d_{\overline{X} }  \right) \) is a metric space. 
\end{proof}
\begin{proof}[c]
  We want to show that for all \(\left\{ u_n \right\}_{n=1}^{\infty}  \subseteq \overline{X}  \), there exists \(\left\{ z_n \right\}_{n=1}^{\infty} \subseteq X \) s.t. \(\lim_{n \to \infty} u_n = \mathrm{LIM}_{n \to \infty } z_n \). Since \(\left\{ u_n \right\}_{n=1}^{\infty } \) is a sequence of formal limit of Cauchy sequences in \(X\), so we can define \(u_k = \mathrm{LIM}_{n \to \infty } x_n^{(k)} \) for all \(k \ge 1\). Now we construct \(\left\{ z_n \right\}_{n=1}^{\infty}  \). Since we know for all \(k \ge 1\), \(\left\{ x^{(k)}_n \right\}_{n=1}^{\infty}  \) is a Cauchy sequence in \(X\), so for all \(k \ge 1\), there exists \(N_k > 0\) s.t. \(n \ge N_k\) implies \(d\left( x_n^{(k)}, x_{N_k}^{(k)} \right) < \frac{1}{k} \). Now we let \(z_k = x_{N_k}^{(k)}\) for all \(k \ge 1\). 
  \begin{claim} \label{clm: zk Cauchy}
    \(\left\{ z_k \right\}_{k=1}^{\infty}  \) is a Cauchy sequence in \(X\).  
  \end{claim} 
  \begin{explanation}
    For all \(\varepsilon > 0\), we know there exists \(K \ge 0\) s.t. \(\frac{1}{K} < \frac{\varepsilon}{3}\). Also, since \(\left\{ u_n \right\}_{n=1}^{\infty}  \) is Cauchy, so there exists \(N > 0\) s.t. \(i, j \ge N\) implies \(d_{\overline{X} }(u_i, u_j) < \frac{\varepsilon}{3} \), which can be writen as \(\lim_{n \to \infty} d\left( x_n^{(i)}, x_n^{(j)} \right) < \frac{\varepsilon}{3}  \). To be more precise, there exists \(N > 0\) and \(N^{\prime}  > 0\) s.t. if \(i, j \ge N\) and \(n \ge N^{\prime} \), then \(d\left( x_n^{(i)}, x_n^{(j)} \right) < \frac{\varepsilon}{3} \). Now for all \(p, q \ge \max \left\{ N, K \right\} \) and \(n \ge \max \left\{ N_p, N_q, N^{\prime}  \right\} \), we have 
    \begin{align*}
       d(z_p, z_q ) &= d\left( x_{N_p}^{(p)}, x_{N_q}^{(q)} \right) \le d\left( x_{N_p}^{(p)}, x_n^{(p)} \right) + d\left( x_n^{(p)}, x_{N_q}^{(q)} \right) \\
       &\le d\left( x_{N_p}^{(p)}, x_n^{(p)} \right) + d\left( x_n^{(p)}, x_n^{(q)} \right) + d\left( x_n^{(q)}, x_{N_q}^{(q)} \right) \\
       &< \frac{1}{p} + \frac{\varepsilon}{3} + \frac{1}{q} < \frac{1}{K} + \frac{\varepsilon}{3} + \frac{1}{K} < \frac{\varepsilon}{3} + \frac{\varepsilon}{3} + \frac{\varepsilon}{3} = \varepsilon.  
    \end{align*}    
    Hence, we know \(\left\{ z_k \right\}_{k=1}^{\infty}  \) is Cauchy.           
  \end{explanation}

  \begin{claim} \label{clm: cauchy converge to z}
    \(\lim_{n \to \infty} u_n = \mathrm{LIM}_{n \to \infty } z_n\). 
  \end{claim}
  \begin{explanation}
    Suppose \(L = \mathrm{LIM}_{n \to \infty } z_n \). For all \(\varepsilon > 0\), we want to show there exists \(N > 0\) s.t. \(m \ge N\) implies \(d_{\overline{X} }\left( u_m, L \right) < \varepsilon  \), which is equivalent to \(\lim_{n \to \infty} d\left( x_n^{(m)}, z_n \right) < \varepsilon   \). To be more precise, we want to show there exists \(N \ge 0\) and \(N^{\prime} > 0\) s.t. if \(m \ge N\) and \(n \ge N^{\prime} \), then \(d\left( x_n^{(m)}, z_n \right) < \varepsilon  \). Note that \(d\left( x_n^{(m)}, z_n \right) \le d\left( x_n^{(m)}, z_m \right) + d(z_m, z_n)  \). Suppose \(K > 0\) has \(\frac{1}{K} < \frac{\varepsilon}{2}\), we know such \(K\) exists. Also, since \(\left\{ z_n \right\}_{n=1}^{\infty}  \) is Cauchy, so we know there exists \(N_1^{\prime} > 0\) s.t. for all \(p, q \ge N_1^{\prime} \), we have \(d\left( z_p, z_q \right) < \frac{\varepsilon}{2} \). Hence, if we pick \(m \ge \max \left\{ K, N_1^{\prime}  \right\} \) and \(n \ge \max \left\{ N_m, N_1^{\prime}  \right\} \), then 
    \begin{align*}
      d\left( x_n^{(m)}, z_n \right) &\le d\left( x_n^{(m)}, z_m \right) + d(z_m, z_n) < \frac{1}{m} + \frac{\varepsilon}{2} \\
      &\le \frac{1}{K} + \frac{\varepsilon}{2} < \frac{\varepsilon}{2} + \frac{\varepsilon}{2} = \varepsilon ,
    \end{align*}
    and we're done.           
  \end{explanation}
  By \autoref{clm: zk Cauchy} and \autoref{clm: cauchy converge to z}, we know every Cauchy sequence in \(\overline{X} \) converges to a formal limit of a Cauchy sequence of \(X\), which means it converges in \(\overline{X} \), and thus \(\left( \overline{X}, d_{\overline{X} }  \right) \) is complete.
\end{proof}

\begin{proof}[d]
  We first show that \(x = y \iff \mathrm{LIM}_{n \to \infty } x = \mathrm{LIM}_{n \to \infty } y  \). If \(x = y\), then we know 
  \[
    \lim_{n \to \infty} d(x, y) = \lim_{n \to \infty} d(x, x) = 0,  
  \]  which means \(\mathrm{LIM}_{n \to  \infty } x = \mathrm{LIM}_{n \to \infty }y  \). Now we prove the converse, if \(\mathrm{LIM}_{n \to \infty } x = \mathrm{LIM}_{n \to \infty } y  \), then we know \(\lim_{n \to \infty} d(x, y) = d(x, y) = 0 \), so \(x = y\). 
  
  Now we show that \(d(x, y) = d_{\overline{X} }(x, y)\). Note that 
  \[
    d_{\overline{X} }(x, y) = \lim_{n \to \infty} d(x, y) = d(x, y), 
  \] so this is true.
\end{proof}

\begin{proof}[e]
  Since we know \(\mathrm{cl}_{\overline{X}} (X) \subseteq \overline{X} \), we only need to show \(\overline{X} \subseteq \mathrm{cl}_{\overline{X} }(X)  \). Suppose \(x \in \overline{X} \), then \(x = \mathrm{LIM}_{n \to \infty } x_n\) where \(\left\{ x_n \right\}_{n=1}^{\infty}  \) is a Cauchy sequence in \(X\). Now we want to show that \(x \in \mathrm{cl}_{\overline{X} } (X) \), which is equivalent to show for all \(\varepsilon > 0\), there exists \(y \in X\) s.t. \(y \in B_{\overline{X} }(x, \varepsilon )\). If such \(y\) exists, then \(d_{\overline{X} }(x, y)< \varepsilon \), which means \(\lim_{n \to \infty} d(x_n, y) < \varepsilon  \). However, \(\left\{ x_n \right\}_{n=1}^{\infty}  \) is a Cauchy sequence, so there exists \(N > 0\) s.t. \(i, j \ge N\) implies \(d(x_i, x_j) < \frac{\varepsilon}{2}\). Thus, we can pick \(y = x_N\), and then we have for all \(n \ge N\), \(d(x_n, y) < \frac{\varepsilon}{2} < \varepsilon \) Hence, we have \(\lim_{n \to \infty} d(x_n, y) < \varepsilon \), and we're done.              
\end{proof}

\begin{proof}[f]
  Since \(\left\{ x_n \right\}_{n=1}^{\infty}  \) can be seen as a sequence of elements in \(\overline{X} \), and notice that \(\left\{ x_n \right\}_{n=1}^{\infty}  \) is still Cauchy in \(\overline{X} \) since for all \(\varepsilon > 0\), we know there exists \(N > 0\) s.t. \(p, q \ge N\) implies \(d(x_p, x_q) < \varepsilon  \), so under same circumstances, we know 
  \[
    d_{\overline{X} }(x_p, x_q) = \lim_{n \to \infty} d(x_p, x_q) < \varepsilon , 
  \] and we're done. Now since we have proved \(\overline{X} \) is complete in (c), so we know there exists \(L \in \overline{X} \) s.t. \(\lim_{n \to \infty} x_n = L \). Also, since \(L \in \overline{X} \), so \(L = \mathrm{LIM}_{n \to \infty }a_n \) for some Cauchy sequence \(\left\{ a_n \right\}_{n=1}^{\infty}  \) in \(X\). Now we want to show \(\mathrm{LIM}_{n \to \infty } a_n = \mathrm{LIM}_{n \to \infty } x_n  \). Hence, we want to show \(\lim_{n \to \infty} d(a_n, x_n) = 0 \), which is equivalent to prove \(\forall \varepsilon > 0\), \(\exists N > 0\) s.t. \(n \ge N\) implies \(d(a_n, x_n) < \varepsilon \). 
  \begin{itemize}
    \item Notice that since \(\lim_{n \to \infty} x_n = L \in \overline{X} \), so \(\forall \varepsilon >0\), \(\exists N_1 > 0\) s.t. \(p \ge N_1\) implies \(d_{\overline{X} }(x_p, L) < \frac{\varepsilon}{2}\), and thus \(\lim_{n \to \infty} d(x_p, a_n) < \frac{\varepsilon}{2} \). Hence, there exists \(M > 0\) s.t. if \(p \ge N_1\) and \(n \ge M\), then \(d(x_p, a_n) < \frac{\varepsilon}{2}\). 
    \item Also, since \(\left\{ x_n \right\}_{n=1}^{\infty}  \) is Cauchy in \(X\), so there exists \(N_2 > 0\) s.t. \(p, q \ge N_2\) implies \(d(x_p, x_q) < \frac{\varepsilon}{2}\).             
  \end{itemize}            
  Use the above two properties, we know for all \(n \ge \max \left\{ M, N_2 \right\} \) we can choose \(s \ge \max \left\{ N_1, N_2 \right\} \) so that 
  \[
    d(a_n, x_n) \le d(a_n, x_s) + d(x_s, x_n) < \frac{\varepsilon}{2} + \frac{\varepsilon}{2} = \varepsilon,
  \] and we're done.
\end{proof}

\begin{problem}
    In the following, all the sets are subsets of a metric space $(X,d)$.

 \begin{enumerate}
  \item[(a)] If $\overline{A}\cap\overline{B}=\varnothing$, then 
  \[
  \partial(A\cup B) = \partial A \cup \partial B.
  \]

  \item[(b)] For a finite family $\{A_i\}_{i=1}^n\subseteq X$, show that
  \[
  \operatorname{int}\!\Bigl(\bigcap_{i=1}^n A_i\Bigr)
  \;=\;
  \bigcap_{i=1}^n \operatorname{int}(A_i).
  \]

  \item[(c)] For an arbitrary (possibly infinite) family $\{A_\alpha\}_{\alpha\in F}\subseteq X$, prove that
  \[
  \operatorname{int}\!\Bigl(\bigcap_{\alpha\in F} A_\alpha\Bigr)
  \;\subseteq\;
  \bigcap_{\alpha\in F}\operatorname{int}(A_\alpha).
  \]

  \item[(d)] Give an example where the inclusion in part \textup{(c)} is strict (i.e., equality fails).

  \item[(e)] For any family $\{A_\alpha\}_{\alpha\in F}\subseteq M$, prove that
  \[
  \bigcup_{\alpha\in F}\operatorname{int}(A_\alpha)
  \;\subseteq\;
  \operatorname{int}\!\Bigl(\bigcup_{\alpha\in F} A_\alpha\Bigr).
  \]

  \item[(f)] Give an example of a finite collection $F$ in which equality does not hold in part \textup{(e)}.
\end{enumerate}

\end{problem}
\begin{proof}[a]
  If \(x \in \partial (A \cup B)\), then for all \(r > 0\), we have 
  \[
    \begin{dcases}
      B_X(x, r) \cap (A \cup B) = \left( B_X(x, r) \cap A \right) \cup \left( B_X(x, r) \cap B \right) \neq \varnothing. \\
      B_X(x, r) \cap \left( X \setminus (A \cup B) \right) = B_X(x, r) \cap \left( X \setminus A \right) \cap \left( X \setminus B \right) \neq \varnothing.
    \end{dcases}
  \] 
  Hence, either \(B_X(x, r) \cap A\) or \(B_X(x, r) \cap B\) is not empty. Also, we have \(B_X(x, r) \cap \left( X \setminus A \right) \neq \varnothing  \) and \(B_X(x, r) \cap \left( X \setminus B \right) \neq \varnothing  \). Thus, \(x \in \partial A \cup \partial B\), which means \(\partial (A \cup B) \subseteq \partial A \cup \partial B\).    
  
  Now we show that \(\partial A \cup \partial B \subseteq \partial (A \cup B)\). If \(x \in \partial A \cup \partial B\), then we first give a claim: 
  
  \begin{claim}
    If \(x \in \partial A\), then \(x \notin \partial B\), and vice versa.   
  \end{claim}
  \begin{explanation}
    If \(x \in \partial A \cap \partial B\), then since \(x \in \partial A \subseteq \overline{A} \) and \(x \in \partial B \subseteq \overline{B} \), so \(x \in \overline{A} \cap \overline{B} = \varnothing  \), which is a contradiction.     
  \end{explanation}
  Without lose of generality, we can suppose \(x \in \partial A\) and \(x \notin \partial B\), then we know 
  \begin{align*}
    &\forall r > 0 \text{ we have }
    \begin{dcases}
      B_X(x, r) \cap A \neq \varnothing \\
      B_X(x, r) \cap \left( X \setminus A \right) \neq \varnothing
    \end{dcases}, \\
    &\exists r^{\prime} > 0 \text{ s.t. exactly one of } 
    \begin{dcases}
      B_X \left( x, r^{\prime}  \right) \subseteq B \\
      B_X \left( x, r^{\prime}  \right) \subseteq (X\setminus B)
    \end{dcases} \text{ occurs.}
  \end{align*}
  However, if \(B_X \left( x, r^{\prime}  \right) \subseteq B  \), then \(x \in B_X \left( x, r^{\prime}  \right) \subseteq B \subseteq \overline{B}  \). However, \(x \in \partial A \subseteq \overline{A} \), so \(x \in \overline{A} \cap \overline{B} = \varnothing   \), which is a contradiction. Thus, we know \(B_X \left( x, r^{\prime}  \right) \subseteq B \). Now since \(x \in \partial A\), so \(\forall r > 0\), we have \(\varnothing \neq B_X (x, r) \cap A \subseteq B_X(x, r) \cap (A \cup B)\). Now we want to show \(B_X(x, r) \cap (X \setminus A) \cap (X \setminus B) \neq \varnothing \).  
  \begin{itemize}
    \item Case 1: \(r \ge r^{\prime} \), then we have \(B_X(x, r) \subseteq B_X \left( x, r^{\prime}  \right) \subseteq X \setminus B\) and thus 
    \[
      B_X (x, r) \cap \left( X \setminus A \right) \subseteq X \setminus B \implies B_X(x, r) \cap \left( X \setminus A \right) \cap \left( X \setminus B \right) \neq \varnothing    
    \] since \(B_X(x, r) \cap \left( X \setminus A \right) \neq \varnothing  \). 
    \item Case 2: \(r^{\prime} < r\), then we know \(B_X \left( x, r^{\prime}  \right) \subseteq (X \setminus B) \) and \(B_X \left( x, r^{\prime}  \right) \subseteq B_X(x, r) \). Now if we can show \(B_X \left( x, r^{\prime}  \right) \cap (X \setminus A) \cap (X \setminus B) \neq \varnothing \), then since \(B_X \left( x, r^{\prime}  \right) \subseteq B_X(x, r)\), so we know 
    \[
      \varnothing \neq  B_X \left( x, r^{\prime}  \right) \cap (X \setminus A) \cap (X \setminus B) \subseteq B_X \left( x, r \right) \cap (X \setminus A) \cap (X \setminus B).
    \]  Now we show that \(B_X \left( x, r^{\prime}  \right) \cap (X \setminus A) \cap (X \setminus B) \neq \varnothing \). Note that since \(B_X \left( x, r^{\prime}  \right) \subseteq (X \setminus B) \), so in fact 
    \[
      B_X \left( x, r^{\prime}  \right) \cap \left( X \setminus A \right) \cap \left( X \setminus B \right) = B_X \left( x, r^{\prime}  \right)  \cap \left( X \setminus A \right)  \neq \varnothing 
    \] since \(x \in \partial A\), and thus we're done.  
  \end{itemize}         
\end{proof}
\begin{proof}[b]
  If \(x \in \mathrm{Int} \left( \bigcap_{i=1}^{n} A_i \right)  \), then \(\exists r_1 > 0\) s.t. \(B_X (x, r_1) \subseteq \bigcap_{i=1}^{n} A_i \). Hence, \(B_X(x, r_1) \subseteq A_i\) for all \(1 \le i \le n\), which means \(x \in \mathrm{Int}(A_i) \) for al \(1 \le i \le n\), and thus \(x \in \bigcap_{i=1}^{n} \mathrm{Int}(A_i)  \). This shows \(\mathrm{Int}\left( \bigcap_{i=1}^{n} A_i  \right) \subseteq \bigcap_{i=1}^{n} \mathrm{Int}(A_i) \). This shows \(\mathrm{Int}\left( \bigcap_{i=1}^{n} A_i  \right) \subseteq \bigcap_{i=1}^n \mathrm{Int}(A_i)    \). Now we show that \(\bigcap_{i=1}^{n} \mathrm{Int}(A_i) \subseteq \mathrm{Int}\left( \bigcap_{i=1}^{n} A_i  \right)    \). Suppose \(x \in \bigcap_{i=1}^{n} \mathrm{Int}(A_i)  \), for each \(i\) s.t. \(1 \le i \le n\), we know there exists \(r_i > 0\) s.t. \(B_X(x, r_i) \subseteq A_i\), so if we pick \(r^{\prime} = \min \left\{ r_1, r_2, \dots , r_n \right\} \), then \(B_X(x, r^{\prime} )\subseteq \bigcap_{i=1}^n A_i \), and thus \(x \in \mathrm{Int}\left( \bigcap_{i=1}^{n} A_i  \right) \).                 
\end{proof}
\begin{proof}[c]
  If \(x \in \mathrm{Int} \left( \bigcap_{\alpha \in F} A_\alpha  \right)  \), then \(\exists r_1 > 0\) s.t. \(B_X (x, r_1) \subseteq \bigcap_{\alpha \in F} A_\alpha  \). Hence, \(B_X(x, r_1) \subseteq A_\alpha \) for all \(\alpha \in F\), which means \(x \in \mathrm{Int}(A_\alpha ) \) for all \(\alpha \in F\), and thus \(x \in \bigcap_{\alpha \in F} \mathrm{Int}(A_\alpha )  \). This shows \(\mathrm{Int}\left( \bigcap_{\alpha \in F} A_\alpha   \right) \subseteq \bigcap_{\alpha \in F} \mathrm{Int}(A_\alpha ) \).
\end{proof}
\begin{proof}[d]
  Suppose \(\left\{ A_\alpha  \right\}_{\alpha \in F} = \left\{ \left( 1 - \frac{1}{n}, 2 + \frac{1}{n} \right)  \right\}_{n \in \mathbb{N} }  \), then \(\bigcap_{\alpha \in F} A_\alpha = [1, 2] \), and \(\mathrm{Int}\left( [1, 2] \right) = (1, 2)  \). Besides, \(\mathrm{Int} \left( 1 - \frac{1}{n}, 2 + \frac{1}{n} \right) = \left( 1 - \frac{1}{n}, 2 + \frac{1}{n} \right)   \), and \(\bigcap_{n \in \mathbb{N} } \left( 1 - \frac{1}{n}, 2 + \frac{1}{n} \right) = [1, 2]  \). Hence, in this case, the equality fails.      
\end{proof}
\begin{proof}[e]
  If \(x \in \bigcup_{\alpha \in F} \mathrm{Int}(A_\alpha )  \), then \(x \in \mathrm{Int}(A_i) \) for some \(i \in F\), and thus there exists \(r_i > 0\) s.t. \(B(x, r_i) \subseteq A_i\). Hence, \(B(x, r_i) \subseteq \bigcup_{\alpha \in F} A_i \), and thus \(x \in \mathrm{Int}\left( \bigcup_{\alpha \in F}A_i  \right)  \).       
\end{proof}
\begin{proof}[f]
  Suppose the family is \(\left\{ [1, 2], [2, 3] \right\} \), then 
  \[
    \mathrm{Int}[1, 2] \cup \mathrm{Int}[2, 3] = (1, 2) \cup (2, 3).  
  \]
  Also, \([1, 2] \cup [2, 3] = [1, 3]\), so \(\mathrm{Int}\left( [1,2] \cup [2, 3] \right) = \mathrm{Int}[1, 3] = (1, 3)   \). This is the case the equality fails. 
\end{proof}

\begin{problem}
    Let $(X, d)$ be a metric space and $Y \subset X$ be an open subset. For any subset $A \subset Y$, show
that $A$ is open in $Y$ if and only if it is open in $X$.
\end{problem}
\begin{proof}
  \vphantom{text}
  \begin{itemize}
    \item [\((\implies )\)] Since \(A\) is open in \(Y\), so there exists open \(O \subseteq X\) s.t. \(A = O \cap Y\). Since \(O\) and \(Y\) are both open sets in \(X\), so there exists \(r_1, r_2 > 0\) s.t. 
    \[
      B_X (x, r_1) \subseteq O \quad \text{and} \quad B_X(x, r_2) \subseteq Y.
    \]
    Now let \(r_3 = \min \left\{ r_1, r_2  \right\} \), then \(B_X(x, r_3) \subseteq O \cap Y = A\), which shows \(A\) is open in \(X\). 
    \item [\((\impliedby )\)]  Now if \(A\) is open in \(X\), then for all \(x \in X\), there exists \(B_X(x, r) \subseteq A\), but \(B_Y (x, r) \subseteq B_X(x, r)\), so we have \(B_Y(x, r) \subseteq A\), and thus \(A\) is open in \(Y\).                     
  \end{itemize}
\end{proof}

\begin{problem}
    On the space $(0,1]$, we may consider the topology induced by the metric space $(\mathbb{R},d)$ defined by
$d(x,y)=|x-y|$ . Alternatively, we may also define a distance $d'$ on $(0,1]$, given by
\[
d'(x,y) = \left| \frac{1}{x} - \frac{1}{y} \right|, \qquad \forall x,y \in (0,1].
\]

\begin{enumerate}
 \item[(a)] Show that $d'$ is a metric on $(0,1]$
 \item[(b)] Let $x \in (0,1]$ and $\varepsilon>0$. Let $B = B_{d}(x,\varepsilon)=\{y | |y-x| < \varepsilon \} \cap (0,1]$  be the open ball centered at $x$ of radius $\varepsilon$ for the metric $d$ in $(0,1]$.  
  Show that for any $y \in B$, we may find $\varepsilon'>0$ such that
  \[
  B_{d'}(y,\varepsilon') \subseteq B = B_{d}(x,\varepsilon).
  \]

 \item[(c)]Show that an open ball in $((0,1],d')$ is also an open ball in $((0,1],d)$.

 \item[(d)] Conclude that the metric spaces $((0,1],d)$ and $((0,1],d')$ are topologically equivalent, that is, a set $A$ is open in one space if and only if it is also open in the other one.

 \item[(e)] Is $((0,1],d')$ a complete metric space? How about $((0,1],d)$?
\end{enumerate}
\end{problem}

\begin{proof}[(a)]
    We verify the following properties:
    \begin{itemize}
        \item $d'(x,y) \geq 0$ since $\lvert \cdot \rvert \geq 0$.
        \item $d'(x,y) = 0 \iff \lvert\frac{1}{x} - \frac{1}{y}\rvert \iff x=y$.
        \item $d'(x,y) = \lvert\frac{1}{x} - \frac{1}{y}\rvert = \lvert\frac{1}{y} - \frac{1}{x}\rvert = d'(y,x)$.
        \item We know if $a, b \in \mathbb{R}$, the triangular inequality $\lvert a \rvert + \lvert b \rvert \geq \lvert a+b \rvert$ holds.
        Then we can plug $a = \frac{1}{x} - \frac{1}{y}$ and $b = \frac{1}{y} - \frac{1}{z}$ in. Then we can get $\lvert \frac{1}{x} - \frac{1}{y} + \frac{1}{y} - \frac{1}{z} \rvert \leq \lvert \frac{1}{x} - \frac{1}{y} \rvert + \lvert \frac{1}{y} - \frac{1}{z} \rvert$. Hence $d'(x,z) \leq d'(x,y) + d'(y,z)$.
    \end{itemize}
    Hence, $d'$ is indeed a metric on $(0.1]$.
\end{proof}

\begin{proof}[(b)]
    We assume that $d(x,y) = \varepsilon'' < \varepsilon$.
    \begin{claim}
        We claim that if we pick $\varepsilon' = \varepsilon - \varepsilon'' > 0$, then $B_{d'}(y,\varepsilon') \subseteq B_d(x,\varepsilon)$
    \end{claim}
    \begin{explanation}
        $\forall z \in B_{d'}(y, \varepsilon')$, $\lvert \frac{1}{y} - \frac{1}{z} \rvert < \varepsilon'$. Then $\lvert \frac{z-y}{zy} \rvert < \varepsilon'$, $\lvert z-y \rvert < \varepsilon'zy$ (Since $z > 0$ and $y > 0$). \\
        Hence we know that $d(y,z) = \lvert y-z \rvert < \varepsilon'zy < \varepsilon'$. \\
        Since $d$ is metric, so $d(x,z) \leq d(x,y) + d(y,z) < \varepsilon'' + (\varepsilon - \varepsilon'') = \varepsilon$. \\
        Hence $z \in B_d(x, \varepsilon)$.
    \end{explanation}
    By the claim above, we're done.
\end{proof}

\begin{proof}[(c)]
    Suppose $E = B_{d'}(x, r)$, we first analyze the properties of the elements in $E$ first, $y \in E$ if $\lvert \frac{1}{y} - \frac{1}{x} \rvert < r$, we can solve $\frac{1}{x} - r < \frac{1}{y} < \frac{1}{x} + r$:
    \begin{itemize}
        \item Condition 1: $\frac{1}{y} < \frac{1}{x} +r$, so $y > \frac{x}{1+rx} > 0$
        \item Condition 2 (Case 1): If $\frac{1}{y} > \frac{1}{x} - r$ and $\frac{1}{x} - r \leq 0$, then any $y \in (0,1]$ satisfy the condition.
        \item Condition 2 (Case 2): If $\frac{1}{y} > \frac{1}{x} - r$, $\frac{1}{x} - r > 0$ and $x \geq 1-rx$, then $y \leq 1 < \frac{x}{1-rx}$
        \item Condition 2 (Case 3): If $\frac{1}{y} > \frac{1}{x} - r$, $\frac{1}{x} - r > 0$ and $x < 1-rx$, then $y < \frac{x}{1-rx} < 1$
    \end{itemize}
    Then we can construct open ball which is also equal to $E$ in another metric space by selecting center point and radius:\\
    For case 1 and 2, we may choose $c = 1 \in (0,1]$ and $r' = 1 - \frac{x}{1+rx}$, since
    \[
    B_{((0,1],d)}(c,r') = B_{(\mathbb{R},d)}(c,r') \cap (0,1] = \{ z \in \mathbb{R} \mid 1-r' < z < 1+r' \} \cap (0,1] = \{ z \in (0,1] \mid 1-r' < z\}
    \]
    \[
    B_{((0,1],d)}(c,r) = \{ z \in (0,1] \mid \frac{x}{1+rx} < z\}
    \]
    This is indeed same as the condition requirement.\\
    For case 3, let $a = \frac{x}{1+rx}$, $b = \frac{x}{1-rx}$, we choose $c = \frac{a+b}{2}$ and $r' = \frac{b-a}{2}$, since
    \[
    B_{((0,1],d)}(c,r') = B_{(\mathbb{R},d)}(c,r') \cap (0,1] = \{ z \in \mathbb{R} \mid c-r' < z < c+r' \} \cap (0,1] = \{ z \in (0,1] \mid a < z < b\}
    \]
    \[
    B_{((0,1],d)}(c,r) = \{ z \in (0,1] \mid \frac{x}{1+rx} < z < \frac{x}{1-rx}\}
    \]
    This is also same as the condition requirement.
    Hence we proved.
    
\end{proof}

% \begin{proof}[(c)(Discard fuck!!!)]
%     To prove this, we need to show given $x \in (0,1], \varepsilon >0, B = B_{d'}(x, \varepsilon)$, for every $y \in B$, we can find $\varepsilon' > 0$ such that $B_d(y, \varepsilon') \subseteq B = B_{d'}(x, \varepsilon)$,
%     We may assume that $d'(x,y) = \varepsilon'' < \varepsilon$. \\
%     First, we show that the value of $y$ has a lower bound.
%     Since $\lvert \frac{1}{y} - \frac{1}{x} \rvert < \varepsilon''$, $\frac{1}{y} < \varepsilon'' + \frac{1}{x}$, so we know that $y > \frac{x}{1+\varepsilon''x}$, to make the following proof easier, we may assume that $m = \frac{x}{1+\varepsilon''x} > 0$. 
%     \begin{claim}
%         We claim that if we pick $\varepsilon' = \frac{m^2(\varepsilon - \varepsilon'')}{m(\varepsilon - \varepsilon'') + 1} > 0$, then then $B_{d}(y,\varepsilon') \subseteq B_{d'}(x,\varepsilon)$
%     \end{claim}
%     \begin{explanation}
%         First, notice that $y > m = \frac{m^2(\varepsilon - \varepsilon'') + m}{m(\varepsilon - \varepsilon'') + 1} > \varepsilon' = \frac{m^2(\varepsilon - \varepsilon'')}{m(\varepsilon - \varepsilon'') + 1}$, so $y - \varepsilon' > 0$. \\
%         $\forall z \in B_{d}(y, \varepsilon')$, $\lvert y - z \rvert < \varepsilon'$, so $z > y - \varepsilon' > 0$ Then $\lvert \frac{1}{y} - \frac{1}{z} \rvert =\lvert \frac{y-z}{yz} \rvert < \frac{\varepsilon'}{yz} < \frac{\varepsilon'}{m(m-\varepsilon')}$. \\
%         Hence we know that $d'(y,z) < \frac{\varepsilon'}{m(m-\varepsilon')}$. \\
%         Since $d'$ is metric, so 
%         \begin{align*}
%             d'(x,z) &\leq d'(x,y) + d'(y,z) < \frac{\varepsilon'}{m(m-\varepsilon')} + \varepsilon'' \\ &= \frac{\frac{m^2(\varepsilon - \varepsilon'')}{m(\varepsilon - \varepsilon'') + 1}}{m(m-\frac{m^2(\varepsilon - \varepsilon'')}{m(\varepsilon - \varepsilon'') + 1})} + \varepsilon'' = \frac{m(\varepsilon - \varepsilon'')}{(m(m(\varepsilon-\varepsilon'')+1)-m^2(\varepsilon - \varepsilon'') )} + \varepsilon'' \\ &= \frac{m(\varepsilon - \varepsilon'')}{m} + \varepsilon'' = \varepsilon,
%         \end{align*}
%          and thus $z \in B_{d'}(x, \varepsilon)$.
%     \end{explanation}
%     And the claim above is sufficient to show that for every open ball in $E$ in $((0,1],d')$, there exist some $r > 0$ such that $d'(x,r) \subseteq E$, so $E$ is also the open set in $((0,1],d)$, hence we proved.
% \end{proof}

\begin{proof}[(d)]
    % Trivial since from (b) it is equivalent that given any open set $E$ in $((0,1],d)$, $E$ is also an open set in $((0,1],d')$, and from (c) we show that given any open set $E$ in $((0,1],d')$, $E$ is also an open set in $((0,1],d)$. \\
    % So given any $E$, $E$ is open in $((0,1],d')$ if and only if $E$ is open in $((0,1],d)$, hence we prove that $((0,1],d')$ and $((0,1],d)$ are topologically equivalent.
    \vphantom{text}
    \begin{itemize}
        \item [\((\implies )\)] Suppose A is open in $((0,1],d)$, then $\forall x\in A, \exists r_x > 0$ such that $B_d(x, r_x) \subseteq A$. So $A = \bigcup_{x\in A} B_d(x, r_x)$, and from (b), we know any open ball in $((0,1],d)$ is also open in $((0,1],d')$, so $A$ is also the union of infinitely many open set in $((0,1],d')$. By the proposition we have shown in class, we conclude $A$ is also a open set in $((0,1],d')$.
        \item [\((\impliedby )\)] Suppose A is open in $((0,1],d')$, then $\forall x\in A, \exists r_x > 0$ such that $B_{d'}(x, r_x) \subseteq A$. So $A = \bigcup_{x\in A} B_{d'}(x, r_x)$. and from (c), we know for any open ball in $((0,1],d')$ is also open in $((0,1],d)$, so $A$ is also the union of infinitely many open set in $((0,1],d)$. By the proposition we have shown in class, we conclude $A$ is also a open set in $((0,1],d)$.
    \end{itemize}
    Hence, we proved that $A$ is open set in $((0,1],d')$ if and only if it is open in $((0,1],d)$.
\end{proof} 

\begin{proof}[(e)]
    We first show that $((0,1],d')$ is complete metric space. \\
    Given Cauchy sequence $(x_n)_{n=m}^\infty$ in $((0,1],d')$, $\forall \varepsilon > 0, \exists N$ such that $\forall n,m \geq N$, $d'(x_n,x_m) = \lvert \ \frac{1}{x_n} - \frac{1}{x_m} \rvert < \varepsilon$. \\
    Then we can construct another sequence $(y_n)_{n=m}^\infty$ such that $y_n = \frac{1}{x_n}$ and since $x_n \in (0, 1]$, $y_n \in [1,\inf)$. \\
    $(y_n)$ is Cauchy sequence since $\forall \varepsilon > 0, \exists N' = N$ such that $\forall n,m \geq N'$, $d(y_n, y_m) = \lvert y_n - y_m\rvert = \lvert \ \frac{1}{x_n} - \frac{1}{x_m} \rvert < \varepsilon$ by previous proof. \\
    Notice that $[1, \infty)$ is a closed subset in $\mathbb{R}$ and we know $(\mathbb{R}, d)$ is complete, so $([1, \infty ),d)$ is also complete. Hence $(y_n)$ converge to some $L \in [1,\infty)$, and this will imply that $(x_n)$ converge to $\frac{1}{L} \in (0,1]$, so given any Cauchy sequence in $((0,1],d')$, it converges to some $L' \in (0,1]$, so $((0,1],d')$ is complete.

    We show that $((0,1],d)$ is \textbf{NOT} complete by showing an example.
    Consider $(z_n)_{n=m}^\infty$, $z_n = \frac{1}{n}$. It is Cauchy sequence since $\forall \varepsilon = \frac{1}{k}> 0, \exists N = \lceil k \rceil$ such that $\forall n, m \geq N$, WLOG suppose $n \geq m$,
    \[
      d(x_n, x_m) = x_n - x_m < x_n \leq \frac{1}{\lceil k \rceil} \leq  \frac{1}{k} = \varepsilon.
    \]
    However, it converges to $0$, which is not in $(0,1]$, so the series $(z_n)$ doesn't converge in $((0,1],d)$, hence $((0,1],d)$ is not complete.
    
\end{proof}

\begin{problem}
    \begin{enumerate}

  \item[(a)] 
  We say that a family of closed balls 
\[
\bigl(\overline{B}(x_n,r_n)\bigr)_{n\ge 1}
\]
is a \emph{decreasing sequence of closed balls} if 
the nesting condition
\[
\overline{B}(x_{n+1},r_{n+1}) \;\subseteq\; \overline{B}(x_n,r_n)
\quad\text{for all } n\in\mathbb{N}
\]
is satisfied. Give an example of a decreasing sequence of closed balls in a complete metric space with empty intersection. 

  \item[(b)]  We say that a family of closed balls 
\[
\bigl(\overline{B}(x_n,r_n)\bigr)_{n\ge 1}
\]
is a \emph{decreasing sequence of closed balls with radii tending to zero} if 
\[
r_n \;\to\; 0 \quad\text{as } n\to\infty,
\]
and the nesting condition
\[
\overline{B}(x_{n+1},r_{n+1}) \;\subseteq\; \overline{B}(x_n,r_n)
\quad\text{for all } n\in\mathbb{N}
\]
is satisfied.
  Show that a metric space $(M,d)$ is complete if and only if every decreasing sequence of closed balls with radii going to zero has a nonempty intersection. \end{enumerate}
\end{problem}

\begin{proof}[(a)]
    Consider the following example:
    $X = \mathbb{N}$, and 
    \[
    d(x,y)=
    \begin{cases}
    0, & \text{if } x=y,\\[4pt]
    1+\dfrac{1}{\min\{x,y\}}, & \text{if } x\ne y .
    \end{cases}
    \]
    and $C_n = \overline{B}_d\!\left(n,\,1+\frac{1}{n}\right) = \{\, x \in X \mid d(x,n) \le 1+\tfrac{1}{n} \,\}$.
    \begin{claim}
        $\forall n \in \mathbb{N}, C_n = \{n, n+1, n+2...\}$
    \end{claim}
    \begin{explanation}
        $\forall m < n$, $d(n,m) = 1 + \frac{1}{m} > 1 + \frac{1}{n}$, so $m \notin C_n$.
        $\forall m \geq n$, $d(n,m) = 1 + \frac{1}{n} \leq 1 + \frac{1}{n}$, so $m \in C_n$
        So $C_n$ is indeed $\{n, n+1, ...\}$.
    \end{explanation}
    
    $\forall n, C_{n+1} \subseteq C_{n}$, So $\bigl(C_n)_{n\ge 1}$ is a decreasing sequence of closed balls. However, $\forall n, n \notin C_{n+1}$, so $\bigcap_{n=1}^{\infty} C_n = \emptyset$.

    Then we show that $(X, d)$ is complete metric space.
    For every Cauchy sequence $(x_n)_{n=1}^\infty$ in $(X, d)$, $\forall \varepsilon > 0, \exists N$ such that $\forall n,m \geq N$, $d(x_n, x_m) < \varepsilon$. \\
    Then we can take $\varepsilon = 0.48763$, by definition, exists $N$ such that $\forall n,m \geq N$, $d(x_n, x_m) < 0.48763$. However if $x_n \neq x_m$, then $d(x_n,x_m) = 1 + \dfrac{1}{\min\{x_n,x_m\}} > 1$, so we know $\forall n \geq N, x_n = x_N$, and this will make the sequence coverage to $x_N \in \mathbb{N}$ since this is a constant sequence.

    Hence, every Cauchy sequence in $X$ coverage to some point in $X = \mathbb{N}$, so $(X, d)$ is complete metric space.
    
\end{proof}

\begin{proof}[(b)]
    First, we show that if the nested condition is satisfied and the radii goes to \(0\), then $(x_n)_{n=1}^\infty$ is Cauchy sequence. \\
    Since $r_n \;\to\; 0 \quad\text{as } n\to\infty$, $\forall \varepsilon' > 0, \exists N'$ such that $\forall n \geq N', r_n < \frac{\varepsilon}{2}$. \\
    And since $\overline{B}(x_{n+1},r_{n+1}) \;\subseteq\; \overline{B}(x_n,r_n) \quad\text{for all } n\in\mathbb{N}$, so $\{x_{n+1},x_{n+2}, ...\} \subseteq \overline{B}(x_n,r_n)$ hence $\forall n, m \geq N', d(x_n, x_m) \leq d(x_n, x_{N'}) + d(x_{N'}, x_m) \leq \frac{\varepsilon}{2} + \frac{\varepsilon}{2} = \varepsilon$.
    So by definition. $(x_n)_{n=1}^\infty$ is Cauchy sequence.

    Then we can start our proof:
    \begin{itemize}
        \item [\((\implies )\)] Since $(M,d)$ is complete and $(x_n)$ is Cauchy sequence, $(x_n)$ will converge to some $x' \in M$. \\
        \begin{claim}
            $x' \in \overline{B}(x_m,r_m)$ for all $m \in \mathbb{N}$
        \end{claim}
        \begin{explanation}
            Since $(x_n)_{n=1}^\infty$ is Cauchy sequence, the subsequence $(x_n)_{n=m}^\infty$ is also Cauchy sequence and also converge to $x'$. \\
            And since $\overline{B}(x_m,r_m)$ is an close ball in $(M, d)$, $x' \in \overline{B}(x_m,r_m)$ by properties we have shown in class.
        \end{explanation}
        Since $x' \in \overline{B}(x_m,r_m)$ for all $m \in \mathbb{N}$, the intersection of these ball are not empty, hence we proved.

        \item [\((\impliedby )\)] First we do a big claim below:
        \begin{claim}
            For every Cauchy sequence $(x_n)_{n=1}^\infty$, we can construct $(r_n)_{n=1}^\infty$ such that there is a subsequence $(x'_n, r'_n)_{n=1}^\infty$.
            $\overline{B}(x'_n,r'_n)$ satisfy nested condition. And further more, $r'_n \;\to\; 0 \quad\text{as } n\to\infty$.
        \end{claim}
        \begin{explanation}
            Since $(x_n)$ is Cauchy sequence, $\forall \varepsilon > 0, \exists N$ such that $\forall n, m \geq N, d(x_n,x_m) < \varepsilon$. \\
            Then we might take $\varepsilon = 1$ and discard those term that $(x_n)_{n=1}^{n=N}$, and do the following construction:\\
            For those $r_n, n \geq N$ terms, let $T_n = \{x_n, x_{n+1}, ...\}$. \\
            We define $d_n = \sup \{\, d(x_j,x_k) \mid x_j,x_k \in T_n \,\}$, since given any pair, the distance is smaller than 1 by the definition, so $d_n$ has a upper bound so supremum exists. \\
            The let $r_n = 2d_n$, then $T_n \subseteq \overline{B}(x_n,d_n) \subseteq \overline{B}(x_n,r_n)$. \\
            Since $x_n$ is Cauchy sequence, $d_n \to 0 \quad\text{as } n\to\infty$, hence $r_n \to 0 \quad\text{as } n\to\infty$.
            Then we show that $\overline{B}(x_k,r_k) \subseteq \overline{B}(x_n,r_n) , \text{for some } k > n$. \\
            First, we need to carefully pick $k$. Since $x_n$ is Cauchy, again, for $\varepsilon = \frac{d_n}{2}$, exist some $k = N'$ such that $d_k < \varepsilon$. \\
            Then $\forall z \in \overline{B}(x_k,r_k), d(x_n, z) \leq d(x_n, x_k) + d(x_k, z) \leq d_n + r_k \leq d_n + d_n = r_n$. \\
            Then we start from $N$-th terms and recursively do the above construction, we can get a sequence $(x'_n, r'_n)_{n=1}^\infty$, which is the subsequence of $(x_n, r_n)_{n=1}^\infty$, satisfy the nest condition and $r'_n \to 0 \;\text{as } n\to\infty$ since $r_n \to 0$.
        \end{explanation}
        Then we know the intersection of $\overline{B}(x'_n,r'_n)$ is not empty by hypothesis, let it $x'$, we know that $\lim_{n \to \infty} x'_n$ lie in infinitely many nested close ball, so $\lim_{n \to \infty} x'_n \in \lim_{n \to \infty} \overline{B}(x'_n,r'_n)$ $0 \leq \lim_{n \to \infty} d(x'_n, x') = 0 \leq r'_n \text{ since } \lim_{n \to \infty} r'_n = 0,\text{and this implies } x'_n \to x'.$
        Hence, we show that $(x'_n)$ converge to $x'$, and we know if a Cauchy sequence's subsequence converge to some point $x'$, the original Cauchy sequence also converge to $x'$, so $(x_n)$ converge to $x$.\\
        since for every Cauchy sequence converge, so $(M,d)$ is complete.
        
    \end{itemize}
\end{proof}