\lecture{10}{2 Oct. 10:20}{}
\begin{proposition}
    Let \((X, d)\) be a compact metric space and let \(f: X \to \mathbb{R} \) be a continuous map. Then 
    \begin{itemize}
        \item [(1)] \(f\) is bounded on \(X\)
        \item [(2)] \(\exists x_{\text{max}}, x_{\text{min}}\) s.t. 
        \[
            f \left( x_{\text{max}} \right) = \max _{x \in X} f(x) \quad f \left( x_{\text{min}} \right) = \min _{x \in X} f(x). 
        \]
    \end{itemize}  
\end{proposition}

\begin{proof}
    Since \(X\) is compact and \(f\) is continuous, so \(f(X)\) is compact in \(\mathbb{R} \) and thus \(f(X)\) is bounded by \autoref{cl: compact subset of X is closed and bounded}. Now since \(\left\{ f(x) \mid x \in X \right\} \) is bounded in \(\mathbb{R} \), so \(\exists p_1, p_2\) s.t. \(p_1 \le f(x) \le p_2\) for all \(x \in X\). Let \(M = \sup_{x \in X} f(x)\) and \(P = \inf _{x \in X} f(x)\). Thus, there exists \(\left\{ y_n \right\}_{n=1}^{\infty} \subseteq f(X) \) s.t. \(y_n \le M\) and \(\lim_{n \to \infty} y_n = M \) for all \(n\). Hence, \(\exists \left\{ x^{(n)} \right\}_{n=1}^{\infty}  \) s.t. \(y_n = f \left( x^{(n)} \right) \) for all \(n\). Since \(X\) is compact, so there exists a subsequence \(\left\{ x^{(n_k)} \right\}_{k=1}^{\infty}  \) s.t. 
    \[
        \lim_{k \to \infty} x^{(n_k)} = x_{\ast} \in X. 
    \]  Since \(f\) is continuous, so \(\lim_{k \to \infty} f \left( x^{(n_k)} \right) = f \left( x_{\ast}  \right)   \), which means \(\lim_{k \to \infty} y^{(n_k)} = f \left( x^* \right)  \). Since \(\lim_{n \to \infty}y_n = M \), so \(f(x_{\ast} ) = M\), and thus \(f \left( x_{\ast}  \right) = \max _{x \in X} f(x) \).                        
\end{proof}

\begin{eg}
    This proposition is false if \(X\) is not compact.  
\end{eg}
\begin{explanation}
    Consider \(f:(0, 1) \to \mathbb{R} \) and \(f(x) =x\), then \(f\) can't achieve its sup on \((0, 1)\).    
\end{explanation}

\begin{definition}[uniformly continuous] \label{def: uniformly conti}
    Let \(f:X \to Y\) be a map between metric spaces \((X, d_X)\) and \((Y, d_Y)\), we say \(f\) is uniformly continuous if \(\forall\varepsilon > 0\), \(\exists \delta > 0\) s.t. 
    \[
        d_Y (f(x), f \left( x^{\prime}  \right) ) < \varepsilon \text{ whenever } d_X \left( x, x^{\prime}  \right) < \delta. 
    \]      
\end{definition}

\begin{remark}
    This \(\delta \) is independent of \(x\). 
\end{remark}

\begin{eg}
    \(f(x)= \frac{1}{x}\) is continuous on \((0, 1]\), then \(f\) is not uniformly continuous on \([0, 1]\).    
\end{eg}
\begin{explanation}
    Let \(\varepsilon = 10\). Suppose \(\exists \delta > 0\) s.t. 
    \[
        \left\vert f(x) - f(y) \right\vert < \varepsilon \text{ if } \vert x - y \vert < \delta. 
    \]  
    We may assume \(\delta < 1\). Choose \(x = \delta \) and \(y = \frac{\delta}{11}\), then \(\vert x - y \vert = \frac{10}{11} \delta < \delta  \), and 
    \[
        \vert f(x) - f(y)  \vert = \frac{1}{\delta } - \frac{11}{\delta } = \frac{10}{\delta } > 10.
    \]    
\end{explanation}

\begin{theorem}
    Suppose \(X\) is compact, then \(f:X \to Y\) is continuous iff \(f\) is uniformly continuous.   
\end{theorem}
\begin{proof}
    If \(f\) is uniformly continuous, then \(f\) is continuous. Now we show the other direction. If \(f\) is continuous, then by contradiction, if it is not uniformly continuous, then \(\exists \varepsilon > 0\) s.t. no matter how small \(\delta \) is, then \(\exists p, q\) s.t. \(d_X(p, q) < \delta \) and \(d_Y(f(p), f(q)) \ge \varepsilon \). Choose \(\delta = \frac{1}{n}\), then exists \(p^{(n)}, q^{(n)} \in X\) s.t. 
    \[
        d_X \left( p^{(n)}, q^{(n)} \right) < \frac{1}{n} \text{ and } d_Y \left( f \left( p^{(n)}\right), f \left( q^{(n)}  \right)   \right) \ge \varepsilon.  
    \]  Since \(X\) is compact, so \(\left\{ p^{(n)} \right\} \) has a convergent subsequence, say 
    \[
        \lim_{k \to \infty} p^{(n_k)} = p \in X.  
    \] Also, there exsits \(\left\{ q^{(n_k)} \right\} \) s.t. \(\lim_{k \to \infty} q^{(n_k)} = p \) since \(\lim_{k \to \infty} d_X \left( p^{(n_k)}, q^{(n_k)} \right) = 0  \). (By triangle inequality). Thus, we have 
    \[
        \lim_{n \to \infty} d_Y \left( f \left( p^{(n_k)} \right) , f \left( q^{(n_k)} \right)  \right) = 0 
    \] since \(f\) is continuous. 
    Hence, it is a contradiction, since
    \[
        \lim_{n \to \infty} d_Y \left( f \left( p^{(n)} \right), f \left( q^{(n)} \right)   \right) \ge \varepsilon. 
    \]
\end{proof}

\section{Connectedness}
\begin{definition}[disconnected/connected] \label{def: disconnected/connected}
    Let \((X, d)\) be a metric space. We say \(X\) is disconnected iff \(\exists \) non-empty open \(V, W\) in \(X\) s.t. \(V \cup W = X\) and \(V \cap W = \varnothing \). Also, we called \(X\) is connected if it is non-empty and not disconnected.       
\end{definition}

\begin{remark}
    \(X\) is disconnected iff \(X\) has a nonempty proper subset which is both open and closed.  
\end{remark}
\begin{proof}
    \vphantom{text}
    \begin{itemize}
        \item [\((\implies )\)] \(V \cup W = X\) and \(V \cap W = \varnothing \), so \(X \setminus V = W\) is open, and thus \(V\) is closed. Since we already know \(V\) is open, so \(V\) is a proper subset of \(X\) that is clopen. 
        \item [\((\impliedby )\)] Suppose \(V\) is clopen in \(X\) and \(V \neq X\). Let \(W = X \setminus V \neq \varnothing \), then we know \(W\) is open since \(V\) is closed and thus \(V \cup W = X\) and \(V \cap W = \varnothing \) and \(V, W\) are both non-empty open in \(X\), so \(X\) is disconnected.               
    \end{itemize}
\end{proof}

\begin{eg}
    Suppose \(X = [1, 2] \cup [3, 4]\), then \(X\) is disconnected.  
\end{eg}
\begin{explanation}
    Since 
    \[
        [1, 2] = (-\infty , 2.5) \cap X,
    \] so \([1, 2]\) is open in \(X\), and similarly we can show \([3, 4]\) is open in \(X\), and \([1, 2] \cup [3, 4] = X\), and \([1, 2] \cap [3, 4] = \varnothing \), so \(X\) is disconnected.       
\end{explanation}

\begin{definition}
    Let \((X, d)\) be a metric space and \(Y \subseteq X\). We say \(Y\) is connected (resp. disconnected) iff the metric space \(\left( Y, d \vert_{Y \times Y} \right) \) is connected (resp. connected).    
\end{definition}

\begin{remark}
    \(Y\) is disconnected iff \(\exists U, V\) open in \(X\) s.t. \(Y \subseteq U \cup V\) with \(U \cap Y \neq \varnothing \) and \(V \cap Y \neq \varnothing \) and \(U \cap V \cap Y = \varnothing \).        
\end{remark}
\begin{proof}
    \vphantom{text} 
    \begin{itemize}
        \item [\((\implies )\)] If \(Y\) is disconnected, then \(\exists \) open sets \(O_1, O_2\) in \(Y\) s.t. \(Y \subseteq O_1 \cup O_2\) and \(O_1 \neq \varnothing \) and \(O_2 \neq \varnothing \) and \(O_1 \cap O_2 = \varnothing \). Since \(O_1\) is open in \(Y\), so there exists open set \(U_1\) in \(X\) s.t. \(O_1 = U_1 \cap Y \neq \varnothing \). Similarly, we know there exists \(U_2\) open in \(X\) s.t. \(O_2 = U_2 \cap Y \neq \varnothing \). Since 
        \[
            Y \subseteq O_1 \cup O_2 = (U_1 \cap Y) \cup (U_2 \cap Y) = (U_1 \cup U_2) \cap Y \subseteq U_1 \cup U_2,
        \] and we know
        \[
            \varnothing = O_1 \cap O_2 = (U_1 \cap Y) \cap (U_2 \cap Y) = U_1 \cap U_2 \cap Y.
        \]
        \item [\((\impliedby )\)] Choose \(O_1 = U \cap Y\) and \(O_2 = V \cap Y\). Note that \(O_1, O_2 \) are non-empty and open in \(Y\) , and we can easily check that \(O_1 \cup O_2 = Y\) and \(O_1 \cap O_2 = \varnothing \), so \(Y\) is disconnected.       
    \end{itemize}
\end{proof}

\begin{theorem}
    Let \(f : X \to Y\) be a continuous map between metric spaces. Let \(E\) be any connected subset of \(X\). Then \(f(E)\) is connected in \(Y\).     
\end{theorem}
\begin{proof}
    Suppose not, \(f(E)\) is disconnected in \(Y\), then \(\exists \) open set \(U, V\) in \(Y\) s.t. \(f(E) \subseteq U \cup V\) and \(U \cap f(E) \neq \varnothing \) and \(v \cap f(E) \neq \varnothing \) and \(U \cap V \cap f(E) = \varnothing \). Since \(f\) is continuous, so \(f^{-1}(U)\) and \(f^{-1}(V)\) are both open and non-empty and \(f^{-1}(U) \cap f^{-1}(V) = \varnothing \), and note that \(E \subseteq f^{-1}(U) \cup f^{-1}(V)\), so \(E\) is disconnected.              
\end{proof}

\begin{theorem}
    Let \(X\) be a nonempty subset of \(\mathbb{R} \), then TFAE: 
    \begin{itemize}
        \item [(a)] \(X\) is connected.  
        \item [(b)] Whenever \(x, y \in X\) and \(x < y\), we have \([x, y] \subseteq X\).    
        \item [(c)] \(X\) is an interval. 
    \end{itemize}  
\end{theorem}
\begin{proof}[proof from (a) to (b)]   
    Suppose not, then there exists \(z \notin X\) s.t. \(x < z < y\). Hence, we can pick \(U = (-\infty , z) \cap X\) and \(V = (z, \infty ) \cap X\), then \(U \neq \varnothing \) and \(V \neq \varnothing \) and \(U, V\) both open in \(X\) and \(U \cap V = \varnothing \) and \(X \subseteq U \cup V\), so \(X\) is disconnected, which is a contradiction.           
\end{proof} 
\begin{proof}[proof from (b) to (a)]
    Suppose not, then \(X\) is disconnected, which means there exists \(V, W\) open in \(\mathbb{R} \) s.t. \(X \subseteq V \cup W\) and \(V \cap X \neq \varnothing \) and \(W \cap X \neq \varnothing \) and \(V \cap W \cap X \neq \varnothing \). Fix \(x \in V \cap X\), and \(y \in W \cap X\), note that \(x \neq y\) since \(V \cap W \cap X = \varnothing \). Assume \(x < y\) (\(x > y\) can be proved similarly). By our assumption, \([x, y] \subseteq X \subseteq V \cup W\). Consider \(S = [x, y] \cap V\), then \(x \in S\), and suppose \(z = \sup S\). 
    \begin{itemize}
        \item Case 1: \(z \in V\), then \(z < y\). Since \(V\) is open in \(X\), so there exists \(\varepsilon > 0\) s.t. 
        \[
            B(z, \varepsilon ) = (z - \varepsilon , z + \varepsilon ) \subseteq V, 
        \] so we know \(\left[ z, z + \frac{\varepsilon}{2} \right] \subseteq [x, y] \subseteq V \), which is impossible since \(z = \sup S\). 
        \item Case 2: \(z \in W\), and since \(W\) is open in \(X\), so \(\exists \varepsilon > 0\) s.t. \((z - \varepsilon , z + \varepsilon ) \subseteq W\), so we know \(\left[ z - \frac{\varepsilon}{2}, z \right] \cap V = \varnothing  \) since \(W \cap V = \varnothing \), which is a contradiction since \(z = \sup S\).          
    \end{itemize}
\end{proof}