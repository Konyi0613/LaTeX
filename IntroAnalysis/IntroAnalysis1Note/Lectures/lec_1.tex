\chapter{Basic Things}
\lecture{1}{2 Sep. 09:10}{}

\section{Natural Numbers}
The set of natural numbers is denoted by \(\mathbb{N} = \left\{ 1,2,\dots  \right\} \). There exists an addition operation 
\[
  1+1 = 2 \quad 1+1+1=3 \quad \underbrace{1+1+\dots +1}_{n \text{ times}}=n.
\] 
\section{Integers}
The set of integers is \(\mathbb{Z} = \left\{ 0, \pm 1, \pm 2, \dots  \right\} \). There is a zero element \(0\) such that \(z + 0 = z\) for any \(z \in \mathbb{Z} \). Also, for \(n \in \mathbb{N} \), we have \(n + (-n) = 0\) and \(n-m = n + (-m)\) for all \(n,m \in \mathbb{N} \).        
% https://q.uiver.app/#q=WzAsMyxbNCwwLCJcXG1hdGhiYntSfSJdLFsyLDAsIlxcbWF0aGJie1F9Il0sWzAsMCwiXFxtYXRoYmJ7Wn0iXSxbMSwwLCJcXHRleHR7Q29tcGxldGVuZXNzIGF4aW9tfSIsMix7ImxldmVsIjoyfV0sWzIsMSwiXFx0ZXh0e2ludHJvZHVjZSBkaXZpc2lvbn0iLDIseyJsZXZlbCI6Mn1dXQ==
\[\begin{tikzcd}[sep=large]
	{\mathbb{Z}} && {\mathbb{Q}} && {\mathbb{R}}
	\arrow["{\text{introduce division}}"', Rightarrow, from=1-1, to=1-3]
	\arrow["{\text{Completeness axiom}}"', Rightarrow, from=1-3, to=1-5]
\end{tikzcd}\]
\section{Field}
Next, we introduce the concept of field. 
\begin{definition}[Fields]\label{dfn: field}
  A field is a set \(F\) together with two binary operations, called addition(\(+\)) and multiplication(\(*\)), such that the following properties hold: 
  \begin{itemize}
    \item [(a)] \(a+b=b+a\), \(a \cdot b = b \cdot a\) for \(a,b \in F\). 
    \item [(b)] \((a+b)+c=a+(b+c)\), \((a \cdot b) \cdot c = a \cdot (b \cdot c)\) for \(a,b,c \in F\). 
    \item [(c)] \(a \cdot (b + c) = a \cdot b + a \cdot c\).  
    \item [(d)] There are distince element \(0\) and \(1\) such that \(a + 0 = a\), \(a \cdot 1 = a\) for \(a \in F\). 
    \item [(e)] For each \(a \in F\), there exists \(-a \in F\) such that \(a + (-a) = 0\). If \(a \neq 0\), there is an element \(\frac{1}{a}\) or \(a^{-1} \) in \(F\) such that \(a \cdot \frac{1}{a} = 1\), or \(a \cdot a^{-1} = 1 \).         
  \end{itemize} 
\end{definition}

\begin{remark}
  If \(a \in F\), then \(a + a \in F\). We denote \(a + a\) by \(2 \cdot a\). Similarly, 
  \[
    \underbrace{a+a+\dots +a}_{n \text{ times}} = n \cdot a,
  \]   
  and 
  \[
    a^n = \underbrace{a \cdot a \cdot \dots \cdot a}_{n \text{ times}}
  \]
  if \(a \in F\) and \(n \in \mathbb{N} \).  
\end{remark}

\begin{remark}
  In a field, we have subtraction and division \(a - b = a + (-b)\) for \(a, b \in F\). If \(b \neq 0\), then \(\frac{a}{b}=a \cdot b^{-1}\) for \(a,b \in  F \).      
\end{remark}

In a field \(F\), we have 
\begin{align*}
  (a+b)^2 &= (a+b)\cdot(a+b) \\
          &= (a+b)\cdot a + (a+b) \cdot b \\
          &= a \cdot a + b \cdot a + a \cdot b + b \cdot b \\
          &= a^2 + ab + ab + b^2 \\
          &= a^2 + 2ab + b^2.
\end{align*} 

\begin{eg}
\[
  \frac{a}{b} + \frac{c}{d} = \frac{ad + bc}{bd}
\] if \(b \neq 0\) and \(d \neq 0\).  
\end{eg}

\begin{proof}
  \begin{align*}
    \frac{a}{b} + \frac{c}{d} &= a \cdot b^{-1} + c \cdot d^{-1} \\
    &= a b^{-1} d d^{-1} + c d^{-1} b b^{-1} \\
    &= ad b^{-1} d^{-1} + cb d^{-1} b^{-1} \\
    &= \frac{ad + bc}{bd}.       
  \end{align*}
  Notice that this is true since we have commutativity in multiplication and 
 \[
  d^{-1}b^{-1} = (bd)^{-1} = \frac{1}{bd}.
 \]
\end{proof}

\begin{eg}
  The set of rational numbers \(\mathbb{Q} = \left\{ \frac{p}{q} \mid  p,q \in \mathbb{Z} , q \neq 0 \right\} \) is a field. 
\end{eg}

\begin{eg}
  The set of real numbers is also a field.
\end{eg}

\begin{eg}
  \(F_2 = \left\{ 0,1 \right\} \) is also a field since we can define addition and multiplication like \(0+0=0, 0+1=1,1+1=0,\) and \(0 \cdot 0 = 0, 1 \cdot 0 = 0, 1 \cdot 1 = 1\).   
\end{eg}

\section{Order Relation}
Next, we introduce the order relation. The real number system is ordered by the relation \(<\), which has the following properties. 
\begin{itemize}
  \item [(f)] For each pair of real numbers \(a\) and \(b\), exactly one of the follwing is true: \(a=b, a < b, b < a\).
  \item [(g)] If \(a<b\) and \(b<c\), then \(a<c\). 
  \item [(h)] If \(a<b\), then \(a+c < b+c\) for any \(c\), and if \(0<c\), then \(a \cdot c < b \cdot c\).  
\end{itemize}

\begin{definition}
  A field with an order relation satisfy (f) to (h) is called an ordered field.
\end{definition}

\begin{eg}
  The set of rational numbers is an ordered field.
\end{eg}

\begin{eg}
  \(F_2\) is not an ordered field. 
\end{eg}
\begin{proof}
  If \(0<1\), then \(1=0+1 < 1+1=0\), which is a contradiction. If \(1<0\), then \(0=1+1<0+1=1\), which is also a contradiction.    
\end{proof}

\begin{notation}
  In an ordered field, we use \(a \le b\) to denote either \(a<b\) or \(a=b\).   
\end{notation}
\section{Absolute Value and Triangle Inequality}
Next, we define the absolute value of a real number 
\[
  \vert a \vert = \begin{dcases}
   a , &\text{ if } a \ge 0  ;\\
   -a , &\text{ if } a \le 0 ;
  \end{dcases} 
\]

\begin{theorem}[Triangle Inequality]
  \[
    \vert a+b \vert \le \vert a \vert + \vert b \vert   
  \] for all \(a,b \in \mathbb{R} \). 
\end{theorem}

\begin{corollary}
  \[
    \left\vert \vert a \vert - \vert b \vert   \right\vert \le \vert a-b \vert \quad \text{and} \quad \left\vert \vert a \vert - \vert b \vert   \right\vert \le \vert a+b \vert 
  \]
\end{corollary}
\begin{proof}
  We write
  \[
    \vert a \vert = \vert a-b+b \vert \le \vert a-b \vert + \vert b \vert.    
  \]
  Similarly we have 
  \[
    \vert b \vert \le \left\vert b-a \right\vert + \vert a \vert.   
  \]
  So 
  \[
    - \vert b-a \vert \le \vert a \vert - \vert b \vert \le \vert a-b \vert.   
  \]
  Thus, 
  \[
    \left\vert \vert a \vert - \vert b \vert   \right\vert \le \vert a-b \vert.
  \]
\end{proof}

\section{Supremum and Infimum}
Next, we introduce the notion of supremum of a subset of real numbers. 
\begin{definition}
  Let \(S\) be a subset of \(\mathbb{R} \), 
  \begin{itemize}
    \item [(1)] we say \(b\) is an upper bound of \(S\) if \(x \le b\) for all \(x \in S\). 
    \item [(2)] If \(B\) is an upper bound of \(S\), and no number smaller than \(B\) is an upper bound of \(S\), then \(B\) is called the supremum or the least upper bound of \(S\). We write \(B = \sup S\).          
  \end{itemize}  
\end{definition}

\begin{corollary}
  If \(B = \sup S\), then 
  \begin{itemize}
    \item [(1)] \(x \in S\) implies \(x \le B\)
    \item [(2)] If \(b < B\), then \(b\) is not an upper bound of \(S\), i.e. there exists \(x_1 \in S\) such that \(b < x_1\).       
  \end{itemize} 
\end{corollary}

\begin{definition}
  Let \(S\) be a subset of \(\mathbb{R} \), 
  \begin{itemize}
    \item [(1)] we say \(b\) is an lower bound of \(S\) if \(x \ge b\) for all \(x \in S\). 
    \item [(2)] If \(\alpha \) is an lower bound of \(S\), and no number bigger than \(\alpha \) is an lower bound of \(S\), then \(\alpha \) is called the infimum or the greatest lower bound of \(S\). We write \(\alpha  = \inf S\).          
  \end{itemize}  
\end{definition}

\begin{corollary}
  If \(\alpha  = \inf S\), then 
  \begin{itemize}
    \item [(1)] \(x \in S\) implies \(x \ge \alpha \)
    \item [(2)] If \(\alpha  < a\), then \(a\) is not an lower bound of \(S\), i.e. there exists \(x_1 \in S\) such that \(x_1 < a\).       
  \end{itemize} 
\end{corollary}

\begin{notation}[Interval Notation]
 \begin{align*}
  (a, b) &= \left\{ x \mid a < x < b \right\} \\
  (a,b] &= \left\{ x \mid a < x \le b \right\} \\ 
  [a, b) &= \left\{ x \mid a \le x < b \right\}  
 \end{align*}
\end{notation}

\begin{eg}
  \(S = \left\{ x \mid x<0 \right\} = (-\infty , 0)\), then \(\sup S = 0\) but \(\inf S\) does not exists.    
\end{eg}

\begin{eg}
  \(S_1 = \left\{ -1, -2, -3, -4, \dots  \right\} = \left\{ -n \mid n \in \mathbb{N}  \right\}  \), then \(\sup S = -1\), but \(\inf S\) does not exist.    
\end{eg}

\begin{definition}[Nonempty Sets]
  A nonempty set is that a set has at least one element. The empty set, written as \(\varnothing \), is the set has no elements at all. 
\end{definition}

\begin{eg}
  \(S = \left\{ x \mid x \in \mathbb{Q} , x < \sqrt{2}  \right\} \) 
\end{eg}

In \(\mathbb{Q} \), \(\sup S\) does not exist. In \(\mathbb{R} \), \(\sup S = \sqrt{2} \).    

\begin{theorem}[Completeness axiom]
  If a nonempty set of real numbers (an ordered field) is bounded above, then it has a least upper bound or \(\sup S\) exists. 
\end{theorem}
\begin{remark}
  This is an extra axiom that can't be derived from the properties of ordered field. 
\end{remark}
\begin{remark}
  Up to "isomorphism", there is exactly one complete ordered field: the field of real numbers.
\end{remark}
\begin{remark}
  From now, we assume \(\mathbb{R} \) satisfies the completeness axiom. Thus, any nonempty subset \(S \subseteq \mathbb{R} \) that is bounded above, we have \(\sup S\) exists.   
\end{remark}
We can prove the following property of \(\sup S\).
\begin{theorem}
  If \(S \subseteq \mathbb{R} \) is bounded above, then \(\sup S\) is the unique real number \(B\) such that 
  \begin{itemize}
    \item [(i)] \(x \le B\) for all \(x \in S\)
    \item [(ii)] for every \(\varepsilon > 0\), there exist an \(x_0 \in S\) such that \(B \varepsilon < x_0\).     
  \end{itemize}   
\end{theorem} 
\begin{proof}
  (i), (ii) follows from the definition. We prove the uniqueness. Suppose \(B_1 = \sup S = B_2\). We want to show \(B_1 = B_2\). Suppose \(B_1 \neq B_2\). Then either \(B_1 < B_2\) or \(B_2 < B_1\). However, if either one is true, then the other one cannot be \(\sup S\).      
\end{proof}
\begin{theorem}[Archimedean Property]\label{thm: Archimedean property}
  If \(p > 0\) and \(\varepsilon > 0\), then there exists an \(n \in \mathbb{N} \) such that \(p < n \varepsilon \).    
\end{theorem}
\begin{proof}
  We prove this contradiction. Suppose it is not true. This implies \(n \varepsilon \le p\) for all \(n \in \mathbb{N} \). Consider \(S = \left\{ n \varepsilon \mid n \in \mathbb{N}  \right\} \), then \(p\) is an upper bound of \(S\), so \(S\) is bounded above by \(p\), so we know \(B = \sup S\) exists. Hence, \(n \varepsilon \le B\) for all \(n \in \mathbb{N} \), so we have \((n+1)\varepsilon \le B\), which means
  \[
    n \varepsilon \le B - \varepsilon 
  \]   for all \(n \in \mathbb{N} \). This implies \(B - \varepsilon \) is also an upper bound of \(S\), which is a contradiction.   
\end{proof}

\begin{theorem}
  Every nonempty subset of the integers that is bounded below has a least element.
\end{theorem}

\begin{proof}
  We first introduce an axiom:
  \begin{theorem}[Well-Ordering principle]
    Every non-empty subset of the natural numbers has a least element.
  \end{theorem}
  \begin{note}
    Here, \(\mathbb{N} \) can be \(\left\{ 0,1,2,\dots  \right\} \) or \(\left\{ 1,2,3,\dots  \right\} \), which is not that important.
  \end{note}
  Now we call this subset of integers as \(S\), and suppose we have \(m\) as a lower bound of \(S\), then define \(S^{\prime} = \left\{ s - m \mid s \in S \right\} \), then we know \(S^{\prime} \) is a nonempty subset of \(\mathbb{N} \), then by well-ordering principle we know there is a least element in \(S^{\prime} \)  and thus there is also a least element in \(S\). 
\end{proof}

\begin{corollary}
  Every nonempty subset of the integers that is bounded above has a greatest element.
\end{corollary}
\begin{proof}
  Suppose \(M\) is an upper bound, then define a set \(S^{\prime} = \left\{ M - s \mid s \in S \right\} \), then by well-ordering principle we know \(M - a\) is the least element of \(S^{\prime} \) for some \(a \in S\), so we have \(M - x \ge M - a\) for all \(x \in S\), which means \(a \ge x\) for all \(x \in S\) and since \(a \in S\), so \(a\) is the greatest element of \(S\).          
\end{proof}

\begin{theorem}
  The set of rational numbers is dense in the real number. That is, if \(a\) and \(b\) are real numbers with \(a<b\), then there exists a rational number \(\frac{p}{q}\) such that \(a < \frac{p}{q} < b\).     
\end{theorem}
\begin{proof}
  Let \(a,b \in \mathbb{R} \), \(a<b\). By \hyperref[thm: Archimedean property]{Archimedean Property}, \(\exists q \in \mathbb{N} \) such that \(q(b-a)>1\). Let \(S = \left\{ m \mid m \text{ is an integer with } m > qa \right\} \), since we know \(S \neq \varnothing \) and \(S\) is bounded below. Hence, \(p=\inf S\) exists and is an integer by the last theorem. So \(qa < p\) and \(p-1\le qa\), which means \(qa<p \le qa+1 < qb\), so we have \(a < \frac{p}{q} < b\). 
\end{proof}
