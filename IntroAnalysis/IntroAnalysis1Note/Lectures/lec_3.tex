\lecture{3}{9 Sep. 09:10}{}
\begin{definition}[Cartesian Product]\label{dfn:Cartesian Product}
    Let \(A, B\) be sets. The cartesian product of \(A\) and \(B\) is defined by 
    \[
        A \times B = \left\{ (a, b) \mid a \in A, b \in B \right\}. 
    \]   
    Similarly, the cartesian product of \(X_1, X_2, \dots , X_n\) is 
    \[
        X_1 \times X_2 \times \dots \times X_n = \left\{ (x_1, x_2, \dots , x_n) \mid x_i \in X_i \ \forall 1 \le i \le n\right\}. 
    \] 
\end{definition}

\begin{definition}[Functions] \label{def: functions}
    Let \(X_1, X_2, \dots , X_n\) be sets and let \(Y\) be another set. A fuction of \(n\) variables with codomains is a map \(f: X_1 \times X_2 \times \dots \times X_n \to Y\) which assigns each \(n\)-tuple \((x_1, x_2, \dots , x_n)\) with \(x_i \in X_i\) a unique element \(f(x_1, x_2, \dots , x_n)\).       
\end{definition}

\begin{definition*}
    We talk about the definition of domain, codomain, and range:
    \begin{definition}
        The domain of \(f\) is \(X_1 \times X_2 \times \dots \times X_n\) and \(Y\) is the codomain of \(f\).   
    \end{definition}
    \begin{definition}
        The range of \(f\) is 
        \[
            \left\{ f(x_1, x_2, \dots ,x_n) \in Y \mid x_i \in X_i \ \forall i \right\}.
        \] 
    \end{definition}
\end{definition*}

In the definition of metric space, we write \((X, d)\) to emphasize our set \(X\) and \(d\) is a distance function defined on \(X \times X\), i.e. 
\[
    d: X \times X \to [0, \infty) \subseteq \mathbb{R},
\]    where 
\[
    d: (x,y) \mapsto d(x,y)
\]
for \(x, y \in X\). Let \((X, d)\) be a metric space and \(Y \subseteq X\). Then \((Y, d \vert _{Y \times Y})\) is also a metric space with distance function defined by 
\[
    d \vert _{Y \times Y} \to [0, \infty)
\]    and 
\[
    d \vert _{Y \times Y}: (\alpha , \beta) \mapsto d(\alpha , \beta ) \text{ for } \alpha , \beta \in Y.
\]

\begin{eg}
    Recall the \hyperref[def: l1-metric]{Taxi-cab metric}, it can be used in cryptography. For example, for two binary strings, we know 
    \[
        d_1((10010), (10101)) = 3 = \text{the number of mismatched bits}.
    \]
\end{eg}

\begin{eg}
    Recall the \hyperref[def: linf-metric]{\(l^{\infty} \)-metric}. Suppose two jobs where each consists of \(3\) tasks, and the time (in hours) to complete each task is represented by a vector 
    \[
        x = (2, 4, 6), \ y = (3, 7, 5),
    \]  so 
    \[
        d_\infty (x, y) = \max \left\{ \vert 2 - 3 \vert, \vert 4 - 7 \vert, \vert 6 - 5 \vert \right\} = 3. 
    \]
\end{eg}

\begin{definition}[Lipschitz equivalent metrics] \label{def: Lipschitz equivalent metric}
    Let \((X, d_1)\) and \((X, d_2)\) be two metrics on \(X\). We say \(d_1\) and \(d_2\) are Lipschitz equivalent if \(\exists c_1, c_2 > 0\) s.t. 
    \[
        c_1 d_1(x, y) \le d_2(x, y) \le c_2 d_1(x, y) \quad \forall x, y \in X
    \] 
\end{definition} 
\begin{remark}
    They will have same topology (defined later).
\end{remark}

\begin{proposition} \label{prop: ineq between d1 d2 doo}
    For all \(x, y \in \mathbb{R} ^n\), 
    \begin{align}
        d_2(x, y) &\le d_1(x, y) \le \sqrt{n} d_2(x,y) \\
        d_\infty (x, y) &\le d_2(x, y) \le \sqrt{n} d_\infty (x, y) 
    \end{align}
\end{proposition}
\begin{remark}
    \begin{align*}
        d_\infty (x, y) &\ge \frac{1}{\sqrt{n}} d_2(x, y) \\
        &\ge \frac{1}{\sqrt{n}} \frac{1}{\sqrt{n}} d_1(x, y) = \frac{1}{n}d_1(x, y).
    \end{align*}
    Also, 
    \[
        d_\infty (x, y) \le d_2(x, y) \le d_1(x, y).
    \]
\end{remark}
\begin{remark}
    \(d_1, d_2, d_\infty \) are all Lipschitz equivalent. 
\end{remark}
\begin{proof}[proof of \autoref{prop: ineq between d1 d2 doo} ]
    Recall \(x = (x_1, \dots , x_n), y = (y_1, \dots , y_n)\), then 
    \[
        d_1(x, y) = \sum_{i=1}^n \vert x_i - y_i \vert, \quad d_2(x,y) = \left( \sum_{i=1}^n (x_i - y_i)^2  \right)^\frac{1}{2}.   
    \] 
    By Cauchy-Schurwatz inequality, 
    \begin{align*}
        d_1(x, y) &= \sum_{i=1}^n \vert x_i - y_i \vert \\
        &\le \left( \sum_{i=1}^n \vert x_i - y_i \vert   \right)^\frac{1}{2} \left( \sum_{i=1}^n 1^2  \right)^\frac{1}{2} = \sqrt{n} d_2(x,y).     
    \end{align*}

    Now we show that \(d_1(x, y) \ge d_2(x, y)\). 
    \begin{align*}
        \left( d_1(x, y) \right)^2 &= \left( \sum_{i=1}^n \vert x_i - y_i \vert   \right)^2 \\
        &= \sum_{i=1}^n \vert x_i - y_i \vert^2 + 2 \sum_{1 \le i < j \le n} \vert x_i - y_i \vert \vert x_j - y_j \vert \\
        & \ge \sum_{i=1}^n \vert x_i - y_i \vert^2 = d_2(x, y)^2.         
    \end{align*} 
    Hence, we have \(d_1(x, y) \ge d_2(x, y)\).
    
    Now we show that \(d_2(x, y) \le \sqrt{n} d_\infty (x, y) \). Note that 
    \[
        d_2(x, y) = \left( \sum_{i=1}^n \vert x_i - y_i \vert^2  \right)^\frac{1}{2}, \quad d_\infty (x, y) = \max_{1 \le i \le n} \vert x_i - y_i \vert.   
    \] For each \(i\), we know 
    \[
        \vert x_i - y_i \vert \le d_\infty (x, y), 
    \] so 
    \[
        d_2(x, y) ^2 \le \sum_{i=1}^n d_\infty (x, y)^2 = n d_\infty (x, y)^2, 
    \] so \(d_2(x, y) \le \sqrt{n} d_\infty (x, y) \). 
\end{proof}

\begin{definition}[Discrete metric] \label{def: Discrete metric}
    Let \(X\) be any set, define the discrete metric:
    \[
        d_{\text{disc}}: X \times X \to \left\{ 0,1 \right\} 
    \] where 
    \[
        d_{\text{disc}}(x, y) = \begin{dcases}
            0 , &\text{ if } x = y ;\\
            1, &\text{ if }  x \neq y.
        \end{dcases}
    \]
\end{definition}
Why this is a metric? Because
\begin{itemize}
    \item \(d_{\text{disc}}(x, y) \ge 0\) for all \(x, y \in X\) and \(d(x, y) = 0\) if and only if \(x = y\). 
    \item \(d_{\text{disc}}(x, y) = d_{\text{disc}}(y, x)\) by definition. 
    \item \(d_{\text{disc}}(x, z) \le d_{\text{disc}} (x, y) + d_{\text{disc}}(y, z)\)?      
\end{itemize}

\begin{proof}[proof of triangle inequality in discrete metric]
    We first consider the case that \(x=z\), then 
    \[
        d_{\text{disc}}(x,z)=0,
    \]
    so it is obviously that the triangle inequality is true. 
    
    Now if \(x \neq z\), then either \(y \neq z\) or \(y \neq x\) must happen, so the triangle inequality must be true.   
\end{proof}

\begin{eg}
    We can define 
    \[
        d(x, x) = 0, \quad d(x, y) = \text{minimal length of a path from }x \text{ to } y,
    \] then this is also a metric.
\end{eg}
\begin{figure}[H]
    \centering
    \incfig{GraphMetric}
    \caption{Graph metrics}
    \label{fig:GraphMetric}
\end{figure}

\begin{definition}[Convergence in metric space] \label{def: convergence in metric space}
    Let \(m\) be an integer, \((X, d)\) be a metric space, and let \(\left( X^{(n)} \right)_{n=m}^{\infty}\) be a sequence of points in \(X\). Let \(x \in X\). We say that \(\left( X^{(n)} \right)_{n=m}^{\infty}  \) converges to \(x\) with respect to \(d\) iff 
    \[
        \lim_{n \to \infty} d\left( X^{(n)}, x \right) = 0,
    \] 
    where \(\lim_{n \to \infty} d\left( X^{(n)} , x\right) = 0 \) iff for every \(\varepsilon > 0\), \(\exists N \ge m\) s.t. \(d \left( X^{(n)}, x \right) \le \varepsilon  \) for all \(n \ge N\). 
    \begin{notation}
        We also write \(\lim_{n \to \infty} X^{(n)} = x\) in \((X, d)\).  
    \end{notation}    
\end{definition}
\begin{remark}
    Suppose \(\left( X^{(n)} \right)_{n=m}^{\infty}  \) converges to \(x\) in \((X, d)\), then \(\left( X^{(n)} \right)_{n = m_1}^{\infty}  \) also converges to \(x\) in \((X, d)\) if \(m_1 \ge m\).       
\end{remark}

\begin{eg}
    Let \(\left( X^{(n)} \right)_{n=1}^{\infty}  \) denote the sequence \(X^{(n)} = (\frac{1}{n}, \frac{1}{n})\) in \(\mathbb{R} ^2\), then what will this sequence converges to  for different metric?   
\end{eg}
\begin{explanation}
    \vphantom{text}
    \begin{itemize}
        \item If the metric is \(d_1\), then 
        \[
            d_1 \left( X^{(n)}, (0, 0) \right) = \left\vert \frac{1}{n} - 0 \right\vert + \left\vert \frac{1}{n} - 0\right\vert = \frac{2}{n},   
        \] so 
        \[
            \lim_{n \to \infty} d_1 \left( X^{(n)}, (0, 0) \right) = \lim_{n \to \infty} \frac{2}{n} = 0.  
        \]
        \item If the metric is \(d_2\), then 
        \[
            d_2 \left( X^{(d)}, (0, 0) \right) = \sqrt{\left( \frac{1}{n} - 0 \right)^2 + \left( \frac{1}{n} - 0 \right)^2 } = \frac{\sqrt{2}}{n}.   
        \]
        Hence, under \(l_2\)-metric \(\left\{ X^{(n)} \right\} \) also converges to \(0\).   
        \item If the metric is \(d_\infty \), then 
        \[
            d_\infty \left( X^{(n)}, (0, 0) \right) = \max \left\{ \left\vert \frac{1}{n} \right\vert, \left\vert \frac{1}{n} \right\vert   \right\} = \frac{1}{n},  
        \] so it also converges to \(0\). 
        \item If the metric is discrete metric, then however, it will not converges to \((0, 0)\) since 
        \[
            \lim_{n \to \infty} d_{\text{disc}} \left( X^{(n)}, (0, 0) \right) = \lim_{n \to \infty}  d_{\text{disc}} \left( \left( \frac{1}{n}, \frac{1}{n} \right) , (0, 0) \right) = 1. 
        \] 
    \end{itemize}
\end{explanation}

\begin{definition*}
    Let \(f: X \to Y\) be a function with domain \(X\) and codomain \(Y\). The range of \(f = \left\{ f(x) \mid x \in X \right\} \subseteq Y\). 
    
    \begin{definition}[injective] \label{def: injective}
        We say \(f\) is injective or one-to-one if for all \(x_1, x_2 \in X\), \(f(x_1) = f(x_2)\) implies \(x_1 = x_2\).    
    \end{definition}

    \begin{definition}[surjective] \label{def: surjective}
        We say \(f\) is surjective or onto if for every \(y \in Y\), \(\exists x \in X\) s.t. \(f(x) = y\).    
    \end{definition}

    \begin{definition}[bijective] \label{def: bijective}
        We say \(f\) is bijective if \(f\) is injective and surjective.  
    \end{definition}
\end{definition*}

\begin{corollary}
    If \(f\) is bijective, then there exists \(f^{-1}: Y \to X\) defined by \(f^{-1}(y) = x\) if \(f(x) = y\). We also have 
    \begin{align*}
        f \left( f^{-1} (y) \right) &= y \ \forall y \in Y \\
        f^{-1} \left( f(x) \right) &= x \ \forall x \in X.
    \end{align*}
\end{corollary}

\begin{eg}
    \(\lim_{n \to \infty} \frac{1}{n} = 0 \) in \((\mathbb{R} , d)\), where \(d\) is the standard metric in \(\mathbb{R} \), which is defined by 
    \[
        d(x, y) = \vert x - y \vert. 
    \]    
    But in different metric, \(\lim_{n \to \infty} \frac{1}{n} \) may not be \(0\).  
\end{eg}
\begin{explanation}
    Define \(f: [0, 1] \to [0, 1]\) defined by
\[
    f(x) = \begin{dcases}
        x, &\text{ if } 0<x<1  ;\\
        1, &\text{ if } x=0;\\
        0, &\text{ if } x=1.
    \end{dcases}
\]
\(f\) is bijective on \([0, 1]\) to \([0, 1]\)  

Define another metric \(d^1\) on \([0, 1]\) by 
\[
    d^1(x, y) = d(f(x), f(y)).
\]  
We want to show that \(d^1\) is also a metric on \([0, 1]\). 
\begin{itemize}
    \item \(d^1(x,y)=d(f(x), f(y)) = \vert f(x) - f(y) \vert \ge 0\)
    \item \(d^1(x, y) = 0\) iff \(f(x) = f(y)\) iff \(x=y\) since \(f\) is injective.  
    \item The triangle inequality is trivially true since we can just use the triangle inequality in \(d\).   
\end{itemize}  

In fact, \(\lim_{n \to \infty} \frac{1}{n} = 1 \) in \(\left( [0, 1], d^1 \right) \) since 
\[
    \lim_{n \to \infty} d^{1}\left( \frac{1}{n}, 1 \right) = \lim_{n \to \infty} d \left( \frac{1}{n}, 0 \right) = \lim_{n \to \infty} \left\vert \frac{1}{n} \right\vert = 0.      
\]  
\end{explanation}

\section{Some point set topology of metric space}
\begin{definition}[ball] \label{def: ball}
    Let \((X,d)\) be a metric space. let \(x_0 \in X\) and \(r > 0\). We define the ball \(B_{(X, d)} (x_0, r)\) in \(X\), centered at \(x_0\) and with radius \(r\) in the metric \(d\), to the set 
    \[
        B_{(X_0, d)} (X_0, Y) \coloneqq \left\{ x \in X \mid d(x_0, x) < r \right\}.
    \]
    Sometimes, we write it as \(B_X(x_0, r)\) or \(B(x_0, r)\).   
\end{definition}

\begin{eg}
    In \(\mathbb{R} ^2\), 
    \begin{align*}
        B_{(\mathbb{R} ^2, d_2)} \left( (0, 0), 1 \right) &= \left\{ (x, y) \mid d_2 ((x, y), (0, 0)) = \sqrt{x^2 + y^2}  < 1 \right\},
    \end{align*}
    and 
    \[
        B_{(\mathbb{R} ^2, d_1)}((0,0), 1) = \left\{ (x, y) \mid d_1((x, y), (0, 0)) = \vert x \vert + \vert y \vert < 1   \right\}, 
    \] and 
    \[
        B_{(\mathbb{R} ^2, d_\infty)} ((0,0), 1) = \left\{ (x, y) \mid d_\infty ((x, y), (0, 0)) = \max \left\{ \vert x \vert, \vert y \vert \right\} < 1  \right\},
    \]
    also we can consider the \(d_{\text{disc}}\) case but I am too lazy to write it down. 
\end{eg}

\begin{notation}
    Let \(E \subseteq X\), we will write 
    \[
        X \setminus E \coloneqq  \left\{ x \in X \mid x \notin E\right\}. 
    \] 
\end{notation}

\begin{definition*}
    Let \((X, d)\) be a metric space and \(E \subseteq X\). For a point \(x_0 \in X\),
    \begin{definition}[interior point] \label{def: interior point}
        \(x_0\) is an interior point of \(E\) if \(\exists r > 0\) s.t. \(B(x_0, r) \subseteq E\).     
    \end{definition}
    \begin{definition}[exterior point] \label{def: exterior point}
        \(x_0\) is an exterior point of \(E\) if \(\exists r > 0\) s.t. \(B(x_0, r) \subseteq X \setminus E\).    
    \end{definition}
    \begin{definition}[boundary point] \label{def: boundary point}
        \(x_0\) is a boundary point of \(E\) if it is neither an interior point nor an exterior point of \(E\).   
    \end{definition}
\end{definition*}

\begin{proposition}
    \(x_0\) is a boundary point of \(E\) iff for all \(r>0\), \(B(x_0, r) \cap E \neq \varnothing \) and \(B(x_0, r) \cap \left( X \setminus E \right) \neq \varnothing\).     
\end{proposition}