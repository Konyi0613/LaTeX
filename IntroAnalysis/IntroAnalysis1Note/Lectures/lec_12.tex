\lecture{12}{9 Oct. 10:20}{}
Last time, we show that the sequence \(\left\{ x_n \right\}_{n=1}^{\infty}  \) with \(x_n = n\) converges to any point \(p\) in \(\mathbb{R} \) in cofinite topology i.e. 
\[
    \mathcal{F} = \left\{ \varnothing  \right\} \cup \left\{ \mathbb{R} \setminus \left\{ \text{finite points} \right\}  \right\}
\] beacuse each non-empty neighborhood of \(p\) is very big.  
In general, a sequence in a topological space may converge to more than one points. 
\begin{definition}[Hausdorff] \label{def: Hausdorff}
    A topological space \((X, \mathcal{F} )\)  is called Hausdorff if given any two distinct points \(x, y \in X\), there exists open sets \(U, V \in \mathcal{F} \) s.t. \(x \in U\) and \(y \in V\) and \(U \cap V = \varnothing \).     
\end{definition}

\begin{eg}
    A metric space is Hausdorff since given \(x \neq y\), \(B_X \left( x, \frac{r}{2} \right) \) and \(B_X \left( y, \frac{r}{2} \right) \) are open and they seperate \(x\) and \(y\) where \(r = \frac{d(x,y) }{2}\).     
\end{eg}

\begin{theorem} \label{thm: Hausdorff implies converge to exactly one point}
    Suppose \((X, \mathcal{J} )\) is a Hausdorff topological space, then the limit of a convergence sequence is unique.  
\end{theorem}
\begin{proof}
    If \(\lim_{n \to \infty} x_n = x\) and \(\lim_{n \to \infty} x_n = y \) for some \(x \neq y\), then since \((X, \mathcal{J} )\) is Hausdorff, so there exists neighborhood \(U\) of \(x\) and \(V\) of \(y\) and \(U \cap V = \varnothing \). Also, there exists \(N_1 > 0\) s.t. \(x_n \in U\) if \(n \ge N_1\), and there exists \(N_2 > 0\) s.t. \(x_n \in V\) if \(n \ge N_2\). Hence, for all \(n \ge \max \left\{ N_1, N_2 \right\} \), we know \(x_n \in U \cap V = \varnothing \), which is a contradiction. Hence, the limit of a convergence sequence is unique.                 
\end{proof}

\begin{definition}[Compact] \label{def: compact in topological space}
    Let \((X, \mathcal{F} )\) be topological space, we say \(X\) is compact if for every open cover 
    \[
        \left\{ U_\alpha  : \alpha \in A \right\} \subseteq \mathcal{F} \text{ with } X \subseteq \bigcup_{\alpha \in A} U_\alpha,  
    \] there exists a finite subcover \(\left\{ U_{\alpha _1}, U_{\alpha _2}, \dots , U_{\alpha _n} \right\} \) s.t. 
    \[
        X \subseteq \bigcup_{i=1}^n U_{\alpha _i}.  
    \] 
\end{definition}

\begin{theorem} \label{thm: continuous map preserves compact in tp space}
    Let \(f: \left( X, \mathcal{F}  \right) \to  (Y, \mathcal{G} ) \) be a continuous map between topological spaces. If \(K \subseteq X\) is compact, then \(f(K)\) is also compact in \((Y, \mathcal{G} )\).  
\end{theorem}
\begin{proof}
    Let \(\left\{ V_\alpha  \right\}_{\alpha \in A} \) be an open cover of \(f(K)\) i.e. \( \left\{ V_\alpha  \right\}_{\alpha \in A} \subseteq \mathcal{G}\) and \(f(K) \subseteq \bigcup_{\alpha \in A} V_{\alpha }\). Since \(f\) is continuous, so \(f^{-1} \left( V_\alpha  \right) \in \mathcal{F}  \) and 
    \[
        K \subseteq \bigcup_{\alpha \in A} f^{-1} \left( V_\alpha  \right). 
    \] Now since \(K\) is compact, so there exists finite subcover, which means \(K \subseteq \bigcup_{i=1}^{n} f^{-1} \left( V_{\alpha_i} \right)  \), so \(f(K) \subseteq \bigcup_{i=1}^{n} V_{\alpha _i} \), and thus \(f(K)\) has a finite subcover of \(\left\{ V_\alpha  \right\}_{\alpha \in A} \), which means \(f(K)\) is compact.   
    \begin{remark}
        In topological space, we also have: \(f:X \to Y\) is continuous if and only if whenever \(V \subseteq Y\) is open (resp. closed), we have \(f^{-1}(V)\) is open (resp. closed) in \(X\).    
    \end{remark} 
    \begin{proof}
        \todo{DIY}
    \end{proof} 
\end{proof}

\begin{proposition}
    Let \((X, \mathcal{F} )\) be a compact topological space and \(f: X \to \mathbb{R} \) is continuous, then 
    \begin{itemize}
        \item [(1)] \(f\) is bounded on \(X\). 
        \item [(2)] If \(X \neq \varnothing \), then \(\exists x_{\text{min}}, x_{\text{max}} \in X\) s.t. \(f \left( x_{\text{max}} \right) = \max _{x \in X} f(x) \) and \(f \left( x_{\text{min}} \right) = \min _{x \in X} f(x) \).      
    \end{itemize} 
\end{proposition}
\begin{proof}
    \vphantom{text}
    \begin{itemize}
        \item [(1)] Since \(X\) is compact, so \(f(X)\) is compact in \(\mathbb{R} \) by \autoref{thm: continuous map preserves compact in tp space}, and since \(\mathbb{R} \) is a metric space, so \(f(X)\) is closed and bounded in \(\mathbb{R} \), which means \(f\) is bounded on \(X\).        
        \item [(2)] Now since \(f(X)\) is bounded, so \(\sup _{x \in X} f(x)\) and \(\inf _{x \in X} f(x)\) exists. Thus, we can pick \(\left( y_n \right) \in f(X) \) and \(\left( z_n \right) \in f(X) \) s.t. \(y_n \to \sup _{x \in X} f(x)\) and \(z_n \to \inf _{x \to X} f(x)\) by \autoref{thm: if sup exists then a sequence in S converges to supS}. Now since \(f(X)\) is closed, so  
        \[
            \sup _{x \in X} f(x) \in \overline{f(X)} = f(X),
        \] so there exists \(x^*\) s.t. \(f \left( x^* \right) = \sup _{x \in X} f(x) \) and similarly the "min" case can be proved.   
    \end{itemize}
\end{proof}

\chapter{Uniform Convergence}
In a metric space \((X, d)\), we define the convergence of a sequence \(\left\{ x^{(n)} \right\}_{n=m}^{\infty}  \), \(\lim_{n \to \infty} x^{(n)} = x \), by "Given any \(\varepsilon > 0\), \(\exists N \ge m\) s.t. \(d \left( x^{(n)}, x \right) < \varepsilon \) for all \(n \ge N\) ". Now suppose 
\[
    f^{(n)}: X \to Y \quad \forall n \in \mathbb{N},
\] where \(X, Y\) are metric spaces, then if we define the convergence of these functions at some point \(x\) to be: 
\[
    f(x) = \lim_{n \to \infty} f^{(n)}(x).
\] Do we have 
\[
    \lim_{n \to \infty} \lim_{x \to x_0} f^{(n)}(x) = \lim_{x \to x_0} \lim_{n \to \infty} f^{(n)} (x)?    
\]
Short answer: Not always true.

In this chapter, we will discuss the concept of limiting function, that is, 
\[
    \lim_{n \to \infty} f^{(n)} = f. 
\]
\section{Limiting values of functions}
\begin{definition}
    Let \((X, d_X)\) and \((Y, d_Y)\) be metric spaces, and let \(E \subseteq X\) and \(f:E \to Y\) be a function. If \(x_0 \in X\) is an adherent point of \(E\), and \(L \in Y\), we say that 
    \[
        \lim_{x \to x_0, x \in E} f(x) = L  
    \] if for every \(\varepsilon > 0\), \(\exists \delta > 0\) s.t. \(d_X (x, x_0) < \delta \) and \(x \in E\) implies \(d_Y \left( f(x), L \right) < \varepsilon  \).            
\end{definition}

\begin{remark}
    In this definition, we need not \(x_0 \in E\), we just need \(x_0 \in \overline{E} \).  
\end{remark}

\begin{remark}
    In other texboock, if 
    \[
        f(x) = \begin{dcases}
            \vert x \vert , &\text{ if }  x \neq 0;\\
            1, &\text{ if }  x = 0,
        \end{dcases}
    \] then \(\lim_{x \to 0} f(x) = 0\) because it does not consider \(x = 0\). More precisely, the definition of
    \[
        \lim_{x \to x_0, x \in E} f(x) = L
    \]
    in other textbook is "\(\forall \varepsilon > 0\), \(\exists \delta > 0\) s.t. \(d_Y \left( f(x), L \right) < \varepsilon  \) for all \(x \in E\) and \(0 < d_X (x, x_0) < \delta \)". Note that it exclude the case \(x = x_0\). 
    
    However, if \(x_0 \in E\), then by Terrence Tao's definition, \(f(x_0) = L\) if \(d_Y \left( f(x), L \right) < \varepsilon  \) for all \(\varepsilon > 0\) and for \(d_X(x, x_0) < \delta \) for the corresponding \(\delta \). Also, if \(x_0 \notin E\), then since \(x_0 \in \overline{E} \), so \(\exists x \in E\) s.t. \(d(x, x_0) < \delta \), so the definition of \(\lim_{x \to x_0, x \in E} f(x)\) is well-defined. In our notation, other textbooks' definition is like
    \[
        \lim_{x \to x_0, x \in E \setminus \left\{ x_0 \right\} } f(x) = L. 
    \]     
\end{remark}

\begin{lemma}
    If \((X, d_X)\) and \((Y, d_Y)\) are metric spaces, then \(f:X \to Y\) is continuous at \(x_0 \in X\) is in fact 
    \[
        \lim_{x \to x_0, x \in X} f(x) = f(x_0).
    \]     
\end{lemma}
\begin{proof}
    Since \(f\) is continuous at \(x_0\) means for all \(\left( x_n \right) \to x_0 \), we have \(f(x_n) \to  f(x_0)\), so this is true.   
\end{proof}

\begin{proposition} \label{prop: converge in E TFAE}
    Let \((X, d_X)\) and \((Y, d_Y)\) be metric spaces, and \(f: X \to  Y\) be a function. Let \(x_0 \in X\) be an adherent point of \(E \subseteq X\) and \(L \in Y\), then TFAE:
    \begin{itemize}
        \item [(a)] \(\lim_{x \to x_0, x \to E} f(x) = L \). 
        \item [(b)] For every sequence \(\left\{ x^{(n)} \right\}_{n=1}^{\infty}  \) in \(E\) converges to \(x_0\), the sequence \(\lim_{n \to \infty} f \left( x^{(n)} \right) = L  \) in \(Y\). 
        \item [(c)] For every open set \(V \subseteq Y\) containing \(L\), there exists an open set \(U \subseteq X\) containing \(x_0\) s.t. \(U \cap E \subseteq f^{-1}(V)\). 
        \item [(d)] If one define \(g: E \cup \left\{ x_0 \right\} \to  Y \) by 
        \[
            g(x) = \begin{dcases}
                f(x), &\text{ if } x \in E \setminus \left\{ x_0 \right\} ;\\
                L, &\text{ if } x = x_0, 
            \end{dcases}
        \] then \(g\) is continuous at \(x_0\) on \(E\).           
    \end{itemize}      
\end{proposition}
\begin{proof}[proof from (a) to (b)]
    We know for all \(\varepsilon > 0\), there exists \(\delta > 0\) s.t. \(x \in E\) and \(d_X(x, x_0) < \delta \) implies \(d_Y (f(x), L) < \varepsilon \). Also, if we have a sequence \(\left\{ x^{(n)} \right\}_{n=1}^{\infty} \subseteq E \) converges to \(x_0\), then there exists \(N > 0\) s.t. \(n \ge N\) implies \(d_X \left( x^{(n)}, x_0 \right) < \delta \), so for all \(n \ge N\), we know \(d_Y \left( f \left( x^{(n)}\right), L  \right) < \varepsilon  \).           
\end{proof}
\begin{proof}[proof from (b) to (c)]
    Suppose by contradiction, \(V \subseteq Y\) is an open neighborhood of \(L\), and there does not exist an open neighborhood \(U \subseteq X\)  of \(x_0\) has
    \[
        U \cap E \subseteq f^{-1}(V),
    \] then for all \(n \in \mathbb{N} \), we know \(\exists y_n \in B_X \left( x_0, \frac{1}{n} \right) \cap E \) and \(y_n \notin f^{-1}(V)\), and note that \(\left( y_n \right)_{n=1}^{\infty}  \) is a sequence converges to \(x_0\) in \(E\), so by (b) we know \(\left( f \left( y_n \right)  \right)_{n=1}^{\infty}  \) must converges to \(L\), which means for the neighborhood \(V\) of \(L\), there exists \(N > 0\) s.t. \(n \ge N\) implies \(f(y_n) \in V\).            
\end{proof}
\begin{proof}[proof from (c) to (d)]
    \todo{DIY}
\end{proof}
\begin{proof}[proof from (d) to (a)]
    \todo{DIY}
\end{proof}

\section{Pointwise and Uniform Convergence}
\begin{definition}[Pointwise convergence] \label{def: Pointwise convergence}
    Let \(\left( f^{(n)} \right)_{n=1}^{\infty}  \) be a sequence of functions from \((X, d_X)\) to \((Y, d_Y)\), and let \(f: X \to Y\) be another function. We say that \(f^{(n)}\) converges pointwise to \(f\) on \(X\) if for every \(x \in X\) and every \(\varepsilon > 0\), \(\exists N_x > 0\) s.t.
    \[
        d_Y \left( f^{(n)}(x), f(x) \right) < \varepsilon \quad \forall n \ge N_x .
    \]        
\end{definition}

\begin{definition}[Uniformly convergence] \label{def: uniformly convergence}
    We say \(f^{(n)}\) converges uniformly to \(f\) on \(X\) if for every \(\varepsilon > 0\), \(\exists N > 0\) s.t. 
    \[
        d_Y \left( f^{(n)}(x), f(x) \right) < \varepsilon  
    \] for all \(n \ge N\) and all \(x \in X\). (\(N\) is independent of \(x\))    
\end{definition}

\begin{eg}
    Suppose \(f_n(x) = \frac{x}{n}\), then \(f_n \to 0\) pointwise (\(0\) is the zero function here). However, \(\left\{ f^{(n)} \right\} \) is not uniformly convergent since given \(\varepsilon = 1\), if \(\exists N > 0\) s.t. 
    \[
        \left\vert f_n(x) - 0 \right\vert < 1 
    \] for all \(n \ge N\) and \(x \in \mathbb{R} \), then 
    \[
        \left\vert \frac{x}{n} \right\vert < 1 
    \] for all \(n \ge N\) and \(x \in X\), but if we pick \(n = N\) and \(x = 2N\), then it gives a contradiction.   
\end{eg}

\begin{eg}
    \(f_n(x) = x^n\) on \([0, 1]\), then we know
    \[
        \lim_{n \to \infty} f_n(x) = \begin{dcases}
            0, &\text{ if } 0 \le x < 1 ;\\
            1, &\text{ if } x = 1,
        \end{dcases} 
    \]  so \(f_n\) continuous and pointwise convergent but not uniformly convergent.  
\end{eg}

\begin{eg}
    Suppose \(f_n(x) = \frac{x}{n}\) on \([0, 1]\), then \(\lim_{n \to \infty} f_n(x) = 0 \) uniformly.    
\end{eg}
\begin{proof}
    \todo{DIY}
\end{proof}

\begin{eg}
    Consider 
    \[
        f_n(x) = \begin{dcases}
            2n, &\text{ if }  x \in \left[ \frac{1}{2n}, \frac{1}{n} \right] ;\\
            0, &\text{ if } x \in \mathbb{R} \setminus \left[ \frac{1}{2n}, \frac{1}{n} \right],
        \end{dcases}
    \] then \(\lim_{n \to \infty} f_n(x) = 0 \) pointwisely, but if we integrate on both side, then 
    \[
        \int _0^1 f^{(n)}(x) \, \mathrm{d} x = 2n \left( \frac{1}{n} - \frac{1}{2n} \right) = 2n \cdot \frac{1}{2n} = 1,  
    \] but \(\int _0^1 0 \, \mathrm{d} x = 0 \), so we know pointwise convergence will not implies they will be equal after integration. 
    \begin{remark}
        We will learn that uniform convergence can ensure integration after taking limit takes same value of taking limit after integration.
    \end{remark} 
\end{eg}