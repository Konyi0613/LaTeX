\lecture{20}{13 Nov. 10:20}{}
We explain that how \(a^x\) is defined for \(a > 0\) and \(x \in \mathbb{R} \). If \(a = 1\), then we define \(1^x = 1\), and if \(a < 1\), then we define \(a^x \coloneqq \left( \frac{1}{a} \right)^{-x} \). So we just need to define \(a^x\) for \(a > 1\). 
\begin{itemize}
    \item \textbf{Step 1:} Rational exponenets. For integers \(m\) and positive integer \(n\), we can define 
    \[
        a^{\frac{m}{n}} \coloneqq \left( a^{\frac{1}{n}} \right)^{m}, 
    \] where \(a^{\frac{1}{n}}\) is the positive number such that \(\left( a^{\frac{1}{n}} \right)^{n} = a \), i.e. the power root of \(x^n = a\). 
    
    Besides, we define \(a^{r + s} = a^r a^s\) for \(r, s \in \mathbb{Q} \). 
    
    Note that if \(a > 1\), then \(a^{\frac{1}{n}} > 1\) for \(n \in \mathbb{N} \). Otherwise \(0 < a^{\frac{1}{n}} \le 1\) gives \(\left( a^{\frac{1}{n}} \right)^n \le 1 \), i.e. \(a \le 1\). Similarly, if \(0 < a < 1\), then \(a^{\frac{1}{n}} < 1\).
    
    Also, we can deduce that \(a^0 = 1\) for all \(a > 0\), and \(r > s\) implies \(a^r > a^s\) for \(a > 0\). Hence, we can well define \(a^x\) for \(x \in \mathbb{Q} \). 

    \item \textbf{Step 2:} Continuity near \(0\) for rational exponent. We first claim that if \(a > 0\) and \(b_n \in \mathbb{Q} \), then \(\lim_{n \to \infty} b_n = 0 \) implies \(\lim_{n \to \infty} a^{b_n} = 1 \). If \(a = 1\), then this is true. If \(0 < a < 1\), then 
    \[
        \lim_{n \to \infty} a^{b_n} = \lim_{n \to \infty} \left( \frac{1}{a} \right)^{-b_n},   
    \] and we know \(\frac{1}{a} > 1\) and \(\lim_{n \to \infty} -b_n = 0\), so it sufficies to prove the case \(a > 1\). First, we prove that \(\lim_{n \to \infty} a^{\frac{1}{n}} = 1 \) for all \(a > 1\). Fix \(\varepsilon > 0\), then \((1 + \varepsilon )^n \ge 1 + n \varepsilon  \to \infty \) as \(n \to \infty \). Thus, there exists \(N > 0\) s.t. 
    \[
        (1 + \varepsilon )^N \ge 1 + N \varepsilon \ge a \iff a^{\frac{1}{N}} \le 1 + \varepsilon,
    \]  
    so \(n \ge N\) implies 
    \[
        0 < a^{\frac{1}{n}} \le a^{\frac{1}{N}} \le 1 + \varepsilon \iff \left\vert a^{\frac{1}{n}} - 1 \right\vert < \varepsilon \quad \text{for } n \ge N.  
    \] 
    Thus, \(\lim_{n \to \infty} a^{\frac{1}{n}} = 1 \). Now for some \(\left( b_n \right) \) s.t. \(\lim_{n \to \infty} b_n = 0 \), we know there exists \(L\) s.t. \(n \ge L\) implies \(\left\vert b_n \right\vert < \frac{1}{N} \). Recall that \(a^{\frac{1}{N}} < 1 + \varepsilon \), so we have \(\frac{1}{1+\varepsilon } < a^{-\frac{1}{N}}\), then since
    \[
        a^{- \vert b_n \vert } \le a^{b_n} \le a^{\vert b_n \vert } \quad \text{when } a > 1 
    \] and thus 
    \[
      a^{-\frac{1}{N}} \le a^{b_n} \le a^{\frac{1}{N}} \quad \text{when } n \ge L.   
    \]
    Thus, 
    \[
        \frac{1}{1+\varepsilon } < a^{b_n} < 1 + \varepsilon ,
    \] which gives 
    \[
        -\varepsilon < \frac{-\varepsilon }{1+\varepsilon } = \frac{1}{1+\varepsilon } - 1 < a^{b_n} - 1 < \varepsilon,
    \] so \(\left\vert a^{b_n} - 1 \right\vert < \varepsilon  \) for \(n \ge L\), and thus \(\lim_{n \to \infty} a^{b_n} = 1 \). This proves the continuity of rational exponent near \(0\).    
    \item \textbf{Step 3:} Continuity on rational exponents. We claim that for \(a > 0\). The function \(f: \mathbb{Q} \to \mathbb{R} \) defined by \(f(r) = a^r\) is continuous on \(\mathbb{Q} \). Fix \(r_0 \in \mathbb{Q} \) and \(\left( r_n \right) \subseteq \mathbb{Q}  \) s.t. \(\lim_{n \to \infty} r_n = r_0 \), then 
    \[
        a^{r_n} - a^{r_0} = a^{r_0} \left( a^{r_n - r_0} - 1 \right). 
    \]
    Note that \(a^{r_0}\) is fixed, and \(\lim_{n \to \infty} r_n - r_0  = 0\), so we know 
    \[
        \lim_{n \to \infty} a^{r_n - r_0} = 1, 
    \] which shows \(\lim_{n \to \infty} a^{r_n} - a^{r_0} = 0 \) and thus \(\lim_{n \to \infty} a^{r_n} = a^{r_0} \). 
    \item \textbf{Step 4:} Real exponenets. Let \(\left( r_n \right) \subseteq \mathbb{Q}  \) with \(\lim_{n \to \infty} r_n = x \). We'll show that \(\left( a^{r_n} \right) \) is Cauchy in \(\mathbb{R} \). We just need to prove the case \(a > 1\). Since \(\lim_{n \to \infty} r_n = x \), so there exists \(r \in \mathbb{Q} _{>0}\) s.t. \(r_n \le r\) for all \(n\). Since \(\lim_{n \to \infty} a^{\frac{1}{n}} = 1 \), and \(\frac{\varepsilon}{a^r} > 0\), so we know there exists \(N > 0\) s.t. 
    \[
        \left\vert a^{\frac{1}{N}} - 1 \right\vert < \frac{\varepsilon }{a^r}. 
    \]
    Now since \(r_n \to x\), so there exists \(L\) s.t. \(m, n \ge L\) implies \(\left\vert r_n - r_m \right\vert < \frac{1}{N} \), so 
    \[
        \left\vert a^{r_n} - a^{r_m} \right\vert = \left\vert a^{r_m} \right\vert \left\vert a^{r_n - r_m} - 1 \right\vert \le a^r \left\vert a^{\frac{1}{N}} - 1 \right\vert < \varepsilon.
    \] Hence, \(\left\{ a^{r_n} \right\} \) is Cauchy. Now we can define
    \[
        a^x \coloneqq \lim_{n \to \infty} a^{r_n} 
    \] for \(x \in \mathbb{R} \). Also, this is independent of the choice of \(r_n \to x\). Since if we have \(\lim_{n \to \infty} r_n = \lim_{n \to \infty} s_n = x  \), where \(\left( r_n \right). \left( s_n \right) \in \mathbb{Q}   \), then \(\lim_{n \to \infty} a^{r_n} = \lim_{n \to \infty} a^{s_0} \). This can be shown by we know \(\left( r_n - s_n \right) \to 0 \) and \(\left( r_n - s_n \right) \subseteq \mathbb{Q}\), so \(\lim_{n \to \infty} a^{r_n - s_n} = 1 \), and since \(\lim_{n \to \infty} a^{r_n} \) and \(\lim_{n \to \infty} a^{s_n} \) both exists, so 
    \[
         \lim_{n \to \infty} a^{s_n} = \left( \lim_{n \to \infty} a^{r_n - s_n}  \right) \left( \lim_{n \to \infty} a^{s_n}  \right) = \lim_{n \to \infty} a^{r_n}.   
    \]         
    \item \textbf{Conclusion:} We can show that exponential function \(a^x\) is strictly increasing on \(\mathbb{R} \) when \(a > 1\) since \(a^{r+s} = a^r a^s\) and \(\left( a^r \right)^s = a^{rs} \).   
\end{itemize}  



Now since we know \(\exp (x) = 1 + \frac{1}{1!} + \dots > 2 \), so \(\exp (x) = e^x\) is strictly increasing on \(\mathbb{R} \), so the inverse function exists.   

\begin{definition}
    We define \(\log : (0, \infty ) \to \mathbb{R} \) (also denoted by \(\ln \)) to be the inverse function of exponential function. Thus, 
    \[
        \exp \left( \log (x) \right) = x, \text{ for }  0 < x < \infty, \quad  \log \left( \exp (x) \right) = x  \text{ for } x \in \mathbb{R} . 
    \] 
\end{definition}

Recall the alternating series test:
\begin{theorem}[Alternating series test]
    Let \(\left( a_n \right) \) be a sequence in \(\mathbb{R} \) with \(a_n \ge 0\) and \(a_{n+1} \le a_n\) for \(n \ge m\) for some \(m\). Then, 
    \[
        \sum_{n=m}^{\infty} (-1)^n a_n 
    \] converges iff \(\lim_{n \to \infty} a_n = 0 \).      
\end{theorem}

\begin{theorem}[Logarithm properties] \label{thm: logarithm properties}
    The logarithm function satisfies the following properties:
    \begin{itemize}
        \item [(a)] For \(x \in (0, \infty )\), we have \(\frac{\mathrm{d}}{\mathrm{d}x} \ln x = \frac{1}{x} \). Hence, \(\int _a^b \frac{1}{x} \, \mathrm{d} x = \ln b - \ln a \) when \([a, b] \subseteq (0, \infty )\). 
        \item [(b)] For \(x, y \in (0, \infty )\), we have \(\ln (xy) = \ln x + \ln y\).     
        \item [(c)] \(\ln 1 = 0\), \(\ln \left( \frac{1}{x} \right) = -\ln x \) for \(x \in (0, \infty )\).
        \item [(d)] For \(x \in (0, \infty )\) and \(y \in \mathbb{R} \), we have \(\ln \left( x^y \right) = y \ln (x) \).    
        \item [(e)] For \(x \in (-1, 1)\), 
        \[
            \ln (1-x) = -\sum_{n=1}^n \frac{x^n}{n}, 
        \] while for \(x \in (0, 2)\), 
        \[
            \ln x = \sum_{n=1}^{\infty} \frac{(-1)^{n+1} (x-1)^n}{n}. 
        \] 
    \end{itemize}
\end{theorem}

\begin{proof}[proof of (a)]
    Recall \(\exp ^{\prime} (x) = \exp (x)\) and \(\exp (\ln (x)) = x\). Thus, by chain rule, 
    \[
        1 = (x)^{\prime} = \left[ \exp (\ln (x)) \right]^{\prime} = \exp ^{\prime} \left( \ln (x) \right) \cdot \left( \ln x \right) ^{\prime},  
    \] which gives 
    \[
        \frac{1}{x} = \frac{1}{\exp (\ln (x))} = \left( \ln x \right)^{\prime}.  
    \]
\end{proof}

\begin{proof}[proof of (b)]
    \[
       \exp \left( \ln (xy) \right) = xy = \exp \left( \ln (x) \right) \exp \left( \ln (y) \right) = \exp \left( \ln x + \ln y \right).   
    \]
    Since \(\exp \) is injective (since strictly increasing), so \(\ln (xy) = \ln (x) + \ln (y)\).
\end{proof}

\begin{proof}[proof of (c)]
    Since \(\exp (\ln (1)) = 1\) and \(\exp (0) = 1\) and \(\exp \) is injective, so \(\ln (1) = 0\). Besides, 
    \[
        0 = \ln 1 = \ln \left( x \cdot \frac{1}{x} \right) = \ln x + \ln \left( \frac{1}{x} \right)  
    \] for all \(x \in (0, \infty )\), so \(\ln \left( \frac{1}{x} \right) = - \ln x \).  
\end{proof}

\begin{proof}[proof of (d)]
    Note that 
    \begin{align*}
        &\exp \left( y \ln (x) \right) = e^{y \ln x} = \left( e^{\ln x} \right)^y = x^y. \\
        & \exp \left( \ln \left( x^y \right)  \right) = x^y, 
    \end{align*}
    so we know \(y \ln x = \ln \left( x^y \right) \). 
\end{proof}

\begin{proof}[proof of (e)]
    Note that 
    \[
        \frac{\mathrm{d}}{\mathrm{d}x} \ln (1-x) = -\frac{1}{1-x} = (-1) \sum_{n=0}^{\infty} x^n \quad \text{when } \vert x \vert < 1.   
    \]
    Hence, if \(t \in (-1, 1)\), 
    \[
        \int _0^t \left( \frac{\mathrm{d}}{\mathrm{d}x} \ln (1-x)  \right) \, \mathrm{d} x = \int _0^t (-1) \sum_{n=0}^{\infty} x^n \, \mathrm{d}x,     
    \] 
    so we have 
    \[
        \ln (1-x) \left.  \right]_0^t = (-1) \sum_{n=0}^{\infty} \frac{t^{n+1}}{n+1},  
    \] which gives 
    \[
        \ln (1-t) = (-1) \sum_{n=0}^{\infty} \frac{t^{n+1}}{n+1} = (-1) \sum_{n=1}^{\infty} \frac{t^n}{n} \quad \text{when } \vert t \vert < 1.    
    \]
    Now replace \(t\) by \(1-x\), then 
    \[
        \ln \left( 1 - (1-x) \right) = - \sum_{n=1}^{\infty} \frac{(1-x)^n}{n} = \sum_{n=1}^{\infty} \frac{(-1)^{n+1}(x-1)^n}{n}  
    \] when \(\left\vert 1-x \right\vert < 1 \), i.e. \(x \in (0,2)\).  
\end{proof}

\begin{corollary}
\[
    \lim_{x \to 2^-} \ln x = \sum_{n=1}^{\infty} \frac{(-1)^{n+1}}{n}.  
\] 
\end{corollary}
\begin{proof}
    When \(x = 2\), we know the power series of \(\ln 2\) is 
    \[
        \sum_{n=1}^{\infty} \frac{(-1)^{n+1} (x-1)^n}{n} = \sum_{n=1}^{\infty} \frac{(-1)^{n+1}}{n} = (-1) \cdot \sum_{n=1}^{\infty} \frac{(-1)^n}{n},  
    \] and since \(\frac{1}{n+1} < \frac{1}{n}\) and \(\lim_{n \to \infty} \frac{1}{n} = 0 \), so by alternating series test \(\sum_{n=1}^{\infty} \frac{(-1)^{n+1}}{n} \) converges, and thus by Abel's theorem we have
    \[
        \lim_{x \to 2^-} \ln x = \sum_{n=1}^{\infty} \frac{(-1)^{n+1}}{n}. 
    \]   
\end{proof}