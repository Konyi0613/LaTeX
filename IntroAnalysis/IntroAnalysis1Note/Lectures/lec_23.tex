\lecture{23}{25 Nov. 09:10}{}
On \(C(\mathbb{R} / \mathbb{Z} , \mathbb{C} )\), we can use \(L^2\) norm to define a metric on \(C(\mathbb{R} / \mathbb{Z} , \mathbb{C} )\): 
\[
    d_2(f, g) = \lVert f - g \rVert_2, 
\]  
but this metric is not complete. See \autoref{sec: L2 norm metric}. 

\section{Trigonometric polynomials}

In the preceding section we introduced an inner product on the space \(C(\mathbb{R} /\mathbb{Z} , \mathbb{C} )\) 
of continuous \(1\)-periodic complex-valued functions. We now use this inner
product to develop the theory of trigonometric polynomials, which play for
Fourier series the same role that ordinary polynomials play for power series.
The basic building blocks will be the complex exponentials \(e^{2 \pi i n x}\), also called
characters.

\begin{definition}
    For each \(n \in \mathbb{Z} \), define the function \(e_n \in C(\mathbb{R} / \mathbb{Z} , \mathbb{C} )\) by 
    \[
        e_n(x) = e^{2\pi i n x},
    \]  then we know \(e_n(x + 1) = e_n(x)\) for all \(x \in \mathbb{R} \). Sometimes we call \(e_n\) the character of frequency \(n\).    
\end{definition}

\begin{definition}
    A function \(f \in C(\mathbb{R} / \mathbb{Z} , \mathbb{C} )\) is called trigonometric polynomial if it can be written in the form 
    \[
        f(x) = \sum_{n = -N}^N c_n e_n(x) 
    \] 
    for some integer \(N \ge 0\). 
\end{definition}

\begin{remark}
\begin{align*}
    \cos (2 \pi n x) &= \frac{e^{2 \pi i n x} + e^{-2 \pi i n x}}{2} = \frac{e_n(x) + e_{-n}(x)}{2} \\
    \sin (2 \pi n x) &= \frac{e^{2 \pi i n x} - e^{-2 \pi i n x}}{2i} = \frac{e_n(x) - e_{-n(x)}}{2i},
\end{align*}
so the (complex) linear combination of \(\cos (2\pi nx)\) and \(\sin (2\pi nx)\) is also a trigonometric polynomial. 
\end{remark}

\begin{lemma}
    Let \(n, m \in \mathbb{Z} \), then 
    \[
        \langle e_n, e_m \rangle = \delta _{nm} = \begin{dcases}
            1, &\text{ if }  n = m;\\
            0, &\text{ if }  n \neq m.
        \end{dcases}
    \] 
\end{lemma}
\begin{proof}
    Since 
    \begin{align*}
        \langle e_n, e_m \rangle &= \int _0^1 e_n(x) \overline{e_m(x)} \, \mathrm{d} x = \int _0^1 e^{2 \pi i n x} \overline{e^{2 \pi i m x}} \, \mathrm{d} x = \int _0^1 e^{2 \pi i n x} e^{-2 \pi i m x} \, \mathrm{d} x \\
        &= \int _0^1 e^{2 \pi i x (n - m)} \, \mathrm{d} x,       
    \end{align*}
    and then 
    \begin{itemize}
        \item Case 1: \(n = m\), then 
        \[
            \langle e_n, e_m \rangle = \int _0^1 e^0 \, \mathrm{d} x = \int _0^1 1 \, \mathrm{d} x= 1.   
        \]
        \item Case 2: \(n \neq m\), then 
        \[
            \langle e_n, e_m \rangle = \frac{e^{2 \pi i x(n-m)}}{2 \pi i (n - m)} \left. \right]_0^1 = \frac{1 - 1}{2 \pi i (n - m)} = 0 
        \]
        since for \(n - m \in \mathbb{Z} \) we have \(e^{2 \pi i (n - m)} = 1\).  
    \end{itemize}
\end{proof}

\begin{corollary}
    Let \(f = \sum_{n=-N}^N c_n e_n \), then \(c_n = \langle f, e_n \rangle \) and 
    \[
        \lVert f \rVert_2^2 = \sum_{n = -N}^N \vert c_n \vert^2   
    \]  
\end{corollary}
\begin{proof}
    Since 
    \[
        \langle f, e_n \rangle = \langle \sum_{k=-N}^N c_k e_k, e_n  \rangle = \sum_{k=-N}^{N} c_k \langle e_k, e_n \rangle = c_n \langle e_n, e_n \rangle = c_n,     
    \]
    so 
    \begin{align*}
        \lVert f \rVert_2^2 &= \langle f, f \rangle = \langle \sum_{k=-N}^N c_k e_k, \sum_{\ell = -N}^N c_{\ell } e_{\ell }   \rangle \\
        &= \sum_{\ell = -N}^N \sum_{k = -N}^N c_k \overline{c_{\ell }} \langle e_k, e_{\ell }  \rangle = \sum_{\ell = -N}^N \sum_{k = -N}^N c_k \overline{c_{\ell } } \delta _{k \ell } = \sum_{\ell = -N}^N c_{\ell } \overline{c_{\ell }} = \sum_{\ell = -N}^{N} \vert c_{\ell } \vert^2.           
    \end{align*}
\end{proof}

Later, we'll show that if \(f \in C(\mathbb{R} / \mathbb{Z} , \mathbb{C} )\), then 
\[
    \lVert f \rVert_2^2 = \sum_{n \in \mathbb{Z} } \vert \langle f, e_n \rangle  \vert^2.   
\] 

\section{Periodic convolutions}
We want to prove the following theorem:
\begin{theorem}
    Let \(f \in C(\mathbb{R} / \mathbb{Z} , \mathbb{C} )\), and let \(\varepsilon > 0\), then \(\exists \) a trigonometric polynomial \(P\) s.t. 
    \[
        \lVert f - P \rVert_{\infty } < \varepsilon. 
    \]    
\end{theorem}

\begin{definition}
    Let \(f, g \in C(\mathbb{R} / \mathbb{Z} , \mathbb{C} )\), then we define their periodic convolution \(f * g: \mathbb{R} \to \mathbb{C} \) by
    \[
        (f * g)(x) = \int _0^1 f(y) g(x - y) \, \mathrm{d} y. 
    \]  
\end{definition}

We first show the below two propositions, which will be useful later.
\begin{proposition} \label{prop: 1-periodic and continuous implies bounded and uniformly continuous}
    For \(F \in C(\mathbb{R} / \mathbb{Z} , \mathbb{C} )\). Viewing \(F\) a \(1\)-periodic continuous function, then \(F\) is bounded and uniformly continuous.
\end{proposition}
\begin{proof}
    We have shown boundedness in \autoref{lm: properties for C(R/Z, C)}. Now we show \(F\) is uniformly continuous. 
    \begin{itemize}
        \item Step 1: Uniform continuity on \([0, 2]\). Since \(F\) is continuous on the compact interval \([0, 2]\), it is uniformly continuous there. Thus, for every \(\varepsilon > 0\) there exists \(\delta _0 > 0\) s.t. 
        \begin{equation} \label{eq: uniform continuity on [0, 2]}
            u, v \in [0, 2], \ \vert u - v \vert < \delta _0 \implies \left\vert F(u) - F(v) \right\vert < \varepsilon. 
        \end{equation}
        Now we can define \(\delta = \min \left\{ \delta _0, \frac{1}{2} \right\} \), then \autoref{eq: uniform continuity on [0, 2]} remains valid with \(\delta \) in place of \(\delta _0\):
        \[
            u, v \in [0, 2], \ \vert u - v \vert < \delta \implies \left\vert F(u) - F(v) \right\vert < \varepsilon. 
        \]   
        \item Step 2: Extend uniform continuity to all of \(\mathbb{R} \). Let \(\varepsilon > 0\) to be fixed and choose \(\delta\) as above. Take any \(x, y \in \mathbb{R} \) s.t. \(\vert x - y \vert < \delta  \). We want to show that 
        \[
            \vert F(x) - F(y) \vert < \varepsilon. 
        \]
        \begin{claim}
            There exists an integer \(n\) s.t. 
            \[
                x, y \in [n, n+2].
            \]
        \end{claim}
        \begin{explanation}
            WLOG, assume \(x \le y\), then choose \(n \in \mathbb{Z} \) s.t. \(n \le x < n + 1\). Since \(\vert x - y \vert = y - x < \frac{1}{2} \), 
            \[
                y < x + \frac{1}{2} < (n + 1) + \frac{1}{2} = n + \frac{3}{2} < n + 2.
            \]
            Thus, \(x, y \in [n, n+2]\).     
        \end{explanation}
        Define \(u \coloneqq x - n\) and \(v \coloneqq y - n\), then \(u, v \in [0, 2]\) and 
        \[
            \vert u - v \vert = \vert x - y \vert < \delta.   
        \] 
        Hence, we know \(\vert F(u) - F(v) \vert < \varepsilon \). Now by the \(1\)-periodicity of \(F\), we know 
        \[
            F(x) = F(u), \quad F(y) = F(v),
        \]   
        so 
        \[
            \vert F(x) - F(y) \vert = \vert F(u) - F(v) \vert < \varepsilon.  
        \]
    \end{itemize}  
\end{proof}

\begin{proposition} \label{prop: 1-periodic and integrable implies int on any length-1 intevral gives same value}
    Let \(F : \mathbb{R} \to \mathbb{C} \) be a \(1\)-periodic integrable function, then for any \(a \in \mathbb{R} \) we have 
    \[
        \int _a^{a+1} F(x) \, \mathrm{d} x=\int _0^1 F(x) \, \mathrm{d} x \text{ for any }  a \in \mathbb{R} .
    \]
\end{proposition}
\begin{proof}
    Fix \(a \in \mathbb{R} \). Choose \(n \in \mathbb{Z} \) s.t. \(n \le a < n+1\) and set \(r \coloneqq a - n \in [0, 1)\). Then 
    \[
        \int _a^{a+1} F(x) \, \mathrm{d} x = \int _{n+r}^{n+1+r} F(x) \, \mathrm{d} x.  
    \]
    Make the change of variables \(x = t + n\), so \(\mathrm{d} x = \mathrm{d} t  \) and 
    \[
        \int _{n+r}^{n+1+r} F(x) \, \mathrm{d} x = \int _r^{r+1} F(t + n) \, \mathrm{d} t.  
    \]     
    Now since \(F(t + n) = F(t)\) for all \(t\), so 
    \[
        \int _r^{r+1} F(t + n) \, \mathrm{d} t = \int _r^{r+1} F(t) \, \mathrm{d} t.   
    \]  
    We now show that 
    \[
        \int _r^{r+1} F(t) \, \mathrm{d} t = \int _0^1 F(t) \, \mathrm{d} t \quad (0 \le r < 1).  
    \]
    Note that 
    \[
        \int _r^{r+1} F(t) \, \mathrm{d} t = \int _r^1 F(t) \, \mathrm{d} t + \int _1^{r+1} F(t) \, \mathrm{d} t,   
    \]
    and in the second term, substitute \(t = s + 1\), so 
    \[
        \int _1^{r+1} F(t) \, \mathrm{d} t = \int _0^r F(s + 1) \, \mathrm{d} s = \int _0^r F(s) \, \mathrm{d} s.   
    \] 
    Hence, 
    \[
        \int _r^{r+1} F(t) \, \mathrm{d} t = \int _r^1 F(t) \, \mathrm{d} t + \int _0^r F(t) \, \mathrm{d} t = \int _0^1 F(t) \, \mathrm{d} t.    
    \]
    Thus, we are done.
\end{proof}

We now list the key algebraic properties of periodic convolution. They resemble the properties of usual convolution, except that periodicity must be checked carefully.


\begin{lemma} \label{lm: properties for periodic convolution}
    Let \(f, g, h \in C(\mathbb{R} / \mathbb{Z} , \mathbb{C} )\) and \(c \in \mathbb{C} \), then 
    \begin{itemize}
        \item (Closure) \(f * g \in C(\mathbb{R} / \mathbb{Z} , \mathbb{C} )\). 
        \item (Commutativity) \(f * g = g * f\). 
        \item (Bilinearity) \((f + g) * h = f * h + g * h\) and \(f * (g + h) = f * g + f * h\) and 
        \[
            (cf) * g = f * (cg) = c(f * g).
        \]
    \end{itemize} 
\end{lemma}
\begin{proof}[proof of (a)]
 Now consider
\[
    (f * g)(x) = \int _0^1 f(y) g(x - y) \, \mathrm{d} y, 
\]  
then since \(f\) is bounded and \(g\) is uniformly continuous, so given \(\varepsilon > 0\) there exists \(M > 0\) s.t. \(\vert f(y) \vert \le M \) and \(\delta > 0\) s.t. 
\[
    \vert s - t \vert < \delta \implies \vert g(s) - g(t) \vert < \frac{\varepsilon}{M}.  
\]      
Also, we have 
\[
    (f * g)(x) - (f * g)(x_0) = \int _0^1 f(y) \left( g(x-y) - g(x_0 - y) \right) \, \mathrm{d} y,  
\]
so if \(\vert x - x_0 \vert < \delta  \), then 
\[
    \vert (x-y) - (x_0 - y) \vert = \vert x - x_0 \vert < \delta,  
\] which gives 
\[
    \vert g(x - y) - g(x_0 - y) \vert < \frac{\varepsilon}{M}. 
\]
Thus, 
\[
    \left\vert (f * g)(x) - (f * g)(x_0) \right\vert \le \int _0^1 \vert f(y) \vert \vert g(x-y) - g(x_0 - y) \vert \, \mathrm{d} y \le \int _0^1 M \cdot \frac{\varepsilon}{M} \, \mathrm{d} y = \varepsilon.     
\]
Thus, \(f * g\) is continuous. Also, since \(g(x + 1 - y) = g(x - y)\), so 
\[
    (f * g)(x + 1) = \int _0^1 f(y) g(x + 1 - y) \, \mathrm{d} y = \int _0^1 f(y) g(x - y) \, \mathrm{d} y = (f * g)(x).  
\]
\end{proof}

\begin{proof}[proof of (b)]
    Since 
\begin{align*}
    (f * g)(x) &= \int _0^1 f(y) g(x - y) \, \mathrm{d} y = -\int _x^{x-1} f(x - z) g(z) \, (\mathrm{d} z) = \int _{x-1}^x f(x-z) g(z) \, \mathrm{d} z \\
    &= \int _0^1 f(x-z) g(z) \, \mathrm{d} z = \int _0^1 g(z) f(x-z) \, \mathrm{d} z = (g * f)(x),     
\end{align*}
so we're done.
\end{proof}
\begin{proof}[proof of (c)]
    \todo{DIY}
\end{proof}
 



\begin{theorem}
    Recall that \(e_n(x) = e^{2 \pi i n x}\), then for any \(f \in C(\mathbb{R} / \mathbb{Z} , \mathbb{C} )\), we have 
    \[
        f * e_n = \hat{f}(n) e_n
    \]
    where
    \[
        \hat{f} (n) = \int _0^1 f(y) e^{-2 \pi i n y} \, \mathrm{d} y = \langle f, e_n \rangle.
    \]
\end{theorem}
\begin{proof}
    \begin{align*}
        (f * e_n)(x) &= \int _0^1 f(y) e_n(x-y) \, \mathrm{d} y = \int _0^1 f(y) e^{2\pi in (x-y)} \, \mathrm{d} y \\
        &= \int _0^1 f(y) e^{2 \pi i n x} \cdot e^{-2 \pi i n y} \, \mathrm{d} y = \left( \int _0^1 f(y) e^{-2 \pi i n y} \, \mathrm{d} y  \right) e^{2 \pi i n x} = \hat{f} (n) e_n(x).    
    \end{align*}
\end{proof}

\begin{theorem}
    Let \(P = \sum_{n=-N}^N c_n e_n \), then 
    \[
        (f * P)(x) = \sum_{n=-N}^N c_n \hat{f} (n) e_n. 
    \] 
\end{theorem}
\begin{proof}
    \begin{align*}
        (f * P)(x) &= \left( f * \left( \sum_{n=-N}^N c_n e_n  \right)  \right)(x) = \sum_{n=-N}^N c_n (f * e_n)(x) \\
        &= \sum_{n=-N}^N c_n \hat{f} (n) e_n.   
    \end{align*}
\end{proof}