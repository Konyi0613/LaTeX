\chapter{Formal Power Series}
\lecture{16}{30 Oct. 10:20}{}
\section{Review of series}
\begin{definition}
    Let \((a_n)_{n=1}^{\infty} \) be a sequence of real numbers,
    \begin{itemize}
        \item [(a)] The limit superior or (\(\limsup\)) of a sequence \((a_n)\) is defined by 
        \[\limsup_{n \to \infty} a_n = \lim_{n \to \infty} \sup _{k \ge n} a_k.  \]
        Let \(S_n = \sup _{k \ge n} a_k\). Note that the index set \(\left\{ k \ge n \right\} \) is larger than \(\left\{ k \ge n+1 \right\} \), so \(S_{n+1} \le S_n\). Equivalently, we can define \(\limsup_{n \to \infty} a_n = \lim_{n \to \infty} S_n  \), and since \(S_n\) is decreasing, so \(\lim_{n \to \infty} S_n \) exists, but it could be \(\infty \) or \(-\infty \). 
        \item [(b)] Similarly we can define 
        \[
            \liminf_{n \to \infty} a_n = \lim_{n \to \infty} \inf _{k \ge n} \inf a_k.  
        \] Let \(I_n = \inf _{k \ge n} a_k\), then we know \(I_n \le I_{n+1}\), so \(I_n\) is increasing, so \(\lim_{n \to \infty} I_n = \liminf_{n \to \infty} a_n  \) exists, but it could be \(\infty \) or \(-\infty \).     
    \end{itemize}
\end{definition}

\begin{eg}
    \begin{align*}
        \limsup_{n \to \infty} n &= \lim_{n \to \infty} \sup _{k \ge n} k = \infty . \\
        \liminf_{n \to \infty} n &= \lim_{n \to \infty} \inf _{k \ge n} k = \lim_{n \to \infty} n = \infty .  
    \end{align*}
\end{eg}

\begin{definition}
    Let \(\sum_{n=1}^{\infty} a_n \) be a series of real numbers, and let \(S_N = \sum_{n=1}^N a_n \), then we say \(\sum_{n=1}^{\infty} a_n \) converges if \(\lim_{N \to \infty} S_N\) exists and 
    \[
        \sum_{n=1}^{\infty} a_n = \lim_{N \to \infty} S_N. 
    \]
\end{definition}

\begin{theorem}
    If \(\sum_{n=1}^{\infty} a_n \) converges, then \(\lim_{n \to \infty} a_n = 0 \).   
\end{theorem}
\begin{proof}
    Suppose \(S_N = \sum_{n=1}^N a_n \), then we know \(S_{n+1} - S_n = a_{n+1}\), then we know 
    \[
        \lim_{n \to \infty} a_n = \lim_{n \to \infty} a_{n+1} = \lim_{n \to \infty} S_{n+1} - S_n = \lim_{n \to \infty} S_{n+1} - \lim_{n \to \infty} S_n = 0.    
    \]  
\end{proof}

\begin{corollary}
    If \(\lim_{n \to \infty} a_n \) doesn't exist or \(\lim_{n \to \infty} a_n \neq 0 \), then \(\sum_{n=1}^{\infty} a_n \) diverges.   
\end{corollary}

\begin{definition}
    We say a series \(\sum_{n=1}^{\infty} a_n  \) is absolutely convergent if \(\sum_{n=1}^{\infty} \vert a_n \vert  \) converges.
\end{definition}

\begin{theorem}
    If \(\sum_{n=1}^{\infty} a_n \) converges absolutely, then \(\sum_{n=1}^{\infty} a_n \) converges.
\end{theorem}
\begin{proof}
    Let \(S_n = \sum_{i=1}^n a_i \), then suppose \(n \ge m\), then we have 
    \[
        \left\vert S_n - S_m \right\vert = \left\vert \sum_{i=m+1}^{n} a_n \right\vert \le \sum_{i=m+1}^n \vert a_i \vert.    
    \]  
    Let \(T_n = \sum_{i=1}^n \vert a_i \vert  \), then we know \(\left\vert T_n - T_m \right\vert = \sum_{i=m+1}^n \vert a_i \vert  \). Since \(\sum_{n=1}^{\infty} a_n \) converges absolutely, so \(\lim_{n \to \infty} T_n \) exists, and thus \(\left\{ T_n \right\} \) is Cauchy. Since
    \[
        \left\vert S_n - S_m \right\vert \le \left\vert T_n - T_m \right\vert, 
    \] so \(\left\{ S_n \right\} \) is also Cauchy, which means it is convergent. 
\end{proof}

\section{Formal Power Series}
\begin{definition}
    Let \(a \in \mathbb{R} \). A formal power series centered at \(a\) is any series of the form 
    \[
        \sum_{n=0}^{\infty} c_n (x - a)^n 
    \] where \(\left\{ c_n \right\}_{n=1}^{\infty}  \) is a sequence of real numbers. We refer \(c_n\) to the \(n\)-th coefficient of the series. Each term \(c_n(x - a)^n\) is a function of \(x\).     
\end{definition}

\begin{eg}
    \(\sum_{n=0}^{\infty} n! (x-2)^n \) is a formal power series centered at \(2\) but \(\sum_{n=0}^{\infty} 2^x (x-3)^n \) is not a formal power series since \(2^x\) depends on \(x\) not on \(n\).      
\end{eg}

\begin{definition}
    Let \(\sum_{n=0}^{\infty} c_n(x-a)^n \) be a formal power series. We define the radius of convergence of \(R \) of this series to be the quantity \(R=\frac{1}{\limsup_{n \to \infty} \vert c_n \vert^{\frac{1}{n}} }\) with the convention \(\frac{1}{0^+} = \infty \) and \(\frac{1}{\infty } = 0\). We'll show that 
    \begin{itemize}
        \item If \(\vert x-a \vert < R\), then this series will converge. 
        \item If \(\vert x - a \vert > R \), then this series will diverge. 
        \item If \(\vert x-a \vert = R \), then no conclusion, and should analize case by case.   
    \end{itemize}   
\end{definition}

\begin{lemma} \label{lm: series approaches limsup}
    Suppose \(L = \limsup_{n \to \infty} a_n \in [0, \infty ) \) and \(L \neq -\infty \), then for any \(\varepsilon > 0\), \(\exists N > 0\) s.t. \(a_n \le L + \varepsilon \) for all \(n \ge N\).      
\end{lemma}
\begin{proof}
    If \(L = \infty \), then this is true. Now if \(L\) is a real number, then \(L = \lim_{n \to \infty} S_n \) where \(S_n = \sup _{k \ge n} a_k\). Given any \(\varepsilon > 0\), \(\exists N > 0\) s.t. \(\vert S_n - L \vert < \varepsilon  \) for all \(n \ge N\), so \(S_n < L + \varepsilon \) for all \(n \ge N\). Hence, 
    \[
        a_n \le \sup _{k \ge n} a_k = S_n < L + \varepsilon \quad \forall n \ge N.
    \]          
\end{proof}

\begin{lemma} \label{lm: given any veps we can find infinitely many an approach limsup in error veps}
    Let \((a_n)\) be a sequence of real numbers, and let \(p \coloneqq \limsup_{n \to \infty} a_n \). Then for every \(\varepsilon > 0\), \(\exists \) infinitely many indices \(n\) s.t. \(a_n > p - \varepsilon \).     
\end{lemma}
\begin{proof}
    We know \(p = \lim_{n \to \infty} S_n \) where \(S_n = \sup _{k \ge n} a_k\), so given \(\varepsilon > 0\), \(\exists N > 0\) s.t. 
    \[
        \left\vert S_n - p \right\vert < \frac{\varepsilon}{2} \quad \forall n \ge N,
    \] so \(p - \frac{\varepsilon}{2} < S_n = \sup _{k \ge n} a_k\), which means \(\exists k_1 \ge n\) s.t. \(S_n - \frac{\varepsilon}{2} < a_{k_n}\), and let \(n = k_1\) and repeat this step to get \(k_2\) and so on, then we know \(\left\{ k_n \right\} \) is an infinite set.     
\end{proof}

\begin{theorem}[Ratio Test] \label{thm: Ratio test}
    Suppose \(\lim_{n \to \infty} \left\vert \frac{a_{n+1}}{a_n} \right\vert = L \in [0, \infty )\). If \(L < 1\), then \(\sum_{n=0}^{\infty} a_n \) converges absolutely. If \(L > 1\), then \(\sum_{n=0}^{\infty} a_n \) diverges.     
\end{theorem}

\begin{theorem}
    Let \(\sum_{n=0}^{\infty} c_n (x - a)^n \) be a formal power series, and let \(R\) be its raduis of convergence \(R = \frac{1}{\limsup_{n \to \infty} \vert c_n \vert^{\frac{1}{n}}  }\), then 
    \begin{itemize}
        \item [(a)] \(\sum_{n=0}^{\infty} c_n (x - a)^n \) diverges if \(\vert x-a \vert > R \). 
        \item [(b)] \(\sum_{n=0}^{\infty} c_n (x - a)^n \) converges absolutely if \(\vert x-a \vert < R \). 
        \item [(c)] \(\sum_{n=0}^{\infty} c_n (x - a)^n \) converges uniformly on \([a-r, a+r]\) when \(0 < r < R\). 
        \item [(d)] Let \(f = \sum_{n=0}^{\infty} c_n (x-a)^n \) if \(\vert x-a \vert < R \), then \(f\) is differentiable and for any \(0<r<R\), \(\sum_{n=1}^{\infty} n c_n (x-a)^{n-1} \) converges uniformly to \(f^{\prime} \) on \([a-r, a+r]\). 
        \item [(e)] For any \([y, z] \subset (a-R, a+R)\), 
        \[
            \int _y^z f(x) \, \mathrm{d} x = \sum_{n=0}^{\infty} \frac{c_n (z-a)^{n+1}}{n+1} - \sum_{n=0}^{\infty } \frac{c_n(y-a)^{n+1}}{n+1}.   
        \]
    \end{itemize}   
\end{theorem}
\begin{proof}[proof from (c) to (e)]
   Trivial. 
\end{proof}
\begin{proof}[proof of (a)]
    Write \(L\coloneqq \limsup_{n \to \infty} \vert c_n \vert^{\frac{1}{n}}  \), so \(L = \frac{1}{R}\). Now if \(\vert x - a \vert > R \), then let \(s = \vert x - a \vert > R \), and we have 
    \[
        \limsup_{n \to \infty} \left\vert c_n (x - a)^n \right\vert^{\frac{1}{n}} = \limsup_{n \to \infty} \vert c_n \vert^{\frac{1}{n}} \vert x - a \vert = \vert x - a \vert \limsup_{n \to \infty} \vert c_n \vert^{\frac{1}{n}} = SL = \frac{S}{R} > 1.  
    \] Choose \(\varepsilon > 0\) s.t. \(\frac{S}{R} - \varepsilon  > 1\). From \autoref{lm: given any veps we can find infinitely many an approach limsup in error veps}, there exists infinitely many \(n\) s.t. 
    \[
        \left\vert c_n \right\vert^{\frac{1}{n}} \vert x - a \vert > \frac{S}{R} - \varepsilon > 1,  
    \] so there are infinitely many \(n\) has \(\left\vert c_n (x-a)^n \right\vert > 1\), so \(\lim_{n \to \infty} c_n (x - a)^n \neq 0 \), and thus we know \(\sum_{n=0}^{\infty} c_n (x - a)^n \) diverges.           
\end{proof}
\begin{proof}[proof of (b)]
    If \(\vert x - a \vert < R \), then \(\frac{\vert x - a \vert }{R} < 1\) i.e. \(\vert x - a \vert L < 1\) (\(L\) is same as it is in proof of (a)). Now we have 
    \(\limsup_{n \to \infty} \vert c_n \vert^{\frac{1}{n}} = L  \), so we can choose \(\varepsilon > 0\) s.t. \(\vert x - a \vert(L + \varepsilon ) < 1 \), so there exists \(N > 0\) s.t. 
    \[
        \left\vert c_n \right\vert^{\frac{1}{n}} < L + \varepsilon \quad \forall n \ge N \iff \vert c_n \vert < (L + \varepsilon )^n \iff \left\vert c_n (x-a)^n \right\vert \le (L + \varepsilon )^n \vert x - a \vert^n     
    \] by \autoref{lm: series approaches limsup}, so 
    \[
        \sum_{n=0}^{\infty} \left\vert c_n (x - a)^n \right\vert \le \sum_{n=0}^{N-1} \left\vert c_n (x-a)^n \right\vert + \sum_{n=N}^{\infty} \left\vert c_n(x-a)^n \right\vert \le \sum_{n=0}^{N-1} \vert c_n \vert \vert x-a \vert^n + \sum_{n=N}^{\infty} \left\vert (L+ \varepsilon )(x-a) \right\vert^n            
    \] and R.H.S. converges since the left term is finite and for the right term we have \(\vert (L + \varepsilon )(x - a) \vert < 1 \). 
\end{proof}

\begin{proof}[proof of (c)]
    If \(\vert x - a \vert \le r < R \), then we can choose \(\varepsilon > 0\) s.t. \(q = (L + \varepsilon )r < 1\). Note that 
    \[
        \left\vert (L + \varepsilon )(x - a) \right\vert^n \le \left( (L + \varepsilon )r \right)^n = q^n,  
    \] and this is independent of \(x\). (Note that \(L > 0\) so we can get rid of the absolute value) Note that this means
    \[
        \sum_{n=0}^{\infty} \left\lVert c_n(x-a)^n \right\rVert_\infty \le \sum_{n=0}^{\infty } q^n, 
    \] which is convergent, so by \hyperref[thm: Weierstrass M-test (bounded and conti)]{Weierstrass \(M\)-test}, we know \(\sum_{n=0}^{\infty} \vert c_n \vert \vert x - a \vert^n   \) converges uniformly.  
\end{proof}
\begin{proof}[proof of (d)]
    \vphantom{text}
    \begin{itemize}
        \item Step 1: first prove \(\sum_{n=1}^{\infty} n c_n (x-a)^{n-1} \) has same convergence radius as \(f\). 
        \item Use \autoref{thm: f_n differentiable and f_n prime conti, then if f_n prime to g uni, and for some x0 lim f_n(x_0) exists, then EE f s.t. lim fn equal f uni and f prime is g}.   
    \end{itemize}
\end{proof}