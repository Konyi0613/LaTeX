\lecture{21}{18 Nov. 10:20}{}
\section{A digression on complex numbers}
\begin{definition}
    The complex numbers \(\mathbb{C} \) are the set of all numbers of the form \(a + bi\), \(a, b \in \mathbb{R} \), where \(i = \sqrt{-1} \), i.e. \(\mathbb{C}  = \left\{ a + bi \mid a, b \in \mathbb{R}  \right\} \), and we can define \(+,-,*,/\) on \(\mathbb{C} \).    
\end{definition}

First, we have \(i^2 = -1\). Now if \(z, w \in \mathbb{C} \) where \(z = a + bi\) and \(w = c + di\), then 
\begin{align*}
    z + w &= (a + c) + (b + d)i \\
    z \cdot w &= (a + bi)(c + di) \\
    &= a(c + id) + bi(c + id) \\
    &= (ac - bd) + i(ad + bc) \\
    (-1) \cdot w &= -a - bi \\
    z - w &= z + (-1) \cdot w = (a-c) + (b-d)i.
\end{align*}    

For each \(z = a + ib\), we can define the complex conjugate of \(z\) by \(\overline{z} = a - ib \). We can also define the absolute value (or modules) of \(z\) by \(\vert z \vert = \sqrt{a^2 + b^2} \) where \(z = a + bi\). Also, 
\[
    \text{(1) } z \cdot \overline{z} = \vert z \vert^2, \quad \text{(2) } \text{if } z \neq 0, \text{ then } \frac{1}{z} = \frac{\overline{z} }{\vert z \vert^2 }.    
\]      
\begin{remark}
    We can identify \(\mathbb{C} \simeq \mathbb{R} ^2\) by \(a + ib \to (a, b)\).  
\end{remark}

\begin{lemma}
    The complex numbers are an addition group, i.e. if \(z_1, z_2, z_3 \in \mathbb{C} \), then the addition saisfies the following properties: 
    \begin{itemize}
        \item Commutativity: \(z_1 + z_2 = z_2 + z_1\). 
        \item Associativity: \((z_1 + z_2) + z_3 = z_1 + (z_2 + z_3) \). 
        \item Identity: \(z_1 + 0 = z_1\) where \(0 = 0 + 0\). 
        \item Inverse: \(0 = z_1 + (-z_1) + (-z_1) + z_1\).     
    \end{itemize}
\end{lemma}

\begin{lemma}
    Complex numbers form a ring, i.e. if \(z_1, z_2, z_3 \in \mathbb{C} \), then 
    \begin{itemize}
        \item [(a)] \(z_1 z_2 = z_2 z_1\). 
        \item [(b)] \((z_1 \cdot z_2) \cdot z_3 = z_1 \cdot (z_2 \cdot z_3)\). 
        \item [(c)] \(z_1 \cdot 1 = z_1\). 
        \item [(d)] \(z_1 (z_2 + z_3) = z_1 z_2 + z_1 z_3\). Also, \((z_1 + z_2) z_3 = z_1 z_3 + z_2 z_3\).    
    \end{itemize} 
\end{lemma}

\begin{lemma}
    We have 
    \begin{itemize}
        \item [(1)] \(\overline{z + w} = \overline{z} + \overline{w}   \). 
        \item [(2)] \(\overline{z \cdot w} = \overline{z} \cdot \overline{w}   \). 
        \item [(3)] \(\overline{\overline{z}} = z \). 
        \item [(4)] \(z = \overline{z} \) iff \(z \in \mathbb{R} \).    
    \end{itemize}
\end{lemma}

\begin{proposition}
    Let \(z, w \in \mathbb{C} \). Then, 
    \begin{itemize}
        \item [(1)] \(\vert z + w \vert \le \vert z \vert + \vert w \vert   \) 
        \item [(2)] \(\vert z \cdot w \vert = \vert z \vert \cdot \vert w \vert   \).
    \end{itemize} 
\end{proposition}
\begin{proof}
    Let \(z = a + bi\) and \(w = c + di\), and do some calculation.  
\end{proof}

\begin{corollary}
    \[
        \left\vert \sum_{i=1}^n z_i  \right\vert \le \sum_{i=1}^n \vert z_i \vert \quad \text{if }  z_1, z_2, \dots , z_n \in \mathbb{C} .    
    \] 
\end{corollary}

On \(\mathbb{C} \), we can introduce a metric \(d(z, w) = \vert z - w \vert \) for \(z, w \in \mathbb{C} \). Hence, if \(z = a + bi\) and \(w = c + di\), then \(z \simeq (a, b) \in \mathbb{R} ^2\) and \(w \simeq (c, d) \in \mathbb{R} ^2\), and thus 
\[
    d(z, w) = d_{\mathbb{R} ^2} (z, w),
\] so \(d\) is a metric on \(\mathbb{C} \).

\begin{definition}
We say 
\[
    \lim_{n \to \infty} z_n = z 
\] if and only if \(\lim_{n \to \infty} d(z_n, z) = \lim_{n \to \infty}  \left\vert z_n - z \right\vert = 0  \) for \(z, z_n \in \mathbb{C} \).  
\end{definition}

\begin{remark}
    \(\mathbb{C} \) is a complete metric space since \(\mathbb{R} ^2\) is complete.  
\end{remark}

\begin{definition}
    If \(z \in \mathbb{C} \), then we can define \(\exp (z)\) by the power series 
    \[
        \exp (z) \coloneqq \sum_{n=0}^{\infty} \frac{z^n}{n!}. 
    \]  
\end{definition}

To show this definition is well-defined, we need to show \(\lim_{n \to \infty} S_n \) exists where \(S_n = \sum_{k=0}^n \frac{z^k}{k!} \). WLOG, suppose \(n > m\), then 
\begin{align*}
    d \left( S_n, S_m \right) &= \left\vert S_n - S_m \right\vert = \left\vert \sum_{k={m+1}}^n \left( \frac{z^k}{k!} \right)   \right\vert \le \sum_{k=m+1}^n \left\vert \frac{z^k}{k!} \right\vert = \sum_{k=m+1}^n \frac{\vert z \vert^k }{k!}.      
\end{align*}  
Recall that
\[
    \exp \left( \vert z \vert  \right) = \sum_{n=0}^{\infty} \frac{\vert z \vert^n }{n!}  
\] is well-defined for all \(z \in \mathbb{C} \) since \(\vert z \vert \in \mathbb{R}  \). Now let \(T_n = \sum_{k=0}^n \frac{\vert z \vert^k }{k!} \), then 
\[
    d(S_n, S_m) \le \left\vert T_n - T_m \right\vert \to 0 \text{ as } n,m \to \infty.  
\] 

We also have the following properties:
\begin{itemize}
    \item [(1)] \(\exp (z + w) = \exp (z) \cdot \exp (w)\) for \(z, w \in \mathbb{C} \). 
    \item [(2)] \(\exp \left( \overline{z}  \right) = \overline{\exp (z)}  \).   
\end{itemize}
In the real case, \(\exp \) is a function from \(\mathbb{R} \) to \((0, \infty )\) and is injective. However, \(\exp : \mathbb{C} \to \mathbb{C}  \setminus \left\{ 0 \right\} \) is not injective. Note that 
\[
    \exp (z + 2 \pi i) = \exp (z) \exp (2\pi i) = \exp (z) \left( \cos (2\pi ) + i \sin (2 \pi ) \right) = \exp (z). 
\]    
It's more complicated to define the inverse of \(\exp (z)\).

\section{Trigonometric functions}
\begin{definition}
    If \(z \in \mathbb{C} \), then we define 
    \[
        \cos (z) = \frac{e^{iz} + e^{-iz}}{2}, \quad \sin (z) = \frac{e^{iz} - e^{-iz}}{2i}.
    \] 
\end{definition}

\begin{proposition}
    For \(z \in \mathbb{C} \), 
    \[
        e^{iz} = \cos (z) + i \sin (z), \quad e^{-iz} = \cos (z) - i\sin (z).
    \] 
\end{proposition}
\begin{proof}
It can be shown by
    \begin{align*}
        \cos z + i \sin z &= \frac{e^{iz} + e^{-iz}}{2} + i\left( \frac{e^{iz} - e^{-iz}}{2i} \right) = e^{iz} \\
        \cos (z) - i \sin z &= \frac{e^{iz} + e^{-iz}}{2} - i \left( \frac{e^{iz} - e^{-iz}}{2i} \right) = e^{-iz} 
    \end{align*}
\end{proof}

\begin{theorem}
\begin{align*}
    \cos (z) &= \sum_{n=0}^{\infty} \frac{(-1)^n z^{2n}}{(2n)!} \\
    \sin (z) &= \sum_{n=0}^{\infty} \frac{(-1)^n z^{n+1}}{(2n+1)!}
\end{align*}
for \(z \in \mathbb{C} \). 
\end{theorem}
\begin{proof}
Recall \(e^z = \sum_{n=0}^{\infty} \frac{z^n}{n!} \) for \(z \in \mathbb{C} \). Hence, 
\begin{align*}
    e^{iz} &= \sum_{n=0}^{\infty} \frac{(iz)^n}{n!} = \sum_{k=0}^{\infty} \frac{(iz)^{2k}}{(2k)!} + \sum_{k=0}^{\infty} \frac{(iz)^{2k+1}}{(2k+1)!} = \sum_{k=0}^{\infty} \frac{(-1)^k z^{2k}}{(2k)!} + i \left( \sum_{k=0}^{\infty} \frac{(-1)^k z^{2k+1}}{(2k+1)!}  \right) \\
    e^{-iz} &= e^{i(-z)} = \sum_{k=0}^{\infty} \frac{(-1)^k z^{2k}}{(2k)!} - i \left( \sum_{k=0}^{\infty} \frac{(-1)^k z^{2k+1}}{(2k+1)!}  \right).      
\end{align*}  
Thus, 
\[
    \cos (z) = \frac{e^{iz} + e^{-iz}}{2} = \sum_{k=0}^{\infty} \frac{(-1)^k z^{2k}}{(2k)!}, \quad \sin (z) = \frac{e^{iz} - e^{-iz}}{2i} = \sum_{k=0}^{\infty} \frac{(-1)^k z^{2k+1}}{(2k+1)!}. 
\]
\end{proof}

\begin{remark}
    In particular, if \(x \in \mathbb{R} \), then 
    \[
       \cos (x) = \frac{e^{ix} + e^{-ix}}{2} = \sum_{k=0}^{\infty} \frac{(-1)^k x^{2k}}{(2k)!}, \quad \sin (x) = \frac{e^{ix} - e^{-ix}}{2i} = \sum_{k=0}^{\infty} \frac{(-1)^k x^{2k+1}}{(2k+1)!}. 
    \] 
\end{remark}

\begin{theorem}[Trigonometric Identities]
    Let \(x, y \in \mathbb{R} \). Then the following properties hold: 
    \begin{itemize}
        \item [(a)] \(\frac{\mathrm{d}}{\mathrm{d}x} \sin (x) = \cos (x) \), and \(\frac{\mathrm{d}}{\mathrm{d}x} \cos (x) = -\sin (x) \).
        \item [(b)] \(\sin ^2(x) + \cos ^2(x) = 1\). In particular, \(\sin (x) \in [-1, 1]\) and \(\cos (x) \in [-1, 1]\).  
        \item [(c)] \(\sin (-x) = - \sin (x)\), and \(\cos (-x) = \cos (x)\). 
        \item [(d)] \(\cos (x+y) = \cos (x) \cos (y) - \sin (x) \sin (y)\), and \(\sin (x+ y) = \sin (x) \cos (y) + \cos (x) \sin (y)\). 
        \item [(e)] \(\sin (0) = 0\) and \(\cos (0) = 1\). 
        \item [(f)] \(e^{ix} = \cos (x) + i \sin (x)\), and \(e^{-ix} = \cos (x) - i \sin (x)\).          
    \end{itemize} 
\end{theorem}

\begin{proof}[proof of (a)]
    \begin{align*}
        \frac{\mathrm{d}}{\mathrm{d}x} \cos (x) &= \frac{\mathrm{d}}{\mathrm{d}x} \left( \frac{e^{ix} + e^{-ix}}{2} \right) = \frac{1}{2} \left( e^{ix} \cdot (ix)^{\prime} + e^{-ix} (-ix)^{\prime}  \right) \\
        &= \frac{1}{2} \left( e^{ix} \cdot i + e^{-ix} \cdot (-i) \right) = \frac{i}{2} \left( e^{ix} - e^{-ix} \right) = \frac{i^2 \left( e^{ix} - e^{-ix} \right)  }{2i} = -\sin x.     
    \end{align*}
    \begin{align*}
        \frac{\mathrm{d}}{\mathrm{d}x} \sin x &= \frac{\mathrm{d}}{\mathrm{d}x} \left( \frac{e^{ix} - e^{-ix}}{2i} \right) = \frac{1}{2i} \left( e^{ix} \cdot i - e^{-ix} \cdot (-i) \right) \\
        &= \frac{i}{2i} \left( e^{ix} + e^{-ix} \right) = \frac{1}{2} \left( e^{ix} + e^{-ix} \right) = \cos (x).       
    \end{align*}
\end{proof}

\begin{proof}[proof of (b)]
    Since 
    \begin{align*}
        \frac{\mathrm{d}}{\mathrm{d}x} \left( \sin ^2(x) + \cos ^2(x) \right) &= 2 \sin (x) \left( \sin (x) \right)^{\prime} + 2 \cos (x) \left( \cos (x) \right)^{\prime}
        \\ &= 2 \sin (x) \cos (x) + 2 \cos (x) (-\sin (x)) = 0, 
    \end{align*}
    and \(\sin ^2(0) + \cos ^2(0) = 1\), so we know \(\sin ^2(x) + \cos^2(x) = 1\) for all \(x \in \mathbb{R} \).   
\end{proof}

\begin{proof}[proof of (c)]
    \todo{DIY}
\end{proof}

\begin{proof}[proof of (d)]
    Since \(e^{i(x+y)} = \cos (x+y) + i \sin (x+y)\), and 
    \[
        e^{i(x+y)} = e^{ix} e^{iy} = \left[ \cos(x) \cos (y) - \sin (x) \sin (y)\right] + i \left[ \sin (x) \cos (y) + \cos (x) \sin (y) \right],  
    \] so we're done.
\end{proof}

\begin{proof}[proof of (e)]
    \todo{DIY}
\end{proof}

\begin{proof}[proof of (f)]
    \todo{DIY}
\end{proof}

\begin{lemma}
    There exists a positive number \(x\) s.t. \(\sin (x) = 0\).  
\end{lemma}
\begin{proof}
    Since \(\sin (0) = 0\), and we know \(\sin (x)\) is continuous and differentiable on \(\mathbb{R} \), and \(\frac{\mathrm{d}}{\mathrm{d}x} \sin (x) = \cos (x) \), so
    \[
        \frac{\mathrm{d}}{\mathrm{d}x} \sin (x) \vert_{x = 0} = \cos (0) = 1 > 0. 
    \]
    Hence, \(\sin (x)\) is strictly increasing near \(0\). Thus, \(\exists \delta > 0\) s.t. \(\sin (x) > 0\) on \((0, \delta]\). Suppose \(\sin (x) \neq 0\) for all \(x > 0\), then \(\sin (x) > 0\) for all \(x > 0\). We will show that this is impossible. 
    \begin{itemize}
        \item Step 1: \(\cos (x) \neq 0\) on \((0, \infty )\). Since we know 
        \[
            \sin (2x) = 2 \sin (x) \cos (x),
        \]  
        so if \(\cos (x_0) = 0\) for some \(x_0 > 0\), then \(\sin (2 x_0) = 0\), which is a contradiction. 
        \item Step 2: \(\cos (x) > 0\) on \((0, \infty )\). Since \(\cos (0) = 1 > 0\), and \(\cos (x) \neq 0\) on \((0, \infty )\), and since \(\cos (x)\) is continuous, so \(\cos (x) > 0\) on \((0, \infty )\). 
        \item Step 3: \(\sin (x) > 0\) for all \(x > 0\). 
        \item Step 4: \(\cot (x) = \frac{\cos (x)}{\sin (x)} > 0\) on \((0, \infty )\). 
        \item Step 5: 
        \[
            \frac{\mathrm{d}}{\mathrm{d}x} \cot (x) = \frac{\mathrm{d}}{\mathrm{d}x} \left( \frac{\cos (x)}{\sin (x)} \right) = \frac{- \sin (x) \sin (x) - \cos (x) \cos (x)}{\sin ^2(x)} = -\frac{1}{\sin ^2(x)} < 0.   
        \]
        Since \(\sin (x) > 0\) and \(\sin (x) \in [-1, 1]\), and we know \(0 < \sin (x) \le 1\) when \(x > 0\), which means \(0 < \sin ^2(x) \le 1\), and this gives 
        \[
            \frac{\mathrm{d}}{\mathrm{d}x} \cot (x) = -\frac{1}{\sin ^2(x)} \le -1. 
        \]  
        Note that 
        \begin{align*}
            \cot (x+s) - \cot (x) &= \int _x^{x+s} \frac{\mathrm{d}}{\mathrm{d}x} \cot (t) \, \mathrm{d} t = \int _x^{x+s} -\frac{1}{\sin ^2(t)} \, \mathrm{d} t \\
            &\le \int _x^{x+s} (-1) \, \mathrm{d}t = -s.   
        \end{align*}  
        Hence, we have 
        \[
            \cot (x+s) - \cot (x) \le -s \text{ when } s > 0, x>0, 
        \] and this gives 
        \[
            \cot (x+s) \le \cot (x) - s \text{ when } s>0, x>0, 
        \]
        Note that 
        \[
            \lim_{s \to \infty} \cot (x) - s = -\infty 
        \] since \(\cot (x)\) is fixed. Hence, 
        \[
            \lim_{s \to \infty} \cot (x+s) = -\infty, 
        \] which means \(\cot (x) < 0\) when \(x\) is big enough, which is a contradiction.  
    \end{itemize}            
\end{proof}