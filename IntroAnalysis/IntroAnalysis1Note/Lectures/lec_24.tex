\lecture{24}{2 Dec. 10:20}{}
Our goal is to prove 
\begin{theorem}[Weiertrass Approximation Theorem by trigonometric polynomial]
    Let \(f \in C(\mathbb{R} / \mathbb{Z} , \mathbb{C} )\) be continuous \(1\)-periodic function, and let \(\varepsilon > 0\). Then there exists a trigonometric polynomial \(P \in C(\mathbb{R} / \mathbb{Z} , \mathbb{C} )\) s.t. 
    \[
        \lVert f - P \rVert_\infty \le \varepsilon \text{ where } \vert f - P \vert_\infty = \sup _{x \in [0, 1]} \vert f(x) - P(x) \vert.
    \]    
    This implies the set of trigonometric polynomial is dense in \(C(\mathbb{R} / \mathbb{Z} , \mathbb{C} )\) w.r.t. \(\lVert \cdot \rVert_\infty  \).  
\end{theorem}

\begin{definition}
    Let \(\varepsilon > 0\) and \(0 < \delta < \frac{1}{2}\). A function \(f \in C(\mathbb{R} / \mathbb{Z} , \mathbb{C} )\) is a periodic \((\varepsilon , \delta )\) approximation to the identity if the following two conditions hold:
    \begin{itemize}
        \item [(a)] \(f(x) \ge 0\) for all \(x \in \mathbb{R} \) and \(\int _0^1 f(x) \, \mathrm{d} x = 1 \). 
        \item [(b)] For all \(\delta \le \vert x \vert \le 1 - \delta  \), one has \(f(x) < \varepsilon \).     
    \end{itemize}
\end{definition}

%fig
\begin{figure}[H]
    \centering
    \includegraphics[width=0.65\textwidth]{./Figures/IMG_0770.jpg}
    \caption{A \((\varepsilon , \delta )\) approximation to the identity}
    \label{fig: A veps delta approximation to the identity}
\end{figure}

\begin{lemma}[Periodic approximation to the identity]
    Let \(\varepsilon > 0\) and \(0 < \delta < \frac{1}{2}\). Then there exists a trigonometric polynomial \(P\) which is a periodic \((\varepsilon , \delta )\) approximation to the identity, that is, \(P \in C(\mathbb{R} / \mathbb{Z} , \mathbb{C} )\), \(P(x) \ge 0\) for all \(x \in \mathbb{R} \), and \(\int _0^1 P(x) \, \mathrm{d} x = 1  \), and \(P(x) < \varepsilon \) for all \(\delta \le \vert x \vert \le 1 - \delta  \).         
\end{lemma}
\begin{proof}
    If we can prove the below claim, then we can show this lemma.
    \begin{claim}
        We can choose \(N\) large enough s.t.
        \[
            F_N(x) = \frac{1}{N} \left\vert \sum_{k=0}^{N-1} e_k(x)  \right\vert^2 \text{ where } e_k(x) = e^{2 \pi i k x}  
        \] 
        and pick \(P\) to be this \(F_N(x)\) with enough large \(N\).  
    \end{claim}
    The geometric meaning is that 
    \[
        \left\{ e_0(x), \dots , e_{N - 1}(x) \right\} 
    \]
    is an orthonormal set w.r.t. \(L_2\)-inner product, which has been shown before. Now 
    \[
        v(x) = \frac{1}{\sqrt{N} } \sum_{i=0}^{N-1} e_i(x) 
    \]
    is a unit vector under \(L_2\)-inner product, so we know 
    \[
        \int _0^1 \frac{1}{N} \left\vert \sum_{k=0}^{N-1} e_k(x)  \right\vert^2 \, \mathrm{d} x = \int _0^1 v(x) \overline{v(x)} \, \mathrm{d} x = \lVert v(x) \rVert_2^2 = 1.     
    \]   
    \begin{claim}
        \[
            F_N(x) = \sum_{n=-N}^N \left( 1 - \frac{\vert n \vert }{N} \right) e_n(x) \text{ where } e_n(x) = e^{2 \pi i n x}.   
        \]
    \end{claim}
    \begin{explanation}
        Note that it is equivalent to show
        \[
           F_N(x) = \sum_{n=-(N-1)}^{N-1} \left( 1 - \frac{\vert n \vert }{N} \right) e_n(x). 
        \]
        Recall that 
        \begin{align*}
            F_N(x) &= \frac{1}{N} \left\vert \sum_{k=0}^{N-1} e_k(x)  \right\vert^2 = \frac{1}{N} \left( \sum_{k=0}^{N-1} e_k(x)  \right) \overline{\left( \sum_{\ell = 0}^{N - 1} e_{\ell }(x)  \right) } = \frac{1}{N} \left( \sum_{k=0}^{N-1} e_k(x)  \right) \left( \sum_{\ell = 0}^{N - 1} \overline{e_{\ell }(x)} \right)  \\
            &= \frac{1}{N} \sum_{k=0}^{N - 1} \sum_{\ell = 0}^{N - 1} e_k(x) \overline{e_{\ell}(x) } = \frac{1}{N} \sum_{k=0}^{N - 1} \sum_{\ell = 0}^{N - 1} e_{k - \ell }(x).        
        \end{align*}
        Note that
        \[
            \begin{dcases}
                0 \le k \le N - 1 \\
                0 \le \ell \le N - 1
            \end{dcases} \implies -(N - 1) \le k - \ell \le N - 1.
        \]
        Now fix \(m = k - \ell \), then we can count that there are \(N - \vert m \vert \) \((k_1, \ell_1 )\) pairs s.t. \(k_1 - \ell _1 = m\). Hence, we know 
        \[
            F_N(x) = \frac{1}{N} \sum_{k=0}^{N - 1} \sum_{\ell = 0}^{n - 1} e_{k - \ell }(x) = \frac{1}{N} \sum_{m = -(N-1)}^{N-1} (N - \vert m \vert ) e_m(x) = \sum_{m = -(N-1)}^{N - 1} \left( 1 - \frac{\vert m \vert}{N} \right)e_m(x).      
        \]   
    \end{explanation}

    Now we already know \(F_N(x) \ge 0\) for all \(x \in \mathbb{R} \) and \(\int _0^1 F_N(x) \, \mathrm{d} x = 1\), and we know \(F_N(x)\) is a trigonometric polynomial, so \(F_N(x) \in C(\mathbb{R} / \mathbb{Z} , \mathbb{C} )\).   
    
    Now we simplify \(F_N\). Note that if \(x \notin \mathbb{Z} \), then  
    \begin{align*}
        \sum_{k=0}^{N - 1} e_k(x) &= \sum_{k=0}^{N - 1} e^{2 \pi i k x} = \sum_{k=0}^{N - 1} \left( e^{2 \pi i x} \right)^k = \frac{e^{2\pi i N x} - 1}{e^{2 \pi i x} - 1} \\
        &= \frac{e^{\pi i N x} \left( e^{\pi i N x} - e^{-\pi i N x} \right) }{e^{\pi i x} \left( e^{\pi i x} - e^{- \pi i x} \right) } = e^{\pi i (N - 1) x} \frac{\sin (\pi N x)}{\sin (\pi x)}   
    \end{align*} 
    Thus, we have 
    \[
        F_N(x) = \frac{1}{N} \left\vert \sum_{k=0}^{N - 1} e_k(x)  \right\vert^2 = \frac{1}{N} \frac{\sin ^2 (\pi N x)}{\sin^2 (\pi x)} \text{ if } x \notin \mathbb{Z} . 
    \]
    Hence, we know 
    \[
        F_N(x) = \begin{dcases}
            N, &\text{ if } x \in \mathbb{Z}  ;\\
            \frac{1}{N} \frac{\sin ^2(\pi N x)}{\sin ^2 (\pi x)}, &\text{ if }  x \notin \mathbb{Z} .
        \end{dcases}
    \]
    Fix \(0 < \delta < \frac{1}{2}\) and \(\delta \le \vert x \vert \le \frac{1}{2} \), then 
    \[
        \pi \delta \le \pi \vert x \vert \le \frac{\pi}{2},
    \] 
    then we know 
    \[
        \left\vert \sin (\pi x) \right\vert \ge \sin (\pi \delta ). 
    \]
    Hence, 
    \[
        \left\vert \frac{1}{\sin (\pi x)} \right\vert^2 \le \frac{1}{\sin ^2 (\pi \delta )}, 
    \]
    and since \(\left\vert \sin ^2 (\pi N x) \right\vert \le 1 \), so  
    \begin{align*}
        F_N(x) &= \frac{1}{N} \left\vert \frac{\sin  (\pi N x)}{\sin  (\pi x)} \right\vert^2 \le \frac{1}{N} \frac{1}{\sin ^2 (\pi \delta )} \text{ where } \delta \le \vert x \vert \le \frac{1}{2} \le 1 - \delta.  
    \end{align*}
    Hence, we can pick \(N\) large enough so that 
    \[
        \frac{1}{N} \frac{1}{\sin ^2 (\pi \delta )} < \varepsilon,
    \] 
    and we're done.
    \begin{remark}
        It seems that we only prove the case for \(\delta \le \vert x \vert \le \frac{1}{2} \) but not \(\frac{1}{2} \le \vert x \vert \le 1 - \delta  \), but in fact since the period of \(F_N\) is \(1\), so by the below figure, we can just prove the case \(\delta \le \vert x \vert \le \frac{1}{2} \) and extend to \(\delta \le x \le 1 - \delta \). 
        \begin{figure}[H]
            \centering
            \includegraphics[width=0.55\textwidth]{./Figures/IMG_1647.png}
            \caption{The black segments are the parts we have already proved, while \(F_N\) on the red segements is in fact same as the part of the leftmost black segment.}
            \label{fig:periodic approximation to the identity}
        \end{figure}     
    \end{remark} 
\end{proof}

Now we can prove the main theorem.
\begin{theorem}[Weiertrass Approximation Theorem by trigonometric polynomial] \label{thm: weierstrass approximation thm by trigonometric polynomial}
    Let \(f \in C(\mathbb{R} / \mathbb{Z} , \mathbb{C} )\) be continuous \(1\)-periodic function, and let \(\varepsilon > 0\). Then there exists a trigonometric polynomial \(P \in C(\mathbb{R} / \mathbb{Z} , \mathbb{C} )\) s.t. 
    \[
        \lVert f - P \rVert_\infty \le \varepsilon \text{ where } \left\lVert f - P \right\rVert _\infty = \sup _{x \in [0, 1]} \vert f(x) - P(x) \vert.
    \]    
    This implies the set of trigonometric polynomial is dense in \(C(\mathbb{R} / \mathbb{Z} , \mathbb{C} )\) w.r.t. \(\lVert \cdot \rVert_\infty  \).  
\end{theorem}
\begin{proof}
    Let \(f \in C(\mathbb{R} / \mathbb{Z} , \mathbb{C} )\). We know that \(\exists M > 0\) s.t. \(\vert f(x) \vert \le M \). Also, \(f\) is uniformly continuous, so there exists \(\delta > 0\) s.t. 
    \[
        \vert x - y \vert < \delta \implies \left\vert f(x) - f(y) \right\vert < \frac{\varepsilon}{4}.  
    \]     
    Let \(P\) be a trigonometric polynomial which is a \(\left( \frac{\varepsilon }{4M}, \delta  \right) \) approximation to the identity. By definition we know \(P(x) \ge 0\) for \(x \in \mathbb{R} \) and \(\int _0^1 P(x) \, \mathrm{d} x = 1 \) and 
    \[
        P(x) < \frac{\varepsilon }{4M} \text{ when } \delta \le \vert x \vert \le 1 - \delta.
    \]      
    Then, \(f * P\) is a trigonometric polynomial. This is because 
    \[
        P = \sum_{k = -N}^N c_k e_k,
    \] 
    and 
    \begin{align*}
        f * P &= f * \left( \sum_{k=-N}^N c_k e_k  \right) = \sum_{k=-N}^N c_k \left( f * e_k \right) \\
        &= \sum_{k = -N}^N c_k \hat{f}(k) e_k,     
    \end{align*}
    which is definitely a trigonometric polynomial. Also, 
    \begin{align*}
        \left\vert f(x) - (f * P)(x) \right\vert &= \left\vert f(x) - \int_0^1 f(x - y) P(y) \, \mathrm{d} y  \right\vert \\
        &= \left\vert \int _0^1 \left( f(x) - f(x - y) \right)P(y) \, \mathrm{d} y   \right\vert \\
        &\le \int _0^1 \left\vert f(x) - f(x - y) \right\vert \vert P(y) \vert  \, \mathrm{d} y = \int _0^1 \vert f(x) - f(y) \vert P(y) \, \mathrm{d} y.      
    \end{align*}
    since \(P(y) \ge 0\) for all \(x \in \mathbb{R} \). Also, we have 
    \begin{align*}
        \int _0^{\delta } \left\vert f(x) - f(x - y) \right\vert P(y) \, \mathrm{d} y &\le \frac{\varepsilon }{4} \int _0^{\delta } P(y) \, \mathrm{d} y \le \frac{\varepsilon}{4} \\
        \int _{1 - \delta }^1 \left\vert f(x) - f(x - y) \right\vert P(y) \, \mathrm{d} y &\le \frac{\varepsilon}{4} \int_{1 - \delta }^1 P(y) \, \mathrm{d} y \le \frac{\varepsilon}{4}.
    \end{align*} 
    On \([\delta , 1 - \delta ]\), we use \(\left\vert f(x) - f(x - y) \right\vert \le 2M \) and \(P(y) < \frac{\varepsilon}{4M}\), so 
    \[
        \int _\delta ^{1 - \delta } \left\vert f(x) - f(x - y) \right\vert P(y) \, \mathrm{d} y \le 2M \frac{\varepsilon }{4M} = \frac{\varepsilon}{2}.
    \]
    Hence, 
    \[
        \left\vert f(x) - (f * P)(x) \right\vert < \varepsilon,
    \]
    and we can pick \(G = f * P\), which is a trigonometric polynomial, and that 
    \[
        \vert f(x) - G(x) \vert < \varepsilon  
    \] for all \(x \in [0, 1]\), and thus we know 
    \[
        \sup_{x \in [0, 1]} \vert f(x) - G(x) \vert \le \varepsilon. 
    \]
\end{proof}