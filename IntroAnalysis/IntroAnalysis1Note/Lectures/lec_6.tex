\lecture{6}{18 Sep. 10:20}{}
\section{Relative topology}
Let \((X,d)\) be a metric space and \(Y \subseteq X\), then \(\left( Y, d\vert_{Y \times Y} \right) \) is also a metric space. 
\begin{eg}
    Consider \((\mathbb{R} ^2, d_2)\) and \(X = \left\{ (x, 0) \mid x \in \mathbb{R}  \right\} \), then on \(\left( X, d_2\vert_{X \times X} \right) = (X, d)\), it is also a metric space. 
\end{eg} 
\begin{explanation}
    Since 
    \[
        d((x,0),(y,0)) = \sqrt{(x-y)^2 + 0 ^2} = \vert x - y \vert, 
    \] so it is obvious that \(d\) is a metric. 
    
    Note that \(X\) is not open in \(\mathbb{R} ^2\). Also, if \(E = \left\{ (x, 0) \mid -1 < x < 1 \right\} \), then \(E\) is not open in \(\mathbb{R} ^2\), but \(E\) is open in \(\left( X, d_2 \vert_{X \times X} \right) \).       
\end{explanation}

\begin{eg}
    Suppose \(X = (-1, 1) \subseteq \mathbb{R} \), then \(\left( X, d \vert_{X \times X} \right) \) is a metric space. Consider \(E = [0, 1)\), then we know \(E\) is not closed in \((\mathbb{R} , d)\) since \(1 \notin \overline{E} \). But \(E\) is closed in \(\left( X, d\vert_{X \times X} \right) \) since \(\overline{E} = X\) in \((X, d\vert_{X \times X})\).          
\end{eg}

\begin{definition}[relatively open/close]
    Let \((X, d)\) be a metric space and \(Y \subseteq X\). We say \(E\) is relatively open (resp. closed) in \(Y\) if \(E\) is open (resp. closed) in \((Y, d\vert_{Y \times Y})\).     
\end{definition}

\begin{note}
    In the following context, if we say \(E\) is open in \(Y\), then we mean \(E\) is "relatively" open, and if we say \(E\) is closed in \(Y\), then we mean \(E\) is relatively closed in \(Y\).       
\end{note}

\begin{note}
    If \(Y\) is open/closed in \(E\), then \(Y \subseteq E\). Otherwise, we cannot define \(d\vert_{Y \times Y}(a, b)\) for \(a, b \in E\setminus Y\).     
\end{note}

\begin{remark} \label{rmk: subball} If \(Y \subseteq X\), and \((X, d), \left( Y, d\vert_{Y \times Y} \right) \) are both metric spaces, then 
   \[
            B_Y(x, r) = \left\{ y \in Y \mid d(y, x) < r \right\} = B_X(x, r) \cap Y. 
    \] 
\end{remark}

\begin{remark}
    If \(E\) is relatively open in \(Y\), then given \(x_0 \in E\), \(\exists r_0 > 0\) s.t. \(B_X(x_0, r_0) \cap Y \subseteq E\). This is because by \autoref{rmk: subball}, we have 
    \[
        B_X(x_0, r_0) \cap Y = B_Y(x_0, r_0) \subseteq E.
    \]   
\end{remark}
\begin{remark}
    A set \(E \subseteq Y\) is relatively closed in \(Y\) if given any \(r>0\) and \(x_0 \in Y\),
    \[
        B_Y(x_0, r) \cap E \neq \varnothing,
    \] then \(x_0 \in E\). This is because "closed" gives \(E = \overline{E}_Y \). Note that this statement is equivalent to 
    \begin{center}
        If \(x_0 \in \overline{E}_Y \), then \(x_0 \in E = E_Y\).  
    \end{center}
\end{remark}

\begin{proposition}
    Let \((X, d)\) be a metric space, and \(Y \subseteq X\) and \(E \subseteq Y\), then 
    \begin{itemize}
        \item [(1)] \(E\) is relatively open in \(Y\) iff \(\exists \) open set \(V\) in \((X, d)\) s.t. \(E = V \cap Y\).      
        \item [(2)] \(E\) is relatively closed in \(Y\) iff \(\exists \) closed set \(K\) in \((X, d)\) s.t. \(E = K \cap Y\).     
    \end{itemize}   
\end{proposition}
\begin{proof}[proof of (1)]
    \vphantom{text}
    \begin{itemize}
        \item [\((\implies )\)] Given any \(x \in E\), \(\exists r_x > 0\) s.t. \(B_X(x, r_x) \cap Y \subseteq E\). Let \(V = \bigcup_{x \in E} B_X(x, r_x) \). Obviously, \(V \cap Y = E\) and \(V\) is open.    
        \item [\((\impliedby )\)] Suppose \(E = V \cap Y\), then given any \(x \in E\), since \(V\) is open, so there exists \(r > 0\) s.t. \(B_X(x, r) \subseteq V\), and then \(B_X(x, r) \cap Y \subseteq V \cap Y = E\). Since \(x\) is an interior point of \(E\) in \(Y\), so \(\mathrm{Int}_Y(E) = E\), and thus \(E\) is open in \(Y\).             
    \end{itemize}
\end{proof}
\begin{proof}[proof of (2)]
    \vphantom{text}
    \begin{itemize}
        \item [\((\implies )\)] \(E\) is relatively closed in \(Y\), then \(Y \setminus E\) is relatively open, so there exists \(V\) open in \(X\) s.t. \(Y \setminus E = V \cap Y\). Hence,
        \begin{align*}
            E &= Y \setminus (Y \setminus E) = (X \setminus (Y \setminus E)) \cap Y = \left( X \setminus (V \cap Y) \right) \cap Y \\
            &= \left( (X \setminus V) \cup (X \setminus Y) \right) \cap Y \\
            &= ((X \setminus V) \cap Y ) \cup ((X \setminus Y) \cap Y) \\
            &= (X \setminus V) \cap Y
        \end{align*}
        Let \(E = (X \setminus V) \cap Y = K \cap Y\), then since \(K = X \setminus V\) is closed in \(X\), so we're done.   
        \item [\((\impliedby )\)] Suppose \(E = K \cap Y\) for some closed \(K\), then \(Y \setminus E = (X \setminus K) \cap Y\), which means \(Y \setminus E\) is relatively open in \(Y\) since \(X \setminus K\) is open and by (a), so \(E\) is closed in \(Y\).        
    \end{itemize}
\end{proof}

\begin{eg}
    Let \(X = [0, 1] \cup [2, 3] \subseteq \mathbb{R} \) with the standard metric \(d(x,y) = \vert x-y \vert \) with \(x,y \in X\), then 
    \begin{itemize}
        \item [(i)] \([0, 1]\) is open and closed in \(X\). 
        \item [(ii)] \(\partial _X [0, 1] = \varnothing \).  
    \end{itemize}  
\end{eg}
\begin{explanation}
    \vphantom{text}
    \begin{itemize}
        \item [(i)] We want to find \(V\) open in \(\mathbb{R} \) s.t.
    \[
        [0, 1] = V \cap \overbrace{([0, 1] \cup [2, 3])}^{X},
    \]  we can choosed \(V = \left( -\frac{1}{2}, \frac{3}{2} \right) \), so \([0, 1]\) is open in \(X\). 
    
    We want to find \(K\) closed in \(\mathbb{R} \) and 
    \[
        [0, 1] = K \cap \left( [0, 1] \cup [2, 3] \right),
    \]and we can choosed \(K = \left[ -\frac{1}{2}, \frac{3}{2}\right] \), so \([0, 1]\) is closed in \(X\).
        \item [(ii)] If \(x \in \partial _X [0, 1]\), then \(B_X(x, r) \cap [0, 1]\) and \(B_X(x, r) \cap [2, 3]\) are both nonempty for any \(r>0\). However, this is impossible for any \(x\) in \(X\), so \(\partial _X[0, 1] = \varnothing \).  
    \end{itemize}
\end{explanation}
\section{Cauchy sequence and complete metric space}
\begin{definition}[subsequence]
    Suppose \(\left( X^{(n)} \right)_{n=m}^{\infty}  \) is a sequence in \((X, d)\). Suppose \(m \le n_1 < n_2 < \dots \), then \(\left( X^{(n_j)} \right)_{j=1}^{\infty}  \) is called a subsequence of \(\left( X^{(n)} \right)_{n=m}^{\infty}  \).   
\end{definition}

\begin{eg}
    \(X^{(n)} = (-1)^n\) for all \(n \in \mathbb{N} \).  
\end{eg}
\begin{explanation}
    \[
        \left\{ X^{(2n)} \right\}_{n=1}^{\infty } 
    \]is a subsequence of \(\left\{ X^{(n)} \right\}_{n=1}^{\infty}  \). 
\end{explanation}

\begin{lemma}
    Let \(\left\{ X^{(n)} \right\}_{n=m}^{\infty}  \) be a convergent sequence with \(\lim_{n \to \infty} X^{(n)} = x \), then every subsequence of \(\left\{ X^{(n)} \right\}_{n=m}^{\infty}  \) also converges to \(x_0\).    
\end{lemma}

\begin{definition}[limit points]
    Suppose \(\left( X^{(n)} \right)_{n=m}^{\infty}  \) is a sequence in \((X, d)\), then we say \(L\) is a limit point of \(\left( X^{(n)} \right)_{n=m}^{\infty}  \) if for every \(N \ge m\) and every \(\varepsilon > 0\), there exists \(n \ge N\) s.t. \(d\left( X^{(n)}, L \right) \le \varepsilon  \).        
\end{definition}

\begin{proposition}
    \(L\) is a limit point of \(\left( X^{(n)} \right)_{n=m}^{\infty}  \) iff there exists a subsequence 
    \[
        \left( X^{(n_j)} \right)_{j=1}^n 
    \]  converges to \(L\). 
\end{proposition}
\begin{proof}
    \vphantom{text}
    \begin{itemize}
        \item [\((\implies )\)] Assume \(L\) is a limit point, now we build a subsequence converges to \(L\) by an inductive method. Our goal is to build a subsequence \(\left\{ X^{(n_j)} \right\} _{j=1}^{\infty} \) so that 
        \[
            d\left( X^{(n_j)}, L \right) < \frac{1}{j} \quad \forall 1 \le j.
        \]
        For \(j=1\), pick \(N = m\), and pick \(\varepsilon < \frac{1}{1}\) to pick \(n_1 \ge N\) s.t.
        \[
            d\left( X^{(n_1)}, L \right) \le \varepsilon < \frac{1}{1}. 
        \] Now suppose \(n_1, n_2, \dots , n_{k-1}\) are all chosen, then now we can pick \(N = n_{k-1} + 1\) and \(\varepsilon < \frac{1}{k}\), so that we can pick \(n_k \ge N\) s.t. \(d\left( X^{(n_k)}, L \right) \le \varepsilon < \frac{1}{k}\), so we're done. Now we show that this subsequence converges to \(L\). For every \(\varepsilon > 0\), we know there exists \(0 < \frac{1}{k} < \varepsilon \), so for all \(K \ge k\), we have 
        \[
            d\left( X^{(K)}, L \right) < \frac{1}{K} \le \frac{1}{k} < \varepsilon,
        \] so we're done.
        \item [\((\impliedby)\)] Left as exercise to the reader.
    \end{itemize}
\end{proof}

\begin{proposition}
    \(L\) is a limit point iff \(L \in \bigcap_{N=1}^{\infty} \overline{S_N}  \) where \(S_N  = \left\{ X^{(K)} \right\}_{K \ge N} \). 
\end{proposition}

\begin{definition}[Cauchy sequence]
    Let \(\left( X^{(n)} \right)_{n=m}^{\infty}  \) be a sequence in \((X, d)\). We say this sequence is a Cauchy sequence if for every \(\varepsilon > 0\), there exists \(N \ge m\) s.t. \(d\left( X^{(j)}, X^{(k)} \right) < \varepsilon  \) for all \(j, k \ge N\).      
\end{definition}

\begin{lemma} \label{lm: converge equal Cauchy}
    Suppose \(\left( X^{(n)} \right)_{n=m}^{\infty}  \) converges in \((X, d)\), then \(\left( X^{(n)} \right)_{n=m}^{\infty}  \) is a Cauchy sequence in \((X, d)\).  
\end{lemma}
\begin{proof}
    Suppose \(\lim_{n \to \infty} X^{(n)} = X_0\), then for every \(\frac{\varepsilon}{2} > 0\), there exists \(N \ge m\) s.t. \(d\left( X^{(n)}, X_0 \right) < \frac{\varepsilon}{2}\) for all \(n \ge N\). If \(j, k \ge N\), then 
    \[
        d\left( X^{(j)}, X^{(k)} \right) \le d\left( X^{(j)}, X_0 \right) + d\left( X^{(k)}, X_0 \right) <\frac{\varepsilon}{2} + \frac{\varepsilon}{2} = \varepsilon.   
    \]      
\end{proof}

\begin{eg}
A sequence in \(\mathbb{Q} \) may not converges in \(\mathbb{Q} \).  
\end{eg}
\begin{explanation}
    See teacher's note.
\end{explanation}

\begin{definition}[Complete space] \label{def: complete}
    A metric space \((X, d)\) is complete iff every Cauchy sequence converges to some points in \(X\). 
\end{definition}

\begin{remark}
    \( \mathbb{Q} \subseteq \mathbb{R} \), then \((\mathbb{Q} ,d)\) is not complete. 
\end{remark}

\begin{remark}
    The limit of a convergent sequence in metric space is unique. If 
    \[
        \lim_{n \to \infty} x^{(n)} = y \quad \text{and} \quad \lim_{n \to \infty} x^{(n)} = z,
    \] then suppose by contradiction, \(y \neq z\). Then, 
    \begin{align*}
        0 \le d(y, z) \le d\left( y, x^{(n)} \right) + d\left( z, x^{(n)} \right) .
    \end{align*} 
    By squeeze theorem, we know \(d(y, z) = 0\) and thus \(y = z\).  
\end{remark}

\begin{proposition}
    Let \((X, d)\) be a metric space and let \(\left( Y, d\vert_{Y \times Y} \right) \) be a subspace of \((X, d)\). If \(\left( Y, d\vert_{Y \times Y} \right) \) is complete, then \(Y\) is closed in \(X\).
\end{proposition}
\begin{proof}
    We want to show that \(Y = \overline{Y} \), so we want to show for all \(y \in \overline{Y} \), we have \(y \in Y\). Now for every \(y \in \overline{Y} \), then by \autoref{prop: adherent point TFAE}, we know there exists a convergent sequence \(\left\{ Y^{(n)} \right\}_{n=1}^{\infty}  \) in \(Y\) and converges to \(y\). However, every convergent sequence is Cauchy, and since \((Y, d\vert_{Y \times Y})\) is complete, so \(\left\{ Y^{(n)} \right\}_{n=1}^{\infty}  \) converges in \(Y\), which means \(y \in Y\), and we're done.        
\end{proof}

\begin{proposition}
    If \((X, d)\) is complete and \(Y \subseteq X\) is closed, then \(\left( Y, d\vert_{Y \times Y} \right) \) is complete.   
\end{proposition}
\begin{proof}
    Given a Cauchy sequence \(\left( X^{(n)} \right)_{n=1}^{\infty}  \) in \(Y\), so this is also a Cauchy sequence in \(X\), so it converges in \(X\). If \(\exists x_0 \in X\) s.t. \(\lim_{n \to \infty} X^{(n)} = x_0 \). Since \(Y\) is closed, so \(Y = \overline{Y} \), and by \autoref{prop: adherent point TFAE}, we know \(x_0 \in \overline{Y} = Y \), so \(x_0 \in Y\), and thus \(\left( X^{(n)} \right)_{n=1}^{\infty }\) also converges in \(Y\).         
\end{proof}
