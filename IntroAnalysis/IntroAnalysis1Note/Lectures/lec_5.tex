\lecture{5}{16 Sep. 10:20}{}
\begin{definition}
    Let \((X,d)\) be a metric space, \(E \subseteq X\) and \(x_0 \in E\). We say \(x_0\) is an adherent point if for every \(r > 0\), \(B(x_0, r) \cap E \neq \varnothing \), and we denote \(\overline{E} \) to the set of all adherent points.      
\end{definition}
\vphantom{text}
\begin{remark} \label{rmk: E in Ebar}
    \(E \subseteq \overline{E} \), since given any \(x_0 \in E\) and \(r > 0\), \(x_0 \in B(x_0, r)\), so \(B(x_0, r) \cap E \neq \varnothing \), and thus \(E \subseteq \overline{E} \).      
\end{remark}
\begin{remark} \label{rmk: pt E in Ebar}
    \(\partial E \subseteq \overline{E} \). Given \(x_0 \in \partial E\), we know for any \(r > 0\), \(B(x_0, r) \cap E \neq \varnothing \), so \(x_0 \in \overline{E} \).    
\end{remark}
\begin{proposition}
    \(x_0 \in \overline{E} \) if and only if there exists \(\left( X^{(n)} \right)_{n=1}^{\infty} \subseteq E\) s.t. \(\lim_{n \to \infty} X^{(n)} \) exists and \(\lim_{n \to \infty} X^{(n)} = x_0\).    
\end{proposition}
\begin{proof}[proof of \((\implies )\)]
    Given \(n \in \mathbb{N} \). Consider \(B\left( x_0, \frac{1}{n} \right) \). We know \(B\left( x_0, \frac{1}{n} \right) \cap E \neq \varnothing  \). Choose \(X^{(n)} \in B\left( x_0, \frac{1}{n} \right) \cap E \), then \(d\left( x_0, X^{(n)} \right) < \frac{1}{n} \), which means \(\lim_{n \to \infty} d\left( x_0, X^{(n)} \right) = 0  \). Hence, there exists \(\left( X^{(n)} \right) \subseteq E \) s.t. \(\lim_{n \to \infty} X^{(n)} = x_0 \).     
\end{proof}
\begin{proof}[proof of \((\impliedby)\)]
    There exists \(N\) s.t. \(X^{(n)} \in B(x_0, r)\) when \(n \ge N\). Given any \(r > 0\), since \(\lim_{n \to \infty} X^{(n)} = x_0 \), so \(\lim_{n \to \infty} d \left( X^{(n)}, x_0 \right) = 0  \). Hence, there exists \(N\) s.t. \(d\left( X^{(n)}, x_0 \right) < r \) when \(n \ge N\). Hence, when \(n \ge N\), we have \(X^{(n)}\subseteq B(x_0, r)\). Since we know \(X^{(n)} \in E\) for all \(n\), so we know \(B(x_0, r) \cap E \neq \varnothing \), so \(x_0 \in \overline{E} \).               
\end{proof}

\begin{proposition}
    Let \((X, d)\) be a metric space and \(E \subseteq X\), then 
    \[
        X = \mathrm{Int}(E) \cupdot \mathrm{Ext}(E) \cupdot \partial E. 
    \]  
\end{proposition}

\begin{corollary}
    Let \((X,d)\) be a metric space and \(E \subseteq X\). Then, 
    \[
        \overline{E} = \mathrm{Int}(E) \cup \partial E = X \setminus \mathrm{Ext}(E) = E \cup \partial E.    
    \] 
\end{corollary}
\begin{proof}
    Since
    \begin{align*}
        \overline{E} &= \overline{E} \cap X = \overline{E} \cap (\mathrm{Int}(E) \cup \mathrm{Ext}(E) \cup \partial E) \\
        &= \left( \overline{E} \cap \mathrm{Int}(E)   \right) \cup \left( \overline{E} \cap \mathrm{Ext}(E)   \right) \cup \left( \overline{E} \cap \partial E  \right) = \mathrm{Int}(E) \cupdot \partial E.   
    \end{align*}
    Also,
    \[
        X \setminus \mathrm{Ext}(E) = \left( \mathrm{Int}(E) \cup \mathrm{Ext}(E) \cup \partial E   \right) \setminus \mathrm{Ext}(E) = \mathrm{Int}(E) \cup \partial E = \overline{E}.   
    \]  
    Besides, we know \(\mathrm{Int}(E) \subseteq E \subseteq \overline{E}  \), so 
    \[
        \overline{E} = \mathrm{Int}(E) \cup \partial E \subseteq E \cup \partial E.    
    \] Also, by \autoref{rmk: E in Ebar} and \autoref{rmk: pt E in Ebar}, we know \(E \cup \partial E \subseteq \overline{E} \), so we know \(\overline{E} = E \cup \partial E \).    
\end{proof}

\begin{definition}
    Let \((X, d)\) be a metric space and \(E \subseteq X\). We say \(E\) is open iff \(\partial E \cap E \neq \varnothing \). We say \(E\) is closed iff \(\partial E \subseteq E\).      
\end{definition}

\begin{proposition} \label{proposition: open set equivalent condition}
    \[
        E \text{ is open} \iff \mathrm{Int}(E) = E \iff X \setminus E \text{ is closed.} 
    \]
\end{proposition}

\begin{proof}[proof of \(E\) is open \(\iff \mathrm{Int}(E) = E\)]
    \vphantom{text}
    \begin{itemize}
        \item [\((\implies )\)] Since \(E\) is open, so \(\partial E \cap E = \varnothing \). Hence, 
        \begin{align*}
            E &= E \cap X = E \cap (\mathrm{Int}(E) \cup \mathrm{Ext}(E) \cup \partial E) \\
            &= (E \cap \mathrm{Int}(E) ) \cup (E \cap \mathrm{Ext}(E)) \cup (E \cap \partial E) = \mathrm{Int}(E) \cup (E \cap \partial E) = \mathrm{Int}(E)
        \end{align*}
        since \(E \cap \mathrm{Ext}(E) =\varnothing  \) and we know \(\partial E \cap E = \varnothing \).  
        \item [\((\impliedby )\)] Since \(\mathrm{Int}(E) = E \), and \(\mathrm{Int}(E) \cap \partial E = \varnothing \), so \(E \cap \partial E = \varnothing \), and thus \(E\) is open.    
    \end{itemize}
\end{proof}
\begin{proof}[proof of \(E\) is open \(\iff X\setminus E\) is closed]
    \vphantom{text}
    \begin{itemize}
        \item [\((\implies )\)]
            \(X = \mathrm{Int}(E) \cup \mathrm{Ext}(E) \cup \partial E\), so
            \[
                X \setminus E = (\mathrm{Int}(E) \cup \mathrm{Ext}(E) \cup \partial E ) \setminus \mathrm{Int}(E) = \mathrm{Ext}(E) \cup \partial E = \mathrm{Int}(X\setminus E) \cup \partial (X \setminus E). 
            \] 
            Hence, \(\partial (X \setminus E) \subseteq X\setminus E\), which means \(X \setminus E\) is closed. 
        \item [\((\impliedby )\)] \(X \setminus E\) is closed, then \(\partial (X \setminus E) \subseteq X \setminus E\), but \(\partial E = \partial (X \setminus E)\), so \(\partial E \subseteq X \setminus E\), and thus \(\partial E \cap E = \varnothing \).     
    \end{itemize}
\end{proof}

\begin{remark}
    If \(\partial E = \varnothing \), then \(E\) is open and closed.   
\end{remark}

\begin{definition}[Clopen] \label{def: clopen}
    If a set \(S\) is closed and open, then \(S\) is clopen.  
\end{definition}

\begin{remark}
    Let \((X, d)\) be a metric space, then \(\varnothing \) is clopen, and we can deduce that \(X\) is also clopen since \(X\) is the complement of \(\varnothing \) and we know \(S\) is open iff \(X\setminus S\) is closed.      
\end{remark}

\begin{remark}
    In \((\mathbb{R} , d)\), where \(d\) is the standard metric, then the only clopen set is \(\mathbb{R} \) or \(\varnothing \).    
\end{remark}

\begin{remark}
    Let \((X, d_{\mathrm{disc} })\) be the discrete metric space on \(X\). Let \(E\) be any set, then \(E\) is open and closed. Given \(x_0 \in E\), we know \(B_{\mathrm{disc} }\left( x_0, \frac{1}{2} \right) \subseteq E\), so \(x_0 \in \mathrm{Int}(E) \), which means \(E = \mathrm{Int}(E) \) , so \(E\) is open. Now since \(X \setminus E\) is also open, so \(E\) is closed. Thus, \(E\) is clopen.        
\end{remark}

\begin{proposition} \label{prop: many properties of open and closed set}
    The following hold: 
    \begin{itemize}
        \item [(a)] \(E\) is open iff \(E = \mathrm{Int}(E) \). 
        \item [(b)] \(E\) is closed iff for every convergent sequence \(\left( X^{(n)} \right)_{n=1}^{\infty}  \) in \(E\), then the limit \(\lim_{n \to \infty} X^{(n)} \in E\). 
        \item [(c)] Let \(r > 0\), then 
        \begin{itemize}
            \item [(i)]         \[
            \overline{B}(x_0, r) = \left\{ x \in X \mid d(x, x_0) \le r \right\}  \text{ is closed.}
        \]
            \item [(ii)]        \[
            B(x_0, r) = \left\{ x \in X \mid d(x, x_0) < r \right\} \text{ is open.}
        \]
        \end{itemize}

        \item [(d)] Any singleton \(\left\{ x_0 \right\} \) where \(x_0 \in X\) is closed. 
        \item [(e)] \(E\) is open iff \(X\setminus E\) is closed. 
        \item [(f)]
        \begin{itemize}
            \item [(i)]
        If \(E_1, \dots , E_n\) are open sets in \(X\), then \(E_1 \cap E_2 \cap \dots \cap E_n\) is open.
            \item [(ii)] If \(F_1, \dots , F_n\) are closed, then \(F_1 \cup \dots \cup F_n\) is closed. 
        \end{itemize}
        \item [(g)]
        \begin{itemize}
            \item [(i)] If \(\left\{ E_\alpha  \right\}_{\alpha  \in I} \) is any collection of open sets in \(X\), then \(\bigcup_{\alpha \in I} E_\alpha  \) is open.
            \item [(ii)] If \(\left\{ F_\alpha  \right\}_{\alpha \in I} \) is any collection of closed sets in \(X\), then \(\bigcap_{\alpha  \in I} F_\alpha  \) is closed. 
        \end{itemize}
        \item [(h)] 
        \begin{itemize}
            \item [(i)] If \(E \subseteq X\), then \(\mathrm{Int}(E) \) is the largest open set that contained in \(E\) i.e. \(\mathrm{Int}(E) \) is open and if \(V \subseteq E\) and \(V\) is open, then \(V \subseteq \mathrm{Int}(E) \).
            \item [(ii)] If \(E \subseteq X\), then \(\overline{E} \) is the smallest closed set containing \(E\) i.e. \(\overline{E} \) is closed and if \(E \subseteq K\) and \(K\) is closed, then \(\overline{E} \subseteq K\). 
        \end{itemize}                   
    \end{itemize}
\end{proposition}
\begin{proof}[proof of (b)]
    \vphantom{text}
\begin{itemize}
    \item [\((\implies )\)] Since \(E\) is closed, so \(\overline{E} = E \), and we know every convergent sequence \(\left( X^{(n)} \right)_{n=1}^{\infty}  \) converges to \(x_0\) with \(x_0 \in \overline{E} \) by \autoref{prop: adherent point TFAE}. Thus, we have \(x_0 \in E\). 
    \item [\((\impliedby )\)] Assume that every convergent sequence in $E$ has its limit in $E$.  
We want to prove that $E$ is closed, i.e.\ that $X \setminus E$ is open.

Take any point $y \in X \setminus E$.  
Suppose, for contradiction, that every ball around $y$ meets $E$.  
That is, for each $k \in \mathbb{N}$ there exists a point 
\[
x^{(k)} \in E \cap B\!\left(y, \tfrac{1}{k}\right).
\]
Then, by construction, we have $x^{(k)} \to y$.  

By our assumption, the limit of any convergent sequence from $E$ 
must lie in $E$. Hence $y \in E$, contradicting the fact that $y \in X \setminus E$.

Therefore, there must exist some radius $r > 0$ such that
\[
B(y,r) \cap E = \varnothing,
\]
which means $B(y,r) \subseteq X \setminus E$.  
Thus every point of $X \setminus E$ is an interior point, so $X \setminus E$ is open.  
Hence $E$ is closed.
\end{itemize}
\end{proof}
\begin{proof}[proof of (c)]
    \vphantom{text}
    \begin{itemize}
        \item [(i)] To show that \(\overline{B}(x_0, r) \) is closed, it sufficies to show that \(X \setminus \overline{B}(x_0, r) \) is open. Note that 
    \[
        X \setminus \overline{B}(x_0, r) = \left\{ x \in X \mid d(x, x_0) > r \right\}. 
    \]      Let \(y \in X \setminus \overline{B}(x_0, r) \), then define \(\varepsilon = d(x_0, y) - r > 0\), then we can similarly prove that \(B(y, \varepsilon ) \subseteq X \setminus \overline{B}(x_0, r)\).    Hence, \(X \setminus \overline{B}(x_0, r) = \mathrm{Int}(X \setminus \overline{B}(x_0, r) ) \), and thus it is open. 
        \item [(ii)] If \(y \in B(x_0, r)\), then \(d(x_0, y) < r\). Let \(\varepsilon = r - d(x_0, y) > 0\), then we claim that \(B(y, \varepsilon ) \subseteq B(x_0, r)\).  Given \(z \in B(y, \varepsilon )\), then \(d(z, y) < \varepsilon \), then use triangle inequality we know \(z \in B(x_0, r)\).
    \end{itemize}   
\end{proof}
\begin{proof}[proof of (d)]
    It sufficies to show that \(X \setminus \left\{ x_0 \right\} \) is open. Given \(y \in X \setminus \left\{ x_0 \right\} \), so we can show that 
    \[
        B\left( y, \frac{d(y,x_0)}{2} \right) \subseteq X \setminus \left\{ x_0 \right\}. 
    \]
    Hence, \(y \in \mathrm{Int}(X \setminus \left\{ x_0 \right\} ) \), and thus \(X \setminus \left\{ x_0 \right\} \) is open.  
\end{proof}
\begin{proof}[proof of (f)]
    \vphantom{text}
    \begin{itemize}
        \item [(i)] Given \(x_0 \in E_1 \cap E_2 \cap \dots \cap E_n\), then \(x_0 \in E_i\) for all \(1 \le i \le n\). Thus, there exists \(r_i > 0\) s.t. 
    \[
        B(x_0, r_i) \subseteq E_i \quad \text{for each } 1 \le i \le n.
    \]   Let \(r = \min \left\{ r_1, \dots , r_n \right\} > 0\), then we know \(B(x_0, r) \subseteq B(x_0, r_i) \subseteq E_i\) for all \(1 \le i \le n\). Hence, \(B(x_0, r) \subseteq E_1 \cap E_2 \cap \dots \cap E_n\), and thus \(E_1 \cap \dots \cap E_n\) is open.
        \item [(ii)] Now if \(F_1, \dots , F_n\) are closed, then \(X\setminus F_1, \dots , X\setminus F_n\) are open. Since we know \(\bigcap_{i=1}^{n} (X \setminus F_i) \) is open, and 
    \[
        \bigcap_{i=1}^n \left( X \setminus F_i \right) = X \setminus \left( \bigcup_{i=1}^n F_i  \right),   
    \]
    so \(X\setminus \left( \bigcup_{i=1}^{n}F_i \right)  \) is open, which means \(\bigcup_{i=1}^{n}F_i\) is closed.
    \end{itemize}        
\end{proof}
\begin{proof}[proof of (g)]
    \vphantom{text}
    \begin{itemize}
        \item [(i)] Suppose \(x_0 \in \bigcup_{\alpha \in I} E_\alpha  \), then there exists \(\mathcal{B} \in I\) s.t. \(x_0 \in E_{\mathcal{B} }\). Now since \(E_{\mathcal{B} }\) is open, so there exists \(r_{x_0} > 0\) s.t. 
    \[
        B(x_0, r_{x_0}) \subseteq E_{\mathcal{B} } \subseteq \bigcup_{i \in \alpha } E_\alpha . 
    \] Hence, \(\bigcup_{\alpha \in I} E_\alpha  \) is open.
        \item [(ii)]     \[
        \left( X \setminus \left( \bigcap_{\alpha \in I} F_\alpha \right)  \right) = \bigcup_{\alpha \in I} \left( X \setminus F_\alpha \right) 
    \] is open since \(X \setminus F_\alpha\) is open for all \(\alpha \in I\), so we have \(\bigcap_{\alpha \in I} F_\alpha  \) is closed. 
    \end{itemize}

    \begin{remark}
        \vphantom{text}
        \begin{itemize}
            \item [(1)] \(\bigcap_{\alpha \in I} E_\alpha  \) may NOT be open. For example,
            \[
                \bigcap_{i=1}^{\infty} \left( - \frac{1}{n}, \frac{1}{n} \right) = \left\{ 0 \right\},
            \]
            which is closed. 
            \item [(2)] \(\bigcup_{\alpha \in I} F_\alpha \) may NOT be closed.  For example, 
            \[
                \bigcup_{i=1}^{\infty} \left[ -1 + \frac{1}{n}, 1 - \frac{1}{n}\right] = (-1, 1),
            \] which is open.
        \end{itemize}
    \end{remark}
    \begin{note}
        In the proof of (f), if the index set \(I\) is infinite, then we can not pick \(\min \left\{ r_1, \dots , r_n \right\} \), so we can not deduce that (f) is correct when there infinitely many open sets or closed sets.  
    \end{note}
\end{proof}
    \begin{proof}[proof of (h)]
        \vphantom{text}
        \begin{itemize}
            \item [(i)] 
            We first claim that \(\mathrm{Int}(E) \) is open. 
            \begin{proof}
                Since for all \(x \in \mathrm{Int}(E) \), \(\exists r_x > 0\) s.t. \(B(x, r_x) \subseteq E\), so   
        \[
            \mathrm{Int}(E) = \bigcup_{x \in \mathrm{Int}(E) } B(x, r_x),  
        \] and by (ii) of (c) and (i) of (g) in \autoref{prop: many properties of open and closed set}, we know \(\mathrm{Int}(E)\) is open.  
    \end{proof}
        Now if we have \(V \subseteq E\) and \(V\) is open, then \(y \in V\) implies there exists \(s > 0\) s.t. \(B(y, s) \subseteq V\), and thus \(B(y, s) \subseteq E\) since \(V \subseteq E\). Hence, we know \(y \in \mathrm{Int(E)} \), and thus \(V \subseteq \mathrm{Int}(E) \). 
        \item [(ii)] To show \(\overline{E} \) is closed, it sufficies to show that \(X \setminus \overline{E} \) is open. Note that
\[
    \overline{E} = X \setminus \mathrm{Ext}(E) = X \setminus \underbrace{ \mathrm{Int}(X \setminus E) }_{\text{open}},
\]  so \(\overline{E} \) is closed. Now if \(E \subseteq K\) and \(K\) is closed, then if \(x \in \overline{E} \), we have \(B(x, r) \cap E \neq \varnothing \) for all \(r > 0\). Hence, \(B(x, r) \cap K \neq  \varnothing \) since \(E \subseteq K\), so \(x \in \overline{K} = K \) (since \(K\) is closed). Thus, \(\overline{E} \subseteq K\). 
    \end{itemize}
           
\end{proof}

