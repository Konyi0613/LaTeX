\lecture{19}{11 Nov. 10:20}{}
\section{Multiplication of Power series}
\begin{theorem}[Fubuni's theorem for infinite sum, Analysis I] \label{thm: double series absolutely conv then can exchange index}
    Let \(f: \mathbb{N} \times \mathbb{N} \to \mathbb{R} \) be a function such that the double series 
    \[
        \sum_{(n, m) \in \mathbb{N} \times \mathbb{N} } f(n, m) 
    \] 
    is absolutely convergent, i.e. 
    \[
        \sum_{(m, n) \in \mathbb{N} \times \mathbb{N} } \left\vert f(n, m) \right\vert  
    \] converges. Then 
    \[
        \sum_{n=1}^{\infty} \sum_{m=1}^{\infty} f(n, m) = \sum_{m=1}^{\infty} \sum_{n=1}^{\infty} f(n, m).    
    \]
    In particular, if \(\sum_{n, m} a_n a_m \) absolutely convergent, then
    \[
        \sum_{m} \sum_{n} a_n a_m = \sum_{n} \sum_{m} a_n a_m.    
    \] 
\end{theorem}

\begin{theorem}
    Let \(f: (a-r,a+r) \to \mathbb{R} \) and \(g : (a-r,a+r) \to \mathbb{R} \) be functions analytic on \((a-r,a+r)\), with power series expansions 
    \[
        f(x) = \sum_{n=0}^{\infty} c_n (x-a)^n, \quad \text{and } g(x) = \sum_{n=0}^{\infty} d_n (x-a)^n,    
    \]  then \(f \cdot g : (a-r,a+r) \to \mathbb{R} \) is also analytic on \((a-r, a+r)\), with power series expansion 
    \[
        f(x) g(x) = \sum_{n=0}^{\infty} e_n (x-a)^n,
    \] where 
    \[
        e_n = \sum_{m=0}^n c_m d_{n - m}. 
    \]
\end{theorem}
\begin{proof}
    We need to show that the series 
    \[\sum_{n=0}^{\infty} e_n (x-a)^n \]
    converges to \(f(x) g(x)\) for all \(x \in (a-r,a+r)\). 
    \begin{itemize}
        \item \textbf{Step 1:} Absolute convergence of \(f\) and \(g\). By our assumption \(f\) and \(g\) are analytic on \((a-r,a+r)\), so \(f(x) = \sum_{n=0}^{\infty} c_n (x-a)^n \) and \(g(x) = \sum_{n=0}^{\infty} d_n(x-a)^n \) converges absolutely when \(\vert x-a \vert < r \). In particular, for each fixed \(x \in (a-r, a+r)\), the sum 
        \[
            C \coloneqq \sum_{n=0}^{\infty} \left\vert c_m (x-a)^m \right\vert \quad \text{and} \quad D\coloneqq \sum_{n=0}^{\infty} \left\vert d_n (x-a)^n \right\vert   
        \] are finite. 
        \item \textbf{Step 2:} Bounding the double series. Consider any \(N \ge 0\), the sum 
        \begin{align*}
            \sum_{n=0}^N \sum_{m=0}^{\infty} \left\vert c_m (x-a)^m d_n (x-a)^n \right\vert &= \left\vert d_0 \right\vert \left( \sum_{m=0}^{\infty} \left\vert c_m (x-a)^m \right\vert   \right) + \dots + \left\vert d_N (x-a)^N\right\vert \left( \sum_{m=0}^{\infty} \left\vert c_m (x-a)^m \right\vert \right) \\
            &= \left( \sum_{n=0}^N \left\vert d_n (x-a)^n \right\vert   \right) \cdot \left( \sum_{m=0}^{\infty} \left\vert c_m (x-a)^m \right\vert   \right) \le D \cdot C,  
        \end{align*}
        where \(D, C\) are fixed constants independent of \(N\). This implies 
        \[
            \sum_{n=0}^{\infty} \sum_{m=0}^{\infty} \left\vert c_m (x-a)^m d_n (x-a)^n \right\vert   
        \] converges absolutely. By Fubini's theorem, 
        \[
            \sum_{m=0}^{\infty} \sum_{n=0}^{\infty} c_m d_n (x-a)^m (x-a)^n = \sum_{n=0}^{\infty} \sum_{m=0}^{\infty} c_m d_n (x-a)^m (x-a)^n.    
        \] 
        \item \textbf{Step 3:} First computations of the double series. We compute the sum in two different ways. Note that 
        \begin{align*}
            f(x) g(x) = \sum_{n=0}^{\infty} \sum_{m=0}^{\infty} c_m (x-a)^m d_n(x-a)^n.     
        \end{align*}
        \item \textbf{Step 4:} Second computation of the double series. 
        \begin{align*}
            f(x) g(x) &= \sum_{n=0}^{\infty} \sum_{m=0}^{\infty} c_m d_n (x-a)^{m+n} = \sum_{m=0}^{\infty} \sum_{n=0}^{\infty} c_m d_n (x-a)^{m+n}    
        \end{align*}
        by Fubini's theorem.
        \item \textbf{Step 5:} Change of variables. Let \(n^{\prime} = m + n\), so 
        \[
            f(x) g(x) = \sum_{m=0}^{\infty} \sum_{n^{\prime} = m}^{\infty} c_m d_{n^{\prime} - m} (x-a)^{n^{\prime} },  
        \] where we adopt the convention \(d_j = 0\) if \(j < 0\). Hence, 
        \begin{align*}
            f(x) g(x) &= \sum_{m=0}^{\infty} \sum_{n^{\prime} = m}^{\infty} c_m d_{n^{\prime} - m} (x-a)^{n^{\prime} } + \sum_{m=0}^{\infty} \sum_{n^{\prime} = 0}^{m-1} c_m d_{n^{\prime} - m} (x-a)^{n^{\prime} } \\
            &= \sum_{m=0}^{\infty} \sum_{n^{\prime} = 0}^{\infty} c_m d_{n^{\prime} - m} (x-a)^{n^{\prime} }    
        \end{align*}  
        \item \textbf{Step 6:} Interchanging the order of summation. Since we have shown that this series converges absolutely (before changing subscript and change the order of double series, but this does not affect the convergence), so by Fubini's theorem we have
        \[
            f(x) g(x) = \sum_{n^{\prime} = 0}^{\infty} \left( \sum_{m=0}^{\infty} c_m d_{n^{\prime} - m}  \right)(x-a)^{n^{\prime} } = \sum_{n^{\prime} = 0}^{\infty} \left( \sum_{m=0}^{n^{\prime} } c_m d_{n^{\prime} - m}  \right) (x-a)^{n^{\prime} }    
        \]
    \end{itemize}  
\end{proof}

\section{The exponential and Logarithm functions}
\begin{definition}
    For every real number \(x\), we define 
    \[
        \exp (x) \coloneqq \sum_{n=0}^{\infty} \frac{x^n}{n!}. 
    \] 
\end{definition}

\begin{theorem}
    The exponential function satisfies the following properties: 
    \begin{itemize}
        \item [(a)] For every \(x \in \mathbb{R} \), \(\sum_{n=0}^{\infty} \frac{x^n}{n!} \) is absolutely convergent. The radius of convergence of \(\sum_{n=0}^{\infty} \frac{x^n}{n!} \) is \(\infty \), so \(\exp (x)\) is analytic on \((-\infty , \infty )\). 
        \item [(b)] \(\exp (x)\) is differentiable on \(\mathbb{R} \) and \(\left( \exp (x) \right)^{\prime} = \exp (x) \). 
        \item [(c)] \(\exp \) is continuous on \(\mathbb{R} \) and for every interval \([a, b]\), we have 
        \[
            \int _a^b \exp (x) \, \mathrm{d} x = \exp (b) - \exp (a). 
        \]
        \item [(d)] For \(x, y \in \mathbb{R} \), we have \(\exp (x + y) = \exp (x) \exp (y)\). 
        \item [(e)] We have \(\exp (0) = 1\), \(\exp (x) > 0\) for all \(x\) and \(\exp (-x) = \frac{1}{\exp (x)}\). 
        \item [(f)] \(\exp (x) > \exp (y)\) if \(x > y\).        
    \end{itemize}
\end{theorem}
\begin{proof}[proof of (a)]
    By ratio test, 
    \[
        \lim_{n \to \infty} \left\vert \frac{\frac{x^{n+1}}{(n+1)!}}{\frac{x^n}{n!}} \right\vert = \lim_{n \to \infty} \left\vert \frac{n! x}{(n+1)!} \right\vert = \lim_{n \to \infty} \left\vert \frac{x}{n+1} \right\vert = 0      
    \] for every \(x \in \mathbb{R} \), so it is absolutely convergent for every \(x \in \mathbb{R} \), and the radius of convergence is \(\infty \) by definition, so \(\exp (x) = \sum_{n=0}^{\infty} \frac{x^n}{n!} \) for all \(x \in \mathbb{R} \) and \(\exp (x)\) is analytic on \(\mathbb{R} \).     
\end{proof}

\begin{proof}[proof of (b)]
    Since \(\exp (x)\) is analytic on \(\mathbb{R} \), so it is infinitely differentiable on \(\mathbb{R} \), and    
\begin{align*}
    \left( \exp (x) \right)^{\prime} &= \frac{\mathrm{d}}{\mathrm{d}x} \left( \sum_{n=0}^{\infty} \frac{x^n}{n!}  \right) = \sum_{n=0}^{\infty} \left( \frac{\mathrm{d}}{\mathrm{d}x} \frac{x^n}{n!}  \right) = \sum_{n=1}^{\infty} \frac{n x^{n-1}}{n!} = \sum_{n=1}^{\infty} \frac{x^{n-1}}{(n-1)!} = \exp (x)    
\end{align*}
for all \(x \in \mathbb{R} \). 
\end{proof}

\begin{proof}[proof of (c)]
    Since \(\exp \) is differentiable, so it is continuous.  
    Since 
    \[
        \frac{\mathrm{d}}{\mathrm{d}x} \exp (x) = \exp (x), 
    \] so by Fundamental Theorem of Calculus, we know 
    \[
        \int _a^b \exp (x) \, \mathrm{d} x = \exp (b) - \exp (a). 
    \]
\end{proof}

\begin{proof}[proof of (d)]
    \begin{align*}
        \exp (x) \exp (y) &= \left( \sum_{m=0}^{\infty} \frac{x^m}{m!}  \right) \left( \sum_{n=0}^{\infty} \frac{y^n}{n!}  \right) \\
        &= \sum_{k=0}^{\infty} \sum_{m=0}^k \frac{1}{m! (k-m)!} x^m y^{k-m} = \exp (x+y)    
    \end{align*}
    by binomial theorem.
\end{proof}

\begin{proof}[proof of (e)]
    \(\exp (0)\) is trivial. If \(x > 0\), then 
    \[
        \exp (x) = \sum_{n=0}^{\infty} \frac{x^n}{n!} > 1, 
    \] and since \(\exp (x + (-x)) = \exp (0) = 1\), and thus by (d) we know 
    \[
        \exp (x) \exp (-x) = \exp (x + (-x)) = \exp (0) = 1.
    \]
\end{proof}

\begin{proof}[proof of (f)]
    Trivial.
\end{proof}

\begin{definition}[Euler number]
    \[
        e \coloneqq \exp (1) = \sum_{n=0}^{\infty} \frac{1}{n!} = 1 + \frac{1}{1!} + \frac{1}{2!} + \dots  
    \]
\end{definition}

\begin{proposition}
    \[
        \exp (x) = e^x \quad \forall x \in \mathbb{R} .
    \]
\end{proposition}

\begin{proof}
    If \(n \in \mathbb{Z} \), then \(\exp (n) = e^n\), which can be shown by induction. When \(n = 0\), this is trivial. If \(n > 0\), then 
    \begin{align*}
        \exp (n) &= \exp (1 + 1 + \dots + 1) = \exp (1) \exp (1) \dots \exp (1) = \left( \exp (1) \right)^n = e^n, 
    \end{align*}    
    so this is true. Now if \(n < 0\), then \(\exp (-n) = e^{-n}\), and thus 
    \[
        \exp (n) = \frac{1}{\exp (-n)} = \frac{1}{e^{-n}} = e^n.
    \] 
    Next, we show \(\exp \left( \frac{p}{q} \right) = e^{\frac{p}{q}} \) for \(p \in \mathbb{Z} \) and \(q \in \mathbb{N} \). Note that we have 
    \[
        \left( \exp \left( \frac{p}{q} \right)  \right)^q = \exp \left( \frac{p}{q} + \dots + \frac{p}{q} \right) = \exp (p) = e^p,  
    \] so 
    \[
        \exp \left( \frac{p}{q} \right) = \left( e^p \right)^{\frac{1}{q}} = e^{\frac{p}{q}}.   
    \] Hence, \(\exp (x) = e^x\) for \(x \in \mathbb{Q} \). Now if \(x \in \mathbb{R} \setminus \mathbb{Q} \), then there exists \((x_n) \in \mathbb{Q} \) s.t. \(\lim_{n \to \infty} x_n = x \). Since \(\exp(x) \) is continuous, so 
    \[
        \lim_{n \to \infty} \exp (x_n) = \exp(x), 
    \] and since \(x_n \in \mathbb{Q} \), so \(\lim_{n \to \infty} e^{x_n} = \exp (x) \). Also, since \(e^x\) is continuous, so 
    \[
        \lim_{n \to \infty} e^{x_n} = e^x, 
    \] so combining these two results we have \(\exp (x) = e^x\). 
\end{proof}