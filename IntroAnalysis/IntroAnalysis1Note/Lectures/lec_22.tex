\lecture{22}{20 Nov. 10:20}{}
\begin{definition}
    We define \(\pi \) to be the smallest positive zero(零根) of \(\sin (x)\), i.e. 
    \[
        \pi \coloneqq \inf \left\{ x \in (0, \infty ) : \sin (x) = 0 \right\}. 
    \]  
    Hence, \(\sin (x) > 0\) on \((0, \pi)\) since \(\sin (x)\) is increasing near \(0\) and continuous.    
\end{definition}

By definition, \(\sin (\pi ) = 0\) and \(\sin ^2(\pi ) + \cos ^2 (\pi ) = 1\), so \(\cos \pi = \pm 1\). However, 
\[
    \frac{\mathrm{d}}{\mathrm{d}x} \cos (x) = -\sin (x) < 0 
\] on \((0, \pi )\), so \(\cos (x)\) is decreasing on \((0, \pi )\), and since \(\cos (0) = 1\), so we know \(\cos (\pi ) = -1\). 

\begin{proposition}
    \(\sin \left( \frac{\pi}{2} \right) = 1 \) and \(\cos \left( \frac{\pi}{2} \right) = 0 \).  
\end{proposition}
\begin{proof}
    We know \(\cos (\pi ) = -1\), and since 
    \[
        \cos (\pi - x) = \cos (\pi ) \cos (x) + \sin (\pi ) \sin (x) = -\cos (x),
    \] so by plugging \(x = \frac{\pi}{2}\) we know \(\cos \left( \frac{\pi}{2} \right) = -\cos \left( \frac{\pi}{2} \right)  \), so \(\cos \left( \frac{\pi}{2} \right) = 0 \). Also, since 
    \[\sin ^2 \left( \frac{\pi}{2} \right) + \cos ^2 \left( \frac{\pi}{2} \right) = 1,  \] so \(\sin \left( \frac{\pi}{2} \right) = \pm 1 \). Since \(\sin (x) > 0\) on \(\left( 0, \pi \right) \), so \(\sin \left( \frac{\pi}{2} \right) = 1 \).
\end{proof}

\begin{theorem}
    Let \(x \in \mathbb{R} \). Then \(\sin (x)\) and \(\cos (x)\) has the following properties: 
    \begin{itemize}
        \item [(1)] 
        \begin{align*}
            &\cos (x + \pi ) = -\cos (x), \quad \sin (x + \pi ) = -\sin (x) \\
            &\cos (x + 2\pi ) = \cos (x), \quad \sin (x + 2\pi ) = \sin (x).
        \end{align*}
        \item [(2)] \(\sin (x) = 0\) iff \(\frac{x}{\pi } \in \mathbb{Z} \), i.e. \(x = k \pi \) for \(k \in \mathbb{Z} \).    
        \item [(3)] \(\cos (x) = 0\) iff \(\frac{x}{\pi } = k + \frac{1}{2}\) where \(k \in \mathbb{Z} \), i.e. \(x = \left( k + \frac{1}{2} \right)\pi  \) for \(k \in \mathbb{Z} \).     
    \end{itemize}   
\end{theorem}
\begin{proof}[proof of (1)]
    Note that 
    \begin{align*}
        \cos (x + \pi ) &= \cos (x) \cos (\pi ) - \sin (x) \sin (\pi ) = - \cos (x) \\
        \sin (x + \pi ) &= \sin (x) \cos (\pi ) + \sin (\pi ) \cos (x) = -\sin (x).
    \end{align*}
    Also, we have 
    \begin{align*}
        \cos (x + 2\pi ) &= \cos \left( (x + \pi ) + \pi  \right) = -\cos (x + \pi ) = (-1) \cdot (-1) \cdot \cos (x) = \cos (x) \\
        \sin (x + 2\pi ) &= \sin \left( (x + \pi ) + \pi  \right) = -\sin (x + \pi ) = (-1) \cdot (-1) \cdot \sin (x) = \sin (x).  
    \end{align*}
\end{proof}

\chapter{Fourier series}

In the previous two chapters, we discussed the issue of how certain functions
(for instance, compactly supported continuous functions) could be approximated by polynomials. Later, we showed how a different class of functions
(real analytic functions) could be written exactly (not approximately) as an
infinite polynomial, or more precisely a power series.
Power series are already immensely useful, especially when dealing with
special functions such as the exponential and trigonometric functions discussed earlier. However, there are some circumstances where power series
are not so useful, because one has to deal with functions (e.g. \(\sqrt{x} \)) which are
not real analytic, and so do not have power series.
Fortunately, there is another type of series expansion, known as Fourier
series, which is also a very powerful tool in analysis (though used for slightly
different purposes). Instead of analyzing compactly supported functions, it
instead analyzes periodic functions; instead of decomposing into polynomials, it decomposes into trigonometric polynomials. Roughly speaking, the
theory of Fourier series asserts that just about every periodic function can
be decomposed as an (infinite) sum of sines and cosines.

\begin{definition}
    Let \(L > 0\) and \(L \in \mathbb{R} \). A complex function \(f: \mathbb{R} \to \mathbb{C} \) is \(L\)-periodic if \(f(x + L) = f(x)\) for all \(x \in \mathbb{R} \).      
\end{definition}

\begin{remark}
    \(f(x + L) = f(x)\) for all \(x\) implies \(f(x + kL) = f(x)\) for all \(k \in \mathbb{Z} \).   
\end{remark}

\begin{eg}
    For any integer \(n \in \mathbb{Z} \), \(\cos (2\pi n x)\), \(\sin (2 \pi n x)\), and 
    \[
        e^{2\pi i n x} = \cos (2\pi n x) + i \sin (2 \pi n x)
    \] all has period \(1\). 
\end{eg}

\begin{remark}
    A \(1\)-periodic function is also called \(\mathbb{Z} \)-periodic since \(f(x+1) = f(x)\) means \(f(x+k)=f(x)\) for all \(k \in \mathbb{Z} \).     
\end{remark}

\begin{remark}
    If \(f\) is \(L\)-periodic, then \(g(x) = f(Lx)\) is \(1\)-periodic. Conversely, if \(g(x)\) is \(1\)-periodic, then \(f(x) \coloneqq g \left( \frac{x}{L} \right) \) is \(L\) -periodic.        
\end{remark}

\begin{remark}
    Suppose \(f\) is \(\mathbb{Z} \)-periodic, then \(f(x+1)=f(x)\) for all \(x \in \mathbb{R} \). We regard \(f\) as functions on \(\mathbb{R} / \mathbb{Z} \). Now we explain what is \(\mathbb{R} / \mathbb{Z} \). On \(\mathbb{R} \), we define \(x \sim y\) iff \(x - y \in \mathbb{Z} \), and \(\sim \) is an equivalence relation, so 
    \[
        [x] = \left\{ x + k \mid k \in \mathbb{Z}  \right\}, 
    \] and thus 
    \[
        \mathbb{R} / \mathbb{Z} = \left\{ [x] \mid x \in \mathbb{R}  \right\} = \left\{ [y] \mid y \in [0, 1) \right\}.  
    \]         
    Hence, if \(f\) is \(\mathbb{Z} \)-periodic, then we can equivalently define \(f\) as 
    \[
        f: \mathbb{R} / \mathbb{Z} \to \mathbb{C}, \quad [y] \to f(y) 
    \] since \([y_1] = [y_2]\) iff \(y_1 - y_2 = k \in \mathbb{Z} \) iff \(f(y_1) = f(y_2 + k) = f(y_2)\).   
\end{remark}

\begin{definition}
    The set of continuous complex-valued \(1\)-periodic function on \(\mathbb{R} /\mathbb{Z} \) is denoted by \(C \left( \mathbb{R} / \mathbb{Z} , \mathbb{C}  \right) \), i.e. 
    \[
        C \left( \mathbb{R} /\mathbb{Z}, \mathbb{C}   \right) = \left\{ f \mid f:\mathbb{R} \to \mathbb{C} , f\text{ is continuous}, f(x+1)=f(x) \quad \forall x \in \mathbb{R}   \right\}.  
    \]   
\end{definition}

\begin{definition}
    Every function \(f \in C \left( \mathbb{R} / \mathbb{Z} , \mathbb{C}  \right) \) can be written as 
    \[
        f(x) = f^1(x) + i f^2(x)
    \] where \(f^1, f^2 \in C\left( \mathbb{R} / \mathbb{Z} , \mathbb{R}  \right) \), i.e. \(f^1, f^2\) are real-valued and continuous and \(1\)-periodic. Now for \(f = f^1 + i f^2\) and \(g = g^1 + i g^2\) in \(C \left( \mathbb{R} / \mathbb{Z} , \mathbb{C}  \right) \), we define 
    \[
        d_\infty (f, g) = \sup _{x \in \mathbb{R} } \left\vert f(x) - g(x) \right\vert = \sup _{x \in [0,1]} \vert f(x) - g(x) \vert = \sup _{x \in [0, 1]} \sqrt{\left( f^1(x) - g^1(x) \right)^2 + \left( f^2(x) - g^2(x) \right)^2  },  
    \]     
    and we know \(d_\infty \) define a metric on \(C \left( \mathbb{R} / \mathbb{Z} , \mathbb{C}  \right) \).  
\end{definition}

\begin{definition}
    We say \(f_n \to f\) uniformly on \(C(\mathbb{R} / \mathbb{Z} , \mathbb{C} )\) iff \(\lim_{n \to \infty} d_\infty (f_n, f) = 0 \).  
\end{definition}

\begin{theorem}
    On \(C(\mathbb{R} / \mathbb{Z} , \mathbb{C} )\), TFAE:
    \begin{itemize}
        \item [(a)] \(d_\infty (f_n, f) \to 0\) as \(n \to \infty \). 
        \item [(b)] \(f_n \to f\) uniformly on \(\mathbb{R} \).
        \item [(c)] \(f_n \to f\) uniformly on \([0, 1]\). 
        \item [(d)] \(\Re (f_n) \to \Re (f)\) uniformly and \(\Im (f_n) \to \Im (f)\) uniformly.        
    \end{itemize} 
\end{theorem}

\begin{lemma} \label{lm: properties for C(R/Z, C)}
    The following properties hold on \(C \left( \mathbb{R} / \mathbb{Z} , \mathbb{C}  \right) \): 
    \begin{itemize}
        \item [(a)] If \(f \in C(\mathbb{R} / \mathbb{Z} , \mathbb{C} )\), then \(f\) is bounded, i.e. \(\exists M > 0\) s.t. 
        \(\left\vert f(x) \right\vert \le M \) for all \(x \in \mathbb{R} \).  
        \item [(b)] If \(f, g \in C \left( \mathbb{R} / \mathbb{Z} , C \right) \), then \(f + g, f - g,\) and \(f \cdot g \in C (\mathbb{R} / \mathbb{Z} , \mathbb{C} )\). Also, for any \(c \in \mathbb{C} \), \(c \cdot f \in C(\mathbb{R} /\mathbb{Z} , \mathbb{C} )\). This implies that \(C (\mathbb{R} / \mathbb{Z} , \mathbb{C} )\) is a vector space over \(\mathbb{C} \) and commutative algebra. 
        \item [(c)] If \(\left\{ f_n \right\}_{n=1}^{\infty}  \) with \(f_n \in C \left( \mathbb{R} / \mathbb{Z} , \mathbb{C}  \right) \) and \(\lim_{n \to \infty} d_\infty (f_n, f) = 0\), then \(f \in C \left( \mathbb{R} / \mathbb{Z} , \mathbb{C}  \right) \). This implies that \((C (\mathbb{R} / \mathbb{Z} , \mathbb{C} ), d_\infty )\) is a complete metric space.
    \end{itemize} 
\end{lemma}
\begin{proof}[proof of (a)]
    Let 
    \[
        f(x) = f^1(x) + i f^2(x)
    \] where \(f^1, f^2: \mathbb{R} \to \mathbb{R} \) and continuous and \(1\)-periodic, then since
    \[
        \sup _{x \in \mathbb{R} } f^1(x) = \sup _{x \in [0, 1]} f^1(x)
    \] and \(f^1\) is continuous on \([0, 1]\), so there exists \(M_1\) s.t. \(\left\vert f^1(x) \right\vert \le M_1 \) for all \(x \in [0, 1]\). Similarly, we can show there exists \(M_2\) s.t. \(\left\vert f^2(x) \right\vert \le M_2\) for all \(x \in  [0, 1]\) and thus 
    \[
        \vert f(x) \vert = \left\vert f^1(x) + i f^2(x) \right\vert = \sqrt{\left( f^1(x) \right)^2 + \left( f^2(x) \right)^2  } \le \sqrt{M_1^2 + M_2^2},    
    \]      
    so \(\sup _{x \in \mathbb{R} }\vert f(x) \vert \le \sqrt{M_1^2 + M_2^2}  \). 
\end{proof}

\begin{proof}[proof of (b)]
    \todo{DIY}
\end{proof}

\begin{proof}[proof of (c)]
    Note that 
    \begin{align*}
        \left\vert f(x+1) - f(x) \right\vert &= \left\vert f(x+1) - f_n(x+1) + f_n(x+1) - f_n(x) + f_n(x) - f(x) \right\vert \\
        &\le \left\vert f(x+1) - f_n(x+1) \right\vert + \left\vert f_n(x+1) - f_n(x) \right\vert + \left\vert f_n(x) - f(x) \right\vert \\
        &\to 0 + 0 + 0 = 0 \text{ as } n\to \infty.     
    \end{align*} 
    This means \(f(x+1) = f(x)\) since \(\vert f(x+1) - f(x) \vert \) can be arbitrarily small. Note that \(f\) is continuous since \(f_n \to f\) uniformly and uniform convergence preserve the continuity.    
\end{proof}

\section{Inner Products on Periodic functions}
\begin{definition}
    If \(f, g \in C(\mathbb{R} / \mathbb{Z} , \mathbb{C} )\) , then we define their inner product by 
    \[
        \langle f, g \rangle = \int _0^1 f(x) \overline{g(x)} \, \mathrm{d} x,  
    \] and if we suppose \(f = f^1 + i f^2\) and \(g = g^1 + i g^2\) for \(f^1, f^2, g^1, g^2 \in C(\mathbb{R} / \mathbb{Z} , \mathbb{R} )\), then 
    \[
        \int _0^1 f(x) \cdot \overline{g(x)} \, \mathrm{d} x = \int _0^1 \left( f^1 g^1 + f^2 g^2 \right) \, \mathrm{d} x + i \int _0^1 \left( f^2 g^1 - f^1 g^2 \right) \, \mathrm{d} x.     
    \]   
\end{definition}

\begin{eg}
    If \(f(x) = 1\) and \(g(x) = e^{2 \pi i x}\), then 
    \[
        \langle f, g \rangle = \int _0^1 1 \cdot \overline{\left( e^{2 \pi i x} \right) } \, \mathrm{d} x = \int _0^1 1 \cdot e^{-2 \pi i x} \, \mathrm{d} x = \left. \frac{e^{- 2 \pi i x}}{-2 \pi i}  \right]_0^1 = \frac{1 - 1}{-2 \pi i} = 0,    
    \] so \(f \perp g\). 
\end{eg}

\begin{lemma} \label{lm: inner product int}
    Let \(f, g, h \in C(\mathbb{R} / \mathbb{Z} , \mathbb{C} )\), then 
    \begin{itemize}
        \item [(1)] \(\langle f, g \rangle = \overline{\langle g, f \rangle }  \). 
        \item [(2)] \(\langle f, f \rangle \ge 0 \), and \(\langle f, f \rangle = 0 \) iff \(f = 0\).   
        \item [(3)] \(\langle f + g, h \rangle = \langle f, h \rangle + \langle g, h \rangle   \), and \(\langle cf, g \rangle = c\langle f,g \rangle \) for all \(c \in \mathbb{C} \). 
        \item [(4)] \(\langle f, g + h \rangle = \langle f, g \rangle + \langle f, h \rangle   \), and \(\langle f, cg \rangle = \overline{c} \langle f, g \rangle   \) for all \(c \in \mathbb{C} \).   
    \end{itemize}
\end{lemma}

\begin{definition}
    We define the \(L^2\) norm of \(f\) to be 
    \[
        \lVert f \rVert_2 = \sqrt{\langle f, f \rangle } = \sqrt{\int _0^1 \vert f(x) \vert^2 \, \mathrm{d}x  }. 
    \] 
\end{definition}

\begin{lemma} \label{lm: norm property}
    If \(f, g \in C(\mathbb{R} / \mathbb{Z} , \mathbb{C} )\), then we have 
    \begin{itemize}
        \item [(a)] \(\lVert f \rVert_2 = 0 \) iff \(f = 0\).
        \item [(b)] \(\vert \langle f, g \rangle  \vert \le \lVert f \rVert_2 \lVert g \rVert_2   \).
        \item [(c)] \(\left\lVert f + g \right\rVert_2 \le \lVert f \rVert_2 + \lVert g \rVert_2   \). 
        \item [(d)] If \(\langle f, g \rangle = 0 \), then \(\lVert f+g \rVert_2^2 = \lVert f \rVert _2^2 + \lVert g \rVert_2^2   \). 
        \item [(e)] If \(c \in \mathbb{C} \), then \(\lVert cf \rVert_2 = \vert c \vert \lVert f \rVert_2   \).   
    \end{itemize} 
\end{lemma}
\begin{proof}[proof of (a)]
    Since 
    \[
        \lVert f \rVert_2 = 0 \iff \int _0^1 \vert f(x) \vert^2 \, \mathrm{d} x = 0,   
    \]
    and since \(\vert f(x) \vert^2 \) is continuous and \(\vert f(x) \vert^2 \ge 0 \), so \(\int _0^1 \vert f(x) \vert^2 \, \mathrm{d} x = 0  \) if and only if \(\vert f(x) \vert^2 = 0 \) for all \(x \in [0, 1]\) if and only if \(f(x) = 0\) for all \(x \in [0, 1]\).     
\end{proof}

\begin{proof}[proof of (b)]
    First, we prove the real case. Let \(u, v \in C([0, 1], \mathbb{R} )\). We want to show that 
    \[
        \left\vert \int _0^1 u(x) v(x) \, \mathrm{d} x  \right\vert \le \left( \int _0^1 u^2(x) \, \mathrm{d} x  \right)^{\frac{1}{2}} \cdot \left( \int _0^1 v^2(x) \, \mathrm{d} x   \right)^{\frac{1}{2}}.  
    \]
    For any real number \(\lambda \), 
    \[
        \int _0^1 \left( u(x) - \lambda v(x) \right)^2 \, \mathrm{d} x \ge 0,  
    \] 
    and we have 
    \[
        \int _0^1 \left( u^2(x) - 2\lambda u(x) v(x) + \lambda ^2 v^2(x) \right) \, \mathrm{d} x = \int _0^1 u^2(x) \,\mathrm{d} x - 2 \lambda \left( \int _0^1 u(x) v(x) \, \mathrm{d} x  \right) + \lambda ^2 \left( \int _0^1 v^2(x) \, \mathrm{d} x  \right),  
    \]
    so 
    \[
        \lambda ^2 \left( \int _0^1 v^2(x) \, \mathrm{d} x  \right) - 2 \lambda \left( \int _0^1 u(x) v(x) \, \mathrm{d} x  \right) + \int _0^1 u^2(x) \, \mathrm{d} x \ge 0 \text{ for all } \lambda.    
    \]
    \begin{itemize}
        \item Case 1: \(\int _0^1 v^2(x) \, \mathrm{d} x = 0 \), then \(v(x) = 0\), so 
        \[
            0 = \int _0^1 u(x) v(x) \, \mathrm{d} x \le \left( \int _0^1 u(x) \, \mathrm{d} x  \right)^{\frac{1}{2}} \cdot 0 = 0.  
        \]
        \item Case 2: \(\int _0^1 v^2(x) \, \mathrm{d} x > 0 \), then the minimum of 
        \[
            G(\lambda ) = \lambda ^2 \overbrace{\left( \int _0^1 v^2(x) \, \mathrm{d} x  \right) }^{C} - 2 \lambda \overbrace{\left( \int _0^1 u(x) v(x) \, \mathrm{d} x  \right) }^{B} + \overbrace{\left( \int _0^1 u^2(x) \, \mathrm{d} x  \right) }^{A}
        \]
        is at \(\lambda = \frac{B}{C}\) (\(G^{\prime} (\lambda ) = 0\)) with minimum 
        \[
            \left( \frac{B}{C} \right)^2 C - 2 \frac{B}{C} B + A = \frac{-B^2 + AC}{C} = \int _0^1 (u(x) - \lambda v(x))^2 \, \mathrm{d} x  \ge 0. 
        \]
        Now since \(C = \int _0^1 v^2(x) \, \mathrm{d} x > 0  \), so \(-B^2 + AC \ge 0\), and thus \(B^2 \le AC\), i.e. 
        \[
            \left\vert \int _0^1 u(x) v(x) \, \mathrm{d} x  \right\vert \le \left( \int _0^1 u^2(x) \, \mathrm{d} x  \right)^{\frac{1}{2}} \cdot \left( \int _0^1 v^2(x) \, \mathrm{d} x   \right)^{\frac{1}{2}}. 
        \]  
    \end{itemize}
    
    Now, we first show that 
    \[
        \left\vert \int _0^1 f(x) \, \mathrm{d} x  \right\vert \le \int _0^1 \vert f(x) \vert \, \mathrm{d} x.   
    \]
    Let \(I = \int _0^1 f(x) \, \mathrm{d} x \in \mathbb{C}  \). If \(I = 0\), then this is true. If \(I \neq 0\), then let \(c = \frac{\overline{I} }{\vert I \vert } \in \mathbb{C} \), so we have 
    \[
        cI = \frac{\overline{I} }{\vert I \vert } I = \vert I \vert \in \mathbb{R}. 
    \]    
    Now 
    \[
        cI = c \int _0^1 f(x) \, \mathrm{d}x = \int _0^1 cf(x) \, \mathrm{d}x = \int _0^1 \Re (cf(x)) \, \mathrm{d} x + i \int _0^1 \Im (cf(x)) \, \mathrm{d} x = \int _0^1 \Re (cf(x)) \, \mathrm{d} x \in \mathbb{R}.    
    \]
    So 
    \[
        \vert I \vert = cI = \int _0^1 \Re (cf(x)) \, \mathrm{d} x \le \int _0^1 \vert cf(x) \vert \, \mathrm{d} x = \vert c \vert \int _0^1 \vert f(x) \vert \, \mathrm{d} x = \int _0^1 \vert f(x) \vert \, \mathrm{d} x 
    \]
    since \(\vert c \vert = 1 \), and thus 
    \[
        \left\vert \int _0^1 f(x) \, \mathrm{d} x  \right\vert = \vert I \vert \le \int _0^1 \vert f(x) \vert \, \mathrm{d} x .   
    \] 
    Hence, we finally have 
    \begin{align*}
        \left\vert \int _0^1 f(x) \overline{g}(x)   \, \mathrm{d} x \right\vert&\le \int _0^1 \left\vert f(x) \overline{g}(x)  \right\vert = \int _0^1 \vert f(x) \vert \left\vert \overline{g}(x)  \right\vert \, \mathrm{d} x \\
        &= \int _0^1 \vert f(x) \vert \vert g(x) \vert \, \mathrm{d} x \le \left( \int_0^1 \vert f(x) \vert^2 \, \mathrm{d} x  \right)^{\frac{1}{2}} \left( \int _0^1 \vert g(x) \vert^2 \, \mathrm{d} x  \right)^{\frac{1}{2}}             
    \end{align*}
    since we convert the complex case to the real case.
\end{proof}

\begin{proof}[proof of (c)]
    Since 
    \begin{align*}
        \lVert f + g \rVert_2^2 &= \langle f+g, f+g \rangle = \langle f, f+g \rangle + \langle g, f+g \rangle \\
        &= \langle f, f \rangle + \langle f, g \rangle + \langle g, f \rangle + \langle g, g \rangle \\
        &= \langle f, f \rangle + \langle f, g \rangle + \overline{\langle f, g \rangle } + \langle g, g \rangle \\
        &= \langle f, f \rangle + 2 \Re \langle f, g \rangle + \langle g, g \rangle \\
        &\le \langle f, f \rangle + 2 \vert \langle f, g \rangle  \vert  + \langle g, g \rangle \\
        &\le \langle f, f \rangle + 2 \lVert f \rVert_2 \lVert g \rVert_2 + \langle g, g \rangle \\
        &= \lVert f \rVert_2^2 + 2 \lVert f \rVert_2 \lVert g \rVert_2 + \lVert g \rVert_2^2 = \left( \lVert f \rVert_2 + \lVert g \rVert_2   \right)^2,                           
    \end{align*}
    so we have 
    \[
        \lVert f + g \rVert_2 \le \lVert f \rVert_2 + \lVert g \rVert_2.   
    \]
\end{proof}

\begin{proof}[proof of (d)]
    Since 
    \[
        \lVert f + g \rVert_2^2 = \lVert f \rVert_2^2 + \lVert g \rVert_2^2 + \langle f, g \rangle + \overline{\langle f, g \rangle },     
    \] so \(\langle f, g \rangle = 0 \) implies 
    \[
        \lVert f + g \rVert_2 = \lVert f \rVert_2 + \lVert g \rVert_2.   
    \] 
\end{proof}

\begin{proof}[proof of (e)]
    \todo{DIY}
\end{proof}