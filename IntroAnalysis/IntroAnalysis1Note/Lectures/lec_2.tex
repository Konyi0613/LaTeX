\lecture{2}{4 Sep. 10:20}{}
\begin{definition}[Floor Function]\label{dfn: floor func}
    For any real number \(x\), the floor function of \(x\) is denoted by \(\lfloor x \rfloor\), and is defined by the formula \(\lfloor n \rfloor\) if \(n \le x < n+1\) where \(n \in \mathbb{Z} \).      
\end{definition}
\begin{corollary}
    \[
        \lfloor x \rfloor \le x < \lfloor x \rfloor + 1.
    \]
\end{corollary}
\begin{eg}
    \(\lfloor 3.7 \rfloor = 3\), \(\lfloor -1.2 \rfloor = -2\).  
\end{eg}

Now by floor function, we can reprove \autoref{thm: rational dense in real}. 

\begin{theorem}[Density of rational number in real number Again]\label{thm: rational dense in real again}
  The set of rational numbers is dense in the real number. That is, if \(a\) and \(b\) are real numbers with \(a<b\), then there exists a rational number \(\frac{q}{p}\) such that \(a < \frac{q}{p} < b\).     
\end{theorem}

\begin{proof}[Reprove \autoref{thm: rational dense in real}]
    Since \(a<b\), so we know \(b - a > 0\). Now by \hyperref[thm: Archimedean property]{Archimedean Property}, we know there exists \(q \in \mathbb{N} \) such that \(q(b-a) > 1\). Let \(p = \lfloor qa \rfloor + 1\), we have 
    \[
        \lfloor qa \rfloor \le qa < \lfloor qa \rfloor + 1 = p.
    \]     
    From our construction, \(qb > qa + 1\), so we have 
    \[
        p = \lfloor qa \rfloor + 1 \le qa + 1 < qb, 
    \]  
    hence we have 
    \[
        qa \le p \le qb.
    \]
\end{proof}

\begin{note}
    For some reason, \(p,q\) in \autoref{thm: rational dense in real} and \autoref{thm: rational dense in real again} are reversed.  
\end{note}

\begin{definition}[irrational number]\label{dfn: irrational num}
    \(x\) is called irrational if \(x\) is not rational.  
\end{definition}

\begin{eg}
    \(\sqrt{2} \) is irrational. 
\end{eg}

\begin{theorem} \label{thm: property of irrational1}
    Let \(r \in \mathbb{Q} \) and \(x \in \mathbb{R} \backslash \mathbb{Q} \), then 
    \begin{itemize}
        \item [1.] \(r + x\) is irrational. 
        \item [2.] If \(r \neq 0\), then \(rx\) is irrational.   
    \end{itemize}  
\end{theorem}
\begin{proof}[sketch of proof]
    \vphantom{text}
    \begin{itemize}
        \item [1.] If \(r + x = q \in \mathbb{Q} \), then \(x = q - r \in \mathbb{Q} \), contradiction.  
        \item [2.] If \(rx = q \in \mathbb{Q} \), then \(x = \frac{q}{r} \in \mathbb{Q} \) since \(r \neq 0\).  
    \end{itemize}
\end{proof}

\begin{theorem}[irrational number dense in real number]\label{thm: irrational dense in real}
    The set of irrational number is dense in real number. That is, if \(a, b \in \mathbb{R} \) and \(a < b\), then there exists a irrational number \(t\) such that \(a < t < b\).    
\end{theorem}
\begin{proof}
    By \hyperref[thm: rational dense in real]{density of rational number}, we can find \(a < r_1 < r_2 < b\) where \(r_1, r_2 \in \mathbb{Q} \), and then let \(t = r_1 + \frac{1}{\sqrt{2}}(r_2 - r_1)\), then we know
    \[
        a < r_1 < t < r_2 < b.
    \]    
    \begin{note}
        We should use \autoref{thm: property of irrational1} and the fact that \(\sqrt{2} \) is irrational. 
    \end{note}
\end{proof}

\begin{definition}[bounded set]\label{dfn: bounded set}
    A set \(S \subseteq \mathbb{R} \) is bounded if there are numbers \(a, b\) s.t. \(a \le x \le b\) for all \(x \in S\). 
\end{definition}

\begin{corollary}
    A bounded non-empty set in \(\mathbb{R} \) has a unique supremum and a unique infimum and \(\inf S \le \sup S\).    
\end{corollary}

\section{Extended real number system}
The real number system, together with \(\infty \) and \(-\infty \), then we have the following properties: 
\begin{itemize}
    \item [(a)] If \(a \in \mathbb{R} \), then \(a + \infty = \infty + a = \infty \) and \(a - \infty = -\infty + a = -\infty \), and \(\frac{a}{\infty }=\frac{a}{-\infty } = 0\). 
    \item [(b)] If \(a > 0\), then \(a \cdot \infty  = \infty \cdot a = \infty \) and \(a \cdot (-\infty ) = (-\infty ) \cdot a = -\infty \)
    \item [(c)] If \(a<0\), then \(a \cdot \infty = \infty \cdot a = -\infty \) and \(a \cdot -\infty = -\infty \cdot a = \infty \) and \(\infty +\infty = \infty \cdot \infty  = (-\infty ) \cdot (-\infty )=\infty \) and \(-\infty  - \infty  = \infty \cdot (-\infty ) = (-\infty ) \cdot \infty = -\infty \)  and \(\vert -\infty  \vert = \vert \infty  \vert = \infty \) 
\end{itemize}  
However, there are some indeterminate form: 
\begin{theorem}
    The following things are not defined:
    \[
        \infty - \infty, \ 0 \cdot \infty, \ \frac{\infty}{\infty }, \text{ and } \frac{0}{0}.
    \]
\end{theorem}

\section{Mathematical Induction}
\begin{theorem}[Peano's Postulate]\label{thm: Peano's postulate}
    The natural numbers satisfy the following properties 
    \begin{itemize}
        \item [(a)] \(\mathbb{N} \) is nonempty. 
        \item [(b)] For each natural number \(n\), there exists a unique rational number \(n\) called the successor of \(n\). 
        \item [(c)] There exists a natural number \(\overline{n} \) that is not the sucessor of any natural number.   
        \item [(d)] Different natural numbers have different sucessors, that is, \(n \neq m\) implies \(n^{\prime} \neq m^{\prime} \). 
        \item [(e)] The only subset of \(\mathbb{N} \) that contains \(\overline{n}\) and also contains the sucessor of every one of its element is \(\mathbb{N} \).       
    \end{itemize}
\end{theorem}

\begin{theorem}[Principle of Mathematical Induction]
    Let \(p_1, p_2, \dots , p_n\) be propositions, one for each positive integers, such that 
    \begin{itemize}
        \item [(a)] \(p_1\) is true. 
        \item [(b)] for each positive integer \(n\), \(p_n\) implies \(p_{n+1}\).    
    \end{itemize} 
    then \(p_n\) is true for each \(n \in \mathbb{N}\).  
\end{theorem}
\begin{proof}
    Let \(M= \left\{ n \mid n \in \mathbb{N} \text{ and } p_n \text{ is true} \right\} \), then from (a) we know \(1 \in M\) and from (b) we know \(n \in M\) implies \(n+1 \in M\).  Hence, from (e) of \hyperref[thm: Peano's postulate]{Peano's Postulate}, we know \(M = \mathbb{N} \).   
\end{proof}

\chapter{Metric Space}
\section{Definition and examples}
\begin{definition}
    Suppose \(x_n \in \mathbb{R} \) for \(n \ge m\). We use the notation \(\left( x_n \right)_{n = m}^{\infty}  \) to denote the sequence of numbers 
\[
    x_m, x_{m+1}, \dots 
\]   
\end{definition}

We first recall the definition of a convergent sequence. 
\begin{definition}[Convergent Sequence]\label{dfn: convergent sequence}
    We say that a sequence \(\left( x_n \right)_{n=m}^{\infty}\) of real numbers converges to \(x\) if for every \(\varepsilon > 0\), there exists an \(N \geq m\) s.t. \(\vert x_n - x \vert \le \varepsilon  \) for all \(n \ge N\). 
    \begin{notation}
    We write \(\lim_{n \to \infty} x_n = x\). 
\end{notation}     
\end{definition}

On \(\mathbb{R} \), we can define the distance function between two points \(x,y \in \mathbb{R} \) by \(d(x,y) = \vert x-y \vert \). We'll discuss this more later. 
\begin{lemma}
    Let \((x_n)_{n=m}^{\infty} \) be a sequence of real numbers, and let \(x\) be another real number, then \((x_n)_{n=m}^{\infty} \) converges to \(x\) if and only if \(\lim_{n \to \infty} d(x_n, x) = 0 \).   
\end{lemma}  
\begin{proof}
    Assume \((x_n)_{n=m}^{\infty} \) converges to \(x\). Let \(\varepsilon > 0\) be arbitrary real number. By definition, there exists an \(N \ge m\) such that \(\vert x_n - x \vert \le \varepsilon  \) for all \(n \ge N\). But \(d(x_n, x) = \vert x_n - x \vert \) by the definition. Hence, \(\forall \varepsilon > 0\), \(\exists N \ge m\) such that \(d(x_n, x) \le \varepsilon \) for all \(n \ge N\). This implies that \(\forall \varepsilon > 0\), \(\exists N \ge m\) such that \(\left\vert d(x_n, x) - 0 \right\vert \le \varepsilon  \) for all \(n \ge N\). This implies \(\lim_{n \to \infty} d(x_n, x) = 0 \). 
    
    The proof of the other side is the same but writing the above proof from bottom to top again.
\end{proof}

\begin{definition}[Metric Space]\label{dfn: metric space}
    A metric space \((X, d)\) is the space of \(X\) of objects(called points), together with a distance function or metric \(d: X \times X \to [0, \infty )\) which associates to each \(x, y\) of points in \(X\) a nonnegative number \(d(x,y) \ge 0\). Furthermore, the metric must satisfy \(4\) axioms. 
    \begin{itemize}
        \item [(a)] For any \(x \in X\), \(d(x,x) = 0\). 
        \item [(b)] (Positivity) For any distinct \(x, y \in X\), we have \(d(x,y) > 0\). 
        \item [(c)] (Symmetry) For any \(x,y \in X\), we have \(d(x,y) = d(y,x)\). 
        \item [(d)] (Triangle inequality) For any \(x,y,z \in X\), we have \(d(x,z) \le d(x,y) + d(y,z)\).         
    \end{itemize}       
\end{definition}

\begin{eg}
    On \(\mathbb{R} \), we can define \(d(x,y) = \vert x - y \vert \).
\end{eg}
\begin{explanation}
        \begin{itemize}
        \item \(d(x, y) = \vert x - y \vert \ge 0 \). 
        \item \(d(x,y) = 0\) iff \(\vert x-y \vert = 0 \) iff \(x = y\). 
        \item \(\vert x-y \vert = \vert y-x \vert  \), so \(d(x,y) = d(y,x)\) 
        \item \(\vert x-z \vert \le \vert x-y \vert + \vert y-z \vert   \) for all \(x,y,z \in \mathbb{R} \).        
    \end{itemize}  
\end{explanation}

\begin{eg}
    Let \((X,d)\) be a metric space and \(Y \subseteq X\), then \(Y\) inherits a natural distance function  
    \[
        d \vert _{Y \times Y}: Y \times Y \to [0, \infty )
    \]   defined by \(d \vert _{Y \times Y} (\alpha , \beta ) = d(\alpha , \beta )\) for all \(\alpha , \beta \in Y\).  
\end{eg}
\begin{note}
    \((Y, d \vert _{Y \times Y})\) is called a metric subspace of \((X, d)\). It is obvious that \(d \vert _{Y \times Y}\) is a metric on \(Y\).    
\end{note}

Recall \(\mathbb{R} ^n\). Let \(x = (x_1, x_2, \dots , x_n) , y = (y_1, y_2, \dots , y_n) \in \mathbb{R} ^n\). 

\begin{definition}[\(l^2\)-metric]\label{def: l2metric}
    The \(l^2\)-metric is defined by 
    \[
        d_2(x,y) = \left( \sum_{i=1}^n (x_n - y_n)^2  \right)^{\frac{1}{2}} (\text{ or we called } d_{l_2}(x,y)).
    \] 
\end{definition}
\begin{definition}[\(l^1\)-metric(taxicab metric)]\label{def: l1-metric}
    The \(l^1\)-metric is defined by 
    \[
        d_1(x,y) = \sum_{i=1}^n \vert x_i - y_i \vert (\text{or we called } d_{l_1}(x,y))
    \] 
\end{definition}
    
\begin{definition}[\(l^{\infty} \)-metric ]\label{def: linf-metric}
        The \(l^{\infty} \)-metric is defined by 
        \[
            d_\infty (x,y) = \max_{1 \le i \le n} \vert x_i - y_i \vert 
        \] 
\end{definition}

\begin{exercise}
    Verify they are all metrics.
\end{exercise}
\begin{note}
    Actually we have to define inner product and norm first and then we can use the triangle inequality of norm to prove \(d_2\) is a metric. (See lecture notes by professor)
\end{note}