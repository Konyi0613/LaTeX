\lecture{13}{14 Oct. 9:10}{}
\begin{prev}
    \(\lim_{n \to \infty} f_n = f \) uniformly where \(f_n, f:X \to Y\) iff given any \(\varepsilon > 0\), \(\exists N > 0\) s.t. 
    \[
        d_Y \left( f_n(x), f(x) \right) < \varepsilon \quad \forall x \in X \text{ and } n \ge N. 
    \]    
\end{prev}

\begin{theorem}[考試會考] \label{thm: uniformly convergence preserves continuity at limit}
    Suppose \(\left( f^{(n)} \right)_{n=1}^{\infty}  \) is a sequence of functions from one metric \((X, d_X)\) to \((Y, d_Y)\) and suppose this sequence of functions converge uniformly to another function \(f:X \to Y\). Let \(x_0 \in X\). If each \(f^{(n)}\) is continuous at \(x_0\), then \(f\) is continuous at \(x_0\).        
\end{theorem}

\begin{proof}
    Since \(f_n \to f\) uniformly. Given \(\varepsilon > 0\), \(\exists N > 0\) s.t. 
    \[
        d_Y \left( f^{(n)}(x), f(x) \right) < \frac{\varepsilon}{3} \quad \text{for all } x \in X, n \ge N. 
    \]   
    Since \(f^{(N)}\) is continuous at \(x_0\), so there exists \(\exists \delta > 0\) s.t. if \(d_X(x, x_0) < \delta \), then 
    \[
        d_Y \left( f^{(N)}(x), f^{(N)}(x_0) \right) < \frac{\varepsilon}{3}.
    \] 
    Now if \(d_X(x, x_0) < \delta \), then 
    \begin{align*}
        d_Y ( f(x), f(x_0)) &\le d_Y \left( f(x), f^{(N)}(x) \right) + d_Y \left( f^{(N)}(x), f^{(N)}(x_0) \right) + d_Y \left( f^{(N)}(x_0), f(x_0) \right) \\
        &< \frac{\varepsilon}{3} + \frac{\varepsilon}{3} + \frac{\varepsilon}{3} = \varepsilon. 
    \end{align*} 
    Hence, \(f\) is continuous at \(x_0\).   
\end{proof}

\begin{corollary} \label{cl: uniformly continuous preserve continuity}
    Let \(f^{(n)}, f:X \to Y\). Suppose \(f^{(n)}:X \to Y\) are continuous for all \(n\). If \(\lim_{n \to \infty} f^{(n)} = f \) uniformly, then \(f\) is also continuous.     
\end{corollary}

\begin{eg}
    Suppose \(f_n(x) = x^n\) define on \([0, 1]\) for all \(n \in \mathbb{N} \), then we know 
    \[
        \lim_{n \to \infty} f_n(x) = \begin{dcases}
           0 , &\text{ if }  0 \le x < 1;\\
            1 , &\text{ if } x = 1.
        \end{dcases} 
    \] 
    Now suppose 
    \[
        f(x) = \begin{dcases}
           0 , &\text{ if }  0 \le x < 1;\\
            1 , &\text{ if } x = 1.
        \end{dcases} ,
    \] then \(\lim_{n \to \infty} f_n(x) = f(x) \) pointwise. Note that \(f_n = x^n\) is continuous on \([0, 1]\) but \(f\) is not continuous. Hence, \(f_n\) does not converge uniformly to \(f\).      
\end{eg}

\begin{remark}
    This example tells us if we change converge uniformly to converge pointwise in \autoref{thm: uniformly convergence preserves continuity at limit}. 
\end{remark}

Now we want to know is 
\[
    \lim_{n \to \infty} \lim_{x \to x_0, x \in E} f^{(n)}(x) = \lim_{x \to x_0, x \in E} \lim_{n \to \infty} f^{(n)}(x)?    
\]

We will later show that the equality holds if \(f^{(n)} \to f\) uniformly and \(Y\) compact, where we define \(f^{(n)}, f: X \to Y\). 

\begin{proposition}
    Let \((X, d_X)\) and \((Y, d_Y)\) be metric spaces with \(Y\) complete, and let \(E \subseteq X\). Suppose \(\left( f^{(n)} \right)_{n=1}^{\infty}  \) is a sequence of functions from \(E\) to \(Y\) that converges uniformly to some function \(f:E \to Y\). Let \(x_0 \in X\) be adherent point of \(E\), and suppose that for each \(n\), \(\lim_{x \to x_0, x \in E} f^{(n)}(x)\) exists. Then the limit \(\lim_{x \to x_0, x \in E} f(x) \) also exists, and moreover, 
    \[
        \lim_{n \to \infty} \lim_{x \to x_0, x \in E} f^{(n)}(x) = \lim_{x \to x_0, x \in E} \lim_{n \to \infty}  f^{(n)}(x). 
    \]             
\end{proposition}
\begin{proof}
    Since \(\lim_{x \to x_0, x \in E} f^{(n)}(x) \) exosts, so we can let \(L_n \coloneqq \lim_{x \to x_0, x \in E} f^{(n)}(x) \in Y \). First, we show that \(\left\{ L_n \right\}_{n=1}^{\infty }  \) is Cauchy. Since \(\lim_{n \to \infty} f^{(n)} = f \) uniformly on \(E\). Given \(\varepsilon > 0\), \(\exists N > 0\) s.t. 
    \[
        d_Y \left( f^{(n)}(x), f(x) \right) < \frac{\varepsilon}{6} \text{ for all } x \in E, n \ge N. 
    \] 
    Hence, \(n, m \ge N\) implies 
    \[
        d_Y \left( f^{(n)}(x), f^{(m)}(x) \right) \le d_Y \left( f^{(n)}(x), f(x) \right) + d_Y \left( f(x), f^{(m)}(x) \right) < \frac{\varepsilon}{6} + \frac{\varepsilon}{6} = \frac{\varepsilon}{3}.   
    \] Now since \(\lim_{x \to x_0, x \in E} f^{(n)}(x) = L_n \) and \(\lim_{x \to x_0, x \in E} f^{(m)}(x) = L_m\), so there exists \(\delta _n, \delta _m > 0\) s.t. 
    \begin{align*}
        d(x, x_0) < \delta _n &\implies d_Y \left( f^{(n)}(x), L_n \right) < \frac{\varepsilon}{3} \\
        d(x, x_0) < \delta _m &\implies d_Y \left( f^{(m)}(x), L_m \right) < \frac{\varepsilon}{3}. 
    \end{align*}  
    Choose \(\delta = \min \left\{ \delta _n, \delta _m \right\} \) and fix \(x \in E\) with \(d_X (x, x_0) < \delta \) (since \(x_0 \in \overline{E} \) so this is possible), then 
    \[
        d_Y \left( L_n, L_m \right) \le d_Y \left( L_n, f^{(n)}(x) \right) + d_Y \left( f^{(n)}(x), f^{(m)}(x) \right) + d_Y \left( f^{(m)}(x), L_m \right) < \frac{\varepsilon}{3} + \frac{\varepsilon}{3} + \frac{\varepsilon}{3} = \varepsilon.    
    \] 
    Hence, \(\left\{ L_n \right\}_{n=1}^{\infty}  \) is Cauchy. Since \(Y\) is complete, so \(\lim_{n \to \infty} L_n = L \) for some \(L \in Y\). 
    \begin{claim}
        \(\lim_{x \to x_0, x \in E} f(x) = L\). 
    \end{claim}    
    \begin{explanation}
        Fix \(\varepsilon > 0\), then \(\lim_{n \to \infty} L_n = L \) and \(\lim_{n \to \infty} f^{(n)}(x) = f(x) \) uniformly on \(E\), so there exists \(N > 0\) s.t. 
        \[
            d_Y \left( L_n, L \right) < \frac{\varepsilon}{3} \text{ and } d_Y \left( f^{(n)}(x), f(x) \right) < \frac{\varepsilon}{3} \text{ for all } x \in E, n \ge N.  
        \] 
        Now since \(L_n = \lim_{x \to x_0, x \in E} f^{(n)}(x)\), so there exists \(\delta > 0\) s.t. 
        \[
            d_X(x, x_0) < \delta \implies d_Y \left( f^{(n)}(x), L_N \right) < \frac{\varepsilon}{3}.
        \]  
        For this \(\delta \), if \(d_X(x, x_0) < \delta \) and \(x \in E\), then we know 
        \[
            d_Y (f(x), L) \le d_Y \left( f(x), f^{(N)}(x) \right) + d_Y \left( f^{(N)}(x), L_N \right) + d_Y \left( L_N, L \right) < \varepsilon.  
        \]     
    \end{explanation} 
    Hence, 
    \begin{align*}
        \lim_{n \to \infty} \lim_{x \to x_0, x \in E} f^{(n)}(x) &= \lim_{n \to \infty} L_n = L \\
        \lim_{x \to x_0, x \in E} \lim_{n \to \infty} f^{(n)}(x) &= \lim_{x \to x_0, x \in E} f(x) = L. 
    \end{align*}  
    This means \(\lim_{x \to x_0, x \in E}\) and \(\lim_{n \to \infty} \) is exchangable here.
\end{proof}

\begin{proposition}
    Let \(f^{(n)}:X \to Y\) be a sequence of continuous functions. If \(\lim_{n \to \infty} f^{(n)} = f \) uniformly. Let \(\lim_{n \to \infty} x^{(n)} = x \) in \(X\), then \(\lim_{n \to \infty} f^{(n)}\left( x^{(n)} \right) = f(x)  \).     
\end{proposition}
\begin{proof}
    Since \(f^{(n)} \to  f\) uniformly, so given \(\varepsilon > 0\), there exists \(N_1 > 0\) s.t. 
    \[
        d_Y \left( f^{(n)}(x), f(x) \right) < \frac{\varepsilon}{2} \text{ for all } x \in X, n \ge N_1.  
    \]  
    Since \(f^{(n)} \to f\) uniformly, so \(f\) is also continuous by \autoref{cl: uniformly continuous preserve continuity}, so there exists \(\delta _x > 0\) s.t. 
    \[
        d_X(y, x) < \delta _x \implies d_Y \left( f(y), f(x) \right) < \frac{\varepsilon}{2}. 
    \]     
    Since \(\lim_{n \to \infty} x^{(n)} = x \), so there exists \(N_2 > 0\) s.t. \(d_X \left( x^{(n)}, x \right) < \delta _x \) for all \(n \ge N_2\). Let \(N = \max \left\{ N_1, N_2 \right\} \), then \(n \ge N\) implies 
    \[
        d_Y \left( f^{(n)}\left( x^{(n)} \right), f(x)  \right) \le d_Y \left( f^{(n)} \left( x^{(n)} \right), f \left( x^{(n)} \right)   \right) + d_Y \left( f \left( x^{(n)} \right), f(x)  \right) < \varepsilon .  
    \]     
\end{proof}

\begin{definition}[bounded function] \label{def: bounded function}
    A function \(f:X \to Y\) from \((X, d_X)\) to \((Y, d_Y)\) is called bounded if its image \(f(X)\) is bounded in \(Y\) i.e. there exists \(y_0 \in Y\) and \(R > 0\) s.t. 
    \[
        d_Y \left( f(x), y_0 \right) < R \text{ for all } x \in X.  
    \]      
\end{definition}

\begin{proposition}
    Let \(\left( f^{(n)} \right)_{n=1}^{\infty}  \) be a sequence of functions from \((X, d_X)\) to \((Y, d_Y)\). Suppose \(\lim_{n \to \infty} f^{(n)} = f \) uniformly where \(f:X \to Y\). If each \(f^{(n)}\) is bounded on \(X\), then \(f\) is also bounded.       
\end{proposition}
\begin{proof}
    Since \(\lim_{n \to \infty} f^{(n)} = f \) uniformly, so given \(\varepsilon = 1\), there exists \(N > 0\) s.t. 
    \[
        d_Y \left( f^{(n)}(x), f(x) \right) < 1 \text{ for all } x \in X, n \ge N.  
    \] Since \(f^{(N)}\) is bounded, so there exists \(y_0\) and \(R_N > 0\) s.t. 
    \[
        d_Y \left( f^{(N)}(x), y_0 \right) < R_N \text{ for all } x \in X.  
    \] 
    Hence, 
    \begin{align*}
        d_Y \left( f(x), y_0 \right) &\le d_Y \left( f(x), f^{(N)}(x) \right) + d_Y \left( f^{(N)}(x), y_0 \right) \\
        &< 1 + R_N   
    \end{align*}
    for all \(x \in X\), so \(f\) is bounded.    
\end{proof}

\section{The Metric of Uniform Convergence}
\begin{definition}
    Suppose \((X, d_X)\) and \((Y, d_Y)\) be metric spaces. We let \(B(X \to Y)\) denotes the set of all bounded functions from \(X\) to \(Y\) i.e. 
    \[
        B(X \to Y) = \left\{ f \mid f:X \to Y \text{ is  bounded}  \right\}. 
    \] 
    If \(X \neq \varnothing \), we define a metric \(d_\infty \) on \(B(X \to Y)\) by 
    \[
        d_\infty \left( f, g \right) = \sup _{x \in X} d_Y (f(x), g(x)) 
    \] for all \(f, g \in B(X \to Y)\).     
\end{definition}

\begin{proposition}
    If \(X\) is non-empty, then \(d_\infty \) is a metric on \(B(X \to Y)\).  
\end{proposition}
\begin{proof}
    Given \(f, g \in B(X \to Y)\), then there exists \(y_f, y_g \in Y\) and \(M_f, M_g > 0\) s.t.
    \[
        d_Y (f(x), y_f) < M_f \text{ and } d_Y (g(x), y_g) < M_g \text{ for all } x \in X. 
    \]  
    Hence, 
    \[
        d_Y (f(x), g(x)) \le d_Y (f(x), y_f) + d_Y(y_f, y_g) + d_Y(y_g, g(x)) < M_f + M_g + d(y_f, y_g),
    \] which means \(\sup _{x \in X} d_Y(f(x), g(x))\) exists and \(\ge 0\), and thus \(d_\infty \) is well-defined.  
    
    Now since 
    \begin{itemize}
        \item [(1)] \(d_\infty (f, g) = \sup _{x \in X} d_Y (f(x), g(x))\ge 0\). 
        \item [(2)] \(d_\infty (f, g) = d_\infty (g, f)\). 
        \item [(3)] For \(f, g, h \in B(X \to Y)\), 
        \[
            d_Y (f(x), h(x)) \le d_Y(f(x), g(x)) + d_Y (g(x), h(x)) \le d_\infty (f, g) + d_\infty (g, h),
        \]
        so \(\sup _{x \in X} d_Y(f(x), h(x)) \le d_\infty (f, g) + d_\infty (g, h)\). 
        \item [(4)] \(d_\infty (f, g) = 0\) iff \(\sup _{x \in X} d_Y (f(x), g(x)) = 0\) iff \(d_Y(f(x), g(x)) = 0\) for all \(x \in X\) iff \(f(x) = g(x)\) for all \(x \in X\).       
    \end{itemize}
    So \(d_\infty \) is a metric. 
\end{proof}

\begin{proposition}
    \[
        f^{(n)} \to f \text{ in } d_\infty ( = d_{B(X \to Y)}) \iff f^{(n)} \to f \text{ uniformly}.  
    \]
\end{proposition}
\begin{proof}
    \vphantom{text}
    \begin{itemize}
        \item [\((\implies )\)] Since \(\lim_{n \to \infty} d_\infty \left( f^{(n)}, f \right) = 0  \). Given \(\varepsilon > 0\), there exists \(N > 0\) s.t. \(n \ge N\) implies 
        \[
            \sup _{x \in X} \left( f^{(n)}(x), f(x) \right)  = d_\infty \left( f^{(n)}, f \right) < \varepsilon,
        \] so 
        \[
            d_Y \left( f^{(n)}(x), f(x) \right) \le \sup _{x \in X} d_Y\left( f^{(n)}(x), f \right) < \varepsilon 
        \] whenever \(n \ge N\), so \(f^{(n)} \to f\) uniformly.  
        \item [\((\impliedby )\)] \todo{DIY}
    \end{itemize}
\end{proof}

Now let \(C(X \to Y)\) be the set of bounded and continuous function, so \(C(X \to Y) \subseteq B(X \to Y)\). 
\begin{theorem}
    If \(Y\) is complete, then \(C(X \to Y)\) is a complete metric space. (The metric is \(d_\infty \) intersected to \(C(X \to Y)\)).
\end{theorem}  
\begin{proof}
    Given any Cauchy sequence \(\left\{ f^{(n)} \right\}_{n=1}^{\infty}  \) in \(\left( C(X \to Y), d_\infty  \right) \), then \(f^{(n)}:X \to Y\) is continuous and bounded. Given \(\varepsilon > 0\), there exists \(N > 0\) s.t.  
    \[
        \sup _{x \in X} d_Y \left( f^{(n)}(x), f^{(m)}(x) \right)  = d_\infty \left( f^{(n)}, f^{(m)} \right) < \frac{\varepsilon}{2} \text{ for all } n, m \ge N.  
    \] Now fix \(x \in X\). Consider the sequence \(\left\{ f^{(n)}(x) \right\}_{n=1}^{\infty}  \) in \(Y\), so \(\left\{ f^{(n)}(x) \right\}_{n=1}^{\infty}  \) is Cauchy in \(Y\). Now since \(Y\) is complete, so \(\lim_{n \to \infty} f^{(n)}(x) \) exists. Let \(f(x) = \lim_{n \to \infty} f^{(n)}(x) \). If we set up \(f(x)\) similarly for all \(x \in X\) to construct \(f\), then we give a claim. 
    \begin{claim}
        \(\lim_{n \to \infty} d_\infty \left( f^{(n)}, f \right) = 0  \). 
    \end{claim}
    \begin{explanation}
        Fix \(\varepsilon > 0\), choose \(N > 0\) s.t. \(d_\infty \left( f^{(n)}, f^{(m)} \right) < \frac{\varepsilon}{2} \) for all \(n, m \ge N\). Then for all \(n \ge N\) we know for all \(x \in X\)  
        \begin{align*}
            d_Y \left( f^{(n)}(x), f(x) \right) &= \lim_{m \to \infty} d \left( f^{(n)}(x), f^{(m)}(x) \right) \le \sup _{m \ge N, x \in X} d \left( f^{(n)}(x), f^{(m)}(x) \right) \\
            &= \sup _{m \ge N} d_\infty \left( f^{(n)}, f^{(m)} \right) < \frac{\varepsilon}{2}.     
        \end{align*}   
        Hence, \(\lim_{n \to \infty} d_\infty \left( f^{(n)}, f \right) = 0\).  
    \end{explanation}          
    By this claim, we know \(f^{(n)} \to f\), so \(Y\) is complete.  
    (We have to show \(f \in C(X \to Y)\).)       
\end{proof}