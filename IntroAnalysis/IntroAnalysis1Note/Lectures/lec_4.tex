\lecture{4}{11 Sep. 10:20}{}
\begin{theorem}\label{thm: Lipschitz eq implies converge at same place}
    Let \((X, d_1)\) and \((X, d_2)\) be metrics on \(X\), and suppose \(d_1\) and \(d_2\) are \hyperref[def: Lipschitz equivalent metric]{Lipschitz equivalent}, then for any sequence \(\left\{ x^{(n)} \right\}_{n=m}^{\infty} \subseteq X \) and any \(x \in X\), we have
    \[
        \lim_{n \to \infty} x^{(n)} = x \text{ in } (X, d_1) \iff \lim_{n \to \infty} x^{(n)} = x \text{ in } (X, d_2). 
    \]        
\end{theorem}
\begin{proof}
    Since \(d_1, d_2\) are Lipschitz equivalent, so there exists \(c_1, c_2 > 0\) s.t. 
    \[
        c_1 d_1(x,y) \le d_2(x, y) \le c_2 d_1(x, y).
    \]  
    \begin{itemize}
        \item [\((\implies )\)] Given \(\frac{\varepsilon}{c_2} > 0\), since \(\lim_{n \to \infty} x^{(n)} = x \) in \((X, d_1)\), so there exists \(N\) s.t. \(N \ge m\) and 
        \[
            d_1(x^{(n)}, x) \le \frac{\varepsilon}{c_2} \text{ for } n \ge N.
        \]
        This implies \(d_2(x^{(n)}, x) \le c_2 d_1(x^{(n)},x) \le \varepsilon \) for \(n \ge N\), which means 
        \[
            \lim_{n \to \infty} x^{(n)} = x \text{ in } (X,d_2). 
        \]  
        \item [\((\impliedby )\) ] Similar.
    \end{itemize}
\end{proof}

\begin{remark}
    On \(\mathbb{R} ^n\), the metrics \(d_1, d_2, d_\infty \) are Lipschitz equivalent, that is, 
    \[
        \lim_{n \to \infty} x^{(n)} = x \text{ in } (\mathbb{R}^n, d_1) \iff \lim_{n \to \infty} x^{(n)} = x \text{ in } (\mathbb{R}^n, d_2) \iff \lim_{n \to \infty} x^{(n)} = x \text{ in } (\mathbb{R}^n, d_\infty )  
    \]   
\end{remark}

\begin{proposition}
    Let \((X, d_{\text{disc}})\) be a discrete metric space, and \(\left\{ x^{(n)} \right\}_{n=m}^{\infty} \subseteq X \) . Then 
    \[
        \lim_{n \to \infty} x^{(n)} = x \text{ in } (X, d_{\text{disc}}) \iff \exists N \ge m \text{ s.t. } x^{(n)}=x \text{ for } n \ge N. 
    \] 
\end{proposition}
\begin{proof}
    \begin{itemize}
        \item [\((\impliedby )\) ] Easy. 
        \item [\((\implies )\) ] Given \(\frac{1}{2} > 0\), there exists \(N \ge m\) s.t. \(d(x_n, x) < \frac{1}{2}\) for \(n \ge N\), but \(d(x_n, x) < \frac{1}{2}\) implies \(d(x_n, x) = 0\), which means \(x_n = x\) for all \(n \ge N\).       
    \end{itemize}
\end{proof}

\begin{definition*}
    We define the interior, exterior, and boundary point again. 
    \begin{definition}
        The set of interior points is denoted by
        \[
            \mathrm{Int} (E) = \left\{ x \in X \mid \exists r > 0 \text{ s.t. } B_X(x, r) \subseteq E \right\}.
        \] 
    \end{definition}
    \begin{definition}
       The set of exterior points is denoted by
        \[
            \mathrm{Ext} (E) = \left\{ x \in X \mid \exists r > 0 \text{ s.t. } B_X(x, r) \subseteq X \setminus E \right\}.
        \]  
    \end{definition}
    \begin{definition}
        A point is a boundary points if it is neithe an interior point nor an exterior point, and we define
        \[
            \partial E = \left\{ x \in X \mid x \notin \mathrm{Int}(E) \text{ and } x \notin \mathrm{Ext}(E) \right\}. 
        \]
    \end{definition}
\end{definition*}

\begin{remark}
    \vphantom{text}
    \begin{itemize}
        \item [1.]
        \[
            x_0 \notin \mathrm{Int} (E) \iff \forall r > 0, B_X(x_0, r) \cap (X \setminus E) \neq \varnothing.
        \]
        \item [2.]
        \[
           x_0 \notin \mathrm{Ext} (E) \iff \forall r > 0, B_X(x_0, r) \cap (E) \neq \varnothing.
        \]
        \item [3.] \(\mathrm{Int} (X \setminus E) = \mathrm{Ext} (E)\). 
        \item [4.] \(\partial E = \partial (X \setminus E) \) since 
        \[
            x_0 \in \partial E \iff x \notin \mathrm{Int}(E) \text{ and } \mathrm{Ext}(E) \iff x_0 \notin \mathrm{Ext}(X \setminus E) \text{ and } x_0 \notin \mathrm{Int}(X \setminus E).    
        \]  
        Also,
        \[
           x_0 \in \partial (X\setminus E) \iff x \notin \mathrm{Int}(X \setminus  E) \text{ and } \mathrm{Ext}(X \setminus E) \iff x_0 \notin \mathrm{Ext}(E) \text{ and } x_0 \notin \mathrm{Int}(E).  
        \]
        Hence, acutually \(\partial E = \partial (X \setminus E)\). 
    \end{itemize}
\end{remark}

\begin{proposition}
    \[
        x_0 \in \partial E \iff \text{ For any } r > 0, B_X(x_0, r) \cap E \neq \varnothing \text{ and } B_X(x_0, r) \cap (X \setminus E) \neq \varnothing 
    \]
\end{proposition}

\begin{eg}
    Let \((\mathbb{R} , d)\) be the usual metric on \(\mathbb{R} \), where 
    \[
        d(x,y) = \vert x-y \vert. 
    \]  
    Then, we know in this space,
    \begin{align*}
        B_\mathbb{R} (x_0, r) &= \left\{ x \in \mathbb{R} \mid d(x, x_0) < r \right\} \\
        &= \left\{ x \in \mathbb{R} \mid \vert x - x_0 \vert < r \right\} \\
        &= \left\{ x \in \mathbb{R} \mid -r + x_0 < x < r + x_0 \right\}.
    \end{align*}

    Hence, suppose \(E = [1, 2)\), then \(\mathrm{Int}(E) = (1,2) \) since we know \(B(x_0, r) = (x_0 - r, x_0 + r)\), so for all \(x \in (1,2)\), we know there is an open ball \(B(x_0, r) \subseteq [1, 2)\) for some \(r > 0\). Also, consider the endpoint \(1,2\), we can verify that these two points are not interior points. Besides, consider the points not in \([1, 2]\), it is trivial that they cannot be interior points.  
\end{eg}

\begin{eg}
    We consider \((X, d_{\text{disc}})\). Let \(E \subseteq X\). If \(x \in E\), we know 
    \[
        B\left( x, \frac{1}{2} \right) = \left\{ y \mid d(y,x) < \frac{1}{2} \right\} = \left\{ x \right\} \subseteq E.  
    \]   
    Hence, \(E \subseteq  \mathrm{Int}(E)  \).  Besides, for all \(x \in \mathrm{Int}(E) \), we know there exists \(r > 0\) s.t. \(B(x_0, r) \subseteq E\), also we know \(x_0 \in B(x_0, r) \subseteq E\), so \(x_0 \in E\), and thus \(\mathrm{Int}(E) \subseteq E \). Hence, \(E = \mathrm{Int}(E) \).  Similarly, \(\mathrm{Int}(X \setminus E) = X \setminus E\).  Suppose there is a \(x \in X\) s.t. \(x \in \partial E\), then \(x \notin \mathrm{Int}(E) = E\) and \(x \notin \mathrm{Ext}(E) = \mathrm{Int}(X \setminus E) = X \setminus E\), so such \(x\) does not exist.    
\end{eg}

\begin{definition}[Closure] \label{def: closure}
    Let \((X, d)\) be a metric space, and let \(E \subseteq X\) and \(x_0 \in X\). We say \(x_0\) is an adherent point of \(E\) if for every \(r > 0\), \(B(x_0, r) \cap E \neq \varnothing \). The set of adeherent points is called the closure of \(E\), and denoted by \(\overline{E} \).        
\end{definition}

\begin{proposition}[TFAE] \label{prop: adherent point TFAE}
    \vphantom{text}
    \begin{itemize}
        \item [(a)] \(x_0\) is an adherent point of \(E\). 
        \item [(b)] \(x_0\) is either an interior point or a boundary point of \(E\). 
        \item [(c)] \(\exists \) a sequence \(\left\{ X^{(n)} \right\}_{n=1}^{\infty}  \) in \(E\) which converges to \(x_0\) in \((X, d)\).        
    \end{itemize}
\end{proposition}
\begin{proof}[proof from (a) to (b)]
    Suppose \(x_0 \in \overline{E} \), then \(B(x_0, r) \cap E \neq \varnothing \) for all \(r > 0\). If \(\exists s > 0\) s.t. \(B(x_0, s) \subseteq E\), then \(x_0 \in \mathrm{Int}(E)\). If such \(s\) does not exists, then we know
    \[
        B(x_0, r) \cap E \neq \varnothing \text{ and } B(x_0, r) \cap (X \setminus E) \neq \varnothing \text{ for all } r>0,
    \]
    so we can use \autoref{prop: boundary equivalent} to conclude that \(x_0\) must be a boundary point.        
\end{proof}
\begin{proof}[proof from (b) to (c)]
    Since either \(x_0 \in \mathrm{Int}(E) \) or \(x_0 \in \partial E\). If \(x_0 \in \mathrm{Int}(E) \), then \(x_0 \in E\), then we can choose \(X^{(n)} = x_0\) for all \(n \ge 1\). If \(x_0 \in \partial E\), then given \(n \in \mathbb{N} \), \(\exists x_n \in B\left( x_0, \frac{1}{n} \right) \cap E \neq \varnothing \). Hence, \(x_n \in E\) and \(d(x_n, x_0) < \frac{1}{n}\). Pick such \(x_n\) to form \(\left\{ X^{(n)} \right\}_{n=1}^{\infty}  \), then we know this sequence converges to \(x_0\).       
\end{proof}
\begin{proof}[proof from (c) to (a)]
    Suppose \(\left\{ X^{(n)} \right\}  \subseteq E\) s.t. \(\lim_{n \to \infty} d\left( X^{(n)}, x_0 \right) = 0  \), then we want to show \(x_0 \in \overline{E} \). Given any \(r > 0\), choose \(N \ge 1\) s.t. 
    \[
        d \left( X^{(n)}, x_0 \right) < r \text{ when } n \ge N. 
    \]     
    This implies for \(n \ge N\), \(X^{(n)} \in E\) and \(X^{(n)} \in B(x_0, r)\), so we know \(E \cap B(x_0, r) \neq \varnothing \) for all \(r > 0\), which means \(x_0 \in \overline{E} \).   
\end{proof}

\begin{remark}
    The equation (a) and (b) implies \(\overline{E} = \mathrm{Int}(E) \cup \partial E  \). 
\end{remark}
\begin{proof}[An alternative proof]
    Since we know \(X = \mathrm{Int}(E)  \cupdot \mathrm{Ext}(E) \cupdot \partial E \) by \autoref{thm: X is int and ext and boundary}, and \(\overline{E} \subseteq X \), so 
    \begin{align*}
        \overline{E} &= \overline{E} \cap X = \overline{E} \cap (\mathrm{Int}(E) \cup \mathrm{ext}(E) \cup \partial E )   \\
        &= (\overline{E} \cap \mathrm{Int}(E) ) \cup (\overline{E} \cap \mathrm{Ext}(E) ) \cup (\overline{E} \cap \partial E). 
    \end{align*}
    Also, notice that 
    \[
        \overline{E} \cap \mathrm{Int}(E) = \mathrm{Int}(E) \quad \overline{E} \cap \mathrm{Ext}(E) = \varnothing \quad \overline{E} \cap \partial E = \partial E,    
    \] so \(\overline{E} = \mathrm{Int}(E) \cup \partial E\). 
\end{proof}

\begin{corollary} \label{cl: Ebar is IntE cup ptE}
    \(\overline{E} = \mathrm{Int}(E) \cup \partial E  \). 
\end{corollary}

\begin{theorem} \label{thm: X is int and ext and boundary}
    Let \((X,d)\) be a metric space and \(E \subseteq X\). Then, 
    \[
        X = \mathrm{Int}(E) \cupdot \mathrm{Ext}(E) \cupdot \partial E  
    \]  
\end{theorem}

\begin{remark}
    \(\partial E\) could be empty. (See previous example.) 
\end{remark}

\begin{corollary}
    Let \((X,d)\) be a metric space and \(E \subseteq X\). Then 
    \[
        \overline{E} = \mathrm{Int}(E) \cup \partial E = X \setminus \mathrm{Ext}(E).   
    \]  
\end{corollary}

\begin{lemma} \label{lm: Ebar is E cup pt E}
    \(\overline{E} = E \cup \partial E \) 
\end{lemma}
\begin{proof}
    We first show that \(E \cup \partial E \subseteq \overline{E} \). For every point \(x \in E\), we know \(x \in B(x, r)\) for all \(r > 0\), so \(B(x,r) \cap E \neq \varnothing \). Also, by definition, we know \(\partial E \subseteq \overline{E}\), so we're done. 
    
    Next, we show that \(\overline{E} \subseteq E \cup \partial E \). For every \(x \in \overline{E} \), if \(x \in E\), then \(x \in E \cup \partial E\). If not, since \(x \in \overline{E} \), so \(B(x, r) \cap E \neq \varnothing \) for all \(r>0\). Also, since \(x \notin E\), and \(x \in B(x, r)\), so \(B(x, r) \cap (X \setminus E) \neq \varnothing \), otherwise \(x \in B(x, r) \subseteq E\), which is a contradiction. Now we know for every \(r>0\), \(B(x,r) \cap E \neq \varnothing \) and \(B(x,r) \cap (X \setminus E) \neq  \varnothing \), so \(x \in \partial E\).           
\end{proof}

\begin{lemma}[Discarded] \label{lm: x is interior means x in E}
    If \(x \in \mathrm{Int}(E) \), then \(x \in E\). In other words, \(\mathrm{Int}(E) \subseteq E \). 
\end{lemma}
\begin{proof}
    If \(x \in \mathrm{Int}(E) \), then there exists \(r > 0\) s.t. \(B(x, r) \subseteq E\), and thus \(x \in B(x, r) \subseteq E\), which means \(x \in E\).     
\end{proof}

\begin{note}
    I thought we need \autoref{lm: x is interior means x in E} to prove \autoref{thm: E is closed iff barE = E}, but I found it needless. Nevertheless, I still want to keep it since I think it is useful in some elsewhere.
\end{note}

\begin{definition}
    Let \((X, d)\) be a metric space and \(E \subseteq X\). We say \(E\) is closed if \(\partial E \subseteq E\).  
    We say \(E\) is open if it doesn't contain any boundary points i.e. \(\partial E \cap E\ = \varnothing \).  
\end{definition}

\begin{theorem} \label{thm: E is closed iff barE = E}
    \(E\) is closed if and only if \(\overline{E} = E \).  
\end{theorem}
\begin{proof}
    \begin{align*}
        E \text{ is closed } &\implies  \partial E \subseteq E \implies \overline{E} = E \cup \partial E = E. \\
        E = \overline{E} = E \cup \partial E &\implies \partial E \subseteq E \implies E \text{ is closed}.  
    \end{align*}  
\end{proof}

\begin{theorem} \label{thm: E open iff int is E}
    \(E \text{ is open.} \iff \mathrm{Int}(E) = E \). 
\end{theorem}
\begin{proof}[proof of \((\implies )\) ]
    \(E\) is open means \(\partial E \cap E = \varnothing \). Fix \(x \in E\), since \(x \notin \partial E\), so \(\exists r > 0\) s.t. \(B(x,r) \cap E = \varnothing \) or \(B(x,r) \cap (X \setminus E) = \varnothing \). Since \(x \in E\) and \(x \in B(x,r)\), so \(B(x,r) \cap (X \setminus E) = \varnothing \), which means \(B(x,r) \subseteq E\), so \(x \in \mathrm{Int} (E)\). Now we know \(E \subseteq \mathrm{Int}(E) \). Also, we know \(\mathrm{Int}(E) \subseteq E \) by \autoref{lm: x is interior means x in E}. Hence, \(\mathrm{Int}(E) = E \).           
\end{proof}
\begin{proof}[proof of \((\impliedby )\) ]
    If \(\mathrm{Int}(E) = E \), then given any \(x \in E = \mathrm{Int}(E) \), there exists \(r>0\) s.t. \(B(x,r) \subseteq E\). Hence, \(B(x,r) \cap (X \setminus E) = \varnothing \), so \(x \notin \partial E\), and thus \(E \cap \partial E = \varnothing \).     
\end{proof}

\begin{theorem} \label{thm: E open iff X--E closed}
    If \(E \subseteq X\), then \(E\) is open \(\iff X \setminus E\) is closed.  
\end{theorem}
\begin{proof}[proof of \((\implies )\)]
    Since we can write \(X = \mathrm{Int}(E) \cupdot \mathrm{Ext}(E) \cupdot \partial E\), and \(E\) is open, so 
    \[
        X \setminus E = (\mathrm{Int}(E) \cupdot \mathrm{Ext}(E) \cupdot \partial E) \setminus E = (\mathrm{Int}(E) \cupdot \mathrm{Ext}(E) \cupdot \partial E) \setminus \mathrm{Int}(E) = \mathrm{Ext}(E) \cupdot \partial E.
    \] by \autoref{thm: E open iff int is E}. Now we want to show that \(\partial (X \setminus E) \subseteq X \setminus E\), and we know 
    \[
        X \setminus E = \mathrm{Ext}(E) \cupdot \partial E = \mathrm{Ext}(E) \cupdot \partial (X \setminus E)  
    \] since \(\partial E = \partial (X \setminus E)\). Hence, we have \(\partial (X \setminus E) \subseteq X\setminus E\).  
\end{proof}
\begin{proof}[proof of \((\impliedby) \)]
    Suppose \(X \setminus E\) is closed, then \(\partial (X \setminus E) \subseteq X \setminus E\), and since \(\partial E = \partial (X \setminus E)\), so \(\partial E \subseteq X \setminus E\), and thus \(\partial E \cap E = \varnothing\), which means \(E\) is open.     
\end{proof}