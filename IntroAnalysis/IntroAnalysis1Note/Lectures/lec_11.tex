\lecture{11}{7 Oct. 10:20}{}
\begin{theorem}[Review of \autoref{thm: continuous map preserve connectedness}]
    Let \(f: X \to Y\) be a continuous map and let \(E\) be a connected subset of \(X\). Then \(f(E)\) is connected in \(Y\).     
\end{theorem}
\begin{corollary}[Intermediate value theorem] \label{cl: Intermediate value theorem}
    Let \(f: X \to \mathbb{R} \) be a continuous map from \((X, d_X)\) to \(\mathbb{R} \). Let \(E \subseteq X\) be any connected subset, and let \(a, b \in E\). Suppose \(y\) is a real number between \(f(a)\) and \(f(b)\) i.e. 
    \[
        \min \left\{ f(a), f(b) \right\} \le y \le \max \left\{ f(a), f(b) \right\},  
    \] then \(\exists c \in E\) s.t. \(f(c) = y\).  
\end{corollary}
\begin{proof}
    There are \(3\) cases: 
    \begin{itemize}
        \item Case 1: \(f(a) = f(b)\), then trivial. 
        \item Case 2: \(f(a) < f(b)\), Since \(E\) is connected and \(f\) is continuous, so \(f(E)\) is connected in \(\mathbb{R} \). Hence, for \(f(a), f(b) \in f(E)\), we know \(\left( f(a), f(b) \right) \subseteq f(E) \) by \autoref{thm: connectedness in R TFAE}, so if \(f(a) < y < f(b)\), then \(\exists c \in E\) s.t. \(f(c) = y\).           
        \item Case 3: \(f(a) > f(b)\), then let \(a^{\prime} = b\) and \(b^{\prime} = a\), then \(f \left( a^{\prime}  \right) < f \left( b^{\prime}  \right)  \) and use the result of Case 2.
    \end{itemize} 
\end{proof}
\section{Topological space}
In metric space \((X, d_X)\), we define open ball
\[
    B_X(x, r) = \left\{ y \mid d_X (y, x) < r \right\},
\]
and a set \(u\) is open if for any \(x \in u\), \(\exists r_x > 0\) s.t. \(B_X(x, r_x) \subseteq u\), so \(u = \bigcup_{x \in u} B_X(x, r_x) \). Hence, in metric space open sets are in fact union of open balls. We also proved that 
\begin{itemize}
    \item \(\varnothing  \) and \(X\) are open. 
    \item  If \(u_1, \dots , u_n\) are open in \(X\), then \(\bigcap_{i=1}^{n} u_i \) is open in \(X\). 
    \item If \(\left\{ u_i \right\}_{i \in A} \) are open in \(X\), then \(\bigcup_{i \in A} u_{i} \) is also open.       
\end{itemize}    

Now we want to extend this concept. 
\begin{definition}[Power sets] \label{def: power sets}
    For a given set \(X\), we define \(2^X\) the power set of \(X\) i.e. 
    \[
        2^X \coloneqq \left\{ A:A\subseteq X \right\} 
    \] the collection of all subsets of \(X\). 
\end{definition}

\begin{eg}
    \(X = \left\{ a, b \right\} \) for \(a \neq b\), then 
    \[
        2^X = \left\{ \varnothing , \left\{ a \right\}, \left\{ b \right\}, \left\{ a, b \right\}    \right\}. 
    \]  
\end{eg}

\begin{definition}[Topological space] \label{def: topological space}
    A topological space is a pair \(\left( X, \mathcal{F}  \right) \), where \(X\) is a set and \(\mathcal{F} \subseteq 2^X \) is a collection of subsets of \(X\), called the open sets. The collection \(\mathcal{F} \) must satisfy 
    \begin{itemize}
        \item \(\varnothing  \) and \(X\) are all in \(\mathcal{F} \). 
        \item  If \(u_1, \dots , u_n\) are in \(\mathcal{F} \), then \(\bigcap_{i=1}^{n} u_i \) is in \(\mathcal{F} \). 
        \item If \(\left\{ u_i \right\}_{i \in A} \) are in \(\mathcal{F} \), then \(\bigcup_{i \in A} u_{i} \) is in \(\mathcal{F} \).       
\end{itemize}    
\end{definition}

\begin{remark}
    In a metric space, let 
    \[
        \mathcal{F} = \text{ the set of open sets in } (X, d_X) = \left\{ u \mid \forall x \in u, \exists r_x > 0 \text{ s.t. } B_X(x, r_x) \subseteq u \right\}, 
    \] then \((X, \mathcal{F}) \) is a topological space.  
\end{remark}

\begin{eg}
    On any set \(X \neq \varnothing \), we have a trivial topology on \(X\) i.e. \(\mathcal{F} = \left\{ \varnothing , X \right\} \), which means \((X, \mathcal{F} )\) is a topological space in \(X\).  
\end{eg}

\begin{eg}
    Consider \(\mathcal{F} = 2^X\), then \((X, 2^X)\) is also a topological space on \(X\).    
\end{eg}

\begin{definition}[Neighborhood] \label{def: neighborhood}
    Let \((X, \mathcal{F} )\) be a topological space, and let \(x \in X\). A neighborhood of \(x\) is any open set \(u \in \mathcal{F} \) s.t. \(x \in u\).     
\end{definition}

\begin{eg}
    \(X = \left\{ a, b \right\} \) and \(a \neq b\), \(\mathcal{F} = \left\{ \varnothing , \left\{ a \right\}, \left\{ b \right\}, \left\{ a, b \right\}    \right\} \), then \(\left\{ a \right\}, \left\{ a, b \right\}  \) are neighborhoods of \(a\) and \(\left\{ b \right\}, \left\{ a, b \right\}  \) are neighborhoods of \(b\).      
\end{eg}

\begin{definition}[Interior/Exterior/Boundary point]
    Let \((X, \mathcal{F} )\) be a topological space, and \(E \subseteq X\) be a subset. We say that 
    \begin{itemize}
        \item \(x_0\) is an interior point of \(E\) if \(\exists \) a neighborhood \(V\) of \(x_0\) s.t. \(V \subseteq E\). 
        \item \(x_0\) is an exterior point of \(E\) if \(\exists \) a neighborhood \(V\) of \(x_0\) s.t. \(V \subseteq X \setminus E\). 
        \item \(x_0\) is a boundary point if it is neither interior or exterior.             
    \end{itemize}  
\end{definition}

\begin{corollary}
    \[
        X = \mathrm{Int}(E) \cupdot \mathrm{Ext}(E) \cupdot \partial E.  
    \]
\end{corollary}
\begin{proof}
    \todo{DIY}
\end{proof}

\begin{definition}[Adherent point] \label{def: adherent point}
    We say \(x_0\) is an adherent point of \(E\) if every neighborhood \(V\) of \(x_0\) has a nonempty intersection with \(E\), and we called \(\overline{E} \) the set of all adherent points.     
\end{definition}

\begin{corollary}
    \(\overline{E} = \mathrm{Int}(E) \cup \partial E  \). 
\end{corollary}
\begin{proof}
    We first show that \(\mathrm{Int}(E) \cup \partial E \subseteq \overline{E}  \). Suppose \(x_0 \in \mathrm{Int}(E) \), then \(x_0 \in E\), so \(x_0 \in \overline{E} \) since \(E \subseteq \overline{E} \). Now if \(x_0 \in \partial E\), then any neighborhood \(V\) of \(x_0\) contains points in \(E\), so \(x_0 \in \overline{E} \).  
    Now we show that \(\overline{E} \subseteq \mathrm{Int}(E) \cup \partial E \). Since \(\overline{E} \subseteq X = \mathrm{Int}(E) \cup \mathrm{Ext}(E) \cup \partial E  \), so we want to show for all \(x_0 \in \overline{E} \), we have \(x_0 \notin \mathrm{Ext}(E) \). By the definition of exterior point, we can easily show this.    
\end{proof}

\begin{definition*}[Open and Closed Sets]\label{def:open-closed topological space ver}
    Suppose \((X, \mathcal{F})\) is a topological space. Then:
    \begin{itemize}
        \item A set \(O \subseteq X\) is called \emph{open} if \(O \in \mathcal{F}\).
        \item A set \(F \subseteq X\) is called \emph{closed} if its complement \(X \setminus F\) is open, i.e., \(X \setminus F \in \mathcal{F}\).
    \end{itemize}
\end{definition*}

\begin{corollary}
    A set \(E \subseteq X\) is open if and only if \(E = \mathrm{Int}(E) \). 
\end{corollary}
\begin{proof}
    \vphantom{text}
    \begin{itemize}
        \item [\((\implies )\)] If \(E\) is open, then since 
        \begin{equation} \label{eq: Int in tp space}
            \mathrm{Int}(E) = \bigcup\left\{ O \subseteq E : O \in \mathcal{F} \right\}, 
        \end{equation}
        we know \(E \subseteq \mathrm{Int}(E) \).  
        \item [\((\impliedby )\)] By the definition of topological space and \autoref{eq: Int in tp space}, we know \(\mathrm{Int}(E) \) is open, so \(E = \mathrm{Int}(E) \) implies \(E\) is open.    
    \end{itemize}
\end{proof}

\begin{corollary}
    A set \(F \subseteq X\) is closed if and only if \(\overline{F} = F\).
\end{corollary}
\begin{proof}
    If \(F\) is closed, then \(X \setminus F\) is open, so \(X \setminus F = \mathrm{Int}(X \setminus F) \), so 
    \[
        F = X \setminus (X \setminus F) = X \setminus \mathrm{Int}(X \setminus F) = X \setminus \mathrm{Ext}(F) = \mathrm{Int}(F) \cup \partial (F) = \overline{F}.    
    \] 
    The other direction is similar.  
\end{proof}

\begin{definition}[Topological subspace]
    Let \((X, \mathcal{F} )\)  be a topological space and \(Y \subseteq X\). We define \(\mathcal{F} _Y = \left\{ V \cap Y \mid V \in \mathcal{F}  \right\} \) and call \((Y, \mathcal{F} _Y)\) the topological subspace of \((X, \mathcal{F} )\) induced by \(Y\). We can show that \(\mathcal{F} _Y\) is a topology on \(Y\).       
\end{definition}

\begin{definition}[Continuous map]
    Let \((X, \mathcal{F} )\) and \((Y, g)\) be topological spaces ane let \(f:X \to  Y\) be a function. We say \(f\) is continuous at \(x_0 \in X\) if for every neighborhood \(V \in g\) of \(f(x_0)\), there exists a neighborhood \(u \in \mathcal{F} \) of \(x_0\) s.t. \(u \subseteq f^{-1}(V)\).          
\end{definition}

\begin{definition}[Convergence]
    Let \(m\) be an integer, \((X, \mathcal{F} )\) be a topological space, and let \(\left( x^{(n)} \right)_{n=m}^{\infty}  \) be a sequence of points in \(X\). We say \(x^{(n)} \to x\) iff for every neighborhood \(V\) of \(x\), \(\exists N \ge m\) s.t. \(x^{(n)} \in V\) for all \(n \ge N\).          
\end{definition}

\begin{remark}
    In a topological space, a sequence may converge to more than one points. 
\end{remark}

\begin{eg}
    Let \(X = \left\{ a, b \right\} \) and \(a \neq b\). Consider the trivial topology \(\mathcal{F} = \left\{ \varnothing , X \right\} \). If we have the constant sequence \(\left\{ x^{(n)} \right\}_{n=m}^{\infty}  \) with \(x^{(n)} = a\), then \(\lim_{n \to \infty} x^{(n)} = a \) and \(\lim_{n \to \infty} x^{(n)} = b \).      
\end{eg}

\begin{explanation}
    Any neighborhood of \(a\) must be \(\left\{ a, b \right\} \), and any neighborhood of \(b\) must also be \(\left\{ a, b \right\} \), so \(\lim_{n \to \infty} x^{(n)} = a \) and \(\lim_{n \to \infty}  x^{(n)} = b\).     
\end{explanation}

\begin{eg} \label{eg: R setminus u is finite point tp}
    On \(\mathbb{R} \), we consider 
    \[
        \mathcal{F} = \left\{ \varnothing  \right\} \cup \left\{ u: \mathbb{R} \setminus  u \text{ is finite points} \right\},  
    \] then \(\mathcal{F} \) is a topology on \(\mathbb{R} \).   
\end{eg}
\begin{explanation}
    Since \(\varnothing, \mathbb{R}  \in \mathcal{F} \), so it satisfies the first rule of topology. Now if \(u_1 \in \mathcal{F} \) and \(u_2 \in \mathcal{F} \), then we want to show \(u_1 \cap u_2 \in F\). We assume \(u_1, u_2\) are non-empty, otherwise it is trivial. Consider \(\mathbb{R} \setminus \left( u_1 \cap u_2 \right) = \left( \mathbb{R} \setminus u_1 \right) \cup \left( \mathbb{R} \setminus u_2 \right)   \), since  \(\mathbb{R} \setminus u_1\) and \(\mathbb{R} \setminus u_2\) are both finite points, so \(u_1 \cap u_2\) is finite points. Hence, we know for finitely many \(u_1, \dots , u_n\), \(\bigcap_{i=1}^{n} u_i \) is in \(\mathcal{F} \). Now we know for \(\mathcal{F} _\alpha \in \mathcal{F} \), 
    \begin{align*}
        \mathbb{R} \setminus \left( \bigcup_{\alpha } \mathcal{F} _{\alpha }  \right) = \bigcap_{\alpha } \left( \mathbb{R} \setminus \mathcal{F} _{\alpha } \right) \subseteq \mathbb{R} \setminus \mathcal{F}_{\alpha _i}  
    \end{align*}           
    for some \(\alpha _i\) in the index set, so \(\mathbb{R} \setminus \left( \bigcup_{\alpha } \mathcal{F} _\alpha   \right) \) is also finite points, and we're done. 
\end{explanation}

\begin{remark}
    In the topological space induced by the topology in \autoref{eg: R setminus u is finite point tp}. If we consider \(\left\{ x^{(n)} \right\}_{n=1}^{\infty}  \) with \(x^{(n)} = n\), then \(\lim_{n \to \infty} x^{(n)} = p \) for any \(p \in \mathbb{R} \).     
\end{remark}
\begin{proof}
    Since any neighborhood \(u\) of \(p\) has
    \[
        \mathbb{R} \setminus u = \left\{ p_1, \dots , p_k \right\} 
    \] with \(p_1 < p_2 < \dots < p_k\), so we have \(x^{(n)} \in u\) for \(n > p_k\).  
\end{proof}