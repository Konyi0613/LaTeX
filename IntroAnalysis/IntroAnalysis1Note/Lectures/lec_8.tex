\lecture{8}{25 Sep. 10:20}{}
\begin{theorem}[Review \autoref{thm: compact to subcover}]
    Let \(Y\) be a compact subset of a metric space \((X, d)\) and let \(\left\{ V_\alpha  \right\}_{\alpha \in A} \) be an open cover of \(Y\). Then \(\exists \) a finite subcover of \(\left\{ V_\alpha  \right\}_{\alpha \in A} \) i.e. \(\exists \alpha _1, \dots ,\alpha _n \in A\) s.t. \(Y \subseteq \bigcup_{i=1}^{n} V_{\alpha _i}. \)        
\end{theorem}

\begin{remark}
    \begin{align*}
        Y \text{ is compact } &\iff \text{ Any open cover of } Y \text{ has a finite subcover.}
    \end{align*}
\end{remark}
\begin{proof}
    The \((\implies )\) direction is proved. Now we proved the other direction.
    \begin{claim} \label{clm: compact complete open cover ver}
        If \((X, d)\) is a metric space and for all open cover of \(X\), there exists finite subcover of \(X\), then \(X\) is complete.    
    \end{claim} 
    \begin{explanation}
        We will prove this by contradiction, using the definition of compactness (the open cover property).

        \textbf{1. Assumption}
        Assume for the sake of contradiction that $X$ is \textbf{compact} but \textbf{not complete}.
        Since $X$ is not complete, there exists a \textbf{Cauchy sequence} $\{x_n\}_{n=1}^{\infty}$ in $X$ that \textbf{does not converge} to any point $p \in X$.

        \textbf{2. Constructing the Open Cover}
        Since $\{x_n\}$ does not converge to a point $p \in X$, for every $p \in X$, the point $p$ is not the limit of the sequence. This means there exists some $\epsilon_p > 0$ such that the open ball $B_{\epsilon_p}(p)$ contains only a \textbf{finite number of terms} of the sequence $\{x_n\}$.

        To see why, suppose for a contradiction that there was some $p \in X$ such that for all $\epsilon > 0$, the ball $B_{\epsilon}(p)$ contains an infinite number of terms of $\{x_n\}$. Let $\{x_{n_k}\}$ be a subsequence with $x_{n_k} \in B_{1/k}(p)$. This subsequence converges to $p$. Since $\{x_n\}$ is a Cauchy sequence and has a convergent subsequence, the entire sequence $\{x_n\}$ must converge to the same limit $p$, which contradicts our initial assumption.
        Therefore, the property holds: for every $p \in X$, there is an $\epsilon_p > 0$ such that $B_{\epsilon_p}(p)$ contains $x_n$ for only finitely many $n$.

        Consider the collection of open balls
        $\mathcal{U} = \{B_{\epsilon_p}(p) : p \in X\}.$
        Since the union of these balls covers every point $p \in X$, $\mathcal{U}$ is an \textbf{open cover} of $X$:
        $$X \subseteq \bigcup_{p \in X} B_{\epsilon_p}(p).$$

        \textbf{3. Using Compactness to Find a Finite Subcover}
        Since $X$ is \textbf{compact}, the open cover $\mathcal{U}$ must have a \textbf{finite subcover}. That is, there exist a finite number of points $p_1, p_2, \dots, p_k \in X$ such that
        $$X \subseteq B_{\epsilon_{p_1}}(p_1) \cup B_{\epsilon_{p_2}}(p_2) \cup \dots \cup B_{\epsilon_{p_k}}(p_k) = \bigcup_{i=1}^k B_{\epsilon_{p_i}}(p_i).$$

        \textbf{4. Reaching the Contradiction}
        By the definition of $\epsilon_{p_i}$, each ball $B_{\epsilon_{p_i}}(p_i)$ contains $x_n$ for only a \textbf{finite number} of indices $n$.
        The union of a finite number of finite sets is a finite set. Therefore, the finite union
        $\bigcup_{i=1}^k B_{\epsilon_{p_i}}(p_i)$
        can contain $x_n$ for only a finite number of indices $n$.
        However, since this finite union covers all of $X$ (step 3), it must contain \textbf{all} terms of the sequence $\{x_n\}_{n=1}^{\infty}$.
        Since the sequence $\{x_n\}$ is an infinite set of points, this is a \textbf{contradiction}.

        The initial assumption that $X$ is not complete must be false.
        Thus, every compact metric space is complete.
    \end{explanation}      
    Suppose any open cover of \(Y\) has a finite subcover, then given any sequence \(\left( y^{(n)} \right)_{n=1}^{\infty}  \). Consider 
    \[
        \bigcup_{x \in Y} B_Y \left( x, 1 \right),  
    \] then this is an open cover of \(Y\), and now we know there is a finite subcover
    \[
        \bigcup_{i=1}^{k} B_Y \left( x_{i} , 1 \right) 
    \] of \(Y\) where \(x_i \in Y\) for all \(i\). Now since \(\left( y^{(n)} \right)_{n=1}^{\infty}  \) has infinitely many terms, so we know for some \(i\), we have infinitely many terms of \(\left( y^{(n)} \right)_{n=1}^{\infty} \subseteq B_Y \left( x_i, 1 \right)  \) by Pigeonhole principle. Hence, there are infinitely many terms of \(\left( y^{(n)} \right)_{n=1}^{\infty}  \) are in 
    \[
        \left\{ y \in Y: 0 \le d(y, x_i) < \frac{1}{2} \right\} \cup \left\{ y \in Y: \frac{1}{2} \le d(y, x_i) < 1 \right\}.  
    \]  Thus, again, by Pigeonhold principle we know there are infinitely many terms of \(\left( y^{(n)} \right)_{n=1}^{\infty}  \) are in either one of the above two sets. By repeating split the space into half as what we do above, we know for all \(k \ge 0\), there are infinitely many terms of \(\left( y^{(n)} \right)_{n=1}^{\infty}  \) has the following property: Every two terms of these "infinitely many terms" has distance less than \(\frac{1}{2^k}\). Note that this means we can pick a subsequence of \(\left( y^{(n)} \right)_{n=1}^{\infty}  \) so that it is Cauchy, and since every Cauchy sequence converges in \(Y\) (Since \autoref{clm: compact complete open cover ver}), so we're done.
    
    
\end{proof}
\begin{corollary} \label{cl: compact nesting then intersection nonempty}
    Let \((X, d)\) be a metric space and let \(K_1, K_2, \dots \) be a sequence of nonempty compact subsets of \(X\) s.t. \(K_{i+1} \subseteq K_i\) for \(i \in \mathbb{N} \), that is, \(K_1 \supseteq K_2 \supseteq K_3 \supseteq \dots  \) for \(i \in \mathbb{N} \), then 
    \[
        \bigcap_{i=1}^{\infty} K_i \neq \varnothing. 
    \]       
\end{corollary}
\begin{proof}
    Suppose \(\bigcap_{i=1}^{\infty} K_i = \varnothing  \). Since \(K_i\)'s are compact, so they are closed. Also, we have 
    \[
        \bigcup_{i=1}^{\infty} \left( K_1 \setminus K_n \right) = K_1 \setminus \left( \bigcap_{i=1}^{\infty} K_n \right) = K_1.
    \]  
    Let \(V_i = K_1 \setminus K_i = K_1 \cap K_i^C\). Note that \(K_i^C\) is open in \(X\). Hence, we have \(V_i\) is open in \(K_1\), and thus \(\left\{ V_i \right\}_{i=1}^{\infty}  \) is an open cover of \(K_1\) in \(K_1\). (\((K_1, d\vert_{K_1 \times K_1})\) is compact.) By \autoref{thm: compact to subcover}, we know there exists \(\alpha _1, \alpha _2, \dots , \alpha _l\) with \(\alpha _1 < \alpha _2 < \dots < \alpha _l\) s.t. 
    \begin{align*}
        K_1 &\subseteq \bigcup_{i=1}^{l} V_{\alpha _i} = \bigcup_{i=1}^{l} \left( K_1 \setminus K_{\alpha _i} \right) \\
            &= K_1 \setminus \bigcap_{i=1}^l K_{\alpha _i} = K \setminus K_{\alpha _l} 
    \end{align*} since \(K_{\alpha _1} \supseteq K_{\alpha _2} \supseteq \dots \supseteq K_{\alpha _l}\). However, \(K_{\alpha _l} \subseteq K_1\) and \(K_{\alpha _l} \neq \varnothing \). Thus, we have a contradiction.  
\end{proof}

\begin{eg}
    Consider \(I_1 = [0, 1]\), and \(I_2 = [0, \frac{1}{3}] \cup [\frac{2}{3}, 1]\), and picking \(I_3, I_4, \dots \) with same method, then \(I_{n+1} \subseteq I_n\) for all \(n\) and they are compact, so
    \[
        \bigcap_{i=1}^{\infty} I_i \neq \varnothing . 
    \]      
\end{eg}

\begin{theorem}
    Let \((X, d)\) be a metric space. 
    \begin{itemize}
        \item [(a)] If \(Y\) is a compact subset of \(X\), and \(Z \subseteq Y\), then \(Z\) is compact iff \(Z\) is closed.      
        \item [(b)] If \(Y_1, \dots , Y_n\) are a finite collection of compact subsets of \(X\), then \(\bigcup_{i=1}^{n} Y_i \) are also compact.   
    \end{itemize} 
\end{theorem}
\begin{proof}[proof of (a)]
    If \(Z\) is compact, then by \autoref{cl: compact subset of X is closed and bounded}, we know \(Z\) is closed. Now we show that if \(Z\) is closed, then \(Z\) is comapct. If \(Z\) is closed, then \(Y \setminus Z\) is open in \(Y\), then we know 
    \[
        Y \setminus Z = V \cap Y
    \] for some open set \(V \subseteq Y\), so note that \((Y \setminus Z) \subseteq V\). Now suppose \(\left\{ U_\alpha  \right\}_{\alpha \in A} \) is an open cover of \(Z\). Hence, we know \(\left\{ U_\alpha  \right\}_{\alpha \in A} \cup \left\{ V \right\}  \) is an open cover of \(Y\) since the former covers \(Z\) and the latter covers \(Y \setminus Z\). Now since \(Y\) is compact, so we know for some \(\alpha _1, \alpha _2, \dots , \alpha _n\), there is 
    \[
        Y \subseteq \left( \bigcup_{i=1}^{n} U_{\alpha _i} \right) \cup V, 
    \] and thus we can write
    \[
        Z \subseteq Y \subseteq \left( \bigcup_{i=1}^{n} U_{\alpha _i}  \right) \cup V. 
    \]  
    However, note that \(Z \cap V = \varnothing \) since
    \[
        Z = Y \setminus (Y \setminus Z) = Y \setminus (V \cap Y) = (Y \setminus V) \cup (Y \setminus Y) = Y \setminus V.
    \]
    Hence, we know 
    \[
        Z \subseteq \bigcup_{i=1}^{n} U_{\alpha _i}, 
    \] and thus for any open cover of \(Z\), we know there exists a finite subcover of \(Z\), and we're done.
\end{proof}
\chapter{Continuous functions on metric spaces}
Suppose \((X, d_x)\) and \((Y, d_y)\) are metric space. Let \(f: X \to Y\) be a function from \(X\) to \(Y\). Then we want that if \(x \in X\) is close to \(y \in X\), then, then \(f(x)\in Y\) is close to \(f(y) \in Y\). 

\begin{definition}[Continuous function] \label{def: continuous}
    Let \((X, d_X)\) and \((Y, d_Y)\) be metric spaces and let \(f: X \to Y\) be a function. Suppose \(x_0 \in X\), we will say \(f\) is continous at \(x_0\) iff for every \(\varepsilon > 0\), there exists \(\delta > 0\) s.t. 
    \[
        d_Y(f(x), f(x_0)) < \varepsilon \quad \text{whereas } d_X(x, x_0) < \delta.
    \] We say \(f\) is continuous if \(f\) is continuous at every point \(x \in X\).     
\end{definition}

\begin{definition}[Preimage] \label{def: preimage}
    Let \(f : X \to Y\) be a function from \(X \to Y\) and \(V \subseteq Y\). The preimage (inverse image) of \(V\) is 
    \[
        f^{-1}(V) = \left\{ x\in X \mid f(x) \in V \right\}.
    \]    
\end{definition}

\begin{eg}
    Suppose \(f(x) = x^2\), then what is the preimage of \((1, \infty )\)?  
\end{eg}
\begin{explanation}
    \[
        f^{-1}((1, \infty )) = (-\infty , -1) \cup (1, \infty ).
    \]
\end{explanation}
Now we build an equivalent definition of continuity. If \(f\) is continuous at \(x_0\), then given any \(\varepsilon > 0\), \(\exists \delta > 0\) s.t. 
\[
    f(x) \in B_Y(f(x_0), \varepsilon ) \quad \text{whereas } x \in B_X(x_0, \delta ).
\]  Also, \(f(x) \in B_Y(f(x_0), \varepsilon )\) if and only if
\[
    x \in f^{-1}(B_Y(f(x_0, \varepsilon ), \varepsilon )).
\] Hence, we have 
\begin{corollary}
    \(f\) is continuous at \(x_0\) if and only if 
    \begin{center}
        Given any \(\varepsilon > 0\), \(\exists \delta > 0\), we have \(B_X(x_0, \delta ) \subseteq f^{-1}(B_Y(f(x_0), \varepsilon )).\).   
    \end{center}  
\end{corollary}

\begin{remark} \label{rmk: if f conti then f intersect K conti}
    If \(f: X \to Y\) is continuous and \(K \subseteq X\), then \(f\vert_K : K \to Y\) is continuous.   
\end{remark}
\begin{explanation}
    Given any point \(x_0 \in K \subseteq X\). Since \(f\) is continuous at \(x_0\), so \(\forall \varepsilon > 0\), \(\exists \delta > 0\) s.t. 
    \[
        d\left( f(x), f(x_0) \right) < \varepsilon \quad \text{ if } d_X(x, x_0) < \delta. 
    \]     
    If \(z \in K\) and \(d_K(z, x_0) < \delta \), then \(d_Y(f(z), f(x)) <\varepsilon \), so \(f\) is continuous on \(K\).     
\end{explanation}

\begin{theorem} \label{thm: continuous TFAE 1}
    Suppose that \((X, d_X)\) and \((Y, d_Y)\) be metric spaces, and \(f:X \to Y\) is a function and let \(x_0 \in X\), then TFAE: 
    \begin{itemize}
        \item [(a)] \(f\) is continuous at \(x_0\).  
        \item [(b)] Whenever \(\left( x^{(n)} \right)_{n=1}^{\infty}  \) is a sequence in \(X\) converges to \(x_0\), then 
        \[
            \lim_{n \to \infty} f \left( x^{(n)} \right) = f(x_0) \text{ in } \left( Y, d_Y \right).   
        \]
        \item [(c)] For every open set \(V \subseteq Y\) that contains \(f(x_0)\), \(\exists \) a open set \(U \subseteq X\) containing \(x_0\) s.t. \(f(U) \subseteq V\), or equivalently, \(U \subseteq f^{-1}(V)\) .      
    \end{itemize}  
\end{theorem}
\begin{proof}[proof of (a) \(\implies \) (b)]
    Given any \(\varepsilon > 0\), since \(f\) is continuous at \(x_0\), so \(\exists \delta > 0\) s.t. if \(d_X(x, x_0) < \delta \), then \(d_Y (f(x), f(x_0)) < \varepsilon \). Now if \(\lim_{n \to \infty} x^{(n)} = x_0 \). Then there exists \(N > 0\) s.t. \(n \ge N\) implies \(d_X \left( x^{(n)}, x_0 \right) < \delta  \). Hence, we know \(d_Y \left( f \left( x^{(n)}\right),f(x_0)   \right) < \varepsilon \). Hence, for this \(\varepsilon \), we know there exists \(N\) s.t. \(n \ge N\) implies \(d_Y \left( f \left( x^{(n)} \right), f(x_0)  \right) < \varepsilon  \), and thus \(\lim_{n \to \infty} f \left( x^{(n)} \right) = f(x_0)  \).         
\end{proof}
\begin{proof}[proof of (b) \(\implies \) (c)]
    Let \(f(x_0) \in V \subseteq Y\).
    \begin{claim}
        There exists an open set \(u\) s.t. \(x_0 \in u \subseteq X\) and \(f(u) \subseteq V\).   
    \end{claim}
    \begin{explanation}
        If this is not true, then this implies that for every open set \(u\) with \(x_0 \in u\), consider \(B_X \left( x_0, \frac{1}{n} \right) \), \(\exists x_u \in u\) and \(f(x_u) \notin V\), then pick all of this \(x_u\) to be \(\left\{ x^{(n)} \right\}_{n=1}^{\infty}  \), we know \(\forall x^{(n)}\) we have \(f \left( x^{(n)} \right) \notin V \).    
        Then, \(\left\{ x^{(n)} \right\}_{n=1}^{\infty}  \) is a sequence converges to \(x_0\). By (b), we know \(\lim_{n \to \infty} f \left( x^{(n)} \right) = f(x_0)  \). However, by our choice, \(f \left( x^{(n)} \right) \notin V \), so \(f \left( x^{(n)} \right) \in Y \setminus V \). Since \(V\) is open, so \(Y\setminus V\) is closed. Hence, we must have \(\lim_{n \to \infty} f \left( x^{(n)} \right) = f(x_0) \in Y \setminus V  \), which is a contradiction. 
    \end{explanation}
\end{proof}
\begin{proof}[proof of (c) \(\implies \) (a)]
    Suppose (c) holds, then we want to show \(f\) is continuous at \(x_0\), which means for all \(\varepsilon > 0\), there exists \(\delta > 0\) s.t. \(B_X(x_0, \delta ) \subseteq f^{-1} \left( B_Y(f(x_0), \varepsilon ) \right) \). Now consider \(V = B_Y(f(x_0), \varepsilon ) \subseteq Y\), then by (c) we know there exists open \(U \subseteq X\) s.t. \(x_0 \in U \subseteq X\) s.t. \(U \subseteq f^{-1}(V) \). Now since \(U\) is open and \(x_0 \in U\), so there exists \(B_X(x_0, \delta ) \subseteq U\), and thus 
    \[
        B_X(x_0, \delta ) \subseteq U \subseteq f^{-1}(V) = f^{-1} \left( B_Y (f(x_0), \varepsilon ) \right), 
    \] and we're done.
\end{proof}

\begin{theorem}
    Suppose \(f: X \to Y\), then TFAE 
    \begin{itemize}
        \item [(a)] \(f\) is continuous. 
        \item [(b)] If \(\lim_{n \to \infty} x^{(n)} = x \in (X, d_X)\), then \(\lim_{n \to \infty} f \left( x^{(n)} \right) = f(x)  \) in \(\left( Y, d_Y \right) \). 
        \item [(c)] If \(V\) is open in \(Y\), then \(f^{-1}(V)\) is open in \(X\). 
        \item [(d)] Whenever \(F\) is closed in \(Y\), then \(f^{-1}(F)\) is closed in \(X\).         
    \end{itemize}
\end{theorem}
\begin{proof}[(a) \(\iff \) (b)]
    By \autoref{thm: continuous TFAE 1}, we know it is true. 
\end{proof}
\begin{proof}[(c) is equivalent to (c) in \autoref{thm: continuous TFAE 1}]
    For each \(x \in f^{-1}(V)\), we have \(f(x) \in V\), so there exists an open set \(u_x\) s.t. \(x \in u_x \subseteq f^{-1}(V)\). Hence, 
    \[
        f^{-1}(V) = \bigcup_{x \in f^{-1}(V)} u_x .  
    \] Since \(u_x\) is open, so \(f^{-1}(V)\) is open.     
\end{proof}
\begin{proof}[(c) \(\iff \) (d)]
    If \(F\) is closed in \(Y\), then \(Y \setminus F\) is open in \(Y\), and thus \(F^{-1}(Y \setminus F) \) is open in \(X\), and since \(f^{-1} (Y \setminus F) = X \setminus f^{-1}(F) \) is open, so \(f^{-1}(F) \) is closed.        
\end{proof}

\begin{remark}
    If \(f\) is continuous, then the image of an open set on \(f\) may not be open, and the image of a closed set on \(f\) may not be closed.     
\end{remark}

\begin{eg}
    Consider \(f(x) = x^2\), then \(f(-1, 1) = [0, 1)\) is not open.   
\end{eg}

\begin{eg}
    Consider \(f(x) = \arctan (x)\), then \(f([0, \infty )) = [0, \frac{\pi}{2}]\).  
\end{eg}

Now if \(f: X \to Y\) and \(g: X \to Z\), then consider
\[
    (f, g) : X \to Y \times Z \text{ with } x \mapsto (f(x), g(x)),
\] then \(Y \times Z\) has a natural metric. Note that if \((y_1, z_1) \) and \((y_2, z_2)\) are in \(Y \times Z\), then 
\[
    d_{Y \times Z}\left( (y_1, z_1), (y_2, z_2) \right) = d_Y(y_1, y_2) + d_Z(z_1, z_2). 
\]    

\begin{lemma}
    Consider \(f: X \to Y\) and \(g: X \to Z\) and \((f, g): X \to Y \times Z\), then \(f\) and \(g\) are both continuous if and only if \((f, g)\) is continuous.       
\end{lemma}
\begin{proof}
    Given \(x \in X\), we know \(\lim_{n \to \infty} x^{(n)} = x \in (X, d_X) \) implies 
    \[
        \lim_{n \to \infty} f \left( x^{(n)} \right) = f(x) \text{ and } \lim_{n \to \infty} g \left( x^{(n)} \right) = x.    
    \] 
    \begin{claim}
        \[
            \lim_{n \to \infty} (f, g)(x^n) = (f, g)(x). 
        \]
    \end{claim}
    \begin{explanation}
        Check.
    \end{explanation}
\end{proof}