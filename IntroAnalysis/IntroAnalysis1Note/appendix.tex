\chapter{Some Extra proof}
\begin{theorem} \label{thm: if subseq of cauchy converge, then cauchy converge to same point}
    For a Cauchy sequence \(\left\{ x^{(n)} \right\}_{n=1}^{\infty}  \), if there exists a subsequence \(\left\{ x^{(n_j)} \right\}_{j=1}^{\infty}  \) converges to \(x\), then \(\left\{ x^{(n)} \right\}_{n=1}^{\infty}  \) also converges to \(x\).   
\end{theorem}
\begin{proof}
    For all \(\varepsilon > 0\), we know there exists \(N > 0\) s.t. \(j \ge N\) implies 
    \[
        d\left( x^{(n_j)}, x \right) < \frac{\varepsilon}{2}. 
    \]  
    Also, there exists \(N^{\prime} > 0\) s.t. \(i, j \ge N^{\prime} \) implies
    \[
        d\left( x^{(i)}, x^{(j)} \right) < \frac{\varepsilon}{2}. 
    \]  
    Hence, if we pick some \(d \ge N\) and \(n_d \ge N^{\prime} \), then we know for all \(n \ge N^{\prime} \), we have 
    \[
        d \left( x^{(n)}, x \right) \le d \left( x^{(n)}, x^{(n_d)} \right) + d \left( x^{(n_d)}, x \right) < \frac{\varepsilon}{2} + \frac{\varepsilon}{2} = \varepsilon,
    \] which means \(\left\{ x^{(n)} \right\}_{n=1}^{\infty}  \) converges to \(x\).  
\end{proof}


\begin{definition}
    A sequence of intervals \(I_n\) (\(n \in \mathbb{N} \)) is nested if \(I_n \neq \varnothing \) and \(I_{n+1} \subseteq I_n\) for all \(n \in \mathbb{N}\). (\(I_1 \supseteq I_2 \supseteq \dots\)).     
\end{definition}

Now we want to know \(\bigcap_{n \in \mathbb{N} }^{\infty} I_n \neq \varnothing \)?

Here is some counterexamples. Consider \(I_n = (0, \frac{1}{n})\), \(n \in \mathbb{N} \). We can show that \(\bigcap_{n=1}^{\infty} I_n = \varnothing  \) by Archimedean Property. Besides, if \(I_n = [n, \infty )\), \(n \in \mathbb{N} \), this is trivial that \(\bigcap_{n=1}^{\infty} I_n = \varnothing  \). 

\begin{theorem}[Theorem of nested intervals]\label{thm: nested interval}
    If \(I_n\) (\(n \in \mathbb{N} \)) is a sequence of bounded closed nested intervals, then \(\bigcap_{n=1}^{\infty} I_n \neq \varnothing  \).  
\end{theorem}

\begin{proof}
    Write \(I_n = [a_n,b_n]\) for all \(n \in \mathbb{N} \). First, we know \(I_n\) is nested iff \(a_n \le b_n\) and \(a_n\) is nondecreasing and \(b_n\) is nonincreasing. Hence, \(\forall n,m \in \mathbb{N} \), we have \(a_n \le a_{\max \left\{ n,m \right\} } \le b_{\max \left\{ n,m \right\} } \le b_m\). In other words, for every \(m \in \mathbb{N} \), \(b_m\) is a upper bound of \(\left\{ a_n \right\} \). Hence, we know \(c = \lim_{n \to \infty} a_n  = \sup \left\{ a_n \right\} \).exists. Then, \(c \le b_m\) for all \(m \in \mathbb{N} \). Also, \(c \ge a_n\) for all \(n \in \mathbb{N} \). Hence, \(a_n \le c \le b_n\) for all \(n \in \mathbb{N} \), and thus we know \(c \in I_n\) for all \(n \in \mathbb{N} \). Thus, \(c \in \bigcap_{n=1}^{\infty} I_n \).                     
\end{proof}

\begin{theorem}[Bolzano Weierstrass Theorem] \label{thm: Bolzano Weierstrass thm}
    Suppose we have a bounded infinite sequence \(a_n \in \mathbb{R} ^m\), then \(\exists \) a subsequence \(a_{n(m)}\) such that \(a_{n(m)}\) is convergent.   
\end{theorem}
\begin{proof}
    We just talk about the case \(m=2\), and the higher case is similar. Choose \(M>0\) such that \(a_n \in [-M,M] \times [-M,M]\) for all \(n \in \mathbb{N} \). Suppose \([-M,M] \times [-M,M]\) is called \(Q\). Divide \(Q\) into \(4\) squares with equal size, and choose one, say \(Q_1\) such that \(\left\vert \left\{ n \mid a_n \in Q_1 \right\} = \infty   \right\vert \). Select \(n_1 \in \mathbb{N} \) such that \(a_{n_1} \in Q_1\). Repeat this step, that is, divide \(Q_1\) into \(4\) subparts, then says the one subpart with infinite many \(a_n\) in it is \(Q_2\) (\(Q_2\) must exists). Select \(n_2 \in \mathbb{N} \) such that \(a_{n_2} \in Q_2\) and \(n_2 > n_1\). Keep repeating this step, then by \autoref{thm: nested interval} we know
    \[
        \bigcap_{n=1}^{\infty} Q_n \neq \varnothing.
    \] 
    \begin{note}
        Just think of the nested intervals are in \(x\) and \(y\) directions.  
    \end{note}
    Actually, \(\bigcap_{n=1}^{\infty} Q_n = \left\{ a \right\} \) for some \(a \in \mathbb{R} ^2\), otherwise if there are two points in the intersection, then at some moment we will divide them into different subpart, which is a contradiction. It can been seen that \(\lim_{k \to \infty} a_{n(k)} = a\).   
\end{proof}

\begin{theorem} \label{thm: two seq d converge to 0 implies converge to same point}
    If \((X, d)\) is a metric space and \(\left\{ a_n \right\}_{n=1}^{\infty} , \left\{ b_n \right\}_{n=1}^{\infty} \subseteq X  \). Now if \(\lim_{n \to \infty} d(a_n, b_n) = 0 \) and \(\lim_{n \to \infty} a_n = p \) for some \(p \in X\), then \(\lim_{n \to \infty} b_n = p \).      
\end{theorem}
\begin{proof}
    Since we know for all \(\varepsilon > 0\), \(\exists N > 0\) s.t. \(n \ge N\) implies \(d(a_n, p) < \varepsilon \), and there exists \(N_1, N_2 > 0\) s.t. \(n \ge N_1\) implies \(d(b_n, a_n) < \frac{\varepsilon}{2}\) and \(n \ge N_2\) implies \(d(a_n, p) < \frac{\varepsilon}{2}\), so now for \(n \ge \max \left\{ N_1, N_2 \right\} \), we know 
    \[
        d(b_n, p) = d(b_n, a_n) + d(a_n, p) < \frac{\varepsilon}{2} + \frac{\varepsilon}{2} =\varepsilon .
    \]          
\end{proof}

\begin{theorem} \label{thm: if sup exists then a sequence in S converges to supS}
    If \(S \subseteq \mathbb{R} \), and \(\sup S\) exists for some set \(S\), then there exists a sequence of \(S\) converging to \(\sup S\).    
\end{theorem}
\begin{proof}
    By the definition of sup, we know for all \(\varepsilon > 0\), \(\exists s \in S\) s.t. \(\sup S \ge s > \sup S - \varepsilon \), so pick \(\varepsilon = \frac{1}{n}\) for all \(n \in \mathbb{N} \), we can form \(\left( s^{(n)} \right)_{n=1}^{\infty}  \) convergeing to \(\sup S\).        
\end{proof}




%────────────────────────────────────────────────────────────────────────────────────────────────────────────────────────────────────────────────────

\chapter{TA Class}