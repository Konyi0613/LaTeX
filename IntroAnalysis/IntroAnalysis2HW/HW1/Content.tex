\begin{problem}
    Define $f : \mathbf{R}^2 \to \mathbf{R}$ by
$$
f(x,y)=
\begin{cases}
\displaystyle
\frac{x^2 y}{x^2+y^2}, & (x,y)\neq(0,0),\\[1em]
0, & (x,y)=(0,0).
\end{cases}
$$

\begin{itemize}
\item[(a)] Show that for every fixed direction $v\in\mathbf{R}^2$, 
the limit
$$
\lim_{t\to 0} \frac{f(tv)-f(0)}{t}
$$
exists.

\item[(b)] Show that $f$ is \emph{not} differentiable at $(0,0)$
in the sense of Definition 6.2.2.

\item[(c)] Explain precisely which part of the definition of differentiability fails.
\end{itemize}
\end{problem}

\begin{problem}
    Let $f:\mathbf{R}^n\to\mathbf{R}^m$ and suppose that for some linear map
$L:\mathbf{R}^n\to\mathbf{R}^m$ one has
$$
f(x_0+h)=f(x_0)+L(h)+R(h),
$$
where the remainder satisfies
$$
\|R(h)\| \le C \|h\|^{1+\alpha}
$$
for some constants $C>0$ and $\alpha>0$.

\begin{itemize}
\item[(a)] Prove that $f$ is differentiable at $x_0$ with derivative $L$.

\item[(b)] Show that if $\alpha=0$, the conclusion may fail
by constructing a counterexample.
\end{itemize}
\end{problem}

\begin{proof}[(a)]
    It is suffices to show that 
    \[
        \lim_{h \to 0} \frac{\|f(x_0 + h) - f(x_0) - L(h)\|}{\| h\|} = 0.
    \]
    Notice that 
    \[
        f(x_0 + h) - f(x_0) - L(h) = R(h)
    \]
    and 
    \[
        \|R(h)\| \le C \|h\|^{1+\alpha},
    \]
    so
    \[
        0 \le \lim_{h \to 0} \frac{\|f(x_0 + h) - f(x_0) - L(h)\|}{\| h\|} \le \lim_{h \to 0} \frac{C\| h\|^{1+\alpha}}{\| h\|} = \lim_{h \to 0} C\| h\|^{\alpha} = 0,
    \]
    and hence
    \[
        \lim_{h \to 0} \frac{\|f(x_0 + h) - f(x_0) - L(h)\|}{\| h\|} = 0
    \]
    by squeeze theorem.
\end{proof}

\begin{proof}[(b)]
    Suppose that $f:\mathbb{R} \to \mathbb{R}$ is $f(x) = \lvert x \rvert$, $x_0 = 0$, $L(h) = 0$, $R(h) = \lvert x \rvert$, $C = 1$ and $\alpha = 0$. \\
    Notice that 
    \[
        f(x_0 + h) = \lvert h \rvert = 0 + 0 + \lvert h \rvert = f(x_0)+L(h)+R(h)
    \]
    and
    \[
        \|R(h)\| = h \le 1 \cdot \|h\|^{1+0} = h,
    \]
    so it satisfies the condition.
    However, $f$ is not differentiable at 0 because
    \[
        \lim_{x \to 0^+} \frac{\lvert h \rvert}{h} \ne \lim_{x \to 0^-} \frac{\lvert h \rvert}{h}.
    \]
\end{proof}

\begin{problem}
Let $D=\{(x,y)\in\mathbb{R}^2:\ x\ge 0\ \text{and}\ 0\le y\le x^3\}$ and $x_0=(0,0)$.
Define $f:D\to\mathbb{R}$ by $f(x,y)=2x+\sqrt{y}$.

We say that $f$ is \emph{differentiable at $x_0$ on $D$ with derivative $L$} if
$L:\mathbb{R}^2\to\mathbb{R}$ is linear and
\[
\lim_{\substack{h\to 0\\ x_0+h\in D}}
\frac{|f(x_0+h)-f(x_0)-L(h)|}{\|h\|}=0.
\]

\begin{itemize}
\item[(a)] Explain briefly why $f$ could \emph{never} be differentiable at $(0,0)$ if the domain were the entire first quadrant $\mathbb{R}^2_{\ge 0}=\{(x,y):x\ge 0,\ y\ge 0\}$.

\item[(b)] Prove that $f$ \emph{is} differentiable at $(0,0)$ on the restricted domain $D$.

\item[(c)] Describe the set of all valid derivatives $L(x,y)$ for $f$ at $(0,0)$ on $D$.
\end{itemize}

\end{problem}
\begin{proof}[(a)]
    Note that 
    \[
        \lim_{\substack{h \to 0 \\ x_0 + h \in \mathbb{R} ^2_{\ge 0}}} \frac{\left\vert f((0, 0) + h) - f(0, 0) - L(h) \right\vert }{\lVert h \rVert } = \lim_{\substack{h \to 0 \\ x_0 + h \in \mathbb{R} ^2_{\ge 0}}} \frac{\vert f(h) - L(h) \vert }{\lVert h \rVert }.
    \]
    Now let \(h = (a, b)\) for \(a, b \in \mathbb{R} \), then 
    \[
        \lim_{\substack{h \to 0 \\ x_0 + h \in \mathbb{R} ^2_{\ge 0}}} \frac{\vert f(h) - L(h) \vert }{\lVert h \rVert } = \lim_{\substack{(a, b) \to (0, 0) \\ (a, b) \in \mathbb{R} ^2_{\ge 0}}} \frac{\left\vert 2a + \sqrt{b} - a L(1, 0) - bL(0, 1)  \right\vert }{\sqrt{a^2 + b^2} },
    \]  
    but if we approach \((a, b)\) to \((0, 0)\)  along \(a = 0\) and \(b \ge 0\), then 
    \[
        \frac{\left\vert 2a + \sqrt{b} - a L(1, 0) - bL(0, 1)  \right\vert }{\sqrt{a^2 + b^2} } = \frac{\sqrt{b} - b L(0, 1) }{b} = \frac{1}{\sqrt{b} } - L(0, 1),
    \] 
    but if \((a, b) \to (0, 0)\), then this limit diverges, which shows 
    \[
        \lim_{\substack{h \to 0 \\ x_0 + h \in \mathbb{R} ^2_{\ge 0}}} \frac{\left\vert f((0, 0) + h) - f(0, 0) - L(h) \right\vert }{\lVert h \rVert } \text{ does not exist,} 
    \]
    i.e. \(f\) could never be differentiable at \((0, 0)\) no matter what \(L\) we pick.   
\end{proof}

\begin{proof}[(b)]
    If we pick \(L: \mathbb{R} ^2 \to \mathbb{R} \) s.t. \(L(1, 0) = 2\) and \(L(0, 1) = 0\) and suppose \(h = (a, b)\) , then 
    \[
        \lim_{\substack{h \to 0 \\ x_0 + h \in D}} \frac{\left\vert f((0, 0) + h) - f(0, 0) - L(h) \right\vert }{\lVert h \rVert } = \lim_{\substack{(a, b) \to (0, 0) \\ (a, b) \in D}} \frac{\sqrt{b} }{\sqrt{a^2 + b^2} },
    \]  
    and note that for \((a, b) \in D\), we know \(0 \le b \le a^3\), so \(b^{\frac{1}{3}} \le a\), and thus 
    \[
        0 \le \frac{\sqrt{b} }{\sqrt{a^2 + b^2} } \le \sqrt{\frac{b}{b^{\frac{2}{3}} + b^2}} = \sqrt{\frac{b^{\frac{1}{3}}}{1 + b^{\frac{4}{3}}}},  
    \]   
    where we know 
    \[
        \sqrt{\frac{b^{\frac{1}{3}}}{1 + b^{\frac{4}{3}}}} \to 0 \text{ as } (a, b) \to (0, 0), 
    \]
    so by squeeze theorem, we know 
    \[
        \lim_{\substack{(a, b) \to (0, 0) \\ (a, b) \in D}} \frac{\sqrt{b} }{\sqrt{a^2 + b^2} } = 0,
    \]
    which shows \(f\) is differentiable at \((0, 0)\) on the restricted domain \(D\).   
\end{proof}

\begin{proof}[(c)]
    If \(L(x, y)\) is a valid derivative for \(f\) at \((0, 0)\) on \(D\), then 
    \[
        \lim_{\substack{(a, b) \to (0, 0) \\ (a, b) \in D}} \frac{\left\vert 2a + \sqrt{b} - a L(1, 0) - bL(0, 1)  \right\vert }{\sqrt{a^2 + b^2} } = 0.
    \]    
    Without lose of generality, suppose 
    \[
        2a + \sqrt{b} - aL(1, 0) - bL(0, 1) > 0, 
    \]
    while the \(< 0\) case can be solved similarly. Now note that 
    \[
        \frac{a(2 - L(1, 0) + \sqrt{b} - bL(0, 1) )}{\sqrt{a^2 + b^2} } = \frac{a(2 - L(1, 0))}{\sqrt{a^2 + b^2}} + \frac{\sqrt{b} }{\sqrt{a^2 + b^2} } - \frac{b L(0, 1)}{\sqrt{a^2 + b^2} }.
    \] 
    In (b), we have shown that 
    \[
        \lim_{\substack{(a, b) \to (0, 0) \\ (a, b) \in D}} \frac{\sqrt{b} }{\sqrt{a^2 + b^2}} = 0.
    \]
    Also, we know 
    \[
        \lim_{\substack{(a, b) \to (0, 0) \\ (a, b) \in D}} \frac{b L(0, 1)}{\sqrt{a^2 + b^2} } = \lim_{\substack{(a, b) \to (0, 0) \\ (a, b) \in D}} \sqrt{b} L(0, 1) \frac{\sqrt{b} }{\sqrt{a^2 + b^2} } = 0. 
    \]
    However,
    \[
        \sqrt{\frac{1}{1 + a^4}} = \frac{a}{a^2 + a^6} \le \frac{a}{\sqrt{a^2 + b^2} } \le \frac{a}{\sqrt{a^2} } = 1
    \]  
    when \((a, b) \in D\) and we know 
    \[
        \lim_{\substack{(a, b) \to (0, 0) \\ (a, b) \in D}} \sqrt{\frac{1}{1 + a^4}} = 1,  
    \] 
    so by squeeze theorem we have
    \[
        \lim_{\substack{(a, b) \to (0, 0) \\ (a, b) \in D}} \frac{a}{\sqrt{a^2 + b^2} } = 1.
    \]
    Thus, if \(L(1, 0) \neq 2\), then 
    \[
        \lim_{\substack{(a, b) \to (0, 0) \\ (a, b) \in D}} \frac{a(2 - L(1, 0) + \sqrt{b} - bL(0, 1) )}{\sqrt{a^2 + b^2} } \neq 0,
    \]
    and thus \(L(x, y)\) is a valid derivative for \(f\) at \((0, 0)\) on \(D\) if and only if \(L(1, 0) = 2\), i.e. \(L(x, y) = 2x + by\) for any constant \(b\). 
    
\end{proof}


\begin{problem}
    Let $E \subset \mathbf{R}^n$, let $f : E \to \mathbf{R}^m$, let $x_0$ be an interior point of $E$, and let $1 \le j \le n$.

Show that $\frac{\partial f}{\partial x_j}(x_0)$ exists if and only if
$D_{e_j}f(x_0)$ and $D_{-e_j}f(x_0)$ exist and are negatives of each other.
In this case,
$$
\frac{\partial f}{\partial x_j}(x_0)
=
D_{e_j}f(x_0).
$$
\end{problem}

\begin{problem}
    \textbf{Exercise 6.3.3.}
Let $f : \mathbf{R}^2 \to \mathbf{R}$ be defined by
$$
f(x,y)
=
\frac{x^3}{x^2 + y^2}
\quad \text{for } (x,y) \ne (0,0),
\qquad
f(0,0) = 0.
$$

Show that $f$ is not differentiable at $(0,0)$, even though it is
directionally differentiable in every direction at $(0,0)$.
Explain why this does not contradict Theorem 6.3.8.

\end{problem}

\begin{proof}
    For fixed $v = (a,b) \in \mathbb{R}^2 \setminus \{0\}$,
    \[
        \lim_{t \to 0} \frac{f(tv) - f((0,0))}{t} = \lim_{t \to 0} \frac{\frac{a^3t^3}{a^2t^2+b^2t^2}}{t} = \frac{a^3}{a^2+b^2} = D_vf((0,0)),
    \]
    the limit exists, and hence $f$ is directionally differentiable in every direction at $(0,0)$. However, 
    \[
        D_{(1,0)} f((0,0)) = 1, D_{(0,1)} f((0,0)) = 0, \text{and } D_{(1,1)} f((0,0)) = \frac{1}{2},
    \]
    so 
    \[
        D_{(1,0)} f((0,0)) + D_{(0,1)} f((0,0)) \ne D_{(1,1)} f((0,0)),
    \]
    so there doesn't exist linear map $L$ such that $D_vf(x_0) = L(v)$ at every direction $v$, and hence $f$ is not differentiable.

    The theorem 6.3.8 requires that all partial derivative $\frac{\partial f}{\partial x_i}$ are continuous at $x_0$, however we will show that this hypothesis is not true in this problem, and hence does not contradict the theorem.

    Notice that 
    \[
        \frac{\partial f}{\partial x} = \frac{3x^2(x^2+y^2) - x^3(2x)}{(x^2+y^2)^2} = \frac{x^4+3x^2y^2}{(x^2+y^2)^2}.
    \]
    First, we approach the limit along $y=0$,
    \[
        \lim_{t \to 0} \frac{\partial f}{\partial x}(t,0) = 1.
    \]
    Then, we approach the limit along $x=0$,
    \[
        \lim_{t \to 0} \frac{\partial f}{\partial x}(0,t) = 0.
    \]
    So the limit depends on the path, and hence $\frac{\partial f}{\partial x}$ is NOT continuous at $(0,0)$. 

\end{proof}

\begin{problem}
    Let $E\subset \mathbb{R}^n$, let $f:E\to \mathbb{R}^m$, and let $x_0$ be an interior point of $E$.

Assume that there exists a linear map $A:\mathbb{R}^n\to \mathbb{R}^m$
such that for every unit vector $v\in S^{n-1}$,
\[
\lim_{t\to 0^+}
\frac{f(x_0+t v)-f(x_0)}{t}
=
A(v).
\]

Suppose moreover that the above convergence is \emph{uniform in $v$ on the unit sphere $S^{n-1}$}, meaning that

\[
\lim_{t\to 0^+}
\sup_{v\in S^{n-1}}
\left\|
\frac{f(x_0+t v)-f(x_0)}{t}
-
A(v)
\right\|
=
0.
\]

Equivalently, for every $\varepsilon>0$ there exists $\delta>0$
such that for all $0<t<\delta$ and all $v\in S^{n-1}$,
\[
\left\|
\frac{f(x_0+t v)-f(x_0)}{t}
-
A(v)
\right\|
<
\varepsilon.
\]

Prove that $f$ is differentiable at $x_0$ and that $f'(x_0)=A$.

\medskip
\noindent\textbf{Hint.}
For $h\neq 0$, write
\[
h=\|h\|\,v
\quad \text{with } v=\frac{h}{\|h\|}\in S^{n-1}.
\]
Use linearity of $A$ to rewrite
\[
A(h)=\|h\|A(v),
\]
and compare
\[
\frac{f(x_0+h)-f(x_0)-A(h)}{\|h\|}
\]
with
\[
\frac{f(x_0+\|h\|v)-f(x_0)}{\|h\|}-A(v).
\]
Then apply the assumed uniform convergence.
\end{problem}
\begin{proof}
    We want to show 
    \[
        \lim_{\substack{h \to 0 \\ x_0 + h \in E}} \frac{\lVert f(x_0 + h) - f(x_0) - A(h) \rVert }{\lVert h \rVert } = 0, 
    \]
    or equivalently, for all \(\varepsilon > 0\), there exists \(\delta > 0\) s.t. \(0 < \lVert h \rVert < \delta  \) and \(x_0 + h \in E\) implies 
    \[
        \frac{\lVert f(x_0 + h) - f(x_0) - A(h) \rVert }{\lVert h \rVert } < \varepsilon.
    \]  
    Now fix any \(\varepsilon > 0\), we know there exists \(\delta ^{\prime} > 0\) s.t. for any \(0 < \lVert h \rVert  < \delta ^{\prime}\) and \(v \in S^{n-1}\) we have 
    \[
        \left\lVert \frac{f(x_0 + \lVert h \rVert v ) - f(x_0)}{\lVert h \rVert } - A(v) \right\rVert < \varepsilon 
    \]      
    due to the uniform convergence. Note that \(x_0 \in \mathrm{Int}(E) \), so there exists \(r > 0\) s.t. \(B(x_0, r) \subseteq E\). Now we know for any \(h = \lVert h \rVert v \) and \(v \in S^{n-1}\) s.t. \(0 < \lVert h \rVert < \min \left\{\delta ^{\prime}, r \right\}\), 
    \begin{align*}
        \varepsilon > \left\lVert \frac{f(x_0 + \lVert h \rVert v ) - f(x_0)}{\lVert h \rVert } - A(v) \right\rVert &= \left\lVert \frac{f(x_0 + h) - f(x_0)}{\lVert h \rVert } - A(v) \right\rVert \\
        &= \left\lVert \frac{f(x_0 + h) - f(x_0) - \lVert h \rVert A(v)}{\lVert h \rVert } \right\rVert \\
        &= \left\lVert \frac{f(x_0 + h) - f(x_0) - A(\lVert h \rVert v)}{\lVert h \rVert } \right\rVert \\
        &= \left\lVert \frac{f(x_0 + h) - f(x_0) - A(h)}{\lVert h \rVert } \right\rVert,
    \end{align*}  
    so we can pick \(\delta = \min\{\delta ^{\prime} , r\}\), and we're done.
\end{proof}