\documentclass[a4paper,12pt]{article}
\usepackage{bbding,combelow,textcomp,amsfonts,amsthm,amsmath,amssymb,graphicx,color,hyperref,etoolbox}
\usepackage[utf8x]{inputenc}
\usepackage[lined,boxed,commentsnumbered]{algorithm2e}
\usepackage{pdfpages}
\newtoggle{story}
\usepackage{xeCJK}
\setCJKmainfont{PingFang TC}


%%%%%%%%%%%%%%%%%%%%%%%%%%%%%%%%%%%%%%%%%%
%% Customisation options --- update as necessary
%%%%%%%%%%%%%%%%%%%%%%%%%%%%%%%%%%%%%%%%%%

%% Course details: title, semester, instructors

\newcommand{\courseTitle}{Combinatorics I (Math 7701)}
\newcommand{\semester}{Fall 2025}
\newcommand{\instructorA}{Shagnik Das}
\newcommand{\instructorB}{TA: Lu Yun-Chi}

%% Sheet number

\newcommand{\sheetNumber}{1}


%% Submission details

\newcommand{\dueDate}{15:30, Sep 23rd, to be submitted on COOL.}
\newcommand{\lateWarning}{}


%% Display irrelevant footnotes?  \toggletrue for yes, \togglefalse for no

\toggletrue{story}

%%%%%%%%%%%%%%%%%%%%%%%%%%%%%%%%%%%%%%%%%%
%% End of customisation options
%%%%%%%%%%%%%%%%%%%%%%%%%%%%%%%%%%%%%%%%%%

\pdfpagewidth 8.5in
\pdfpageheight 11in
\topmargin -1in
\headheight 0in
\headsep 0in
\textheight 8.5in
\textwidth 6.5in
\oddsidemargin 0in
\evensidemargin 0in
\headheight 77pt
\headsep 0in
\footskip .75in

\makeatletter
\newcommand{\buildtitle}[4]{
\begin{flushleft}
{\large
#1
\hfill{}
#2
\par
#3
}
\end{flushleft}
\vskip 4pt
\begin{center}
{\large\bfseries#4\par}
\end{center}
\bigskip
}
\makeatother

\renewcommand{\thesection}{\normalsize\arabic{section}}

\renewcommand{\deg}{\textrm{deg}}
\newcommand{\eps}{\varepsilon}
\newcommand{\ceil}[1]{\left \lceil #1 \right \rceil}

\newcommand{\hint}{\begin{flushright} [Hint at \url{\hintURL}.] \end{flushright}}

\newcommand{\card}[1]{\left| #1 \right|}
\newcommand{\mb}[1]{\mathbb{#1}}
\newcommand{\mc}[1]{\mathcal{#1}}
\newcommand{\qbinom}{\genfrac{[}{]}{0pt}{}}


\newcommand{\story}[1]{\iftoggle{story}{\footnote{#1}}{}}
\newcommand{\storymark}[1][42]{\iftoggle{story}{\footnotemark[#1]}{}}
\newcommand{\storytext}[2][42]{\iftoggle{story}{\footnotetext[#1]{#2}}{}}

\newcommand{\bonus}[2]{\paragraph{Bonus (#1 pt\ifstrequal{#1}{1}{}{s})} #2}


\begin{document}


\buildtitle{\courseTitle{}}{\semester}{\instructorA{} \hfill \instructorB{}}{Exercise Sheet \sheetNumber{}}

\vspace{-0.2in}

\begin{center}

{\bf Due date: \dueDate{}}

\end{center}

\noindent Working with your partner, you should try to solve all of the exercises below. You should then submit solutions to four of the problems, with each of you writing two, clearly indicating the author of each solution. Note that each problem is worth $10$ points, and starred exercises represent problems that may be a little tougher, should you wish to challenge yourself. In case you have difficulties submitting on COOL, please send your solutions by e-mail.

\paragraph{Exercise 1}  In a game of Scrabble, there is a bag containing fourteen letter tiles, namely {`ABCDEFGHIJKLMN'}. A player reaches in, pulls out seven of the tiles, and then arranges them in some order on their rack.

\begin{itemize}
	\item[(a)] How many different strings of seven letters can the player have on their rack?
	\item[(b)] What if, instead, the tiles in the bag were `REARRANGEMENTS'?
\end{itemize}
\paragraph{Solution:} (by 黃子恆) See last few pages.

\paragraph{Exercise 2}  Let $q$ be a prime power, and let $\mathbb{F}_q$ be the finite field of order $q$. Let $V = \mathbb{F}_q^n$ be the $n$-dimensional vector space over $\mathbb{F}_q$. We denote by $\qbinom{V}{k}_q$ the set of $k$-dimensional vector subspaces of $V$, and by $\qbinom{n}{k}_q = \card{\qbinom{V}{k}_q}$ the number of such subspaces.

\begin{itemize}
	\item[(a)] By double-counting, or otherwise, give a formula for $\qbinom{n}{k}_q$.
	\item[(b)] Let $\vec{v} \in V$ be a non-zero vector. How many $k$-dimensional subspaces of $V$ contain $\vec{v}$?
\end{itemize}

\paragraph{Exercise 3}  Let $X$ be a set of $n$ elements, and call a sequence $(x_1, x_2, \hdots, x_\ell) \in X^{\ell}$ \emph{non-repetitive} if we have $x_{i+1} \neq x_i$ for all $1 \le i \le \ell-1$.

\begin{itemize}
	\item[(a)] How many non-repetitive sequences of length $\ell$ are there?
\end{itemize}

We call the sequence \emph{cyclically non-repetitive} if we also have $x_1 \neq x_{\ell}$. Let $N_{n,\ell}$ denote the number of cyclically non-repetitive sequences of length $\ell$.

\begin{itemize}
	\item[(b)] For $n \ge 1$ and $\ell \ge 3$, prove that $N_{n,\ell-1} + N_{n, \ell} = n(n-1)^{\ell-1}$.
	\item[(c)] Prove that $N_{n, \ell} = (n-1)^{\ell} + (-1)^{\ell}(n-1)$ for all $n \ge 1$ and $\ell \ge 2$.
\end{itemize}
\paragraph{Solution:} (by 張沂魁)
\begin{itemize}
	\item[(a)] For a non-repetitive sequence \((x_1, x_2, \dots , x_{\ell} ) \in X^{\ell} \), we know we have \(n\) choices for \(x_1\), \(n - 1\) choices for \(x_2\) since \(x_2 \neq x_1\), and \(n - 1\) choices for \(x_3\) since \(x_3 \neq x_2\), and so on, so by product rule, we know there are
	\[
		n\overbrace{(n-1)(n-1)\dots (n-1)}^{\ell -1 \text{ times}} = n (n-1)^{\ell -1}
	\] non-repetitive sequences of length \(\ell \). 
	\item[(b)] We double count on the number of non-repetitive sequences of length \(\ell \). From (a), we know there are \(n(n-1)^{\ell - 1}\) non-repetitive sequences of length \(n\). Now we do case analysis for non-repetitive sequences of length \(n\).
	\begin{itemize}
		\item[]Case 1: \(x_ 1 \neq x_{\ell } \) \\
		If so, then it is a cyclically non-repetitive sequence, so there are \(N_{n, \ell }\) such sequences.  
		\item[]Case 2: \(x_1 = x_{\ell } \) \\
		If so, then there is a bijection between \(\left\{ (x_1, x_2, \dots , x_{\ell } ) \right\} \) and \(\left\{ (x_1, x_2, \dots , x_{\ell -1}) \right\} \). Note that every \((x_1, x_2, \dots , x_{\ell - 1})\) is a cyclically non-repitive sequence of length \(\ell - 1\) since \(x_1 = x_{\ell}\) and \(x_{\ell } \neq x_{\ell  - 1} \), which means 
		\begin{align*}
			&\left\vert \left\{ (x_1, x_2, \dots , x_{\ell } ): x_1 = x_l \right\}  \right\vert \\
			=&\left\vert \left\{ \text{cyclically non-repetitive sequences from } X^{\ell - 1}\right\}  \right\vert \\
			= &N_{n, \ell  - 1}.
		\end{align*}
		
	\end{itemize}
	Now by sum rules, we know there are \(N_{n, \ell } + N_{n, \ell - 1}\) non-repetitive sequences of length \(n\). Thus, by double counting, we know 
	\[
		n(n-1)^{\ell - 1} = N_{n, \ell} + N_{n, \ell - 1}.
	\]  
	\item[(c)]For \(n = 1\), we can see that there aren't any cyclically non-repetitive sequence of any length since we have only one choice for every entry of the sequence, so the formula is correct. Now we first fix \(n \ge 2\), and do induction on \(\ell \). 
	\begin{itemize}
		\item Base case: \(\ell  = 2\), we can first pick two distinct elements from \(X\), say \(a_1, a_2\), so we have \(\binom{n}{2}\) choices, and each \((a_1, a_2)\) and \((a_2, a_1)\) form two different cyclically non-repetitive sequences, and thus we have
		\[
			N_{n, 2} = \binom{n}{2} \cdot 2 = n(n - 1) = (n - 1)^{2} + (-1)^2(n - 1) 
		\], which shows the formula is correct for \(\ell = 2\). 
		\item Now suppose the hypothesis is true for \(\ell = k - 1\), then by (b) we have 
		\[
			N_{n, k-1} + N_{n, k} = n(n - 1)^{k - 1}.
		\] By replacing \(N_{n, k - 1}\) with the hypothesis representation, we get 
		\[
			(n - 1)^{k - 1} + (-1)^{k - 1}(n - 1) + N_{n, k} = n(n - 1)^{k - 1},
		\] and thus 
		\begin{align*}
			N_{n, k} &= n(n - 1)^{k - 1} - (n - 1)^{k - 1} - (-1)^{k - 1}(n - 1) \\
			&= (n - 1)^k + (-1)^k(n - 1),
		\end{align*}which means the hypothesis is correct for \(\ell = k\). 
	\end{itemize}
\end{itemize}

\paragraph{Exercise 4}  Prove the Binomial Theorem, 
\[ (x + y)^n = \sum_{k=0}^n \binom{n}{k} x^k y^{n-k},\]
by induction.
\paragraph{Solution:} (by 黃子恆) See last few pages.


\paragraph{Exercise 5}  
Prove the following binomial identities.
\begin{itemize}
	\item[(a)] $\sum_{j=0}^n \binom{n}{j}^2 = \binom{2n}{n}$.
	\item[(b)] $\sum_{j=0}^k (-1)^j \binom{n}{j} = (-1)^k \binom{n-1}{k}$. (Note that the case $k = n$ generalises our identity from lectures about odd- and even-sized subsets.)
	\item[(c)] $\sum_{j=0}^n \binom{n}{j} j = n2^{n-1}$.
\end{itemize}
\paragraph{Solution:} (by 張沂魁)
\begin{itemize}
	\item[(a)] Suppose there are \(2n\) people, where \(n\) of them are boys and \(n\) of them are girls. Now we want to pick \(n\) people of them, so we know there are \(\binom{2n}{n}\) choices. Also, we can use sum rule on the number of chosen boys, say it's \(j\). Hence, there are 
	\[
		\sum_{j=0}^n \binom{n}{j} \binom{n}{n - j} = \sum_{j=0}^n \binom{n}{j}^2 
	\]   choices since we first pick \(j\) boys and then pick \(n - j\) girls. Hence, we know 
	\[
		\sum_{j=0}^n \binom{n}{j}^2 = \binom{2n}{n}.
	\] 
	\item[(b)] Suppose \(X\) is a set of size \(n\). Now we do case analysis on the parity of \(k\).
	\begin{itemize}
		\item If \(k = 2m\) for some \(m \in \mathbb{N} \cup \left\{ 0 \right\}  \), then we want to check 
		\[
			\left( \binom{n}{0} + \binom{n}{2} + \dots + \binom{n}{2m} \right) - \left( \binom{n}{1} + \binom{n}{3} + \dots + \binom{n}{2m-1} \right) = \binom{n - 1}{2m}.
		\]
		Note that we have a bijection between the set of subsets of \(X\) of even size and the set of subsets of \(X\) of odd size. That is, 
		\[
			S \mapsto S \Delta \left\{ n \right\} = \begin{cases}
				S - \left\{ n \right\} , &\text{ if } n \in S ;\\
				S \cup \left\{ n \right\} , &\text{ if } n \notin S.
			\end{cases}
		\]  
		This is because if \(S\) is a set of even size, then it will be mapped to a set of odd size, and vice versa. Note that some of the subsets of \(X\) of size \(2m\) will be mapped to the subsets of \(X\) of size \(2m + 1\). Hence, we can do double counting on the subsets of \(X\) of odd size \(k\) s.t. \(k \le 2m - 1\). By the directely computing and sum rule, we know there are 
		\[
			\binom{n}{1} + \binom{n}{3} + \dots + \binom{n}{2m - 1}
		\] such subsets. Also, by the previously mentioned map and sum rule, we know there are 
		\[
			\binom{n}{2} + \binom{n}{4} + \dots + \binom{n}{2m} - \binom{n - 1}{2m}
		\] such subsets since we know every counted subset of even size can be mapped to a subset of odd size \(k\) s.t. \(k \le 2m - 1\) except the subset of size \(2m\) and not containing \(n\) (since it will be map to a set of size \(2m + 1\), which is not counted). Hence, we know 
		\[
			\binom{n}{1} + \binom{n}{3} + \dots + \binom{n}{2m - 1} = \binom{n}{2} + \binom{n}{4} + \dots + \binom{n}{2m} - \binom{n - 1}{2m},
		\] which is as desired.
	\end{itemize}
	\item[(c)] If there are \(n\) boys, and we want to choose \(k\) people of them s.t. \(k \ge 1\)  and then give exactly one of the chosen boy the offer from Google while the left \(k - 1\) boys would be in the waiting list, then we can double count on the number of ways to do such choices. We can count on the number of chosen boys, which is from \(1\) to \(n\), so by sum rule, there are 
	\[
		\sum_{j=1}^n \binom{n}{j}j 
	\] ways to do such choice since if we fix \(j\), the number of chosen boys, then we can pick one of them to get the offer from Google, so there are \(\binom{n}{j}j\) ways to choose in total by product rule, and then we can use sum rule to add up all the case of different \(j\), and note that \(\binom{n}{0}0 = 0\), so
	\[
		\sum_{j=1}^n \binom{n}{j} j = \sum_{j=0}^n \binom{n}{j} j. 
	\]  
	On the other hand, we can first choose who to get the offer, so there are \(n\) choices, and then since this people must be chose, so for the left \(n - 1\) people, we can either choose them or not choose them to be in the waiting list, and thus there are \(2^{n - 1}\) ways to pick the boys to be in the waiting list, so there are \(n2^{n - 1}\) ways to choose who get the offer and who is in the waiting list. Hence, we know 
	\[
		\sum_{j=0}^n \binom{n}{j}j = n 2^{n - 1}. 
	\]   
\end{itemize}

% \section*{Hints}  The following QR codes contain hints for some of the homework exercises.  You should be able to decode them using any QR code scanner, including this one: \url{https://zxing.org/w/decode.jspx}.

% \paragraph{Exercise ?} Also available at \url{https://i.ibb.co/???????/S??E?.png}.

\begin{center}
% \includegraphics[scale=0.5]{hints/S??E?.png}
\end{center}
\includepdf[pages=-]{./concatenate.pdf}

\end{document}