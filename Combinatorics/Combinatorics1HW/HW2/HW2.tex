\documentclass[a4paper,12pt]{article}
\usepackage{bbding,combelow,textcomp,amsfonts,amsthm,amsmath,amssymb,graphicx,color,hyperref,etoolbox}
\usepackage[utf8x]{inputenc}
\usepackage[lined,boxed,commentsnumbered]{algorithm2e}
\usepackage{pdfpages}
\usepackage{xeCJK}
\setCJKmainfont{PingFang TC}

\newtoggle{story}


%%%%%%%%%%%%%%%%%%%%%%%%%%%%%%%%%%%%%%%%%%
%% Customisation options --- update as necessary
%%%%%%%%%%%%%%%%%%%%%%%%%%%%%%%%%%%%%%%%%%

%% Course details: title, semester, instructors

\newcommand{\courseTitle}{Combinatorics I (Math 7701)}
\newcommand{\semester}{Fall 2025}
\newcommand{\instructorA}{Shagnik Das}
\newcommand{\instructorB}{TA: Lu Yun-Chi}

%% Sheet number

\newcommand{\sheetNumber}{2}


%% Submission details

\newcommand{\dueDate}{23:59, Oct 9th, to be submitted on COOL.}
\newcommand{\lateWarning}{}


%% Display irrelevant footnotes?  \toggletrue for yes, \togglefalse for no

\toggletrue{story}

%%%%%%%%%%%%%%%%%%%%%%%%%%%%%%%%%%%%%%%%%%
%% End of customisation options
%%%%%%%%%%%%%%%%%%%%%%%%%%%%%%%%%%%%%%%%%%

\pdfpagewidth 8.5in
\pdfpageheight 11in
\topmargin -1in
\headheight 0in
\headsep 0in
\textheight 8.5in
\textwidth 6.5in
\oddsidemargin 0in
\evensidemargin 0in
\headheight 77pt
\headsep 0in
\footskip .75in

\makeatletter
\newcommand{\buildtitle}[4]{
\begin{flushleft}
{\large
#1
\hfill{}
#2
\par
#3
}
\end{flushleft}
\vskip 4pt
\begin{center}
{\large\bfseries#4\par}
\end{center}
\bigskip
}
\makeatother

\renewcommand{\thesection}{\normalsize\arabic{section}}

\renewcommand{\deg}{\textrm{deg}}
\newcommand{\eps}{\varepsilon}
\newcommand{\ceil}[1]{\left \lceil #1 \right \rceil}

\newcommand{\hint}{\begin{flushright} [Hint at \url{\hintURL}.] \end{flushright}}

\newcommand{\card}[1]{\left| #1 \right|}
\newcommand{\mb}[1]{\mathbb{#1}}
\newcommand{\mc}[1]{\mathcal{#1}}


\newcommand{\story}[1]{\iftoggle{story}{\footnote{#1}}{}}
\newcommand{\storymark}[1][42]{\iftoggle{story}{\footnotemark[#1]}{}}
\newcommand{\storytext}[2][42]{\iftoggle{story}{\footnotetext[#1]{#2}}{}}

\newcommand{\bonus}[2]{\paragraph{Bonus (#1 pt\ifstrequal{#1}{1}{}{s})} #2}


\begin{document}


\buildtitle{\courseTitle{}}{\semester}{\instructorA{} \hfill \instructorB{}}{Exercise Sheet \sheetNumber{}}

\vspace{-0.2in}

\begin{center}

{\bf Due date: \dueDate{}}

\end{center}

\noindent Working with your partner, you should try to solve all of the exercises below. You should then submit solutions to four of the problems, with each of you writing two, clearly indicating the author of each solution. Note that each problem is worth $10$ points, and starred exercises represent problems that may be a little tougher, should you wish to challenge yourself. In case you have difficulties submitting on COOL, please send your solutions by e-mail.

\paragraph{Exercise 1}  A professor is getting a room ready for an exam.  The room has $d$ desks in one long row, and there will be $s$ students taking the exam.  Before the students enter the exam room, the professor wants to place the examination papers on the desks in advance.  How many ways can this be done if:
\begin{itemize}
	\item[(a)] the examination papers are identical?
	\item[(b)] the examination papers already have the names of the students printed on them?
	\item[(c)] the examination papers are identical, but the professor wants to have at least two empty desks between each pair of students?
\end{itemize}
\vspace{-1em}
\paragraph{Solution (黃子恆):} Given at the last few pages of this file.

\paragraph{Exercise 2}  Prove that for every $n \ge 1$, the Stirling numbers of the first kind are unimodal in $k$; that is, there is some $m(n)$ such that
\[ s_{n,0} < s_{n,1} < \hdots < s_{n,m(n)-1} \le s_{n,m(n)} > s_{n,m(n)+ 1} > \hdots > s_{n,n}. \]
Moreover, either $m(n) = m(n-1)$ or $m(n) = m(n-1) + 1$.

\paragraph{Exercise 3}  Show that the average number of cycles in a permutation of length $n$ is exactly $H_n$, where $H_n = \sum_{k=1}^{n} \frac{1}{k}$ is the $n$th harmonic number. \\
%\vspace{-1em}
\textbf{Solution (張沂魁):} Note that we want to calculate 
\[
	\frac{\sum_{k=1}^{n} k \cdot s_{n, k} }{n!}
\] since \(s_{n, k}\) represents the number of permutation of \(n\) with exactly \(k\) cycles and \(n!\) is the total number of permutations of \(n\). Now since we know 
\[
	x(x+1)\dots (x+(n-1)) = x^{\overline{n}} = \sum_{k=0}^n s_{n, k} x^k,
\] so if we differentiate both side, we have 
\[
	\sum_{i=0}^{n-1} \frac{x(x+1)\dots (x+n-1)}{x+i} = \sum_{k=1}^n k s_{n, k} x^{k-1},  
\] and if we plug \(x=1\) into both sides, we have 
\[
	\sum_{i=0}^{n-1} \frac{n!}{1+i} = \sum_{k=1}^n k \cdot s_{n, k},  
\] so 
\[
	\frac{\sum_{k=1}^n k \cdot s_{n,k} }{n!} = \frac{\sum_{i=0}^{n-1} \frac{n!}{1+i} }{n!} = \sum_{i=0}^{n-1} \frac{1}{1+i} = \sum_{i=1}^n \frac{1}{i} = H_n.  
\]
\paragraph{Exercise 4*}  Show that the number of partitions of $n$ into odd parts is equal to the number of partitions of $n$ into distinct parts.
\vspace{-1em}
\paragraph{Solution (張沂魁):} We first give the generating function of \((a_n)\), where \(a_n\) is the number of partition of \(n\) into odd parts. Note that it is 
\[
	A(x) = \prod _{j \ge 1} \frac{1}{1 - x^{2j - 1}} = \prod _{j \ge 1} \left( 1 + x^{2j - 1} + \left( x^{2j - 1} \right)^2 + \dots \right) 
\] since for each \(j\), \(\frac{1}{1 - x^{2j - 1}}\) represents the number of parts of size \(2j - 1\), and note that \(2j - 1\) must be odd, so the coefficient of \(x^n\) in \(A(x)\) is just \(a_n\). More precisely, we can contribute \(1\) to the coefficient of \(x^n\) in \(A(x)\) if and only if 
\[
	\left( x^{2 j_1 - 1} \right)^{i_1} \cdot \left( x^{2j_2 - 1} \right)^{i_2} \cdots = x^n,
\] which corresponds to a partition into odd parts: use \(i_p\) many \(2j_p - 1\)s to get a partition of \(n\) for all \(p\), where \(j_1 < j_2 < \dots \).   

Now we talk about the generating function of \((b_n)\), where \(b_n\) is the number of partitions of \(n\) into distinct parts. Note that it is 
\[
	B(x) = \prod _{k \ge 1} \left( 1 + x^k \right) 
\] since we can contribute \(1\) to the coefficient of \(x^n\) in \(B(x)\) if and only if 
\[
	x^{k_1} x^{k_2} \dots = x^n
\] for some \(k_1 < k_2 < \dots \), which corresponds to a parition of \(n\) into distinct parts. Now we show that in fact \(A(x) = B(x)\). Since for all \(j \in \mathbb{N} \), we have 
\[
	\frac{1}{1 - x^{2j - 1}} = \left( 1 + x^{2j - 1} \right) \left( 1 + \left( x^{2j - 1} \right)^2  \right) \left( 1 + \left( x^{2j - 1} \right)^{2^2}  \right) \dots = \prod _{p \ge 0} \left( 1 + \left( x^{2j - 1} \right)^{2^p}  \right),   
\]  
so in fact 
\[
	A(x) = \prod _{j \ge 1} \prod _{p \ge 0} \left( 1 + \left( x^{2j - 1} \right)^{2^p}  \right) = \prod _{j \ge 1} \prod _{p \ge 0} \left( 1 + x^{(2j - 1)2^p} \right), 
\] but if we define \(f: \mathbb{N} \times (\mathbb{N} \cup \left\{ 0 \right\} ) \to \mathbb{N} \) by \(f(j, p) = (2j - 1)2^p\), we can show that \(f\) is bijective: 
\begin{itemize}
	\item Injective: If \(f(j_1, p_1) = f(j_2, p_2)\), then \((2j_1 - 1)2^{p_1} = (2j_2 - 1)2^{p_2}\), but if \(p_1 \neq p_2\), then one side have more \(2\) as its factor than the other side, which is impossible, so \(p_1 = p_2\). However, if \(p_1 = p_2\), then \(2 j_1 - 1 = 2j_2 - 1\), which means \(j_1 = j_2\), so \(f\) is injective. 
	\item Surjective: For \(n \in \mathbb{N} \), if \(n\) is odd, then \(n = n \cdot 2^0 = f \left( \frac{n+1}{2}, 0 \right) \). If \(2 \mid n\), then \(n = \left( \frac{n}{2} \right) \cdot 2^1 \), and if \(\frac{n}{2}\) is odd, then \(f \left( \frac{\frac{n}{2} + 1}{2}, 1 \right) = n \), and if \(\frac{n}{2}\) is even, we can repeat this step. Since \(n\) has finitely many \(2\) as its factor, so this algorithm stops, and we can conclude that \(f\) is surjective.                    
\end{itemize} 
Now we know \(f\) is bijective, so in fact 
\[
	A(x) = \prod _{j \ge 1} \prod _{p \ge 0} \left( 1 + x^{(2j - 1)2^p} \right) = \prod _{k \ge 1} \left( 1 + x^k \right) = B(x),
\] so comparing the coefficient of \(x^n\) in \(A(x)\) and \(B(x)\), we know they are same, and thus the number of partitions of \(n\) into odd parts is equal to the number of partitions of \(n\) into distinct parts.     

\paragraph{Exercise 5}   A group\footnote{In the non-mathematical sense of the word.} of five pirates, called Alice, Bob, Charles, Diana and Erik, have a treasure of $100$ identical gold coins that they need to divide between themselves.
\begin{itemize}
	\item[(a)] How many ways can they divide the coins?
\end{itemize}
Unfortunately, Alice, Bob, Charles, Diana and Erik do not care about your answer to (a).\footnote{After all, they are pirates, not combinators.}  They will divide the coins according to the traditional rules of piracy.  Alice will first suggest a division of the coins - for instance, she might suggest that they each get twenty coins.

Once she has made a suggestion, the pirates (including her) will vote --- either they accept the division, or they do not.  If a strict majority (strictly more than half) accept the division, then that is how they divide the coins, and the matter is settled.

However, if a majority reject the proposal, or the vote is split evenly, then they kill Alice, and the next pirate (Bob --- they proceed alphabetically) makes a proposal instead.  They repeat the same process until a division is agreed upon by the remaining pirates.

The pirates all have the following priorities, which they use when deciding how to vote:
\begin{itemize}
	\item[1.] \textbf{Staying alive:} above all, the pirates want to survive --- they prefer an outcome where they are alive with $0$ coins to one where they die.
	\item[2.] \textbf{Greed:} provided the pirates can stay alive, they want to get as many coins as possible --- they prefer an outcome where they are alive with $n+1$ coins to one where they are alive with $n$ coins.
	\item[3.] \textbf{Violence:} all other things being equal, the pirates would like to kill as many other pirates as possible -- they prefer an outcome where they are alive with $n$ coins and $k+1$ pirates die to one where they are alive with $n$ coins and $k$ pirates die.
\end{itemize}

\begin{itemize}
	\item[(b)] Given these rules, what division should Alice propose?
\end{itemize}

\paragraph{Exercise 6}  In this exercise, you will determine the general solution to recurrence relations where repeated roots are allowed.

\begin{itemize}
	\item[(a)] Suppose we have the polynomial $\prod_{i=1}^q \left( z - \lambda_i \right)^{k_i}$, let $k = \sum_{i=1}^q k_i$, and let $R(z)$ be a polynomial of degree at most $k-1$. Prove that there are unique polynomials $R_i(z)$, $1 \le i \le q$, such that $\deg(R_i) \le k_i - 1$ and 
\[ \frac{R(z)}{\prod_{i=1}^q ( z - \lambda_i )^{k_i}} = \sum_{i=1}^q \frac{R_i(z)}{ (z - \lambda_i )^{k_i} }. \]
	\item[(b)] Let $R_i(z)$ be a polynomial of degree at most $k_i - 1$. Show that there are unique constants $A_{i,j}$, $1 \le j \le k_i$, such that
\[ \frac{R_i(z)}{(z - \lambda_i)^{k_i}} = \sum_{j=1}^{k_i} \frac{A_{i,j}}{(z - \lambda_i)^j}. \]
	\item[(c)] Suppose the sequence $(b_n)_{n \ge 0}$ has the generating function $b(x) = (1 - \lambda x)^{-j}$.  Show that the sequence is given by $b_n = \binom{j + n - 1}{n} \lambda^n$.
	\item[(d)] Let $(a_n)_{n \ge 0}$ be a sequence determined by a recurrence relation with characteristic polynomial $p(z) = \prod_{i = 1}^q (z - \lambda_i)^{k_i}$; that is, the characteristic polynomial has distinct roots $\lambda_i$, each appearing with multiplicity $k_i$.  Show that the solution must take the form\footnote{The solution is often presented in the more convenient (but equivalent) form $\sum_{i=1}^q \left( \sum_{j=0}^{k_i - 1} \tilde{A}_{i,j} n^{j} \right) \lambda_i^n$.}
\[ a_n = \sum_{i=1}^q \left( \sum_{j = 1}^{k_i} A_{i,j} \binom{n-1+j}{n} \right) \lambda_i^n, \]
where the coefficients $A_{i,j}$ can be determined from the initial conditions.
\end{itemize}
\vspace{-1em}
\paragraph{Solution (黃子恆):} Given at the last few pages of this file.
\newpage

\section*{Hints}  The following QR codes contain hints for some of the homework exercises.  You should be able to decode them using any QR code scanner, including this one: \url{https://zxing.org/w/decode.jspx}.

\paragraph{Exercise 6} Also available at \url{https://i.ibb.co/5hDxjwxD/s2e6.png}.

%\begin{center}
%	\includegraphics[scale=0.5]{hints/S2E6.png}
%\end{center}
\includepdf[pages=-]{./concatenate.pdf}
\end{document}