\documentclass[a4paper,12pt]{article}
\usepackage{bbding,combelow,textcomp,amsfonts,amsthm,amsmath,amssymb,graphicx,color,hyperref,etoolbox}
\usepackage[utf8x]{inputenc}
\usepackage[lined,boxed,commentsnumbered]{algorithm2e}
\usepackage{comment}

\newtoggle{story}
\usepackage{pdfpages}
\usepackage{xeCJK}
\setCJKmainfont{PingFang TC}


%%%%%%%%%%%%%%%%%%%%%%%%%%%%%%%%%%%%%%%%%%
%% Customisation options --- update as necessary
%%%%%%%%%%%%%%%%%%%%%%%%%%%%%%%%%%%%%%%%%%

%% Course details: title, semester, instructors

\newcommand{\courseTitle}{Combinatorics I (Math 7701)}
\newcommand{\semester}{Fall 2025}
\newcommand{\instructorA}{Shagnik Das}
\newcommand{\instructorB}{TA: Lu Yun-Chi}

%% Sheet number

\newcommand{\sheetNumber}{3}


%% Submission details

\newcommand{\dueDate}{15:30, Oct 21st, to be submitted on COOL.}
\newcommand{\lateWarning}{}


%% Display irrelevant footnotes?  \toggletrue for yes, \togglefalse for no

\toggletrue{story}

%%%%%%%%%%%%%%%%%%%%%%%%%%%%%%%%%%%%%%%%%%
%% End of customisation options
%%%%%%%%%%%%%%%%%%%%%%%%%%%%%%%%%%%%%%%%%%

\pdfpagewidth 8.5in
\pdfpageheight 11in
\topmargin -1in
\headheight 0in
\headsep 0in
\textheight 8.5in
\textwidth 6.5in
\oddsidemargin 0in
\evensidemargin 0in
\headheight 77pt
\headsep 0in
\footskip .75in

\makeatletter
\newcommand{\buildtitle}[4]{
\begin{flushleft}
{\large
#1
\hfill{}
#2
\par
#3
}
\end{flushleft}
\vskip 4pt
\begin{center}
{\large\bfseries#4\par}
\end{center}
\bigskip
}
\makeatother

\renewcommand{\thesection}{\normalsize\arabic{section}}

\renewcommand{\deg}{\textrm{deg}}
\newcommand{\eps}{\varepsilon}
\newcommand{\ceil}[1]{\left \lceil #1 \right \rceil}

\newcommand{\hint}{\begin{flushright} [Hint at \url{\hintURL}.] \end{flushright}}

\newcommand{\card}[1]{\left| #1 \right|}
\newcommand{\mb}[1]{\mathbb{#1}}
\newcommand{\mc}[1]{\mathcal{#1}}


\newcommand{\story}[1]{\iftoggle{story}{\footnote{#1}}{}}
\newcommand{\storymark}[1][42]{\iftoggle{story}{\footnotemark[#1]}{}}
\newcommand{\storytext}[2][42]{\iftoggle{story}{\footnotetext[#1]{#2}}{}}

\newcommand{\bonus}[2]{\paragraph{Bonus (#1 pt\ifstrequal{#1}{1}{}{s})} #2}


\begin{document}


\buildtitle{\courseTitle{}}{\semester}{\instructorA{} \hfill \instructorB{}}{Exercise Sheet \sheetNumber{}}

\vspace{-0.2in}

\begin{center}
{\bf Due date: \dueDate{}}

\end{center}

\noindent Working with your partner, you should try to solve all of the exercises below. You should then submit solutions to four of the problems, with each of you writing two, clearly indicating the author of each solution. Note that each problem is worth $10$ points, and starred exercises represent problems that may be a little tougher, should you wish to challenge yourself. In case you have difficulties submitting on COOL, please send your solutions by e-mail.

\paragraph{Exercise 1}  In this exercise you will practice building generating functions and decoding their sequences.
\begin{itemize}
	\item[(a)] Determine a closed form for the generating functions, $a(x)$, of the following sequences.
\begin{itemize}
	\item[(i)] $a_n = n^3$ for all $n \ge 0$.
	\item[(ii)] $a_n = \begin{cases}
	2^n &\mbox{if } n \mbox{ is odd} \\
	2^n + 3^{n/2} &\mbox{if } n \mbox{ is divisible by } 4 \\
	2^n - 3^{n/2} & \mbox{if } n \mbox{ is even, but not divisible by } 4
	\end{cases}$.
\end{itemize}
	\item[(b)] Given the following generating functions, find a closed form for the $n$th term, $a_n$, of the corresponding sequences.
\begin{itemize}
	\item[(I)] $a(x) = - \log (1 - 3x^2)$
	\item[(II)] $a(x) = \cos (x^2)$
\end{itemize}
\end{itemize}
\paragraph{Solution:} (黃子恆) See last few pages.

\paragraph{Exercise 2}  We have a corridor that is $3$ metres wide and $n$ metres long, and we want to cover the floor entirely with carpets. We have $n$ identical carpets, measuring $3 \times 1$ in metres, each of which can be placed horizontally or vertically. Let $c_n$ denote the number of different ways of covering the corridor.

\begin{itemize}
	\item[(a)] Find a recurrence relation for $c_n$, and provide sufficient initial conditions (starting from $c_0$) to compute the sequence.
	\item[(b)] Find a closed form for the generating function $c(x) = \sum_{n \ge 0} c_n x^n$.
	\item[(c)] Derive a closed formula for $c_n$.
\end{itemize}

\paragraph{Solution:} (黃子恆) See last few pages.

\paragraph{Exercise 3}   Using the definitions of the derivatives and products of formal power series, show that Leibniz's Rule (otherwise known as the product rule) also holds for formal power series.  That is,
\[ \left( F(x) \cdot G(x) \right)' = F'(x) \cdot G(x) + F(x) \cdot G'(x). \]
\paragraph{Solution:} (張沂魁) Suppose \(F(x) = \sum_{n=0}^{\infty} f_n x^n \) and \(G(x) = \sum_{n=0}^{\infty} g_n x^n \), then we know
\[
	F(x) \cdot G(x) = \sum_{n=0}^{\infty} \left( \sum_{k=0}^n f_k g_{n-k}  \right) x^n.
\]  Thus, 
\[
	\left( F(x) \cdot G(x) \right)^{\prime} = \sum_{n=1}^{\infty} n \left( \sum_{k=0}^n f_k g_{n-k}  \right) x^{n-1} = \sum_{n=0}^{\infty} (n+1)\left( \sum_{k=0}^{n+1} f_k g_{n+1-k}  \right) x^n.    
\]
Also, since we know 
\[
	F^{\prime} (x) = \sum_{n=0}^{\infty} (n+1) f_{n+1} x^n \text{ and }  \quad G^{\prime} (x) = \sum_{n=0}^{\infty} (n+1)g_{n+1}x^n,  
\]
so 
\[  
\begin{cases}\displaystyle
	F^{\prime} (x) \cdot G(x) = \sum_{n=0}^{\infty} \left( \sum_{k=0}^n (k+1)f_{k+1} g_{n-k}  \right) x^n \\ \displaystyle
	 F(x) \cdot G^{\prime} (x) = \sum_{n=0}^{\infty} \left( \sum_{k=0}^{n} f_k (n - k + 1) g_{n-k+1}  \right) x^n.  
\end{cases} 
\]
Thus, we have 
\begin{align*}
	F^{\prime} (x) \cdot G(x) + F(x) \cdot G^{\prime} (x) &= \sum_{n=0}^{\infty} \left( \sum_{k=0}^n (k+1)f_{k+1}g_{n-k} + \sum_{k=0}^n (n-k+1)f_k g_{n-k+1}   \right) x^n \\
	&= \sum_{n=0}^{\infty} \left( \sum_{u=1}^{n+1} u f_u g_{n-u+1} + \sum_{k=0}^n (n-k+1)f_k g_{n-k+1}   \right) x^n \\
	&= \sum_{n=0}^{\infty} \left( \sum_{u=0}^{n+1} u f_u g_{n-u+1} + \sum_{k=0}^{n+1} (n-k+1)f_k g_{n-k+1}   \right) x^n \\
	&= \sum_{n=0}^{\infty} \left( \sum_{k=0}^{n+1} (k + n - k + 1) f_k g_{n-k+1}  \right) x^n \\
	&= \sum_{n=0}^{\infty} \left( \sum_{k=0}^{n+1}(n+1)f_k g_{n-k+1}  \right)x^n = \left( F(x) \cdot G(x) \right)^{\prime} .
\end{align*}

\paragraph{Exercise 4}  One of the reasons that the Catalan numbers are so loved by combinators is that they pop up all over the place.\footnote{Indeed, Richard Stanley's \emph{Enumerative Combinatorics: Volume 2} famously has a set of exercises with no fewer than $66$ different Catalan structures!}  Show that the following sequences are equal to the Catalan sequence $(c_n)_{n \ge 0}$.
\begin{itemize}
	\item[(a)] $(b_n)_{n \ge 0}$, where $b_n$ is the number of rooted full binary trees with $n+1$ leaves.  A rooted full binary tree starts from a root node, and every node either has two descendents (a left child and a right child), or none.  If a node has no descendents, it is called a leaf.  See Figure~\ref{fig:trees} for the case $n = 3$.
\begin{comment}
	\begin{figure}[p]
\includegraphics[width=\linewidth]{catFBT.png}
\caption{Rooted full binary trees with $4$ leaves. (Thanks Wikipedia!)}
\label{fig:trees}
\end{figure}
\end{comment}
	\item[(b)] $(t_n)_{n \ge 0}$, where $t_n$ is the number of triangulations of a convex $(n+2)$-gon; that is, the number of ways a convex polygon with $n+2$ sides can be cut into triangles by connecting its vertices with straight non-crossing lines.\footnote{Note that since there are no convex polygons with $2$ sides, we take $t_0 = 1$, since we have nothing to do, and there is one way of doing nothing.}  See Figure~\ref{fig:triangles} for the case $n = 4$.
\begin{comment}
	\begin{figure}[p]
\includegraphics[width=\linewidth]{catHEX.png}
\caption{Triangulations of hexagons. (Thanks again, Wikipedia!)}
\label{fig:triangles}
\end{figure}
\end{comment}
	\item[(c)] $(p_n)_{n \ge 0}$, where $p_n$ is the number of permutations in $S_n$ that do not contain an increasing subsequence of length $3$; that is, the number of bijections $\pi: [n] \to [n]$ such that there are no $1 \le i < j < k \le n$ with $\pi(i) < \pi(j) < \pi(k)$.
\end{itemize}

\paragraph{Exercise 5}  A diagonal lattice path is a path in the grid $\mathbb{Z}^2$ with steps of the form $(1,1)$ or $(1,-1)$. Recall that the Catalan number $c_n$ counts the number of diagonal lattice paths of length $2n$ from $(0,0)$ to $(2n,0)$ that never go below the $x$-axis.  In this exercise you will give an alternative proof to show $c_n = \frac{1}{n+1} \binom{2n}{n}$.
\begin{itemize}
	\item[(a)] Determine the total number of diagonal lattice paths from $(0,0)$ to $(2n,k)$ for any integer $k \in \mathbb{Z}$.
	\item[(b)] Call a diagonal lattice path from $(0,0)$ to $(2n,0)$ \emph{illegal} if it goes below the $x$-axis.  Show there is a bijection between these illegal lattice paths and diagonal lattice paths from $(0,0)$ to $(2n,-2)$.
	\item[(c)] Deduce that the number of Dyck paths of length $2n$ is $\frac{1}{n+1} \binom{2n}{n}$.
\end{itemize}

\paragraph{Solution:} (張沂魁)
\begin{itemize}
	\item [(a)] If we want to go from \((0,0)\) to \((2n, k)\), then we must have 
	\[
		\begin{cases}
			(\# \text{ of } (1,1)) - (\# \text{ of } (1, - 1)) = k\\
			(\# \text{ of } (1,1)) + (\# \text{ of } (1, -1)) = 2n
		\end{cases},
	\]
	so we know \(\#\) of \((1, 1)\) is \(n + \frac{k}{2}\) and \(\#\) of \((1, -1)\) is \(n - \frac{k}{2}\). Hence, suppose the total number of diagonal lattice paths from \((0, 0)\) to \((2n, k)\) is \(a_{n, k}\), then 
	\[
		a_{n, k} = \begin{cases}
			0, &\text{ if } 2 \nmid k ;\\ \displaystyle
			\binom{2n}{n + \frac{k}{2}}, &\text{ if } 2 \mid k.
		\end{cases}
	\]          
	since any permutation of \(n + \frac{k}{2}\) \((1, 1)\)'s and \(n - \frac{k}{2}\) \((1, -1)\)'s form a diagonal lattice paths from \((0, 0)\) to \((2n, k)\).     
	\item [(b)] Suppose a illegal path \(P\) from \((0, 0)\) to \((2n, 0)\) is illegal, then we suppose \(P\) reaches from \((m, 0)\) to \((m+1, -1)\) for some \(m \in \mathbb{N} \) where this is the first time \(P\) intersects \(y = -1\). Now suppose the subpath of \(P\) from \((m+1, -1)\) to \((2n, 0)\) is called \(P_2\), then \(P = P_1 + P_2\) where \(P_1\) is the subpath from \((0, 0)\) to \((m+1, -1)\). Note that the number of \((1, 1)\) step in \(P_2\) is more than the number of \((1, -1)\) step in \(P_2\) by \(1\) since \(P_2\) goes from \(y=-1\) to \(y=0\). Hence, if we reflect \(P_2\) with respect to \(y = -1\), then \(P_2\) goes from \(y=-1\) to \(y=-2\), then we define a map that sends any illegal \(P\) to this \(P_2\)-reflected map. Now we claim that this map is a bijection between the set of illegal diagonal lattice path from \((0, 0)\) to \((2n, 0)\) and the set of diagonal lattice paths from \((0, 0)\) to \((2n, -2)\). We first show that this map is an injection. If illegal \(P, P^{\prime} \) is mapped to same paths, then \(P, P^{\prime} \) have the same \(P_1\) and \(P_2\) part, but \(P_1\) part is the same as it is in the \(P, P^{\prime} \), so \(P, P^{\prime} \) share same part from \((0, 0)\) to \((m + 1, -1)\). Also, \(P, P^{\prime} \) share same \(P_2\) part after the map, so we can reflect \(P_2\) back, and thus \(P, P^{\prime} \) have same part from \((m+1, -1)\) to \((2n, 0)\). Thus, this map is an injection. Also, this map is a surjection since for any diagonal lattice path from \((0, 0)\) to \((2n, -2)\), there must be a first intersection with \(y=-1\), and thus we can reflect the part after this intersection to get an illegal diagonal lattice path from \((0, 0)\) to \((2n, 0)\), and obviously this path is in the preimage of the original given path which goes from \((0, 0)\) to \((2n, -2)\). Hence, this map is a bijection.                                               
	\item [(c)] Since the number of diagonal lattice path from \((0, 0)\) to \((2n, 0)\) is \(\binom{2n}{n}\) (any permutation of \(n\) \((1, 1)\)'s and \(n\) \((1, -1)\)'s gives a choice), and by (b) we know the number of illegal lattice path from \((0, 0)\) to \((2n, 0)\) is the number of diagonal lattice paths from \((0, 0)\) to \((2n, -2)\), which is \(\binom{2n}{n-1}\) by (a), so 
	\begin{align*}
		\# \text{ of Dyck paths of length } 2n &= \binom{2n}{n} - \binom{2n}{n-1} = \frac{(2n)!}{n!n!} - \frac{(2n)!}{(n-1)!(n+1)!} \\
		&= \frac{(2n)!}{n!n!} - \frac{(2n)!n}{n!n!(n+1)} \\
		&= \frac{(2n)!}{n!n!} - \frac{n}{n+1} \frac{(2n)!}{n!n!} \\
		&= \frac{1}{n+1} \frac{(2n)!}{n!n!} = \frac{1}{n+1} \binom{2n}{n}.
	\end{align*}
\end{itemize}


% \newpage

% \section*{Hints}  The following QR codes contain hints for some of the homework exercises.  You should be able to decode them using any QR code scanner, including this one: \url{https://zxing.org/w/decode.jspx}.

% \paragraph{Exercise ?} Also available at \url{https://i.ibb.co/???????/S??E?.png}.

\includepdf[pages=-]{hw3.pdf}

\end{document}