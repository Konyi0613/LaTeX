\lecture{23}{5 Dec. 12:20}{}
\begin{definition}
    A poset \((P, \le)\) is locally finite if for every \(x, y \in P\), the interval 
    \[
        [x, y] = \left\{ z \in P: x \le z \le y \right\} 
    \]  
    is finite.
\end{definition}

\begin{eg}
    For \((\mathbb{N}, \le ), (\mathbb{Z} , \le), \left( 2^{[n]}, \subseteq  \right), \left( \mathbb{N} , \cdot \mid \cdot  \right)  \), then they are all locally finite. 
\end{eg}

\begin{definition}
    Given a locally finite poset \((P, \le)\), the incidence algebra is the set of functions 
    \[
        I(P) = \left\{ f: P ^2 \to \mathbb{R} \text{ s.t. } f(x, y) = 0 \quad \forall x \not\le y  \right\}. 
    \] 
\end{definition}

\begin{remark}
    We can think of the incidence algebra as the space of functions on the non-empty intervals \([x, y]\). 
\end{remark}

\begin{theorem}
    The incidence algebra is equipped with operations: 
    \begin{itemize}
        \item (pointwise) sums: \(\forall f, g \in I(P)\), 
        \[
            (f + g)(x, y) = f(x, y) + g(x, y). 
        \]
        \item scalar multiplication: \(\forall f \in I(P), \lambda \in \mathbb{R} \), 
        \[
            (\lambda f)(x, y) = \lambda f(x, y). 
        \]
        \item multiplication (convolution): \(\forall f, g \in I(P)\), 
        \[
            (f * g)(x, y) = \sum_{z \in [x, y]} f(x, z) g(z, y).
        \]
    \end{itemize}
\end{theorem}

\begin{remark}
    Let \(<\) be a linear extension of \((P, \le)\) is a total ordering of \(P\), i.e. \(\forall x \neq y\), \(x < y\) or \(y < x\), and if \(x \le y\) but \(x \neq y\), then \(x < y\).          
\end{remark}

\begin{proposition}
    We can write \(f \in I(P)\) as a matrix whose rows/columns are indexed by \(P\), and the corresponding matrix is upper triangular.
\end{proposition}

\begin{figure}[H]
    \centering
    \includegraphics[width=\textwidth]{./Figures/IMG_0838.jpg}
    \caption{An example of \(\left( 2^{[3]}, \subseteq  \right) \)}
\end{figure}

With this correspondence, 
\[
    M_{f * g} = M_f M_g.
\]
Thus, the multiplication (convolution) is not necessarily commutative, i.e. 
\[
    f * g \neq g * f,
\]
but it is associative, i.e. 
\[
    (f * g) * h = f * (g * h).
\]
\subsubsection{Identity}
There is an identity: the delta function 
\[
    \delta (x, y) = \begin{dcases}
        1, &\text{ if }  x = y;\\
        0, &\text{ if }  x \neq y.
    \end{dcases},
\]
then \(M_{\delta } = I_{\vert P \vert }\).

\subsubsection{Inverses}
For every \(f \in I(P)\), there is a unique inverse \(f^{-1} \in I(P)\) s.t. 
\[
    f * f^{-1} = \delta = f^{-1} * f \iff f(x, x) \neq 0 \quad \forall x \in P.
\] 
We can define it recursively: 
\[
    f^{-1} (x, x) = \frac{1}{f(x, x)}.
\]
Then, 
\[
    1 = \delta (x, x) = \left( f * f^{-1} \right)(x) = f(x, x) f^{-1} (x, x). 
\]
What about \(f^{-1}(x, y)\) for \(x \le y\) but \(x \neq y\)? Suppose we have already define \(f^{-1}(z, y)\) for all \(x < z \le y\), then 
\begin{align*}
    0 &= \delta (x, y) = \left( f * f^{-1} \right)(x, y) = \sum_{z \in [x, y]} f(x, z) f^{-1} (z, y) \\
    &= f(x, x) f^{-1}(x, y) + \sum_{x < z \le y} f(x, z) f^{-1}(z, y),   
\end{align*}     
so we can define 
\[
    f^{-1}(x, y) = \frac{- \sum_{x < z \le y} f(x, z) f^{-1}(z, y)}{f(x, x)}.
\]

\begin{remark}
    \(f^{-1}\) is unique since 
    \[
        I = M_{\delta } = M_{f * f^{-1}} = M_f M_{f^{-1}},
    \] 
    and if the inverse of \(M_f\) exists, then this inverse, which is \(M_{f^{-1}}\), is unique, so \(f^{-1}\) is unqiue. Also, since \(M_f\) is upper triangular, so \(\det M_f\) is the product of all entries on the diagonal, which is 
    \[
        \prod _{x \in P} f(x, x),
    \]  
    so we need \(f(x, x) \neq 0\) for all \(x \in P\) to ensure the existence of the inverse.  
\end{remark}

\begin{definition}
    The zeta function \(\zeta _P\) of a locally finite poset \(P\) is 
    \[
        \zeta _P (x, y) = \begin{dcases}
            1, &\text{ if } x \le y ;\\
            0, &\text{ if } x \not\le y.
        \end{dcases},
    \]  
    i.e. \(\zeta _P \equiv 1\) on the non-empty intervals. 
\end{definition}

\begin{definition}
    Given a locally finite poset \((P, \le)\), the mobius function \(\mu _P\) is the inverse of the zeta function:
    \[
      \mu _P = \zeta _P^{-1}.  
    \] 
\end{definition}

Observe that since \(\zeta _P (x, x) = 1 \neq 0\) for all \(x \in P\), so \(\mu _P = \zeta _P^{-1}\) exists: 
\[
    \mu _P (x, x) = \frac{1}{\zeta _P (x, x)} = 1 \quad \forall x \in P
\]  
and for all \(x < y\) we have 
\[
    \mu _P (x, y) = - \sum_{x < z \le y} \mu _P (z, y) = - \sum_{x \le z < y} \mu _P (x, z) . 
\]

\begin{remark}
    For a locally finite poset \(P\), the mobius function \(\mu _P\)  is the inverse of the zeta function \(\zeta_P^{-1} \), that is, 
    \[
        \mu _P(x, y) = \begin{dcases}
            1, &\text{ if } x=y ;\\
            -\sum_{x \le z \lneq y} \mu _P(x, z) = - \sum_{x \lneq z \le y} \mu _P(z, y) , &\text{ if } x \lneq y ;\\
            0, &\text{ if } x \nleq y.
        \end{dcases}
    \] 
\end{remark}