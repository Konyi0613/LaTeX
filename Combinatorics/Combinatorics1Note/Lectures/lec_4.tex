\lecture{4}{12 Sep. 12:20}{}
\section{Number of nonnegative integer solution to \(x_1 + \dots + x_k = n\)}
We can represent solutions using a "stars and bar" diagaram:
\begin{itemize}
    \item \(n\) stars represent the items 
    \item \(k-1\) bars to divides the groups  
\end{itemize}
\begin{eg}
    \(x_1 = 3, x_2 = 1, x_3 = 0, x_4 = 5\). \((k=4, n=9)\)  
\end{eg}
\begin{explanation}
    \[
        \underbrace{* * *}_{x_1} \mid \underbrace{*}_{x_2} \mid \mid \underbrace{* * * * *}_{x_3}
    \]
\end{explanation}

Hence, we can use a projection between solution and diagrams with \(k-1\) bars and \(n\) stars. 

Each diagram consists of \(n + k - 1\) symbols. Once we know which are the bars, we know the full diagram. 

\[
    \text{number of diagrams} = \binom{n + k - 1}{k - 1} = \binom{n + k - 1}{n}
\]

\begin{proposition}
    The number of non-negative integer solutions to \(x_1 + \dots + x_k = n\) is \(\binom{n + k - 1}{k - 1}\). 
\end{proposition}
Now we have a new problem. 
\begin{question}
    How many solutions are there to \(x_1 + \dots + x_k = n\) with \(x_i \ge 1\) for all \(i\)?
\end{question}

We can let \(y_i = x_i - 1\), then \(y_i \ge 0\) and \(y_1 + \dots + y_k = n - k\). Hence, the answer is 
\[
    \binom{(n - k) + (k - 1)}{k-1} = \binom{n-1}{k-1}.
\]   

\begin{definition}[Multisets] \label{def: multiset}
    An unordered collection of elements with repretition allowed. 
    \[
        \left\{ \left\{ 1,1,1,2,3 \right\}  \right\} \neq \left\{ \left\{ 1,2,3 \right\}  \right\} 
    \] can be represented as an ordered tuple in increasing order.
\end{definition}

\begin{eg}
    How many multisets of size \(n\) are there from a set of size \(k\)?  
\end{eg}
\begin{explanation}
    Let \(x_i\) be the multiplicites of the \(i\)-th element in the multiset. Then \(x_i \ge 0\) and 
    \[
        x_1 + \dots + x_k = n.
    \]   
    Hence, the number of multisets is 
    \[
        \binom{n + k - 1}{k - 1}.
    \]
\end{explanation}

Alternatively, multisets are \((a_1, \dots , a_n)\) with \(1 \le a_1 \le \dots \le a_n \le k\). Now if we let \(b_i = a_i + i - 1\), then 
\[
    (b_1, \dots , b_n) = (a_1, a_2 + 1, \dots , a_n + n - 1) \text{ with } 1 \le b_1 < b_2 < \dots < b_n \le n + k - 1.
\]   Note that there is a bijection between \(\left\{ (a_1, \dots , a_n) \right\} \) and \(\left\{ (b_1, \dots , b_n) \right\} \). This shows the number of multisets of size \(n\) from \([k]\) is the number of subsets of \([n + k - 1]\) of size \(n\), which is 
\[
    \binom{n + k - 1}{n} = \binom{n + k - 1}{k - 1}.
\]      
Now we add some new setting. 
\begin{itemize}
    \item Distinguishable items 
    \item Indistinguishable groups 
    \item Groups non-empty.
\end{itemize}
The objects we are counting is 
\[
    \left\{ S_1, S_2, \dots , S_k \right\} 
\] with \(S_1 \cupdot S_2 \cupdot \dots \cupdot S_k = [n]\) and \(S_i \neq \varnothing \) for all \(i\). 

\begin{definition}[The Stirling Number of the second kind] \label{def: stirling num second kind}
\(S(n,k)\) is defined to be number of partitions of \(n\) distinct items into \(k\) indistinguishable non-empty groups.
\end{definition}

\begin{eg}
    \(S(n, 1) = 1\) for all \(n \ge 1\). \(S(n, n) = 1\) for all \(n\). \(S(n, n-1) = \binom{n}{2}\) for all \(n \ge 2\). \(S(n, 2) = 2^{n-1} - 1\).       
\end{eg}
\begin{explanation}
    We just talk about the \(S(n, 2)\) one. Since we can choose any subset of \([n]\), so there are \(2^n\) possibilities, but each partition is counted twice, so we have to divide it by \(2\), and subtract the partition that includes empty group, so it is \(2^{n-1} - 1\).    
\end{explanation}

\begin{proposition}
    For all \(n, k\), 
    \[
        S(n, k) = S(n-1, k-1) + k S(n-1,k).
    \]  
\end{proposition}
\begin{proof}
    Case analysis: 
    \begin{itemize}
        \item Case 1: \(\left\{ n \right\} \) is a group. \\
        This means the remaining \(n-1\) elements are partitioned into \(k-1\) groups, so there are \(S(n-1, k-1)\) possibilities. 
        \item Case 2: \(\left\{ n \right\} \) is not a group.  \\
        \(n - 1\) left elments is first partitioned into \(k\) groups, then we can distribute the \(n\)-th element into each group, so there are \(k S(n-1, k)\) possibilities.    
    \end{itemize}
    By sum rule, we know 
    \[
        S(n, k) = S(n - 1, k - 1) + k S(n - 1, k).
    \]
\end{proof}

\begin{eg}
    Using induction to prove 
    \[
        S(n, n - 1) = \binom{n}{2}.
    \]
\end{eg}
\begin{explanation}
    \begin{align*}
        S(n, n - 1) &= S(n - 1, n - 2) + (n - 1) S(n - 1, n - 1) = S(n - 1, n - 2) + (n - 1) \\
        &= \dots = 1 + 2 + \dots + n - 1 = \binom{n}{2}.
    \end{align*}
\end{explanation}

Now what if the groups are distinguishable? Also, we have
\begin{itemize}
    \item items distinguishable 
    \item groups distinguishable
    \item groups non-empty.
\end{itemize}

Short answer: \(S(n, k) k!\). 