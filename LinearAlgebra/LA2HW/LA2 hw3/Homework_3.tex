%Template by Saintan. 
\documentclass[a4paper]{article}
    %% font & format %%
\usepackage[margin=2.5cm]{geometry}
\usepackage{type1cm, titlesec, fancyhdr, titling}
\usepackage{multicol}
\usepackage[dvipsnames]{xcolor}
\usepackage{ulem}
    %% Math, Logos & symbols %%
\usepackage{amsmath,amsthm,amssymb, mathtools}
\usepackage{yhmath, faktor, dsfont}
\usepackage{academicons, wasysym, marvosym}
\usepackage[scr]{rsfso}
\usepackage{braket}

%% Enhancement %%
\usepackage{graphicx, tabularx}
\usepackage[shortlabels,inline]{enumitem}
%% TikZ %%
\usepackage{tikz-cd}
\usepackage[breakable]{tcolorbox}
\usetikzlibrary{decorations.pathmorphing}
\usetikzlibrary{calc, arrows,matrix}

%% Reference: Make sure these are the last packages included! %%
\usepackage[english]{babel}
\usepackage[nameinlink]{cleveref}


%%%頁面設定 with titling%%%
\setlength{\headheight}{15pt}
\setlength{\droptitle}{-1.5cm}
\parindent=24pt

\newtheoremstyle{mystyle}
  {6pt}{15pt}% 上下間距
  {}%          內文字體
  {}%              縮排
  {\bf}%       標頭字體
  {.}%       標頭後標點
  {1em}% 內文與標頭距離
  {}% Theorem head spec (can be left empty, meaning 'normal')

\theoremstyle{mystyle}	
\newtheorem{theorem}{Theorem}
\newtheorem*{definition}{Definition}
\newtheorem{example}[theorem]{Example}
\newtheorem{exercise}[theorem]{Exercise}
\newtheorem{corollary}[theorem]{Corollary}
\newtheorem{property}[theorem]{Property}
\newtheorem{proposition}[theorem]{Proposition}
\newtheorem{lemma}[theorem]{Lemma}
\newtheorem{problem}[theorem]{Problem}
\newtheorem{answer}{Answer}[section]
\newtheorem{fact}[theorem]{fact}
\newtheorem*{recall}{Recall}
\newtheorem*{remark}{Remark}
\newtheorem*{claim}{Claim}
\newtheorem*{observation}{Observation}

%% New commands %%
\newcommand{\N}{\mathbb{N}}
\newcommand{\Z}{\mathbb{Z}}
\newcommand{\Q}{\mathbb{Q}}
\newcommand{\R}{\mathbb{R}}
\newcommand{\C}{\mathbb{C}}
\newcommand{\F}{\mathbb{F}}
\renewcommand{\Pr}{\mathbb{P}}
\newcommand{\E}{\mathbb{E}}
\newcommand{\B}{\mathcal{B}}
\newcommand{\mcC}{\mathcal{C}}
\renewcommand{\L}{\mathcal{L}}

\DeclareMathOperator{\Dom}{Dom}
\DeclareMathOperator{\id}{id}
\newcommand{\6}{\partial}
\newcommand{\ds}{\displaystyle}

\DeclarePairedDelimiter{\norm}{\lVert}{\rVert} %Use \norm* for \left\| \right\|
\DeclarePairedDelimiter{\gen}{\langle}{\rangle}
\DeclarePairedDelimiter{\innerp}{\langle}{\rangle}
\DeclarePairedDelimiter{\floor}{\lfloor}{\rfloor}
\DeclarePairedDelimiter{\ceil}{\lceil}{\rceil}
\DeclareMathOperator{\vsspan}{span}
\DeclareMathOperator{\image}{Im}
\DeclareMathOperator{\nullity}{nullity}
\DeclareMathOperator{\rank}{rank}
\DeclareMathOperator{\ch}{ch}
\DeclareMathOperator{\tr}{tr}
\DeclareMathOperator{\m}{m}

\usepackage{xeCJK}
    \xeCJKsetup{AutoFakeBold=true, AutoFakeSlant=true}
    \setCJKmainfont{PingFang TC}
    %\setmainfont{Times New Roman}
\begin{document}
\title{\textbf{Homework 3}}
\author{Linear Algebra (II), Spring 2025 \\ B13902024 張沂魁}
\date{\textbf{Deadline: 3/12 (Wed.) 12:10}}
\maketitle

\begin{exercise}
    Let $T:V\to V$ be a linear operator with
    $\ch_T(x)=(x-\lambda)^n$ and
    $$
    V=\bigoplus_{i=1}^rZ(v_i;T)
    $$
    be the decomposition of $V$ into a direct sum of $T$-cyclic subspaces according to Theorem 3.2. Let $s_i=\dim Z(v_i;T)$ and assume that $s_1\geq s_2\geq\cdots\geq s_r$. Determine $\m_T(x)$ and $\dim E_\lambda$. (Express them in terms of $r$ and $s_i$.)
\end{exercise}

\hfill \\ \textbf{Solution: }
Suppose $m_T(x)=(x-\lambda)^d$ and note that
\[
   V = \ker (ch_T(T)) = \ker (T-\lambda I)^n = K_\lambda
\]
so we have $V=K_\lambda$, and we know $m_T(T)(v)=0$ for all $v \in V$. Since we must have
\begin {gather*}
(T-\lambda I)^d(v_1)=0 \\ 
(T-\lambda I)^d(v_2)=0 \\
\vdots \\ 
(T-\lambda I)^d(v_r)=0
\end {gather*}
so we know $d=\max_{1 \le i \le r} p_i$ such that $p_i$ is the smallest positive integer 
with $(T-\lambda I)^{p_i}(v_i)=0$ since if $d$ is smaller than
this number, then there exists some $v_i$ such that $(T-\lambda I)^d(v_i) \neq 0$,
and if $d=\max_{1 \le i \le r} p_i$, then $(T-\lambda I)^d(v_i)=0$ for all $i$ such that $1 \le i \le r$.
Now note that in the process of constructing $V$ in Theorem 3.2, we have $(T-\lambda I)^{s_i}(v_i)=0$ for all $i$, 
and since we know
\[ B_i= \set{v_i, (T-\lambda I)(v_i), (T-\lambda I)^2(v_i), \cdots , (T-\lambda I)^{s_i-1}(v_i)}\]
is a basis of $Z(v_i ; T)$, so $(T-\lambda I)^{x_i}(v_i) \neq 0$ for all $x_i \le s_i-1$, so $s_i$ is the 
smallest positive integer $p_i$ such that $(T-\lambda I)^{p_i}(v_i)=0$. That is,
 $d=\max_{1 \le i \le r} s_i = s_1$, so $m_T(x)=(x-\lambda)^{s_1}$. \\
Now we consider the Jordan form of $T$ being
\[
T_J=
\begin{pmatrix}
J_1 &0 &\cdots &0 &0 \\
0 &J_2 &\cdots &0 &0\\
&&\vdots \\ 
0 &0 &\cdots &0 &J_r
\end{pmatrix}
\]
where $J_i$ is a $s_i \times s_i$ Jordan block for all $i$. Now we claim that for every Jordan block
$J_i$, it has only $\lambda$ for its eigenvalue and the dimension of its eigenspace is $1$. First,
 since we can find a basis $\beta_i$ of $Z(v_i;T)$ so that $T|_{Z(v_i;T)}$'s matrix representation 
 with respect to $\beta_i$ is $J_i$, and since $Z(v_i;T)$ is a $T$-invariant space, so $T|_{Z(v_i;T)}$'s characteristic polynomial divides
 the characteristic polynomial of $T$, and because $T$ only has $\lambda$ for its eigenvalue, so $T|_{Z(v_i;T)}$ also only have $\lambda$ for its eigenvalue.
 Note that
 \[
 J_i=
 \begin{pmatrix}
    \lambda &1  \\
    &\lambda &1  \\
    &&\ddots &1\\ 
    &&&\lambda
 \end{pmatrix}
 \]
 and if
 \[
    \begin{pmatrix}
        \lambda &1  \\
        &\lambda &1  \\
        &&\ddots &1\\ 
        &&&\lambda
     \end{pmatrix}
     \begin{pmatrix}
        a_1 \\
        a_2 \\ 
        \vdots \\
        a_r
     \end{pmatrix}
     =
     \lambda
     \begin{pmatrix}
     a_1 \\
     a_2 \\ 
     \vdots \\
     a_r
     \end{pmatrix}
 \]
 we must have $a_2=a_3=\cdots=a_r=0$ and $a_1$ can be any number, so the dimension of the eigenspace of
 $J_i$ is one. Now suppose we pick some $v_i$ such that $v_i$ is an eigenvector of $J_i$, then we know $\set{v_i}$ is a basis of
 the eigenspace of $J_i$, and we know
 \[
 \begin{pmatrix}
    v_1 \\ 0 \\ 0 \\ \vdots \\ 0
 \end{pmatrix}, 
 \begin{pmatrix}
    0 \\ v_2 \\ 0 \\ \vdots \\ 0
 \end{pmatrix},
 \begin{pmatrix}
    0 \\ 0 \\ v_3 \\ \vdots \\ 0
 \end{pmatrix},
 \begin{pmatrix}
    0 \\ 0 \\ 0 \\ \vdots \\ v_r
 \end{pmatrix}
 \] are all the eigenvectors of $T_J$, which can be easily verified by block matrix multiplication.
 Now we call the set consisting of these eigenvectors is called $U$, and the element of $U$ with $v_i$ embedded in it is called $u_i$, then we know all elements in $U$ are linearly independent, so $\dim E_\lambda \ge r.$
Now we claim that $U$ is a basis of $E_\lambda$. First, it is easy to verify $\vsspan U \subseteq E_\lambda$, now we show
that $E_\lambda \subseteq \vsspan U$. Suppose
\[
w =
\begin{pmatrix}
    w_1 \\ w_2 \\ \vdots \\ w_r
\end{pmatrix} \in E_\lambda
\] with each $w_i$ a $s_i \times s_i$ block matrix, so we must have
\[
T_J w = \begin{pmatrix}
    J_1 &0 &\cdots &0 &0 \\
    0 &J_2 &\cdots &0 &0\\
    &&\vdots \\ 
    0 &0 &\cdots &0 &J_r
    \end{pmatrix} \begin{pmatrix}
        w_1 \\ w_2 \\ \vdots \\ w_r
    \end{pmatrix} = 
    \begin{pmatrix}
        J_1w_1 \\ J_2w_2 \\ \vdots \\ J_rw_r
    \end{pmatrix} =
    \lambda \begin{pmatrix}
        w_1 \\ w_2 \\ \vdots \\ w_r
    \end{pmatrix}
    =
    \begin{pmatrix}
        \lambda w_1 \\ \lambda w_2 \\ \vdots \\ \lambda w_r
    \end{pmatrix}
\] we can notice that $J_i w_i=\lambda w_i$ for all $i$, that is, $w_i$ is an eigenvector of $J_i$ for all $i$,
so $w_i =c_i v_i$ for some scalar $c_i$, so
\[ 
w=\sum_{i=1}^r c_i u_i
\]
which means $w \in \vsspan U$, so now we know $E_\lambda = \vsspan U$, and thus $\dim E_\lambda = \dim U = r$.
 \begin{exercise}
    Let $T$ be a linear operator on a finite-dimensional vector space $V$ with Jordan canonical form
    \[
    \left(
    \begin{array}{cccccccc}
        2 & 1 & \multicolumn{1}{l|}{0} & 0 & 0 & 0 & 0 \\
        0 & 2 & \multicolumn{1}{l|}{1} & 0 & 0 & 0 & 0 \\
        0 & 0 & \multicolumn{1}{l|}{2} & 0 & 0 & 0 & 0 \\ \cline{1-5}
        0 & 0 & \multicolumn{1}{l|}{0} & 2 & \multicolumn{1}{l|}{1} & 0 & 0 \\
        0 & 0 & \multicolumn{1}{l|}{0} & 0 & \multicolumn{1}{l|}{2} & 0 & 0 \\ \cline{4-7}
        0 & 0 & 0 & 0 & \multicolumn{1}{l|}{0} & 3 & 0 \\
        0 & 0 & 0 & 0 & \multicolumn{1}{l|}{0} & 0 & 3
    \end{array}
    \right).
    \]
    \begin{enumerate}[(a)]
        \item Determine $\ch_T(x)$ and $\m_T(x)$.
        \item Determine $\dim E_\lambda$ for each eigenvalue $\lambda$ of $T$.
        \item For each eigenvalue $\lambda$ of $T$, determine the smallest positive integer $p$ for which $K_\lambda=\ker((T-\lambda I)^p)$.
    \end{enumerate}
\end{exercise}

\hfill \\ \textbf{Solution: } First note that $T$ has $2, 3$ as its eigenvalues.
\begin{itemize}
    \item [(a)] Since $V= K_2 \oplus K_3$, and the submatrix with $2$ on its diagonal line is with size $5 \times 5$, while the
    submatrix with $3$ on its diagonal line has size $2 \times 2$, so $ch_T(x)=(x-2)^5(x-3)^2$. Now
    suppose $m_T(x)=(x-2)^{d_2}(x-3)^{d_3}$, then by Exercise 1 we know $d_2$ is the largest size of the Jordan block with respect to eigenvalue 2,
    while $d_3$ is the largest size of the Jordan block with respect to eigenvalue $3$, so $d_2=3$ and $d_3=1$.
    Therefore, $m_T(x)=(x-2)^3(x-3)$.
    \item [(b)] By Exercise 1, we know $\dim E_\lambda$ is the number of Jordan block with respect to eigenvalue $\lambda$, so
    $\dim E_2=2$, and $\dim E_3 = 2$.
    \item [(c)] Claim: For every eigenvalue $\lambda$, if $K_\lambda=\bigoplus_{i=1}^r Z(v_i;T)$, where $Z(v_i; T)$ is the 
    $T$-cyclic subspace generated by $v_i$(We use the method in Theorem 3.2 to determine $v_i$), and $s_i=\dim Z(v_i;T)$, and without lose of generacity we suppose
    $s_1 \ge s_2 \ge \cdots \ge s_r$, now if $p$ is the smallest
    positive integer such that $K_\lambda=\ker ((T-\lambda I)^p)$, then $p=s_1$. \\ 
    Proof: It is easy to prove that $\ker ((T-\lambda I)^s_1) \subseteq K_\lambda$. Now note that
    \[ B=\bigcup_{i=1}^r\set{v_i, T(v_i), T^2(v_i), \cdots , T^{s_i-1}(v_i)}\]
    is a basis of $\bigoplus_{i=1}^r Z(v_i;T)$, so for all $u \in K_\lambda$, we can write
    $u=\sum_{i=1}^r\sum_{j=0}^{s_i-1}\alpha_{ij}T^j(v_i)$ for some scalar $\alpha_{ij}$. Besides, note that $(T-\lambda I)^{s_1}(v_i)=0$ for all $1 \le i \le r$.
    The reason has been explained in Exercise 1. Therefore, we have 
    \[ 
    (T-\lambda I)^{s_1}\left(\sum_{i=1}^r\sum_{j=0}^{s_i-1}\alpha_{ij}T^j(v_i)\right)=\sum_{i=1}^r\sum_{j=0}^{s_i-1}\alpha_{ij}T^j(T-\lambda I)^{s_1}(v_i)=0
    \] for all $\alpha_{ij}$. In other words, $K_\lambda \subseteq \ker((T-\lambda I)^{s_1})$, so 
    $K_\lambda = \ker((T-\lambda I)^{s_1})$. Now we have to prove the minimality of $s_1$. If there is a positive integer 
    $k$ such that $K_\lambda=\ker((T-\lambda I)^{s_1-k})$, then $(T-\lambda I)^{s_1 - k}(v_1)=0$, but since $s_1$ is the smallest positive integer 
    $p$ such that $(T-\lambda I)^p(v_1)=0$, so it is impossible to have such $k$, and we are done. \\
    By this claim, if we say the smallest positive integer $p$ such that $K_\lambda=\ker((T-\lambda I)^p)$ is $p_\lambda$, then
    $p_2=3, p_3=1$.
\end{itemize}

\begin{exercise}
    Let
    $$
    A=\begin{pmatrix}
        11 & -26 & -11 & 14 & 9 & -7\\
        4 & -10 & -3 & 4 & 3 & -2 \\
        11 & -22 & -13 & 12 & 9 & -6 \\
        4 & -8 & -4 & 2 & 4 & -2 \\
        2 & -4 & -2 & 2 & 0 & -1\\
        0 & 0 & 0 & 0 & 0 & -2
    \end{pmatrix}.
    $$
    It is known that $\ch_A(x)=(x+2)^6$. 
    \begin{enumerate}[(a)]
        \item Determine $\m_A(x)$ and $\dim E_{-2}$.
        \item Determine the Jordan form of $A$. \\
        \textit{Hint}: What do the properties proved in Exercise 1 tell you?
    \end{enumerate} 
\end{exercise}
\hfill \\ \textbf{Solution: }
\begin{itemize}
    \item [(a)] Suppose $m_A(x)=(x+2)^d$, then by calculation we know $3$ is the smallest positive integer $p$ such that $(A+2I)^p=0$, that is, $d=3$.
    Besides, we know $\dim ker(A+2I)=3$, which can also be done by easy computation.
    \item [(b)] By $(a)$ and Exercise 1, we know in the Jordan form of $A$, the biggest Jordan block has the size $3$ since $m_A(x)=(x+2)^3$, and the number 
    of Jordan blocks is $3$, now since the sum of the size of the Jordan block is $6$, which is the size of $A$, and the size of every Jordan block must be greater than $0$, 
    so there are $3$ Jordan blocks of the size $3,2,1$, respectively, in the Jordan form of $A$, and since the only eigenvalue of $A$ is $-2$, so the Jordan form of $A$ is
    \[
    \begin{pmatrix}
        -2 &1 &0 &0 &0 &0 \\ 
        0 &-2 &1 &0 &0 &0 \\ 
        0 &0 &-2 &0 &0 &0 \\ 
        0 &0 &0 &-2 &1 &0 \\ 
        0 &0 &0 &0 &-2 &0 \\ 
        0 &0 &0 &0 &0 &-2
    \end{pmatrix} 
    \]
\end{itemize}

\begin{figure}[h]
    \includegraphics[width=\textwidth]{我的人生已經失敗了吧.jpg}
    \caption{Me after doing HW3}
\end{figure}

\end{document}