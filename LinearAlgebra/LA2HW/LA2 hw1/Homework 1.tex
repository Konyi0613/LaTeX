%Template by Saintan. 
\documentclass[a4paper]{article}
    %% font & format %%
\usepackage[margin=2.5cm]{geometry}
\usepackage{type1cm, titlesec, fancyhdr, titling}
\usepackage{multicol}
\usepackage[dvipsnames]{xcolor}
\usepackage{ulem}
    %% Math, Logos & symbols %%
\usepackage{amsmath,amsthm,amssymb, mathtools}
\usepackage{yhmath, faktor, dsfont}
\usepackage{academicons, wasysym, marvosym}
\usepackage[scr]{rsfso} 

%% Enhancement %%
\usepackage{graphicx, tabularx}
\usepackage[shortlabels,inline]{enumitem}
%% TikZ %%
\usepackage{tikz-cd}
\usepackage[breakable]{tcolorbox}
\usetikzlibrary{decorations.pathmorphing}
\usetikzlibrary{calc, arrows,matrix}

%% Reference: Make sure these are the last packages included! %%
\usepackage[english]{babel}
\usepackage[nameinlink]{cleveref}


%%%頁面設定 with titling%%%
\setlength{\headheight}{15pt}
\setlength{\droptitle}{-1.5cm}
\parindent=24pt

\newtheoremstyle{mystyle}
  {6pt}{15pt}% 上下間距
  {}%          內文字體
  {}%              縮排
  {\bf}%       標頭字體
  {.}%       標頭後標點
  {1em}% 內文與標頭距離
  {}% Theorem head spec (can be left empty, meaning 'normal')

\theoremstyle{mystyle}	
\newtheorem{theorem}{Theorem}
\newtheorem*{definition}{Definition}
\newtheorem{example}[theorem]{Example}
\newtheorem{exercise}[theorem]{Exercise}
\newtheorem{corollary}[theorem]{Corollary}
\newtheorem{property}[theorem]{Property}
\newtheorem{proposition}[theorem]{Proposition}
\newtheorem{lemma}[theorem]{Lemma}
\newtheorem{problem}[theorem]{Problem}
\newtheorem{answer}{Answer}[section]
\newtheorem{fact}[theorem]{fact}
\newtheorem*{recall}{Recall}
\newtheorem*{remark}{Remark}
\newtheorem*{claim}{Claim}
\newtheorem*{observation}{Observation}

%% New commands %%
\newcommand{\N}{\mathbb{N}}
\newcommand{\Z}{\mathbb{Z}}
\newcommand{\Q}{\mathbb{Q}}
\newcommand{\R}{\mathbb{R}}
\newcommand{\C}{\mathbb{C}}
\newcommand{\F}{\mathbb{F}}
\renewcommand{\Pr}{\mathbb{P}}
\newcommand{\E}{\mathbb{E}}
\newcommand{\B}{\mathcal{B}}
\newcommand{\mcC}{\mathcal{C}}
\renewcommand{\L}{\mathcal{L}}

\DeclareMathOperator{\Dom}{Dom}
\DeclareMathOperator{\id}{id}
\newcommand{\6}{\partial}
\newcommand{\ds}{\displaystyle}

\DeclarePairedDelimiter{\norm}{\lVert}{\rVert} %Use \norm* for \left\| \right\|
\DeclarePairedDelimiter{\gen}{\langle}{\rangle}
\DeclarePairedDelimiter{\innerp}{\langle}{\rangle}
\DeclarePairedDelimiter{\floor}{\lfloor}{\rfloor}
\DeclarePairedDelimiter{\ceil}{\lceil}{\rceil}
\DeclareMathOperator{\vsspan}{span}
\DeclareMathOperator{\image}{Im}
\DeclareMathOperator{\nullity}{nullity}
\DeclareMathOperator{\rank}{rank}
\DeclareMathOperator{\ch}{ch}
\DeclareMathOperator{\tr}{tr}
\DeclareMathOperator{\m}{m}

\begin{document}
\title{\textbf{Homework 1}}
\author{Linear Algebra (II), Spring 2025}
\date{\textbf{Deadline: 2/26 (Wed.) 12:10}}
\maketitle

\begin{exercise}[Exercise 1.4] 
    Prove that if $I_1$ and $I_2$ are two ideals of $F[x]$, then the set 
    \[
    \{f_1(x)+f_2(x): f_j(x)\in I_j\}
    \]
    is also an ideal of $F[x]$.
\end{exercise}

\begin{exercise}[Exercise 1.5] 
    Let \( T: V \to V \) be a linear operator on \( V \). Check the following sets are ideals of \( F[x] \): 
    \begin{enumerate}[(i)]
        \item The set \( I_T:=\{ f(x) \in F[x] : f(T) = 0 \} \).
        \item The set \( I_T(v):=\{ f(x) \in F[x] : f(T)(v) = 0 \} \),
        where \( v \in V \) is a fixed given vector.
        \item The set \( I_T(v,W) := \{ f(x) \in F[x] : f(T)(v) \in W \} \), where \( W \) is a \( T \)-invariant subspace of \( V \) and \( v \in V \) is a given vector.
        \item In part (iii), if \( W \) is only a subspace of \( V \) but not \( T \)-invariant, does the statement still hold? Prove it or disprove it by giving a counterexample.
    \end{enumerate}
    \begin{remark}
        If \( V \) is finite-dimensional, then we know that the first set
        \[
        \{ f(x) \in F[x] : f(T) = 0 \} = (\m_T(x))
        \]
        is a principal ideal generated by the minimal polynomial of \( T \).
    \end{remark}
\end{exercise}

\begin{exercise}[Exercise 1.6] 
    Prove that if \( W \) is a \( T \)-invariant subspace of \( V \) and \( v_1 - v_2 \in W \), then \( I_T(v_1,W)=I_T(v_2,W) \).
\end{exercise}

\begin{exercise}[Exercise 1.9]
    Prove that $(f(x)) = (g(x))$ if and only if $f(x) = cg(x)$ for some nonzero $c$ in $F$.
\end{exercise}

\begin{exercise} 
    Let $T:V\to V$ be a linear operator on a finite-dimensional vector space $V$, and let $v\in V$ be a nonzero vector in $V$. Denote $W=Z(v;T)$ to be the $T$-cyclic subspace generated by $v$. 
    \begin{enumerate}[(a)]
        \item Show that the $T$-annihilator of $v$ defined in Definition 1.10 is the minimal polynomial of $T|_W$ and that its degree is equal to $\dim W$.
        \item Deduce that the $T$-annihilator of $v$ is equal to the characteristic polynomial of $T|_W$.
        \item Show that the degree of the $T$-annihilator of $v$ is $1$ if and only if $v$ is an eigenvector of $T$.
    \end{enumerate}
\end{exercise}

\begin{center}
    (There are extra exercises in the next page.)
\end{center}
\newpage

\section*{Extra Exercises}
You aren't asked to hand in extra exercises, and solving them will NOT affect your grade.

\begin{definition}
    For a matrix $A\in M_{n\times n}(F)$, one can show that
    \[
    \{ f(x) \in F[x] : f(A) = O \}
    \]
    is an ideal in $F[x]$. Hence it is principal, say, generated by a monic polynomial $g(x)$. Then we define the \textbf{minimal polynomial} of $A$ to be $g(x)$ and denote it by $\m_A(x)$.
\end{definition}
\begin{exercise}
    Given a nonzero matrix $A\in M_{n\times n}(F)$, show that the sequence of matrices
    \[
    I_n, A, A^2, A^3, A^4, \dots
    \]
    spans a subspace of $M_{n\times n}(F)$ of dimension $k$, where $k$ is the degree of minimal polynomial of $A$. 
\end{exercise}

\begin{exercise}
    Let $A\in M_{n\times n}(F)$ and let $\displaystyle \m_A(x)=\prod_{i=1}^k (x-\lambda_i)^{m_i}$ be its minimal polynomial. Find the minimal polynomial of $2n\times 2n$ matrix
    \[
    B=
    \begin{pmatrix}
        A & I_n \\
        O & A
    \end{pmatrix}
    \]
    in terms of $k$, $\lambda_i$, and $m_i$.
\end{exercise}
\end{document}