\chapter{Vector Space}
\lecture{1}{3 Sep. 10:20}{}
\section{Introduction to vector and vector space}
In high school, our vectors are in \(\mathbb{R} ^2\) and \(\mathbb{R} ^3\), and we have define the addition and scalar multiplication of vectors.  
\begin{figure}[H]
    \centering
    \incfig{vec}
    \caption{Vectors in \(\mathbb{R} ^2\)}
    \label{fig:vec}
\end{figure}
\begin{eg}
    \(\mathbb{R} ^n = \left\{ \left( a_1, a_2, \dots , a_n \mid a_i \in \mathbb{R}  \right)  \right\} \) 
\end{eg}
With this type of space, we can define addition and multiplication as 
\begin{align*}
    \left( a_1, a_2, \dots ,a_n \right) + \left( b_1, b_2, \dots , b_n \right) &= \left\{ a_1 + b_1, a_2 + b_2, \dots , a_n + b_n \right\} \\
    \alpha \cdot \left( a_1, a_2, \dots ,a_n \right) &= \left( \alpha a_1, \alpha a_2, \dots , \alpha a_n \right)  
\end{align*}
Also, if we define a space:
\begin{eg}
    \(V = \left\{ \text{function } f:(a,b) \to \mathbb{R}  \right\} \), where \((a,b)\) is an open interval.  
\end{eg}
then this can also be a vector space after defining addtion and multiplication.
\begin{note}
    In a vector space, we have to make sure the existence of \(0\)-element, which means \(0(x) = 0\). 
\end{note}
Now we give a more abstract example:
\begin{eg}
    Suppose \(S\) is any set, then define \(V = \left\{ \text{all functions from } S \text{ to } \mathbb{R} \right\} \)  
\end{eg}
If we define \((f + g)(s) = f(s) + g(s)\) and \((\alpha \cdot f)(s) = \alpha \cdot f(s)\), and \(0(s) = 0\), then this is also a vector space. 

\subsubsection{Put some linear conditions}
\begin{eg}
    In \(\mathbb{R} ^n\), fix \(\vec{a}=\left( a_1, a_2, \dots , a_n \right) \in \mathbb{R} ^n \), if we define 
\[
    W = \left\{ (x_1, x_2, \dots ,x_n) \in \mathbb{R} ^n \mid a_1 x_1 + a_{2} x_2 + \dots + a_n x_n = 0  \right\}, 
\] then this is also a vector space.
\end{eg}
However, if we have 
\[
    W^{\prime} = \left\{ (x_1, \dots , x_n) \in \mathbb{R} ^n \mid a_1 x_1 + \dots +a_n x_n = 1\right\},
\]
then this is not a vector space because it is not close.

\begin{eg}
    In \(V = \left\{ (a,b) \to \mathbb{R}  \right\} \) or \(W_1 = \left\{ \text{polynomial defined on } (a, b) \right\} \), these are both vector space. 
\end{eg}
\begin{remark}
    In the later course, we will learn that \(W_1\) is a subspace of \(V\).  
\end{remark}

\begin{eg}
    If we furtherly defined \(W_1^{(k)} = \left\{ \text{polynomial degree } \le k \right\} \), then this is also a vector space. 
\end{eg}

\begin{remark}
    \(W_1^{(k)}\) is actually isomorphic to \(\mathbb{R} ^{k+1}\) since
    \[
        a_0 + a_{1}x + a_2 x^2 + \dots + a_k x^k \leftrightarrow (a_0, a_1, a_2, \dots , a_n). 
    \]  
\end{remark}

\begin{eg}
    \(W_2 = \left\{ \text{continuous function on } (a,b) \right\} \) and \(W_3 = \left\{  \text{differentiable functions} \right\} \) are also both vector spaces.  
\end{eg}

\begin{eg}
    \(W_4 = \left\{ \frac{\mathrm{d}^2 f}{\mathrm{d}x^2} = 0  \right\} \) and \(W_5 = \left\{ \frac{\mathrm{d} ^2 f}{\mathrm{d}x ^2} = -f  \right\} \) are both vector spaces. 
\end{eg}
\begin{explanation}
    \begin{align*}
        W_4 &= \left\{ a_0 + a_1 x \right\} \\
        W_5 &= \left\{ a_1 \cos x + a_2 \sin x \right\} 
    \end{align*}
\end{explanation}

\section{Formal definition of vector spaces}
\subsection{Vector Spaces Over \(\mathbb{R} \) }
\begin{definition}
    Suppose \(V\) is a non-empty set equipped with 
    \begin{itemize}
        \item addition: \(V \times V \to V\), that is, given \(u,v \in V\), defining \(u + v \in V\)
        \item scalare multiplication: \(\mathbb{R}  \times V \to V\), that is, given \(\alpha \to \mathbb{R} \) and \(v \in V\), we need to have \(\alpha v \in V\)       
    \end{itemize} 
    Also, we need some good properties or conditions 
    \begin{itemize}
        \item For addition, 
        \begin{itemize}
            \item \(u + v = v + u\) 
            \item \((u + v) + w = u + (v + w)\)
        \end{itemize}
        \item There exists \(0 \in V\) such that \(u + 0 = u = 0 + u\)
        \item Given \(v \in V\), there exists \(-v \in V\) such that \(v + (-v) = 0 = (-v) + v\)
        \item For scalar multiplication,
        \begin{itemize}
            \item \(1 \cdot v = v\) for all \(v \in V\) 
            \item \((\alpha \beta ) v = \alpha \cdot (\beta v)\) for all \(\alpha , \beta \in \mathbb{R} \) and \(v \in V\).   
        \end{itemize}
        \item For addition and multiplication, 
        \begin{itemize}
            \item \(\alpha (u + v) = \alpha u + \alpha v\)
            \item \((\alpha + \beta )u = \alpha u + \beta u\)  
        \end{itemize}
    \end{itemize}

\end{definition}