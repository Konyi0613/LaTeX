\lecture{19}{14 Nov. 10:30}{}
Given \(T \in \mathcal{L} (V)\) or \(A \in M_n(F)\), we want to see the structures of \(T\) transparently, e.g. to compute \(A^k\).
\begin{itemize}
    \item [(i)] (constant) recursive sequence. Suppose \(S_0, S_1, \dots ,S_{n-1}\) is given, and 
    \[
        S_{k+n} = \alpha _0 S_k + \alpha _1 S_{k+1} + \dots + \alpha _{n-1} S_{k+n-1}.
    \]
    Let \(v_k = \begin{pmatrix}
         S_k  \\
         \vdots \\
         S_{k+n-1} \\
    \end{pmatrix}\), then 
    \[
        v_{k+1} = \begin{pmatrix}
            0 & 1 &  &  &   \\
             &  & 1 &  &   \\
             &  &  & \ddots &   \\
             &  &  &  & 1  \\
            \alpha _0 & \alpha _1 & \alpha _2 & \cdots & \alpha _{n-1}  \\
        \end{pmatrix} v_k.
    \] 
    \item [(ii)] (linear homogeneous) ODE. (with constant coefficient)
    \[
        y^{(n)} = \alpha _{n-1} y^{(n-1)} + \dots + \alpha _1 y^{\prime} + \alpha _0 y.
    \]
    Let \(f(x) = \begin{pmatrix}
         y \\
         y^{\prime}  \\
         \vdots \\
         y^{(n-1)} \\
    \end{pmatrix}\), then 
    \[
        f^{\prime} (x) = e^{Ax} \cdot C_{n \times 1},
    \] where 
    \[
        e^{Ax} = 1 + Ax + \dots + \frac{1}{k!} A^k x^k + \dots 
    \]
    In fact, each entry of \(A^k = O\left( d^k \right) \).
\end{itemize}    
 Now we can study \(v_{k+1} = Av_k\), \(f^{\prime} (x) = Af(x)\) for any \(A \in M_n(\mathbb{R} )\).    

 Now can we make \(T\) or \(A\) diagonal? Note that the entries in the diagonal must be eigenvalues, which are the roots of \(\mathrm{ch}_A(x) \). So it needs to split, say 
 \[
    \mathrm{ch}_T(x) = \sum_{i=1}^r \left( x - \lambda _i \right)^{m_i}, \quad \lambda _i \neq \lambda _j \text{ for } i \neq j.    
 \]   
Geometrically, see if \(\dim E(\lambda _i) = m_i\). (In general, g-mult \(\le\) a-mult)

Algebraically, see if 
\[
    \prod _{i=1}^r (T - \lambda _i) = 0.
\]
\subsubsection{What we can do?}
Decompose \(V\) into smaller/simpler pieces. Hence, we can use idempotenet deomposition: 
\[
    P_i P_j = \begin{dcases}
        P_i, &\text{ if } i=j ;\\
        0, &\text{ if }  i \neq j.
    \end{dcases}, \quad 1 = P_1 + \dots + P_n, \quad V = W_1 \oplus \dots \oplus W_n.
\] 