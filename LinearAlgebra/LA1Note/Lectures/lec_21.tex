\lecture{21}{26 Nov. 10:20}{}
\begin{remark}
    From now on, the note is not the lecture note since I didn't go to the lecture.
\end{remark}

\section{Cyclic Subspaces and Annihilators}
\begin{definition}
    If \(\alpha \) is a any vector in \(V\), the \(T\)-cyclic subspace generated by \(\alpha \) is the subspce \(Z(\alpha ; T)\) of all vectors of the form \(g(T)\alpha \), \(g \in F[x]\). If \(Z(\alpha ; T) = V\), then \(\alpha \) is called a cyclic vector for \(T\).          
\end{definition}

\begin{remark}
    Another way to describe \(Z(\alpha ; T)\) is that 
    \[
        Z(\alpha ; T) = \mathrm{span} \left\{ T^k \alpha  \right\}_{k \ge 0},  
    \] 
    and thus \(\alpha \) is a cyclic vector for \(T\) if and only if these vectors span \(V\). However, the general operator \(T\) has no cyclic vector.    
\end{remark}

\begin{definition}
    If \(\alpha \) is any vector in \(V\), the \(T\)-annihilator of \(\alpha \) is the ideal \(M(\alpha ; T)\) in \(F[x]\) consisting of all polynomials \(g\) over \(F\) such that \(g(T)\alpha = 0\). The unique moic polynomial \(p_\alpha \) which generates this ideal will also be called the \(T\)-annihilator of \(\alpha \).            
\end{definition}

\begin{remark}
    \(\deg p_\alpha > 0\) unless \(\alpha \) is the zero vector.  
\end{remark}

\begin{theorem}
    Let \(\alpha \) be any non-zero vector in \(V\) and let \(p_\alpha \) be the \(T\)-annihilator of \(\alpha \), then 
    \begin{itemize}
        \item [(i)] The degree of \(p_\alpha \) is equal to the dimension of the cyclic subspace \(Z(\alpha ; T)\). 
        \item [(ii)] If the degree of \(p_\alpha \) is \(k\), then the vectors \(\alpha, T;a, \dots , T^{k-1}\alpha  \) form a basis for \(Z(\alpha ; T)\). 
        \item [(iii)] If \(U\) is the linear operator on \(Z(\alpha ; T)\) induced by \(T\), then the minimal polynomial for \(U\) is \(p_\alpha \).           
    \end{itemize}     
\end{theorem}
\begin{proof}
    Let \(g \in F[x]\), then \(f = p_\alpha q + r\), where either \(r = 0\) or \(\deg(r) < \deg p_\alpha = k\). The polynomial \(p_\alpha q \in \mathrm{Ann}_T (\alpha ) \), so 
    \[
        g(T)\alpha = r(T) \alpha .
    \]      
    Since \(r = 0\) or \(\deg(r) < k\), the vector \(r(T)\alpha \) is a linear combination of the vectors \(\alpha , T \alpha , \dots T^{k-1} \alpha \), and thus 
    \[
        g(T) \alpha = r(T) \alpha \in \mathrm{span}\left\{ \alpha , T \alpha , \dots , T^{k-1} \alpha  \right\}.  
    \]   
    Since \(g\) can be arbitrary polynomial of \(F[x]\), so 
    \[
        Z(\alpha ; T) \subseteq \mathrm{span} \left\{ \alpha , T \alpha , \dots , T^{k-1} \alpha  \right\},
    \]
    and
    \[
        \mathrm{span} \left\{ \alpha , T \alpha , \dots , T^{k-1} \alpha  \right\} \subseteq Z(\alpha ; T)
    \] is trivial, so
    \[
        Z(\alpha ; T) = \mathrm{span} \left\{ \alpha , T \alpha , \dots , T^{k-1} \alpha  \right\}.
    \]
    Note that the set \(\left\{ \alpha , T \alpha , \dots , T^{k-1} \alpha  \right\} \) is linearly independent, otherwise if 
    \[
        \sum_{i=0}^{k-1} \beta _i T^i \alpha = 0, 
    \] where \(\beta _i\) are not all zeros, then 
    \[
        d(x) = \sum_{i=0}^{k-1} \beta _i x^i \in \mathrm{Ann}_T(\alpha ),  
    \] but \(\deg d \le k - 1 < k = \deg p_\alpha \), so this is impossible. Hence, 
    \[
        \left\{ \alpha , T \alpha , \dots , T^{k-1} \alpha  \right\} 
    \] is a basis of \(Z(\alpha ; T)\), and thus we showed (i) and (ii). Now we show (c). Note that 
    \[
        p_\alpha (U) \left( g(T) \alpha  \right) = p_\alpha (T) g(T) \alpha  = g(T) p_\alpha (T) \alpha = g(T) 0 = 0. 
    \]
    Hence, \(p_\alpha (U) = 0\). Now if \(h \in F[x]\) with \(\deg h < k\) and \(h(U) = 0\), then we know \(h(U) \alpha = h(T) \alpha = 0\), so \(h \in \mathrm{Ann}_T(\alpha ) \) and thus \(p_\alpha \mid h\), so this is impossible since \(\deg p_\alpha > \deg h\). Hence, \(\deg m_U \ge k\) and thus \(m_U = p_\alpha \).         
\end{proof}

\begin{corollary}
    If \(V = Z(\alpha ; T)\) and \(T \in \mathcal{L} (V)\), then \(\deg m_T(x) = \dim V\). Also, since \(\dim V = \deg \mathrm{ch}_T(x) \), so we have \(m_T(x) = \mathrm{ch}_T(x) \) since \(m_T(x) \mid \mathrm{ch}_T(x) \) and they are both monic.      
\end{corollary}

Now if \(U \in \mathcal{L} (W)\) where \(\dim W = k\) and \(W = Z(\alpha ; U)\), then we know 
\[
    \alpha , U \alpha , \dots , U^{k-1} \alpha 
\]  
form a basis for \(W\), and the annihilator \(p_\alpha \) of \(\alpha \) is \(m_U(x) = \mathrm{ch}_U(x) \) by the above corollary. If we let \(\alpha _i = U^{i-1} \alpha \) for \(i = 1, 2, \dots , k\), then suppose \(\mathcal{B} = \left\{ \alpha _1, \dots , \alpha _k \right\} \), we know the action of \(U\) on \(\mathcal{B} \) is 
\begin{align*}
    U \alpha _i &= \alpha _{i+1} \text{ for } i = 1, \dots , k - 1 \\
    U \alpha _k &= -c_0 \alpha _1 - c_1 \alpha _2 - \dots - c_{k-1} \alpha _k
\end{align*} 
where \(p_\alpha  = c_0 + c_1 x + \dots + c_{k-1} x^{k-1} + x^k\). The expression for \(U \alpha _k\) is true since \(p_\alpha (U) \alpha = 0\). Hence, 
\[
    [U]_{\mathcal{B} } = \begin{pmatrix}
        0 & 0 & \cdots & 0 & -c_0  \\
        1 & 0 & \cdots & 0 & -c_1  \\
        0 & 1 & \cdots & 0 & -c_2  \\
        \vdots & \vdots & \ddots & \vdots &  \vdots \\
        0 & 0 & \cdots & 1 & -c_{k-1}  \\
    \end{pmatrix},
\]            
which is called the companion matrix of the monic polynomial \(p_\alpha \). 

\begin{theorem}
    If \(U \in \mathcal{L} (W)\) where \(\dim W < \infty \), then \(U\) has a cyclic vector if and only if there is some ordered basis for \(W\) in which \(U\) is represented by the companion matrix of \(m_U(x)\).      
\end{theorem}