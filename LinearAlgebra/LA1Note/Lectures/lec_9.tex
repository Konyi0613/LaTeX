\lecture{9}{1 Oct. 10:20}{}
Consider the system 
\[
    \begin{dcases}
        a_{11} x_1 + \dots + a_{1n} x_n = y_1 \\
        \vdots \\
        a_{m1} x_1 + \dots + a_{mn} x_n = y_m.
    \end{dcases}
\]
We want to solve \(X\) s.t. \(AX = Y\), where \(A = (a_{ij})_{m \times n}\) and \(Y = (y_{i})_{i=1}^m\). Suppose \(P \in M_{m \times m}(F)\) invertible, then if \(B = PA\), we have \(BX=Z\), which means doing row operations on the system. In this case, we call two systems are equivalent. We also call \(A, B\) are row equivalent. 

Now we talk about the types of elementary row operations: 
\begin{itemize}
    \item [(i)] Replace \(i\)-th row with \(c \cdot r_i\) for some \(c \neq 0\). 
    \item [(ii)] Replace \(r_i\) with \(r_i + c r_j\) for some \(j \neq i\). 
    \item [(iii)] Interchange \(r_i\) and \(r_j\) for some \(i \neq j\).       
\end{itemize}

One can use (i) and (ii) in finite steps, making \(A\) into row reduced form(REF) of \(A\), which means   
\begin{itemize}
    \item first entry of a non-zero row is \(1\), we called it leading \(1\)
    \item entries below and above leading \(1\) are \(0\).   
\end{itemize}

If allowing (iii), we can make \(A\) into RREF(row reduced echelon form), which means REF and all zero rows are at the bottom.

Note that \(AX=Y\) gives \(PAX=PY\), so we can write \(P(A \mid Y) = (PA \mid PY)\). Hence, we can do row operations on \((X \mid Y)\) so that the \(X\) part becomes REF or RREF to solve the system. 
The system will be like 
\begin{align*}
    x_{k_1} + &\dots + 0 + \dots = z_1 \\
            &x_{k_2} + \dots + 0 = z_2 \\
    \vdots& 
\end{align*} 
Suppose for the first \(n\) rows, there are \(r\) non-zero rows. If there is some \(z_i \neq 0\) for \(i > r\), the system has no solution. If not, there is at least one solution, and there are \(n - r\) free variables. 

\begin{note}
    If \(n - r = 0\), then the system has unique solution, and if \(n - r > 0\), then it has infinitely many solutions.  
\end{note}

In the homogeneous case (i.e. \(y_1 = y_2 = \dots = y_m = 0\)), we find \(\nu (A) = n - r\). In this case, if \(n > m\), then \(n - r > m - r \ge 0\), so there are non-zero solutions to \(AX = 0\). 

\subsubsection{Some consequences: }
\begin{itemize}
    \item If \(A \in M_n(F)\), then TFAE 
    \begin{itemize}
        \item The system \(AX=0\) has only trivial solution (injective). 
        \item For any \(Y\), \(AX=Y\) has a (unique) solution (surjective). 
        \item \(A\) is invertible.     
    \end{itemize}
\end{itemize}

If \(P, Q\) are invertible, then \((PQ)^{-1} = Q^{-1} P^{-1}   \). Also, by above mentioned things, we know every invertible matrix is a product of many elementary matrix, that is, \(A = (E_1)^{-1} (E_2)^{-1} \dots (E_m)^{-1}\) since we know 
\[
    (E_m \dots E_2 E_1)A = I_m.
\]   
\begin{note}
    If \(A\) is invertible, then \(AX = 0\) has only trivial solution, then its RREF is \(I\), and thus \(A\) can be recovered to \(I\) by some row operations.    
\end{note}

\begin{prev}
    If \(\left\{ v_1, \dots , v_n \right\} \) is linearly independent and \(\left\{ w_1, \dots , w_m \right\} \) spans \(V\), then \(n \le m\).     
\end{prev}

Suppose \(x_1 v_1 + \dots + x_n v_n = 0\), where 
\[
    v_i = a_{1i}w_1 + a_{2i}w_2 + \dots + a_{mi} w_m,
\] then we have 
\[
    a_{i1} x_1 + \dots + a_{im} x_n = 0
\] for all \(1 \le i \le m\). If \(n > m\), then there exists a non-zero solution to this system, which contradicts to the fact that \(x_1 = x_2 = \dots = x_n = 0\).  

\begin{corollary}
    For \(A \in M_{m \times n}(F)\), we know there exists invertible \(P, Q\) s.t. 
    \[
        PAQ = \begin{bmatrix}
            I_r & 0  \\
            0 & 0  \\
        \end{bmatrix}.
    \]  
\end{corollary}

\begin{corollary}
    row rank is equal to col rank.
\end{corollary}

\begin{question}
    How to show \(A\) invertible? 
\end{question}
\begin{answer}
    Check RREF of \(A\) is \(I_n\) or not.  
\end{answer}

\begin{question}
    How to find \(A^{-1}\)? 
\end{question}
\begin{answer}
    Calculate \((A \mid I_n)\). 
\end{answer}

\chapter{Dual space}
Consider a vector space \(V\), and \(V\) is over a field \(F\), then we call 
\[
    V^* = \mathcal{L} (V, F).
\]   

\begin{definition}
    Suppose \(V\) is a vector space over \(F\) (with basis \(\left\{ 1 \right\} \)), then 
    \begin{itemize}
        \item A linear functional \(f\) is a linear map \(f: V \to F\). 
        \item \(V^* = \mathcal{L} (V, F)\) is called the dual space of \(V\).     
    \end{itemize}  
\end{definition}

\begin{eg}
    Suppose \(V = F^n\), then \(V^* = M_{1 \times n}(F)\).  
\end{eg}
Note that Suppose \(f \in V^*\) corresponds to \((a_1, a_2, \dots , a_n)\), then \(f(e_i) = a_i\). 

\begin{eg}
    Suppose \(V = M_{n \times n}(F)\), then the tract map 
    \[
        \mathrm{tr}: M_{n \times n}(F) \to F \quad (a_{ij}) \mapsto \sum_{i=1}^n a_{ii}  
    \] is in \(V^*\). 
\end{eg}

\begin{eg}
    We can define \(E_{pq}^* \in V^*\) by 
    \[
        E_{pq}^*((a_{ij})) = a_{pq},
    \] then \(\left\{ E_{ij}^* \right\} \) is a basis of \(V^*\).   
\end{eg}

\begin{eg}
    Suppose
    \[
        V = \left\{ \text{continuous function }f: [p, q] \to \mathbb{R}  \right\}, 
    \] then we can define \(\mathrm{ev}_s \), the evaluation at \(s\), by 
    \[
        \mathrm{ev}_s(f) = f(s),
    \] and we can define \(I: V \to \mathbb{R} \) with 
    \[
        I(f) = \int_p^q f(x) \, \mathrm{d} x, 
    \] then \(\mathrm{ev}_s \) and \(I\) are both elements of \(V^*\).   
\end{eg}