\lecture{14}{29 Oct. 10:20}{}
\begin{prev}
    There is a unique function 
    \[
        \det : M_n(R) \to R
    \] satisfying \(n\)-linear in rows, alternating, and \(\det (I_n) = 1\). Also, if \(D: M_n(R) \to R\) satisfies \(n\)-linear and alternating, then \(D(A) = D(I) \cdot \det (A)\). Besides, \(\det \) can be constructed inductively:
    \[
        \det (A) = \sum_{i=1}^n a_{ij} c_{ij} 
    \]  where \(c_{ij} = (-1)^{i + j} \det \left( A(i \mid j) \right) \) is the \((i, j)\)-cofactor. 
\end{prev}
If \(\sigma \in S_n\), and let \(\sigma (I) = (e_{\sigma (1)}, e_{\sigma (2)}, \dots , e_{\sigma (n)})\) (permuting the rows), then \(\det (\sigma (I)) = (-1)^m\) if \(\sigma = \tau _1 \tau _2 \dots \tau _m\) where \(\tau _i\) is a transposition since \(\det \) is alternating, so exchange two rows in the function input change the sign of the output. 

\begin{corollary}
    For \(\sigma \in S_n\), if \(\sigma = \tau _1 \tau _2 \dots \tau _p = \tau _1^{\prime} \tau _2^{\prime} \dots \tau _q^{\prime} \), then \(p\) and \(q\) are both even or both odds.    
\end{corollary}

\begin{definition}
    \(\sigma \in S_n\) is called an even(resp. odd) permutation if \(\sigma = \tau _1 \tau _2 \dots \tau _m\) for \(m\) even(resp. odd). Thus, we can define 
    \[
        \sgn : S_n \to \left\{ \pm 1 \right\}, \quad \sigma \mapsto \det (\sigma (I)). 
    \]   
\end{definition}

Hence, we can give a second method to construct \(\det \):
\[
    \det \left( (a_{ij})_{n \times n} \right) = \sum_{\sigma \in S_n} \sgn (\sigma ) \prod _{i=1}^n a_{i, \sigma (i)}.  
\] 

\begin{eg}
    If we want to calculate
    \[
        \det \begin{pmatrix}
            0 & 0 &  & a_n  \\
            a_1 & 0 &  &  0 \\
             &  & \ddots &   \\
            0 & \cdots & a_{n-1} & 0  \\
        \end{pmatrix},
    \] then we have two ways:
    \begin{itemize}
        \item [(1)] expand along the last column. 
        \item [(2)] Suppose \(A = (a_{ij})_{n \times n}\), where \(a_{ii} = a_i\) for all \(i\) and \(a_{ij} = 0\) for all \(i \neq j\), then \(\det A = a_1 a_2 \dots a_n\), and the matrix given in the problem is from exchanging first row and second row of \(A\), then exchange second row and third row, and keep going until exchanging the \(n-1\)-th row and \(n\)-th row, so the answer is \((-1)^{n-1}a_1 a_2 \dots a_n\) since it takes \(n-1\) times exchangement. (exchange rows in the inoput of an alternating function will change the sign of output.)         
    \end{itemize}
\end{eg}

\begin{eg}
    Companion form of \(f(x) = x^n + a_1 x^{n-1} + \dots + a_n\):
    \[
        A_f = \begin{pmatrix}
            0 & 0 & \cdots & -a_n  \\
            1 & 0 & \cdots & -a_{n-1}  \\
            \vdots & \vdots & \ddots &  \vdots \\
            0 & \cdots & 1 & -a_1  \\
        \end{pmatrix}.
    \] We can calculate \(\det (A_f + xI) = f(x)\). 
\end{eg}

\begin{theorem}
    Suppose \(A, B \in M_n(R)\), where \(R\) is a ring with identity, then 
    \[
        \det (AB) = \det (A) \det (B).
    \] Thus, we have \(\det (P^{-1}) = \det (P)^{-1}\). 
\end{theorem}
\begin{proof}
    Let \(D(A) = \det (AB)\), then we can check that \(D\) satisfies \(n\)-linear and alternating. If this were true, then \(D(A) = D(I) \det (A)\), and \(D(I) = \det (IB) = \det (B)\), so \(D(A) = \det (A) \det (B)\) and thus we have 
    \[
        \det (AB) = \det (A) \det (B).
    \]    
    \begin{note}
        Note that 
        \[
            D \begin{pmatrix}
                 u_1 \\
                 \vdots \\
                 v + \alpha w \\
                 \vdots \\
                 u_n
            \end{pmatrix} = \det \left( \begin{pmatrix}
                 u_1 \\
                 \vdots \\
                 v + \alpha w \\
                 \vdots \\
                 u_n
            \end{pmatrix} B \right) = \det \left( \begin{pmatrix}
                 u_1 B  \\
                 \vdots \\
                 vB + \alpha w B \\
                 \vdots \\
                 u_n B \\
            \end{pmatrix} \right) = D \begin{pmatrix}
                 u_1 \\
                 \vdots \\
                 v \\
                 \vdots \\
                 u_n \\
            \end{pmatrix} + \alpha D \begin{pmatrix}
                 u_1 \\
                 \vdots \\
                 w \\
                 \vdots \\
                 u_n \\
            \end{pmatrix},
        \] and alternating can be proved similarly.
    \end{note}
\end{proof}

\begin{theorem}
    If \(A \sim B\), then \(\det A = \det B\).  
\end{theorem}

\begin{theorem}
    \(\det A^t = \det A\). 
\end{theorem}
\begin{proof}
Note that 
\[
    a_{1 \sigma (1)} \dots a_{n \sigma (n)} = a_{\sigma ^{-1}(1), 1} \dots a_{\sigma ^{-1}(n), n},
\]
and \(\sgn (\sigma ) = \sgn \left( \sigma ^{-1} \right)  \).Hence, if we suppose \(B = A^t\), then
\begin{align*}
    \det (B) &= \sum_{\sigma} \sgn (\sigma ) \prod b_{i, \sigma (i)} \\
    &= \sum_{\sigma } \sgn (\sigma ) \prod a_{\sigma (i), i} \\
    &= \sum_{\tau: \tau = \sigma ^{-1} } \sgn (\tau ) \prod a_{i, \tau (i)} = \det (A).  
\end{align*}   
\end{proof}

\begin{exercise}
    Show that 
    \[
        \det \begin{pmatrix}
            A & B  \\
            0 & D  \\
        \end{pmatrix} = \det (A) \det (D).
    \]
\end{exercise}


\begin{theorem}
    Let \(A \in M_n(R)\), then we can define the (classical) adjoint 
    \[
        \mathrm{adj}(A) = \widetilde{A} = \left( \widetilde{a_{ij}}  \right) , 
    \] where 
    \[
        \widetilde{a_{ij} } = (j, i)\text{-cofactor } c_{j,i} = (-1)^{i+j} \det \left( A(j \mid i) \right),   
    \] then \(A \widetilde{A} = \widetilde{A} A = \det (A) I\). This means if \(A\) is invertible, then \(A^{-1} = \frac{1}{\det (A)} \widetilde{A} \).  
\end{theorem}
\begin{proof}
    Note that the \((i, i)\)-entry of \(A \widetilde{A} \) is 
    \begin{align*}
        \sum_{k=1}^n a_{ik} \widetilde{a_{ki}} = \sum_{k=1}^n (-1)^{i+k}a_{ik} \det (A(i \mid k)) = \det (A),   
    \end{align*}
    while the \((i, j)\)-entry for \(i \neq j\) is 
    \begin{align*}
        \sum_{k=1}^n a_{ik} \widetilde{a_{kj}} &= \sum_{k=1}^n (-1)^{j+k} a_{ik} \det \left( A(j \mid k) \right) \\
        &= \det \begin{pmatrix}
             &  &  &   \\
             &  &  &   \\
            a_{i1} & a_{i2} & \dots  & a_{in}  \\
             &  &  &    \\
             &  &  &   \\
        \end{pmatrix} (j\text{-th row}) = 0
    \end{align*}  
    since \(\det \) is alternating. Thus, \(A \widetilde{A} = \det (A) I\). Similarly, we can show \(\widetilde{A} A = \det (A) I\).  
\end{proof}

\begin{theorem}
    Suppose \(A \in M_n(F)\) is invertible, then consider the system
    \[
        A \begin{pmatrix}
             x_1 \\
             \vdots \\
             x_n \\
        \end{pmatrix} = \begin{pmatrix}
             c_1 \\
             \vdots \\
             c_n \\
        \end{pmatrix},
    \] then \( x_i = \frac{1}{\det (A)} \det (C_i)\), where \(C_i\) is the matrix \(A\) but replace the \(i\)-th column with \((c_1, c_2, \dots , c_n)^t\).    
\end{theorem}
\begin{proof}
    In fact,
    \[
        \begin{pmatrix}
             x_1 \\
             \vdots \\
             x_n \\
        \end{pmatrix} = \frac{1}{\det (A)} \widetilde{A} \begin{pmatrix}
             c_1 \\
             \vdots \\
             c_n \\
        \end{pmatrix},
    \] and by comparing the entries, we know 
    \[
        \det (A) x_i = \sum_{j=1}^n (-1)^{i+j} c_j \det (A(j \mid i)) = \det (C_i). 
    \]
\end{proof}

\begin{exercise}
    If \(v_1, v_2, \dots , v_n \in \mathbb{R} ^n\), then 
    \[
        \det (v_1, v_2, \dots , v_n) = \pm \text{volumn}. 
    \] 
\end{exercise}

\begin{definition}
    For finite dimensional vector space \(V\), suppose \(T \in \mathcal{L} (V)\), then one can define \(\det (T)\) by choosing an ordered basis \(B\) of \(V\), and define 
    \[
        \det (T) \coloneqq \det \left( [T]_B \right). 
    \]     
\end{definition}

\begin{remark}
    This \(\det (T)\) does not depend on the choice of \(B\) since
    \[
        [T]_B \sim [T]_{B^{\prime} }
    \]  for any two basis \(B, B^{\prime} \) of \(V\). This is because
    \[
        [T]_{B^{\prime} } = [id]_{B^{\prime} }^B [T]_B [id]_B^{B^{\prime} }.
    \]  
\end{remark}