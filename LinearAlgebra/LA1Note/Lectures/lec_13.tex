\lecture{13}{17 Oct. 10:20}{}
Actually determinant can be defined on ring (we defined it on field before). 
\begin{theorem}
    There is the determinant function 
    \[
        \det : M_n(R) \to R.
    \]
\end{theorem}

Now we talk more about expansion. We do expansion along a column. 
Suppose we have 
\[
    \delta : M_{n-1}(R) \to R,
\]which is \((n-1)\)-linear and alternating and \(\delta (I_{n-1}) = 1\), then if we define \(D_j = D: M_n(R) \to R\), which is the expansion along the \(j\)-th column, and it has 
\[
    D(A = (a_{ij})) = \sum_{i=1}^n (-1)^{i+j} a_{ij} \delta \left( A(i \mid j) \right).  
\] 

\begin{note}
    \(C_{ij} = (-1)^{i+j} \delta \left( A(i \mid j) \right)  \) is called the \((i, j)\)-cofactor. 
\end{note}

\begin{theorem}
    \(D\) is \(n\)-linear and alternating, and \(D(I_n) = 1\).   
\end{theorem}
\begin{proof}
    \todo{DIY}
\end{proof}
\begin{note}
    In the proof of alternating, we may need to use \autoref{lm: alternating property}. 
\end{note}
\begin{note}
    We still regard \(D\) as a function taking \(n\) row vectors as its input.
\end{note}

\begin{prev}
    If \(D: M_n(R) \to R\) is \(n\)-linear, alternating, then 
    \[
        D \left( (a_{ij}) \right) = \sum_{\sigma } \prod _{i} a_{1 \sigma (i)} D \begin{pmatrix}
             e_{\sigma (1)} \\
             \vdots \\
             e_{\sigma (n)} \\
        \end{pmatrix}  
    \]  
\end{prev}

Now we put things together: 
\begin{theorem}
    \vphantom{text}
    \begin{itemize}
        \item [(i)] There is a function \(\det : M_n(R) \to R\) satisfying \(n\)-linear, alternating, and \(\det (I_n) = 1\). 
        \item [(ii)] If \(D: M_n(R) \to R\) is \(n\)-linear, alternating, then \(D(A) = D(I) \cdot \det (A)\).
        \item [(iii)] For a permutation \(\sigma \), if \(\sigma = t_1 t_2\dots t_n = t_1^{\prime} t_2^{\prime} \dots t_m^{\prime} \), where \(t_i, t_i^{\prime} \)'s are transpositions, then \((-1)^n = (-1)^m\).        
    \end{itemize}
\end{theorem}

\begin{remark}
    (ii) needs the fact that 
    \[
        D \left( e_{\sigma (1)}, e_{\sigma (2)}, \dots , e_{\sigma (n)} \right) = (-1)^m D(e_1, e_2, \dots , e_n) 
    \] if \(\sigma \) is the composition of \(m\) traspositions.  
\end{remark}

\begin{remark}
    (i) and (ii) hold for any \(R\). 
\end{remark}

Now we introduce two formulas: 
\begin{itemize}
    \item [(1)] 
    \[
        \det (A) = \sum_{i=1}^n (-1)^{i+j} a_{ij} \det \left( A(i \mid j) \right).  
    \]
    \item [(2)] 
    \[
        \sgn : \left\{ \text{permutation}  \right\} \to \left\{ \pm 1 \right\}, \quad \sigma \mapsto (-1)^m   
    \] if \(\sigma = t_1 t_2 \dots t_m\) if \(t_i\)'s are transpositions.  
\end{itemize}

Thus, we know 
\[
    \det (A) = \sum_{\sigma } \sgn (\sigma ) a_{1 \sigma (1)} a_{2 \sigma (2)} \dots a_{n \sigma (n)}. 
\]