\lecture{6}{19 Sep. 10:20}{}
\begin{prev}
    \(T\) is called an isomorphism if \(T\) is both injective and surjective.  
\end{prev}
\begin{proposition} \label{prop: isomorphism TFAE}
    Suppose \(\dim V = \dim W = n\), then TFAE 
    \begin{itemize}
        \item [(i)] \(T\) is an isomorphism. 
        \item [(ii)] \(T\) is injective.  
        \item [(iii)] \(T\) is surjective.  
        \item [(iv)] \(T\) sends any basis of \(V\) to a basis of \(W\).   
        \item [(v)] \(T\) sends one basis to a basis. 
    \end{itemize} 
\end{proposition}

\begin{eg}
    Suppose \(A \in M_{m \times n}(F)\), say \(A = (c_1, c_2, \dots , c_n)\), then \(T_A\) is injective if and only if \(\left\{ c_1, \dots , c_n \right\} \) is linearly independent. (which means \(n \le m\)).
\end{eg}
\begin{explanation}
    Since \(T_A(e_i) = c_i\) and \(\left\{ e_i \right\}_{i=1}^n \) forms a basis. 
\end{explanation}

\begin{eg}
    Following the last example, \(T_A\) is surjective if and only if \(\left\{ c_1, c_2, \dots , c_n \right\} \) spans \(W\). (which means \(n \ge m\)).   
\end{eg}

\section{Space of linear maps}
Consider 
\[
    \left\{ f:V \to W \right\}, 
\] and then we can define addition and multiplication by 
\[
    (f + g)(v) = f(v) + g(v) \quad (\alpha\cdot f)(v) = \alpha f(v).
\]
Hence, we know it is a vector space. Now if we collect all linear maps, say 
\[
    \mathcal{L} (V, W) = \left\{ \text{linear } T: V \to W \right\}. 
\]
Observe that \(\mathcal{L} (V, W)\) is a vector space since we can similarly define the addition and multiplication. 

Now if we have \(U,V,W\), three vector spaces, and \(f: U \to V\) is a linear map, then if we define a map 
\begin{align*}
    R_f: \mathcal{L} (V, W) &\to \mathcal{L} (U, W) \\
    T &\mapsto T \circ f,
\end{align*} 
then this map is linear. Similarly,
\begin{align*}
    L_f: \mathcal{L} (W, U) &\to \mathcal{L} (W, V) \\
    T &\mapsto f \circ T,
\end{align*}
then this is also a linear map.
\begin{note}
    We just need to check something like 
    \[
        R_f(T+S) = R_f(T) + R_f(S) \quad R_f(\alpha T) = \alpha R_f(T).
    \]
\end{note}
Now if we consider 
\begin{align*}
    \mathcal{L} (V, W) \times \mathcal{L} (U, V) &\to \mathcal{L} (U, W) \\
    (T, S) \mapsto T \circ S,
\end{align*}
then this is also a linear map.

\begin{eg} \label{eg: matrix is linear map}
    \(\mathcal{L} (F^n, F^m) = M_{m \times n}(F)\). 
    
\end{eg}
\begin{explanation}
    Check that 
    \[
        T_A + T_B = T_{A + B}.
    \]
        \begin{note}
        More precisely, they are isomorphic, that is, \(\mathcal{L} (F^n, F^m) \cong M_{m \times n}(F)\). 
    \end{note}
\end{explanation}

\begin{eg}
    Consider 
    \[
        \mathcal{L}(F^n, F^m) \times \mathcal{L} (F^p, F^n) \to \mathcal{L} (F^p, F^m),
    \] we know this is a linear map, and by \autoref{eg: matrix is linear map}, we know 
    \[
        M_{m \times n}(F) \times M_{n \times p}(F) \to M_{m \times p}(F)
    \] is a linear map.
\end{eg}
\begin{explanation}
    Check 
    \[
        (T_A \circ T_B)(v) = T_{AB}(v) \iff A(Bv) = (AB)(v).
    \]
\end{explanation}

\begin{definition}
    We call 
    \[
        V \cong F^n
    \] a basic isomorphisms if \(\dim V = n\). 
\end{definition}

\begin{corollary}
    \(\mathcal{L} (F^n, F^m) \cong M_{m \times n}(F)\). 
\end{corollary}
\begin{remark}
    If you change \(F^n\) to \(V\) and \(F^m\) to \(W\), then this is also correct since \(F^n \cong V\) and \(F^m \cong W\). (We suppose \(\dim V = n\) and \(\dim W = m\).)      
\end{remark}