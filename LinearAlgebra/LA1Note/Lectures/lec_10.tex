\lecture{10}{3 Oct. 10:20}{}
\begin{definition}
    \(A, B \in M_n(F)\) are called similar or \(A \sim B\) iff \(B = P^{-1} A P \).    
\end{definition}

\begin{notation}
    We call \(\mathcal{L} (V, F)\)
    \[
        V^* \quad \text{or} \quad V^{\vee} \quad \text{or} \quad V^t.
    \] 
\end{notation}

\begin{theorem}
    \[
        \dim V = \dim V^*.
    \]
\end{theorem}
\begin{proof}[Matrix relation proof]
    Since \(V^* \simeq M_{1 \times n}(F)\), where \(n = \dim V\), so 
    \[
        \dim V^* = \dim M_{1 \times n}(F) = n = \dim V. 
    \]  
\end{proof}
\begin{proof}
    Suppose \(B = \left\{ v_1, v_2, \dots , v_n \right\} \) is a basis of \(V\), and define \(B^* = \left\{ v_1^*, v_2^*, \dots , v_n^* \right\} \) as 
    \[
        v_i^* \left( v_j \right) = \delta _{ij}. 
    \] Note that \(v_i^* \in \mathcal{L} (V, F)\) for all \(i\). Note that for all \(v = \sum_{i=1}^{n} \alpha _i v_i\), we have 
    \[
        v_i^* \left( v \right) = \alpha _i. 
    \] Check \(B^*\) is linearly independent. Suppose \(f = \sum \alpha _i v_i^* = 0 \), then we know \(f(v_j) = \alpha _j = 0\) for all \(j\). Also, note that \(B^*\) spans \(V^*\).       
\end{proof}

\begin{remark} \label{rmk: coordinate by dual basis}
    \[
        [v]_B = \begin{pmatrix}
             v_1^*(v) \\
              \vdots \\
              v_n^*(v) \\
        \end{pmatrix}
    \]
\end{remark}

\begin{eg}
    Suppose \(V = F^2\) and \(B = \left\{ e_1, e_2 \right\} \), then \(V^*\) is identified with 
    \[
        \mathcal{L} \left( F^2, F \right) = M_{1 \times 2}(F), 
    \] where \(B^* = \left\{ e_1^*, e_2^* \right\} \) with 
    \[
        e_1^* = (1, 0) \quad e_2^* = (0, 1).
    \] 
\end{eg}

Now if we know \(T: V \to W\) is a linear map, then we can define \(T^*: W^* \to V^*\) by 
\[
    T^*: f \mapsto f \circ T,
\] and we called it the transpose of \(T\). We will show that if \([T]_C^B \equiv = M\), then \(\left[ T^* \right]^{C^*}_{B^*} = N = M^t\), which means if \(M = (m_{ij})_{m \times n}\) and \(N = (n_{ij})_{n \times m}\), then \(n_{ij} = m_{ji}\) for all \(i, j\) with \(1 \le i \le n\) and \(1 \le j \le m\). 
\begin{proof}
    Suppose \(T^* \left( w_j^* \right) = \sum_{p=1}^{n} n_{pj} v_p^*  \), then since
    \[
        w_j^* \left( T(v_j) \right) = w_j^* \left( \sum_{q=1}^{m} m_{qi} w_q  \right) = m_{ji},
    \] so \(n_{ij} = m_{ji}\). (See \autoref{rmk: coordinate by dual basis}) Note that the below one is the evaluation of the above equation at \(v_j\).  
\end{proof}         