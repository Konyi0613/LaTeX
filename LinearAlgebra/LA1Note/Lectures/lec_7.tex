\lecture{7}{24 Sep. 10:20}{}
% https://q.uiver.app/#q=WzAsNyxbMiwzLCJWIl0sWzUsMywiVyJdLFsyLDAsIlUiXSxbMywyXSxbMCw0LCJcXHRleHR7bGluZWFyfSJdLFsyLDJdLFszLDNdLFswLDEsIlQiLDJdLFsyLDAsImYiLDJdLFsyLDEsIlQgXFxjaXJjIEYiXSxbMywzXSxbNCw3LCIiLDAseyJvZmZzZXQiOjIsImN1cnZlIjoyLCJzaG9ydGVuIjp7InRhcmdldCI6MTB9LCJsZXZlbCI6MX1dLFs0LDgsIiIsMix7Im9mZnNldCI6LTEsImN1cnZlIjotMiwic2hvcnRlbiI6eyJ0YXJnZXQiOjEwfSwibGV2ZWwiOjF9XV0=
\[\begin{tikzcd}
	&& U \\
	\\
	&& {} & {} \\
	&& V & {} && W \\
	{\text{linear}}
	\arrow[""{name=0, anchor=center, inner sep=0}, "f"', from=1-3, to=4-3]
	\arrow["{T \circ F}", from=1-3, to=4-6]
	\arrow[from=3-4, to=3-4, loop, in=55, out=125, distance=10mm]
	\arrow[""{name=1, anchor=center, inner sep=0}, "T"', from=4-3, to=4-6]
	\arrow[shift left, curve={height=-12pt}, between={0}{0.9}, from=5-1, to=0]
	\arrow[shift right=2, curve={height=12pt}, between={0}{0.9}, from=5-1, to=1]
\end{tikzcd}\]

There is a special case, 
\[
    \mathcal{L} (V, V) \coloneqq \mathcal{L} (V) = \left\{ \text{linear } T:V \to V \right\} ,
\]
which is the space of linear operators on \(V\).

Now consider linear \(T_A: F^n \to F^m, T_B: F^p \to F^m\), then we can define a map \(T_{AB} = T_A \circ T_B\), and it will be a linear map.  

% https://q.uiver.app/#q=WzAsMyxbMCwwLCJGXnAiXSxbMCwyLCJGXm4iXSxbMiwyLCJGXm0iXSxbMCwxLCJUX0IiLDJdLFsxLDIsIlRfQSIsMl0sWzAsMiwiVF9BIFxcY2lyYyBUX0IgPSBUX3tBQn0iXV0=
\[\begin{tikzcd}
	{F^p} \\
	\\
	{F^n} && {F^m}
	\arrow["{T_B}"', from=1-1, to=3-1]
	\arrow["{T_A \circ T_B = T_{AB}}", from=1-1, to=3-3]
	\arrow["{T_A}"', from=3-1, to=3-3]
\end{tikzcd}\]

Also, note that \(T_A, T_B\) corresponds to two matrices \(A, B\), respectively, and it turns out that \(T_{AB}\) corresponds to the matrix \(AB\). (Check)

Hence, \(\mathcal{L} \left( F^n \right) = M_n(F) \). 

A matrix \(P\) is called invertible if \(T_P\) is bijective. In this case,
% https://q.uiver.app/#q=WzAsMixbMCwwLCJcXExhcmdlIEZebiJdLFsyLDAsIlxcTGFyZ2UgRl5tIl0sWzAsMSwiVF9wIl0sWzEsMCwiVF9RIiwwLHsib2Zmc2V0IjotM31dXQ==
\[\begin{tikzcd}
	{\Large F^n} && {\Large F^m}
	\arrow["{T_p}", from=1-1, to=1-3]
	\arrow["{T_Q}", shift left=3, from=1-3, to=1-1]
\end{tikzcd}\]

Hence, there exists \(Q \in M_n(F)\) s.t. \(QP = PQ = I_n\) since we know \(T_P \circ T_Q = T_Q \circ T_P = I\). 

Thus, we have 
\[
    P = (c_1, c_2, \dots , c_n) \text{ invertible} \iff \left\{ c_1, \dots , c_n \right\} \text{ is a basis.} 
\]
by \autoref{prop: isomorphism TFAE}. 

\section{Map/matrix correspondence}
%https://q.uiver.app/#q=WzAsNSxbMCwwLCJWIl0sWzIsMCwiVyJdLFswLDIsIkZebiJdLFsyLDIsIkZebSJdLFsxLDFdLFswLDIsIltcXGNkb3RdX0IiLDJdLFswLDEsIlQiXSxbMSwzLCJbXFxjZG90XV97Qid9Il0sWzIsMywiXFx0ZXh0e1doYXQgaXMgdGhpcz99IiwyXSxbNCw0LCIiLDIseyJvZmZzZXQiOjV9XV0=
\[\begin{tikzcd}[sep=huge]
	V && W \\
	& {} \\
	{F^n} && {F^m}
	\arrow["T", from=1-1, to=1-3]
	\arrow["{[\cdot]_B}"', from=1-1, to=3-1]
	\arrow["{[\cdot]_{B'}}", from=1-3, to=3-3]
	\arrow[shift right=5, from=2-2, to=2-2, loop, in=55, out=125, distance=10mm]
	\arrow["{\text{What is this?}}"', from=3-1, to=3-3]
\end{tikzcd}\]

Take an ordered basis \(B = \left\{ v_1, v_2, \dots , v_n \right\} \) and \(B^{\prime} = \left\{ w_1, \dots , w_m \right\} \), and says 
\[
    T(v_j) = \sum_{i=1}^m \alpha _{ij} w_i \mapsto \begin{pmatrix}
         \alpha _{1j} \\
          \vdots \\
          \alpha _{mj}\\
    \end{pmatrix}. 
\]  

Now consider the matrix 
\[
    A = (\alpha _{ij}) = \left( [T(v_1)]_{B^{\prime} }, [T(v_2)]_{B^{\prime} }, \dots  \right) ,
\] and then we called \(A\) the martix of \(T\) relative to \(B\) and \(B^{\prime} \). (matrix representative of \(T\)), and we denote this by \([T]_{B^{\prime} }^B\). 
\begin{theorem}
    \[
        [T(v)]_{B^{\prime} } = [T]_{B^{\prime} }^B [v]_B.
    \]
\end{theorem}     

\begin{theorem} 
    We have \([\cdot]_{B^{\prime} }^B : \mathcal{L} (V, W) \to M_{m \times n}(F)\), ans this matrix representative \([\cdot]_{B^{\prime} }^B\) is an isomorphism, which means 
    \begin{itemize}
        \item \([T + S]_{B^{\prime} }^B = [T]_{B^{\prime} }^B + [S]_{B^{\prime} }^B\).
        \item It is bijective.  
    \end{itemize} 
\end{theorem}  

\begin{corollary}
    if \(\dim V = n\) and \(\dim W = m\), then   
    \[
        \dim (\mathcal{L} (V, W)) = \dim V \cdot \dim W.
    \]
\end{corollary}

\begin{theorem}
    \[
        [T]_{B^{\prime} }^B [S]_{B}^{B^{\prime\prime} } = [T \circ S]_{B^{\prime} }^{B^{\prime\prime} }.
    \]
\end{theorem}

% https://q.uiver.app/#q=WzAsOCxbMSwwLCJWIl0sWzEsMywiRl5uIl0sWzQsMCwiVyJdLFs0LDMsIkZebSJdLFswLDAsInZfaiJdLFswLDQsImVfaiJdLFs1LDQsImNfaj0oXFxhbHBoYV97MWp9LCBcXGNkb3RzLFxcYWxwaGFfe21qfSledCJdLFs1LDAsIlxcc3VtX3tpPTF9Xm4gXFxhbHBoYV97aWp9d19pIl0sWzAsMV0sWzEsM10sWzAsMl0sWzIsM10sWzQsNSwiIiwyLHsic3R5bGUiOnsidGFpbCI6eyJuYW1lIjoiYXJyb3doZWFkIn19fV0sWzUsNiwiIiwyLHsic3R5bGUiOnsidGFpbCI6eyJuYW1lIjoibWFwcyB0byJ9fX1dLFs2LDcsIiIsMix7InN0eWxlIjp7InRhaWwiOnsibmFtZSI6ImFycm93aGVhZCJ9fX1dXQ==
\[\begin{tikzcd}
	{v_j} & V &&& W & {\sum_{i=1}^n \alpha_{ij}w_i} \\
	\\
	\\
	& {F^n} &&& {F^m} \\
	{e_j} &&&&& {c_j=(\alpha_{1j}, \cdots,\alpha_{mj})^t}
	\arrow[tail reversed, from=1-1, to=5-1]
	\arrow[from=1-2, to=1-5]
	\arrow[from=1-2, to=4-2]
	\arrow[from=1-5, to=4-5]
	\arrow[from=4-2, to=4-5]
	\arrow[maps to, from=5-1, to=5-6]
	\arrow[tail reversed, from=5-6, to=1-6]
\end{tikzcd}\]

Special case: 
\[
    \mathcal{L} (V) \to M_n(F).
\]
Take an ordered basis \(B = \left\{ v_1, \dots , v_n \right\} \). If \(T \in \mathcal{L} (V)\), then we can define \([T]_B = [T]_B^B\). 

\begin{corollary} \label{cl: any T can be written as I 0 0 0 by two bases transformation}
    Given \(T: V \to W\). There are \(B = \left\{ v_1, \dots , v_n \right\} \) and \(B^{\prime} = \left\{ w_1, \dots , w_m \right\} \) where \(B\) is a basis of \(V\) and \(B^{\prime} \) is a basis of \(W\) and 
    \[
        [T]_{B^{\prime} }^B = \begin{pmatrix}
            I_p & 0  \\
            0 & 0  \\
        \end{pmatrix},
    \] where \(p = \rank (T)\).         
\end{corollary}
\begin{proof}
    We can let \(B = \left\{ v_1, \dots , v_r , v_{r + 1}, \dots , v_n \right\} \), where \(\left\{ v_{r+1}, \dots , v_n \right\} \) is a basis of \(\ker T\) and \(T(v_1), \dots , T(v_r)\) is a basis of \(\Im (T)\), (Recall the proof in \autoref{thm: rank and nullity theorem}), then we can let \(B^{\prime} = \left\{ T(v_1), \dots , T(v_r), \dots  \right\} \).       
\end{proof}

\begin{eg}
    Suppose \(V = \left\{ \text{polynomials with degree} \le k \right\} \) and \(W\) is the space of polynomials with degree \(\le k + 1\), then if \(T: V \to W\) and \(p(x) \mapsto \int_{0}^{x} p(t) \, \mathrm{d} t \), then we know an ordered basis \(B = \left\{ 1, x, x^2,\dots , x^k  \right\} \) and \(B^{\prime} = \left\{ 1, x, x^2, \dots , x^{k+1} \right\} \), and then 
    \[
        [T]_{B^{\prime} }^B = \begin{pmatrix}
            0 & 0 &  &  &   \\
            1 & 0 &  &  &   \\
            0 & \frac{1}{2} &  &  &   \\
            \vdots  & 0 & \ddots &  & 0  \\
            0 & 0 &  &  &  \frac{1}{k+1} \\
        \end{pmatrix}.
    \]   
\end{eg}

\begin{eg}
    Suppose \(V\) is the space of polynomials of degree \(\le k\), and \(B = \left\{ 1, x, x^j, \dots , x^k \right\} \), and \(B^{\prime} = \left\{ 1, y, y^2, \dots , y^k \right\} \) with \(y = x-1\). Then, if \(T\) is the identity transformation, note that 
    \[
        x^j = (y + 1)^j = 1 + j \cdot y + \binom{j}{2} y^2 + \dots + \binom{j}{j} y^j. 
    \]
    Hence, we have 
    \[
        [T]_{B^{\prime} }^B = \left( \begin{array}{ccccc}
             \binom{0}{0} & \binom{1}{0} & \binom{2}{0} &  &   \\
             0& \binom{1}{1} & \binom{2}{1} &  &   \\
             0& 0 & \binom{2}{2} & &  \\
             \vdots& \vdots &  & \ddots  &  \\
             0& 0 &  &  &   \\
        \end{array} \right) 
    \]     
\end{eg}

\begin{question}
    Given \(V\), and \(B, B^{\prime} \) are ordered basis, then what is the relation between \([v]_B\) and \([v]_{B^{\prime} }\)?    
\end{question}
\begin{answer}
    Change of bases.
\end{answer}

\begin{corollary}
    \[
        [id]_{B^{\prime} }^{B} [v]_B = [v]_{B^{\prime} }.
    \]
\end{corollary}

\begin{corollary}
    \[
        [id]_{B^{\prime} }^{B}[id]_{B}^{B^{\prime} } = [id]_{B^{\prime} }^{B^{\prime}} .
    \]
\end{corollary}


\begin{corollary}
    Given any \(A \in M_{m \times n}(F)\). There are invertible matrices \(P \in M_m(F)\) and \(Q \in M_n(F)\) s.t. 
    \[
        PAQ = \begin{pmatrix}
            I_p & 0  \\
            0 & 0  \\
        \end{pmatrix},
    \] where \(p\) is the row rank of \(A\).   
\end{corollary}
\begin{proof}
   Suppose \(A = [T]_{B}^{B^{\prime} }\), and by \autoref{cl: any T can be written as I 0 0 0 by two bases transformation}, we know there exists \(b, b^{\prime} \) s.t. \([T]_{b}^{b^{\prime} }\) is the matrix we want, then we can let \(Q = [id]_{b^{\prime} }^{B^{\prime} }\) and \(P = [id]_{b}^{B}\), and we're done.       
\end{proof}