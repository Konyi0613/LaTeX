\lecture{11}{8 Oct. 10:20}{}
\begin{definition}[Annihilator] \label{def: annihilator}
    Let \(S \subseteq V\) be a subset, then the annihilator \(S^0 \subseteq V^*\) is the subset defined by 
    \[
        \left\{ f \in V^* \mid f(x) = 0 \quad \forall x \in V\right\}. 
    \]  
\end{definition}

\begin{proposition}
    For all \(S \subseteq V\), \(S^0 \) is a subspace of \(V^*\).  
\end{proposition}
\begin{proof}
    For all \(f, g \in S^0\), we know 
    \[
        (cf + g)(x) = cf(x) + g(x) = 0 \quad \forall x \in V,
    \] so \(cf + g \in S^0\). 
\end{proof}

\begin{eg}
    \(\left\{ 0 \right\}^0 = V^* \) and \(V^0 = \left\{ 0 \right\} \).  
\end{eg}

\begin{proposition} \label{prop: S1 in S2 then anni of S2 in anni of S1}
    If \(S_1 \subseteq S_2\), then \(S_2^0 \subseteq S_1^0\).  
\end{proposition}
\begin{proof}
    If \(f \in S_2^0\), then \(f(x) = 0\) for all \(x \in S_2\), so \(f(x) = 0\) for all \(x \in S_1\), and thus \(f \in S_1^0\), which means \(S_2^0 \subseteq S_1^0\).       
\end{proof}

\begin{proposition}
    If \(W = \langle S \rangle \), then \(W^0 = S^0\).  
\end{proposition}
\begin{proof}
    Since \(S \subseteq W\), so we know \(W^0 \subseteq S^0\) by \autoref{prop: S1 in S2 then anni of S2 in anni of S1}. Also, for all \(f \in S^0\), we know for all \(x \in \langle S \rangle \), \(x = \sum \alpha _i x_i \) where \(x_i\)'s are elements of \(S\), so
    \[
        f(x) = f \left( \sum \alpha _i x_i  \right) = \sum \alpha _i f(x_i) = 0,  
    \] which means \(S^0 \subseteq W^0\).       
\end{proof}

\begin{eg}
    Suppose \(W_1 \subseteq W_2 \subseteq V\), then \(W_1^0 \supseteq W_2^0 \supseteq V^0\). 
\end{eg}

\begin{proposition}
    Suppose \(V\) is finite dimensional and \(W \subseteq V\), then \(\dim W + \dim W^0 = \dim V = \dim V^*\).  
\end{proposition}
\begin{proof}
    Let \(\dim W = m\) and \(\dim V = n\), and take \(B = \left\{ w_1, \dots , w_m \right\} \) a basis of \(W\) and \(C = \left\{ w_1, \dots , w_m, v_{m+1}, \dots ,v_n \right\} \) as a basis of \(V\). If we take dual of \(C\), suppose 
    \[
        C^* = \left\{ w_1^*, w_2^*, \dots , w_m^*, v_{m+1}^*, \dots , v_n^* \right\},
    \] and now we claim \(\left\{ v_{m+1}^*, \dots , v_n^* \right\} \) is a basis of \(W^0\). For all \(f \in V^*\), we know \(f = \sum_{i=1}^m \alpha _i w_i^* + \sum_{j=m+1}^n \beta _j v_j^*  \). Now if \(f \in W^0\), then we know \(f(w) = 0\) for all \(w \in W\), so \(f(w_i) = 0\) for all \(w_i\)'s, and thus 
    \[
        f(w_i) =  \sum_{i=1}^m \alpha _i w_i^*(w_i) + \sum_{j=m+1}^n \beta _j v_j^*(w_i) = \alpha _i = 0,
    \] so we know \(f = \sum_{j=m+1}^n \beta _j v_j^*\), which means \(f \in \langle v_{m+1}^*, \dots , v_n^* \rangle \). Thus, \(W^0 \subseteq \langle v_{m+1}^*, \dots , v_n^* \rangle\)  Also, \(v_i^*(w) = 0\) for all \(w \in W\), so we know \(\langle v_{m+1}^*, \dots , v_n^* \rangle \subseteq W^0 \), and we're done.      
\end{proof}

\begin{corollary}
    If \(\dim V, \dim W < \infty \) and \(T: V \to W\) is linear, and we define \(T^* : W^* \to V^*\) as \(T\)'s transpose, then \(\rank T = \rank T^*\).  
\end{corollary}
\begin{proof}
    First we show that \(\ker T^* = (\Im T)^0\). Suppose \(f \in \ker T^*\), then 
    \[
        0 = T^*(f) = fT,
    \] so \(fT(v) = 0\) for all \(v \in V\), so \(f(w) = 0\) for all \(w \in \Im T\), so \(f \in \left( \Im T \right)^0 \). Conversely, we can similarly show that \((\Im T)^0 \subseteq \ker T^*\), and we're done. Note that 
    \[
        \dim W^* - \rank T^* = \nu \left( T^* \right) = \dim \left( \Im (T)^0 \right) = \dim W - \dim (\Im T) = \dim W - \rank T,
    \] and since \(\dim W = \dim W^*\), so we know \(\rank T = \rank T^*\).  
\end{proof}

\begin{corollary}
    Suppose \(A\) is a matrix, then its row rank and column rank are same. 
\end{corollary}
\begin{proof}
    By regarding \(A\) as a linear map \(T\)'s corresponding matrix, then  \(T^*\)'s corresponding matrix is \(A^t\), and since we have shown that \(\rank T = \rank T^*\), so \(A\)'s row rank is equal to \(A^t\)'s row rank, which is \(A\)'s column rank.       
\end{proof}

\section{Dual of Dual space/Evaluation}
We first define that \(V^{* *} = \left( V^* \right)^*  \), and we can define a linear map
\[
    \mathrm{ev}:V \to V^{* *}, \quad x \mapsto \widetilde{x},   
\] where \(\widetilde{x} \) is the functional 
\[
    \widetilde{x}: V^* \to F \quad f \mapsto f(x). 
\] 

\begin{theorem}
    \(\mathrm{ev} \) is an isomorphism between \(V\) and \(V^{* *}\).  
\end{theorem}
\begin{proof}
    We can check \(\widetilde{x}, \mathrm{ev} \) are linear easily. \todo{DIY}
    \begin{lemma} \label{lm: v not zero then f(v) neq 0 for some f}
        If \(v \in V\) is not zero, then there exists \(f \in V^*\) s.t. \(f(v) \neq 0\).   
    \end{lemma}
    \begin{proof}
        Take \(B = \left\{ v_1 = v, v_2, \dots , v_n \right\} \) as a basis of \(V\) and take dual \(B^*\), then \(v_1^*(v) = 1\).    
    \end{proof}
    \begin{claim}
        \(\mathrm{ev}:V \to V^{* *} \) is injective. 
    \end{claim}
    \begin{proof}
        Suppose \(v \in \ker \mathrm{ev} \), then \(\widetilde{v} = 0\), which means \(f(v) = 0\) for all \(f \in V^*\), so \(v = 0\) by \autoref{lm: v not zero then f(v) neq 0 for some f}, and thus \(\mathrm{ev} \) is injective.      
    \end{proof}

    Since \(\dim V = \dim V^* = \dim \left( V^* \right)^* = \dim V^{* *} \), so injectivity implies bijectivity. 
\end{proof}

\begin{corollary}
    If \(T:V \to W\) is a linear map with inverse \(S:W \to V\), then \(T^*: W^* \to V^*\)'s inverse is \(S^*: V^* \to  W^*\), where \(S^*\) is the transpose of \(S\).       
\end{corollary}

\begin{corollary}[Matrix ver]
    Suppose \(A \in M_n(F)\) is invertible, then \(A^t\) is invertible, and 
    \[
        \left( A^t \right)^{-1} = \left( A^{-1} \right)^t. 
    \] 
\end{corollary}