\lecture{4}{12 Sep 10:20}{}
In calculus, \(f: \mathbb{R} \to \mathbb{R} \) is called continuous if \(f \left( \lim_{x \to a} x  \right) = \lim_{x \to a}f(x)  \). 

\begin{definition}[Linear transformation] \label{def: linear transformation}
    Suppose \(V, W\) are vector spaces over \(F\). A function 
    \begin{align*}
        T: V &\to W \\
        v &\mapsto T(v)
    \end{align*} 
    is called a linear transformation or a linear map if 
    \[
        T(u+v) = T(u) + T(v) \quad T(\alpha v) = \alpha T(v),
    \]
    or equivalently, 
    \[
        T(\alpha u + v) = \alpha T(u) + T(v).
    \]
\end{definition}

\begin{corollary}
    Suppose \(T\) is a linear transformation, then 
    \[
        T \left( \sum_{i=1}^n \alpha _i u_i  \right) = \sum_{i=1}^n \alpha _i T(u_i).  
    \]
\end{corollary}

\begin{eg}
    Suppose \(V = \left\{ \text{functions from } (-1,1) \text{ to } \mathbb{R}  \right\} \), and define \(T_a(f) = f(a)\), then \(T_a\) is a linear transformation.  
\end{eg}

\begin{eg}
    Consider the space of column vectors,
    \[
        F^n = \left\{ \begin{pmatrix}
             \alpha _1 \\
              \alpha _2\\
              \vdots\\
              \alpha _n\\
        \end{pmatrix}  \mid \alpha _i \in F\right\}, 
    \] and define \(A = \left( a_{ij}  \right) \in M_{n \times n}(F) \) by
    \[
        A = \begin{pmatrix}
            a_{11}  & \cdots & a_{1n}  \\
            \vdots &  &   \\
             a_{m1} & \cdots & a_{mn}  \\
        \end{pmatrix},
    \] then if we have \(T_A: F^n \to F^m\) where
    \[
        \begin{pmatrix}
             x_1 \\
             \vdots \\
             x_n \\
        \end{pmatrix} \mapsto A \cdot \begin{pmatrix}
             x_1 \\
             \vdots \\
             x_n \\
        \end{pmatrix},
    \] then \(T_A\) is a linear map. 

    \begin{note}
        \[
            \begin{pmatrix}
                 & \vdots &   \\
                \alpha _{i1} & \cdots & \alpha _{in}  \\
                 & \vdots &   \\
            \end{pmatrix}
            \begin{pmatrix}
                 x_1 \\
                 \vdots \\
                 x_n \\
            \end{pmatrix} = 
            \begin{pmatrix}
                 \vdots \\
                  \sum_{j=1}^n a_{ij} x_j  \\
                  \vdots \\
            \end{pmatrix}
        \]
    \end{note}
\end{eg}

\begin{eg}
    Consider row of vector space,
    \[
        F^m = \left\{ (\alpha _1, \dots , \alpha _m) \mid \alpha _i \in F \right\} ,
    \] and \(A \in M_{m \times n} (F)\), then if \(T_A: F^m \to F^n\) where
    \[
        T_A: u = (u_1, \dots , u_m) \mapsto (u_1, \dots , u_m) \cdot A
    \]  
    is a linear map.
\end{eg}

Obeserve that a linear map \(T:V \to W\) is determined by \(T(v_i)\), where \(\left\{ v_1, \dots , v_n \right\} \) is a basis of \(V\).

\begin{proposition} \label{prop: pick val of T(vi), linear map unique}
    Suppose \(\left\{ v_1, v_2, \dots , v_n \right\} \) is a basis of \(V\), then pick any \(w_1, \dots , w_n \in W\). Then there is a unique linear map \(T: V \to W\) satisfying \(T(v_i) = w_i\).     
\end{proposition}
\begin{proof}
    Since any \(v \in V\) has a unique representation \(v = \sum_{i=1}^n \alpha _i v_i \). Hence, for a linear map \(T:V \to W\), and for any \(v \in V\), we know
    \[
        T(v) = T \left( \sum_{i=1}\alpha _i v_i  \right) = \sum_{i=1}^n \alpha _i T(v_i) = \sum_{i=1}^n \alpha _i w_i.   
    \] 
    Hence, if such map exists, then it must be unique. Now we have to show the existence of this map. Now if we define a map 
    \[
        T \left( \sum_{i=1}^n \alpha _i v_i  \right) = \sum_{i=1}^n \alpha _i w_i,  
    \] then we can check this is a linear map.
\end{proof}

\begin{eg}
    Suppose \(F^n\) is the span of column vectors, and \(A \in M_{m \times n}(F)\), and define \(T_A(v) = Av\), then we can check \(T_A (e_i) = c_i\), where \(c_i\) is the \(i\)-th column of \(A\).  This is the linear map that sends \(e_i\) to \(c_i \in F^m\). If we pick \(c_1, c_2, \dots , c_n \in F^m\), then there is a unique map sending \(e_i\) to \(c_i\). In fact, this map is 
    \[
        T_A : v \mapsto Av
    \], where the \(i\)-th column of \(A\) is \(c_i\).         

\end{eg}

\begin{definition*}
Given \(T: V \to W\), where \(T\) is linear.
\begin{definition}[Kernel] \label{def: kernel}
The kernel/nullspace of \(T\) is defined as
    \[
        \ker (T) = \left\{ v \in V \mid T(v) = 0 \right\} \subseteq V.
    \]   
\end{definition}

\begin{definition}[Image] \label{def: image}
The image/range of \(T\) is defined as 
    \[
        \Im (T) = \left\{ T(v) \mid v \in V \right\} \subseteq W. 
    \]
\end{definition}
\end{definition*}

\begin{remark}
    Kernel and Image are subspaces.
\end{remark}