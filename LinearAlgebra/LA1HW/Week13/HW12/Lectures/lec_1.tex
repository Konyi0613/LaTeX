\section*{Section 7.1}
\begin{problem*}
\textbf{1.} Let $T$ be a linear operator on $F^2$. Prove that any non-zero vector which is not a characteristic vector for $T$ is a cyclic vector for $T$. Hence, prove that either $T$ has a cyclic vector or $T$ is a scalar multiple of the identity operator.
\end{problem*}
\begin{proof}
    Suppose \(v \in F^2\) and \(v \neq 0\) and \(Tv \notin \mathrm{span}\left\{ v \right\}  \), then \(\left\{ v, Tv \right\} \) is linearly independent and thus a basis of \(F^2\), which means \(v\) is a cyclic vector for \(T\). Now if such \(v\) does not exist, then \(Tv \in \mathrm{span} \left\{ v \right\}  \) for all \(v \in F^2\). Then if \(\left\{ p, q \right\} \) form a basis of \(F^2\), then \(Tp =\lambda p\) and \(Tq = \lambda ^{\prime} q\) for some \(\lambda , \lambda ^{\prime} \in F\), and we claim that \(\lambda = \lambda ^{\prime} \). If \(\lambda \neq \lambda 
    ^{\prime} \), then
    \[
        T(p + q) = T(p) + T(q) = \lambda p + \lambda ^{\prime} q \notin \mathrm{span}\left\{ p + q \right\},  
    \]               
    which is impossible. Hence, \(\lambda = \lambda ^{\prime} \), and thus for all \(v \in V\), we know \(v = \alpha p + \beta q\) for some \(\alpha , \beta \in F\), and thus 
    \[
        Tv = \alpha T p + \beta Tq = \alpha \lambda p + \beta \lambda q = \lambda (\alpha p + \beta q) = \lambda v,
    \]    
    which shows \(T\) is a scalar multiple of the identity operator.  
\end{proof}

\begin{problem*}
\textbf{7.} Let $V$ be an $n$-dimensional vector space, and let $T$ be a linear operator on $V$. Suppose that $T$ is diagonalizable.
\begin{enumerate}[(a)]
    \item If $T$ has a cyclic vector, show that $T$ has $n$ distinct characteristic values.
    \item If $T$ has $n$ distinct characteristic values, and if $\{\alpha_1, \dots, \alpha_n\}$ is a basis of characteristic vectors for $T$, show that $\alpha = \alpha_1 + \cdots + \alpha_n$ is a cyclic vector for $T$.
\end{enumerate}
\end{problem*}
\begin{proof}
    \vphantom{text}
    \begin{itemize}
        \item [(a)] If \(T\) has a cyclic vector, then \(m_T(x) = \mathrm{ch}_T(x) \), and thus 
        \[
            \deg m_T(x) = \deg \mathrm{ch}_T(x) = n, 
        \]
        and since \(T\) is diagonalizable, so we know \(m_T(x) = (x - \lambda _1)(x - \lambda _2) \dots (x - \lambda _n)\) for some \(\lambda _i \in F\). Thus,
        \[
            \mathrm{ch}_T(x) = \mathrm{m}_T(x) = (x - \lambda _1) \dots (x - \lambda _n),  
        \]    
        which means \(T\) has \(n\) distinct characteristic values.  
        \item [(b)] Suppose \(T \alpha _i = \lambda _i \alpha _i\) for all \(i = 1,2, \dots , n\). Then we know 
        \[
            T^i \alpha = T^i \left( \sum_{j=1}^n \alpha _j  \right) = \sum_{j=1}^n T^i \left( \alpha _j \right) = \sum_{j=1}^n \lambda _j^i \alpha _j.    
        \]
        Now we want to show \(\left\{ \alpha , T \alpha , \dots , T^{n-1} \alpha  \right\} \) is a basis of \(V\) so that we know \(\alpha \) is a cyclic vector for \(T\). Note that 
        \begin{align*}
            \alpha &= \alpha _1 + \alpha _2 + \dots + \alpha _n \\
            T \alpha &= \lambda _1 \alpha _1 + \lambda _2 \alpha _2 + \dots + \lambda _n \alpha _n \\
            T^2 \alpha &= \lambda _1^2 \alpha _1 + \lambda _2^2 \lambda _2 + \dots + \lambda _n \alpha _n \\
            &\vdots \\
            T^{n-1} \alpha &= \lambda _1^{n-1} \alpha _1 + \lambda _2^{n-1} \alpha _2 + \dots + \lambda _n^{n-1}\alpha _n,
        \end{align*}    
        so we have 
        \[
            \underbrace{\begin{pmatrix}
                1 & 1 & \cdots & 1  \\
                \lambda _1 & \lambda _2 & \cdots & \lambda _n  \\
                \vdots & \vdots & \ddots & \vdots  \\
                \lambda _1^{n-1} & \lambda _2^{n-1} & \cdots & \lambda _n^{n-1}  \\
            \end{pmatrix}}_{A \in M_n(F)} \underbrace{\begin{pmatrix}
                 \alpha _1 \\
                 \alpha _2 \\
                 \vdots \\
                 \alpha _n \\
            \end{pmatrix}}_{X \in M_n(F)} = \underbrace{\begin{pmatrix}
                 \alpha  \\
                 T \alpha  \\
                 \vdots \\
                 T^{n-1} \alpha  \\
            \end{pmatrix}}_{Y \in M_n(F)},
        \]
        and note that \(\det (A) \neq 0\) since \(\lambda _i \neq \lambda _j\) for all \(i \neq j\) and \(A\) is the Vandermonde matrix. Hence, \(A\) is invertible, and thus 
        \[
            \rank Y = \rank AX = \rank X = n
        \]     
        since the \(n\) rows of \(X\) are linearly independent, and thus the \(n\) rows of \(Y\) are also linearly independent, which shows \(\left\{ \alpha , T \alpha , \dots , T^{n-1} \alpha  \right\} \) is a basis of \(V\), and we're done.      
    \end{itemize}
\end{proof}

\section*{Section 7.3}
\begin{problem*}
\textbf{6.} Let $A$ be the complex matrix
$$
A = \begin{pmatrix}
2 & 0 & 0 & 0 & 0 & 0 \\
1 & 2 & 0 & 0 & 0 & 0 \\
-1 & 0 & 2 & 0 & 0 & 0 \\
0 & 1 & 0 & 2 & 0 & 0 \\
1 & 1 & 1 & 1 & 2 & 0 \\
0 & 0 & 0 & 0 & 1 & -1
\end{pmatrix}.
$$
Find the Jordan form for $A$.
\end{problem*}
\begin{proof}
    Note that \(\mathrm{ch}_A(x) = (x - 2)^5 (x + 1) \), and since 
    \[
        A - 2I = \begin{pmatrix}
            0 & 0 & 0 & 0 & 0 &  0 \\
            1 & 0 & 0 & 0 & 0 & 0  \\
            -1 & 0 & 0 & 0 & 0 & 0  \\
            0 & 1 & 0 & 0 & 0 & 0  \\
            1 & 1 & 1 & 1 & 0 & 0  \\
            0 & 0 & 0 & 0 & 1 & -3  \\
        \end{pmatrix},
    \] 
    so we know \(\rank (A - 2I) = 4\), and thus \(\dim \ker (A - 2I) = 2\). Also, we have 
    \[
        (A - 2I)^2 = \begin{pmatrix}
            0 & 0 & 0 & 0 & 0 & 0  \\
            0 & 0 & 0 & 0 & 0 & 0  \\
            0 & 0 & 0 & 0 & 0 & 0  \\
            1 & 0 & 0 & 0 & 0 & 0  \\
            0 & 1 & 0 & 0 & 0 & 0  \\
            1 & 1 & 1 & 1 & -3 & 9  \\
        \end{pmatrix},
    \] so \(\rank (A - 2I)^2 = 3\), and thus \(\dim \ker (A - 2I)^2 = 3\). Now we know there are two Jordan blocks with characteristic value \(2\) and there is one Jordan block with characteristic value \(2\) and of size larger than \(2\), and since the sum of the size of these two Jordan blocks with characteristic value \(2\) is \(5\), so we know the sizes of these two Jordan blocks are \(1\) and \(4\), and since the sum of size of the Jordan block with characteristic value \(1\) is \(1\), so we know the Jordan form of \(A\) is 
    \[
        \left( \begin{array}{cccc|c|c}
             2 & 1 & 0 & 0 & 0 & 0  \\
             0 & 2 & 1 & 0 & 0 & 0  \\
             0 & 0 & 2 & 1 & 0 & 0  \\
             0 & 0 & 0 & 2 & 0 & 0  \\
             \hline 
             0 & 0 & 0 & 0 & 2 & 0  \\
             \hline 
             0 & 0 & 0 & 0 & 0 & 1  \\
        \end{array} \right).
    \]         
\end{proof}

\begin{problem*}
\textbf{10.} Let $n$ be a positive integer, $n \ge 2$, and let $N$ be an $n \times n$ matrix over the field $F$ such that $N^n = 0$ but $N^{n-1} \ne 0$. Prove that $N$ has no square root, i.e., that there is no $n \times n$ matrix $A$ such that $A^2 = N$.
\end{problem*}
\begin{proof}
    If \(A^2 = N\), then \(A^{2n} = N^n = 0\) and \(A^{2n - 2} = N^{n - 1} \neq 0\), so 
    \[
        m_A(x) \mid x^{2n} \text{ but } m_A(x) \nmid x^{2n - 2}, 
    \]   
    so \(m_A(x) \in \left\{ x^{2n - 1}, x^{2n} \right\} \). Hence, \(\deg m_A(x) \ge 2n - 1\), but \(m_A(x) \mid \mathrm{ch}_A(x) \), so
    \[
        n = \deg \mathrm{ch}_A(x) \ge \deg m_A(x) \ge 2n - 1,
    \] but this gives \(1 \ge n\), so this is impossible.   
\end{proof}

\begin{problem*}
\textbf{13.} If $N$ is a $k \times k$ elementary nilpotent matrix, i.e., $N^k = 0$ but $N^{k-1} \ne 0$, show that $N^t$ is similar to $N$. Now use the Jordan form to prove that every complex $n \times n$ matrix is similar to its transpose.
\end{problem*}
\begin{proof}
    Note that 
    \[
        \left( N^k \right)^t = (N^t)^k \quad \forall k \ge 0. 
    \]
    Hence, \(\left( N^t \right)^k = 0 \) and \(\left( N^t \right)^{k-1} \neq 0 \). Note that since \(N^k = 0\) and \(N^{k-1} \neq 0\), so \(m_N(x) = x^k\), and thus if \(J_N\) is the Jordan form of \(N\), then \(J_N\)'s largest Jordan block has size \(k\), which means \(J_N\) has exactly one Jordan block. Also, we have similar argument on \(N^t\), so \(N\) and \(N^t\) has same Jordan form, and thus 
    \[
        N \sim J_N \sim N^t.
    \]             
     Now for every complex \(n \times n\) matrix \(A\), since \(\mathbb{C} \) is algebraically closed, so the Jordan form of \(A\) exists, say it is \(J_A\), then we know 
     \[
        J_A = \bigoplus_{i=1}^k \bigoplus_{j=1}^{r_i} J_{s_j}(\lambda _i),
     \]  where \(k\) is the number of characteristic value of \(A\), and \(r_i\) is the number of Jordan block with characteristic value \(\lambda _i\), and \(s_j\) is the size of the Jordan block. Note that for all \(i, j\), \(J_{s_j}(\lambda _i) - \lambda _i I\) is a \(s_j \times s_j\) elementary nilpotent matrix, so 
     \begin{align*}
        J_{s_j}(\lambda _i) - \lambda_i I &= Q^{-1} (J_{s_j}(\lambda _i) - \lambda _i I)^t Q = Q^{-1} J_{s_j}(\lambda _i)^t Q - \lambda _i Q^{-1} I Q =  Q^{-1} J_{s_j}(\lambda _i)^t Q - \lambda _i I,
     \end{align*}       
     for some \(Q\) and thus 
     \[
        J_{s_j}(\lambda _i)^t = Q^{-1} J_{s_j}(\lambda _i) Q,
     \] which shows 
     \[
        J_{s_j}(\lambda _i) \sim J_{s_j}(\lambda _i)^t.
     \]
     Hence, we know 
     \[
        J_A = \bigoplus_{i=1}^k \bigoplus_{j=1}^{r_i} J_{s_j}(\lambda _i) \sim \bigoplus_{i=1}^k \bigoplus_{j=1}^{r_i} J_{s_j}(\lambda _i)^t = J_A^t,
     \]
     so 
     \[
        J_A^t = R^{-1} J_A R
     \] for some \(R\). Now if \(A = P^{-1} J_A P\) for some \(P\), then 
     \[
        A^t = P^t J_A^t \left( P^{-1} \right)^t = P^t R^{-1} J_A R \left( P^{-1} \right)^t,  
     \]   
     which shows 
     \[
        A^t \sim J_A \sim A,
     \] and we're done.
\end{proof}