\begin{problem}
    Let \(W\) be a set of all \((x_1, x_2, x_3, x_4, x_5)\) in \(\mathbb{R} ^5\) which satisfy 
    \[
        \begin{aligned}
        2x_1 &- x_2 &+ \tfrac{4}{3}x_3 &- x_4 & &= 0\\
        x_1  &      &+ \tfrac{2}{3}x_3 & &- x_5 &= 0\\
        9x_1 &- 3x_2 &+ 6x_3           &- 3x_4 &- 3x_5& = 0
        \end{aligned}
    \]
Find a finite set of vectors which spans \(W\). 
\end{problem}
\begin{proof}
    We can first write the system of equations into the matrix form and then use Gaussian elimination.
\begin{align*}
    \begin{pmatrix}
        2 & -1 & \frac{4}{3} & -1 & 0& 0  \\
        1 &  0&  \frac{2}{3}&  0& -1& 0  \\
        9 & -3 & 6 & -3 & -3 &0  \\
    \end{pmatrix} \to \begin{pmatrix}
        1 &  -\frac{1}{2}& \frac{2}{3} & -\frac{1}{2} &  0& 0  \\
        0 &  \frac{1}{2}&  0&  \frac{1}{2}&  -1& 0  \\
        0 &  0&  0&  0&  0&  0 \\
    \end{pmatrix}
\end{align*}
Hence, we can need to solve
\[
    \begin{dcases}
        x_1 - \frac{1}{2}x_2 + \frac{2}{3}x_3 - \frac{1}{2}x_4 = 0 \\
        \frac{1}{2}x_2 + \frac{1}{2} x_4 - x_5 = 0.
    \end{dcases}
\]
So we know \((x_1, _2, x_3, x_4, x_5) = \left( t - \frac{2}{3}a, b, a, 2t - b, t \right) \) for some \(a,b,t \in \mathbb{R} ^5\), and thus we know the set 
\[
    S = \left\{ \left( 1,0,0,2,1 \right), \left( -\frac{2}{3}, 0, 1, 0, 0 \right), \left( 0, 1, 0, -1, 0 \right)   \right\} 
\]  spans \(W\). 
\end{proof}

\begin{problem}
Prove that a subspace of \(\mathbb{R} ^2\)  is \(\mathbb{R} ^2\) , or the zero subspace, or consists of all
scalar multiples of some fixed vector in \(\mathbb{R} ^2\) . (The last type of subspace is, intuitively, a straight line through the origin.)
\end{problem}
\begin{proof}
    We first give a claim: 
    \begin{claim}
        Suppose \(V\) is a vector space, and if \(W\) is a subspace of \(V\), then \(\dim W \le \dim V\).  
    \end{claim}
    \begin{explanation}
        Since \(W \subseteq V\), so suppose \(k=\dim V\) and \(B_1\) is a basis of \(V\), then \(W \subseteq V = \Span B_1\), which means if there is a basis of \(W\), say \(B_2\), then \(\vert B_2 \vert \le \vert B_1 \vert\), which means \(\dim W \le \dim V\).         
    \end{explanation}

    Now also we know \(\dim \mathbb{R} ^2 = 2\) since \(\left\{ (0,1), (1, 0) \right\} \) is a linearly independent set and spans \(\mathbb{R} ^2\). Thus, if there is a subspace of \(\mathbb{R} ^2\), say \(W\), then \(\dim W = 0, 1, 2\). If \(\dim W = 0\), then \(W\) is the zero subspace. If \(\dim W = 1\), then \(W\) consists of all scalar multiples of some fixed vector in \(\mathbb{R} ^2\). If \(\dim W = 2\), then we can give a claim first: 
    \begin{claim}
        Suppose \(W \subseteq V\) and they are both vector spaces, then if \(\dim V = \dim W\), then \(V = W\).  
    \end{claim}        
    \begin{explanation}
        Suppose by contradiction, there exists \(v \in V \setminus W\), and suppose \(B\) is a basis of \(W\), then we know \(B \cup \left\{ v \right\} \) is linearly independent in \(V\). However, \(\left\vert B \cup \left\{ v \right\}  \right\vert > \dim V \), which is a contradiction.      
    \end{explanation}  
    By this claim, we know \(W = \mathbb{R} ^2\) if \(\dim W = 2\).    
\end{proof}
\begin{proof}[Alternative proof (not sure about if the above one is legal)]
Suppose \(W\) is a subspace of \(\mathbb{R}^2\). If \(W=\left\{ (0,0) \right\} \), then it is the zero subspace. If not, there is some \(v \neq (0, 0)\) s.t. \(v \in W\). However, since \(W\) is a vector space, so \(\Span \left\{ v \right\} \subseteq  W\). If the equal sign holds, then \(W\) consists of all scalar multiples of some fixed vector in \(\mathbb{R} ^2\). If the equal signs does not holds, then there exists \(v, w \in W\) s.t. \(w\) is not a scalar multiple of \(v\). 
\begin{claim}
    \(\Span \left\{ v, w \right\} = \mathbb{R} ^2\). 
\end{claim}
\begin{explanation}
    It is trivial that \(\Span \left\{ v, w \right\} \in \mathbb{R} ^2 \). Now we show that \(\mathbb{R} ^2 \subseteq \Span \left\{ v,w \right\} \). First assume \(v = (v_1, v_2)\) and \(w = (w_1, w_2)\). For all \(r = (c_1, c_2) \in \mathbb{R} ^2\), we know the systems of equations    
    \[
        \begin{dcases}
            v_1 x + w_1 y = c_1 \\
            v_2 x + w_2 y = c_2
        \end{dcases}
    \]
    has a unique solution by Cramer's rule learnt in high school. Hence, we know \(r = x v + y w \in \Span \left\{ v, w \right\} \), and we're done.
\end{explanation}

Besides, it is trivial that a subspace cannot be bigger than the original vector space, that is, there does not exists \(v \in W\) s.t. \(v \notin \mathbb{R} ^2\). Hence, \(W\) cannot be bigger, and thus we have concluded all the cases of \(W\).    
\end{proof}

\begin{problem}
Let \(W_1\) and \(W_2\) be subspaces of a vector space \(V\) such that the set-theoretic union of \(W_1\) and \(W_2\) is also a subspace. Prove that one of the spaces \(W_i\) is contained in the other.
\end{problem}
\begin{proof}
    Suppose each \(W_i\) is not contained in the other, then there exists \(u, v\) s.t. 
    \[
        u \in W_2 \setminus W_1 \quad v \in W_1 \setminus W_2. 
    \]  
    Thus, we know \(u + v \in W_1 \cup W_2\) since \(W_1 \cup W_2\) is a vector space. Also, we know \(u + v \in W_1\) or \(u + v \in W_2\). Now if \(u + v \in W_1\), then \(u = u+v + (-v) \in W_1\), which is a contradiction, and if \(u + v \in W_2\), we have \(v = u + v + (-u) \in W_2\), which is also a contradiction. Hence, we know either \(W_1 \subseteq W_2\) or \(W_2 \subseteq W_1\) should happen.        
\end{proof}

\begin{problem}
Let \(V\) be the vector space of all functions from \(\mathbb{R} \)  into \(\mathbb{R} \); let \(V_e\) be the subset of even functions, \(f(-x) = f(x)\); let \(V_o\) be the subset of odd functions, \(f(-x) = -f(x)\).
\begin{itemize}
    \item [(a)] Prove that \(V_e\) and \(V_o\) are subspaces of \(V\). 
    \item [(b)] Prove that \(V_e + V_o = V\). 
    \item [(c)] Prove that \(V_e \cap V_o = \left\{ 0 \right\} \).     
\end{itemize}
\end{problem}
\begin{proof}
    \vphantom{text}
\begin{itemize}
    \item [(a)] First note that \(V_e \subseteq V\) and \(V_o \subseteq V\), and they are both non-empty (since zero function is contained in both \(V_e\) and \(V_o\), then   
    \begin{itemize}[\(\bullet\) ]
        \item For all \(f, g \in V_e\) and \(\alpha \in F\), we define \(h(x) = \alpha f(x) + g(x)\), and we know \(h \in V_e\) since
        \[
            h(-x) = \alpha f(-x) + g(-x) = \alpha f(x) + g(x) = h(x).
        \]
        Hence, \(V_e\) is a subspace of \(V\).  
        \item For all \(f, g \in V_o\) and \(\alpha \in F\), we define \(h(x) = \alpha f(x) + g(x)\), and we know \(h \in V_o\) since
        \[
            h(-x) = \alpha f(-x) + g(-x) = -\alpha f(x) - g(x) = -h(x).
        \]
        Hence, \(V_o\) is a subspace of \(V\).  
    \end{itemize}
    \item [(b)] We first show that \(V \subseteq V_e + V_o\). Since for all \(f \in V\), we know 
    \[
        f(x) = \left( \frac{f(x) + f(-x)}{2} \right) + \left( \frac{f(x) - f(-x)}{2} \right), 
    \]  where 
    \[
        f_e(x) = \left( \frac{f(x) + f(-x)}{2} \right) \in V_e \quad f_o(x) = \left( \frac{f(x) - f(-x)}{2} \right) \in V_o
    \] since 
    \begin{align*}
        f_e(-x) &= \left( \frac{f(-x) + f(x)}{2} \right) = f_e(x) \\
        f_o(-x) &= \left( \frac{f(-x) - f(x)}{2} \right) = -f_o(x) .
    \end{align*}
    Now we show that \(V_e + V_o \subseteq V\). This is trivial since for all \(f \in V_e + V_o\) we have \(f = g + h\) for some \(g \in V_e \subseteq V\) and \(h \in V_o \subseteq V\), so \(f = g+h \in V\) since \(V\) is a vector space. Hence, \(V_e + V_o \subseteq V\). \\
    Now we have \(V \subseteq V_e + V_o\) and \(V_e + V_o \subseteq V\), so we can conclude that \(V = V_e + V_o\).   
    \item [(c)]
\end{itemize}   
\end{proof}

\begin{problem}
    Let \(W_1\) and \(W_2\) be subspaces of a vector space \(V\) such that \(W_1 + W_2 = V\) and \(W_1 \cap W_2 = \left\{ 0 \right\} \). Prove that for each vector \(\alpha \) in \(V\) there are unique vectors \(\alpha _1\) in \(W_1\) and \(\alpha _2\) in \(W_2\) such that \(\alpha = \alpha _1 + \alpha _2\).
\end{problem}
\begin{proof}
    Since \(W_1 + W_2 = V\), so we know for each vector \(\alpha \in V\), it can be representend as \(\alpha = \alpha _1 + \alpha _2\) for some \(\alpha _1 \in W_1\) and \(\alpha _2 \in W_2\). If there are two differet \((\alpha_1, \alpha _2)\)-pairs to represent \(\alpha \), say \(\alpha = \alpha _1 + \alpha _2 = \alpha _1^{\prime}  + \alpha _2 ^{\prime} \) where \(\alpha _1, \alpha _1^{\prime} \in W_1\) and \(\alpha _2 , \alpha _2^{\prime}  \in W_2\), then we know \(\alpha _1 - \alpha _1^{\prime}  = \alpha _2^{\prime}  - \alpha _2 \). However, since \(\alpha _1 - \alpha _1^{\prime}  \in W_1\) and \(\alpha _2^{\prime}  - \alpha _2 \in W_2\), so \(\alpha _1 - \alpha _1^{\prime} = \alpha _2^{\prime} - \alpha _2 \in W_1 \cap W_2 = \left\{ 0 \right\} \), which means \(\alpha _1 = \alpha _1^{\prime} \) and \(\alpha _2 = \alpha _2^{\prime} \) and thus the \((\alpha _1, \alpha _2)\)-pair is unique.                
\end{proof}