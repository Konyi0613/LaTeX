\begin{problem}
Let \(V\) be the set of real numbers. Regard \(V\) as a vector space over the field of rational numbers, with the usual operations. Prove that this vector space is not finite-dimensional.
\end{problem}
\begin{proof}
    If \(V\) is finite dimensional, then any real number \(r \in V\) can represented by the linear combination of finite many rational numbers. However, rational numbers are closed under addition and multiplication, so if \(r\) is irrational, then it is impossible to represent \(r\) as the linear combination of finite many rational number. 
\end{proof}

\begin{problem}
Consider the differentiation transformation on \(V=\mathbb{R}[x]\), which is defined by 
\[
  D(f(x)) = f^{\prime} (x) \quad \forall f(x) \in V.  
\] 
Find the range and null space of this transformation.
\end{problem}
\begin{proof}
    Since for all \(f(x) = a_0 + a_1 x + \dots + a_n x^n\), we know 
    \[
        F(x) = C + a_0 x + \frac{a_1}{2} x^2 + \dots + \frac{a_i}{i+1}x^{i+1} + \dots + \frac{a_n}{n+1}x^{n+1}
    \] has \(F^{\prime} (x) = f(x)\), where \(C\) can be any element in \(\mathbb{R} \). Also, we have \(F(x) \in \mathbb{R} [x]\) and \(D(F(x))=f(x)\), so \(f(x) \in \Im (D)\), which means \(\Im (D) = \mathbb{R} [x]\).
    
    Now if for some \(f(x) \in \mathbb{R} [x]\), we have \(D(f(x)) = 0\), then we know \(f(x)\) is a constant function, so
    \[
        \ker D = \left\{ f(x) \in \mathbb{R} [x] \mid f(x) = c \text{ for some } c \in \mathbb{R}  \right\}.
    \]   
\end{proof}

\begin{problem}
    Let \(V\) be the vector space of all \(n \times n\) matrices over the field \(F\), and let \(B\) be a fixed \(n \times n\) matrix. If 
    \[
        T(A) = AB-BA,
    \]verify that \(T\) is a linear transformation from \(V\) into \(V\).   
\end{problem}
\begin{proof}
It is trivial that \(T\) is a map from \(V\) into \(V\). Now we show that \(T\) is a linear map. Note that for all \(A, C \in M_{n \times n}(F)\), we have
\begin{align*}
    T(\alpha A + C) &= (\alpha A + C)B - B(\alpha A + C) \\
    &= \alpha AB + CB - \alpha BA - BC \\
    &= \alpha (AB - BA) + (CB - BC) \\
    &= \alpha T(A) + T(C).
\end{align*}   
Hence, \(T\) is a linear map.
\end{proof}

\begin{problem}
    Let \(V\) be the set of all complex numbers regarded as a vector space over the field of real numbers (usual operations). Find a function from \(V\) into \(V\) which is a linear transformation on the above vector space, but which is not a linear transformation on \(C^1\), i.e., which is not complex linear.
\end{problem}
\begin{proof}
    We can consider \(T:V \to V\), where
    \[
        T(a+ bi) = (a + b) + bi \quad \forall a+bi \in V \text{ with }a,b \in \mathbb{R}.
    \] 
    We first show that it is a linear transformation on \(V\). For any \(\alpha \in \mathbb{R} \), and \(a^{\prime} + b^{\prime} i \in V\) with \(a^{\prime} ,b^{\prime} \in \mathbb{R}\),
    \begin{align*}
        T(\alpha (a + bi) + (a^{\prime} + b^{\prime} i)) &= T((\alpha a + a^{\prime} ) + (\alpha b + b^{\prime} )i) \\
        &= (\alpha (a + b) + a^{\prime} + b^{\prime} ) + (\alpha b + b^{\prime} )i \\
        &= (\alpha (a+b) + \alpha b i) + (a^{\prime} +b^{\prime}  + b^{\prime} i) \\
        &= \alpha T(a + bi) + T(a^{\prime} +b^{\prime} i).
    \end{align*}
    Hence, \(T\) is linear on \(V\). Now we show that \(T\) is not complex linear. 
    For \(\alpha = p + qi \in \mathbb{C} \), where \(p, q \in \mathbb{R} \), we know 
    for any \(a + bi \in V\) with \(a, b \in \mathbb{R} \), we have 
    \begin{align*}
        T((p+qi)(a+bi)) &= T((pa-qb) + (pb+qa)i) \\
        &= (pa-qb+pb+qa)+(pb+qa)i,
    \end{align*}    
    but we also have 
    \begin{align*}
        (p + qi)T(a + bi) &= (p + qi)((a+b) + bi) \\
        &= (p + qi)(a + b) + (p + qi)bi \\
        &= pa + pb + qai + qbi + pbi - qb \\
        &= (pa + pb - qb) + (qa + qb + pb)i
    \end{align*}
    Note that
    \[
        T((p+qi)(a+bi)) - (p+qi)T(a+bi) = qa-qbi=q(a-bi).
    \]
    If we pick \(a \neq b\) and \(q \neq 0\), then \(T((p+qi)(a+bi)) \neq (p+qi)T(a+bi)\). Thus, \(T\) is not complex linear. 
\end{proof}
\begin{problem}
    Let \(V\) be a vector space and \(T\) a linear transformation from \(V\) into \(V\). Prove that the following two statements about \(T\) are equivalent.
    \begin{itemize}
        \item [(a)]The intersection of the range of \(T\) and the null space of \(T\) is the zero subspace of \(V\).
        \item [(b)]If \(T(T(\alpha )) = 0\), then \(T(\alpha )=0\).  
    \end{itemize}
\end{problem}
\begin{proof}
    \begin{align*}
        &\ker T \cap \Im T = \left\{ 0 \right\} \\
        \iff &\text{If } w \in \ker T \text{ for some } w \in \Im T, \text{ then } w = 0. \\
        \iff &\text{If } T(w) = 0 \text{ for some } w \in \Im T, \text{ then } w = 0. \\
        \iff &\text{If } T(T(\alpha )) = 0 \text{ for some } \alpha \in V, \text{ then } T(\alpha ) = 0.   
    \end{align*}    
\end{proof}