\lecture{9}{8 Oct. 13:20}{}
\begin{theorem} \label{thm: m,n coprime means Z/mnZ isomorphic to Z/mZ x Z/nZ}
    Let \(m, n\) be coprime integers, then 
    \[
        \phi : \mathbb{Z} / mn \mathbb{Z} \to \mathbb{Z} / m \mathbb{Z} \times \mathbb{Z} / n \mathbb{Z} 
    \] with \(a + mn \mathbb{Z} \mapsto (a + m \mathbb{Z} , a + n\mathbb{Z} )\) is an isomorphism. 
\end{theorem}

\begin{eg}
    \(m = 2, n = 3\) 
\end{eg}
\begin{table}[H]
    \centering
    \begin{tabular}{c|c}
        \toprule
            \(\mathbb{Z} / 6 \mathbb{Z} \)  & \(\mathbb{Z} / 2 \mathbb{Z}  \times \mathbb{Z} / 3 \mathbb{Z} \)   \\
        \midrule
            \(\overline{0} \)  & \(\left( \overline{0}, \overline{0}   \right) \)   \\
            \(\overline{1} \) & \(\left( \overline{1}, \overline{1}   \right) \)  \\
            \(\overline{2} \) & \(\left( \overline{0}, \overline{2}   \right) \)  \\
            \(\overline{3} \)  & \(\left( \overline{1}, \overline{0}   \right) \)  \\
            \(\overline{4} \)  &\(\left( \overline{0}, \overline{1}   \right) \)  \\
            \(\overline{5} \)  & \(\left( \overline{1}, \overline{2}   \right) \)  \\
        \bottomrule
    \end{tabular}
    \caption{The case \(m=2, n=3\)}
    \label{tab:Z6Z isomorphic to Z2Z Z3Z}
\end{table}

\begin{proof}[proof of \autoref{thm: m,n coprime means Z/mnZ isomorphic to Z/mZ x Z/nZ}]
    We have to show injectivity, surjectivity, and homomorphism. Note that if we have \(\vert G \vert = \vert H \vert  \), then injectivity is equivalent to surjectivity since surjectivity gives \(\vert G \vert \ge \vert H \vert \) and injectivity gives \(\vert H \vert \ge \vert G \vert \). (Suppose the map is \(G \to H\)) Now since 
    \[
        \left\vert \mathbb{Z} / mn \mathbb{Z}  \right\vert = mn = \left\vert \mathbb{Z} / m \mathbb{Z} \times \mathbb{Z} / n \mathbb{Z}  \right\vert,  
    \] so we just need to show the injectivity and group homomorphism. Now if 
    \[
        \phi \left( \overline{x}  \right) = (\overline{0}, \overline{0}),  
    \] then \(x \in m\mathbb{Z} \cap n \mathbb{Z} = mn \mathbb{Z} = \overline{0} \), so \(\ker \phi = \left\{ \overline{0}  \right\} \). 
    
    \begin{exercise}
        Show the homomorphism part.
    \end{exercise}
\end{proof}

\begin{question}
    Now that we know \(\phi \) is an isomorphism, can we construct \(\phi ^{-1}\)?  
\end{question}
\begin{answer}
    First, find integers \(a, b\) s.t. 
    \[
        ma + nb = 1,
    \] then for \(\left( \overline{x}, \overline{y}   \right) \in \mathbb{Z} / m\mathbb{Z} \times \mathbb{Z} / n \mathbb{Z} \), we can set 
    \[
        \phi ^{-1} \left( \overline{x}, \overline{y}   \right) = \overline{may + nbx}.  
    \] This definition works since 
    \[
        nb \equiv 1 \mod{m} \quad ma \equiv 1 \mod{n}.
    \]
    Check that \(\phi \circ \phi ^{-1} \left( \overline{x}, \overline{y}   \right) = \left( \overline{x}, \overline{y}   \right)  \). 
\end{answer}

\begin{question}
    How about the step of finding such \(a, b\)? 
\end{question}
\begin{answer}
    Suppose \(m \ge n\). Let \(r_0 = m, r_1 = n\), then 
    \begin{align*}
        r_0 &= q_1 r_1 + r_2 \quad 0 \le r_2 < r_1 \\
        r_1 &= q_2 r_2 + r_3 \quad 0 \le r_3 < r_2 \\
        r_2 &= q_3 r_3 + r_4 \quad 0 \le r_4 < r_3 \\
        \vdots \\
        r_{n - 2} &= q_{n - 1} r_{n - 1} + r_n \quad 0 \le r_n < r_{n - 1} \\
        r_{n-1} &= q_n r_n.
    \end{align*} 
    Now since for every \(r_i\), \(\gcd(r_i, r_{i+1}) = \gcd(m, n)\), and \(\gcd(r_{n-1}, r_n) = r_n\), so it works. Since \(\gcd(m, n) = 1\), so \(r_n = 1\), and thus 
    \begin{align*}
        1 = r_n &= r_{n - 2} - q_{n-1} r_{n-1} \\
        &= r_{n-2} - q_{n-1} \left( r_{n-3} - q_{n-2} r_{n-2} \right) \\
        &= -q_{n-1} r_{n-3} + (1 + q_{n-1} q_{n-2}) r_{n-2} \\
        &= \dots 
    \end{align*} 
    so we can recover it to \(1 = a r_0 + b r_1 = am + bn\).      
\end{answer}