\lecture{14}{7 Nov. 13:20}{}
Let \(G\) be finite group and \(p\) prime. Suppose \(\left\vert G \right\vert = p^e m \), and \(\gcd(p, m) = 1\), then 
\[
    \mathrm{Syl}_p(G) = \left\{ H < G \mid \vert H \vert = p^e  \right\},  
\] and for \(H \in \mathrm{Syl}_p(G) \), we call it a Sylow \(p\)-subgroup. 
\begin{theorem}[Sylow's theorem]
    \vphantom{text}
    \begin{itemize}
        \item [(1)] \(\mathrm{Syl}_p(G) \) is non-empty i.e. Sylow \(p\)-subgroup exists. 
        \item [(2)] Suppose \(H < G\) has \(\vert H \vert = p^{i} \) for some \(0 \le i \le e\), then there exists \(P \in \mathrm{Syl}_p(G) \) s.t. \(H < P\). 
        \item [(3)] For \(P, P^{\prime} \in \mathrm{Syl}_p(G) \), there exists \(g \in G\) s.t. \(P^{\prime} = g P g^{-1}\) i.e. all Sylow \(p\)-subgroups are conjugate in \(G\). 
        \item [(4)] Let \(n_p \coloneqq \left\vert \mathrm{Syl}_p(G)  \right\vert \), then \(n_p \equiv 1 \mod{p}\) and \(n_p \mid \vert G \vert \).             
    \end{itemize}
\end{theorem}  

\begin{proposition}
    With the same setting, let \(r \le e\), then there exists \(H < G\) s.t. \(\vert H \vert = p^r\).    
\end{proposition}
\begin{proof}
    First consider all subsets of size \(p^r\). Let \(\mathscr{S}  \coloneqq \left\{ S \subseteq G \mid \vert S \vert = p^r  \right\} \). At least, \(h \in \mathscr{S} \) if exists. Suppose \(\vert G \vert = p^e m = p^r M \). First observe that 
    \[
        \left\vert \mathscr{S}  \right\vert = \binom{p^r M}{p^r} = \frac{p^r M \left( p^r M - 1 \right) \dots \left( p^r M - \left( p^r - 1 \right)  \right)  }{p^r \left( p^r - 1 \right) \dots 1 },
    \] and note that all factors \(p\) in the denominators are cancelled since 
    \[
        p^r M - i \equiv p^r - i \mod{p^r} \quad \forall 1 \le i \le p^r - 1.
    \] 
    Hence, \(\mathrm{ord} _p \vert \mathscr{S}  \vert = \mathrm{ord} _p(M) = s\). 
    Now consider a group action of \(G\) on \(\mathscr{S} \) given by 
    \begin{align*}
        G \times \mathscr{S} \to \mathscr{S} , \quad \left( g, S \right) \mapsto g \cdot S \text{ (left-multiplication)}. 
    \end{align*} 
    Let \(\mathscr{S} = \cup _{i} \mathscr{S} _i\) be the decomposition into orbits (cosets). Thus, 
    \[
        \vert \mathscr{S}  \vert = \sum_{i} \vert \mathscr{S} _i \vert,   
    \] and \(\vert \mathscr{S}  \vert \) is divisible by \(p\) exactly \(s\) times, and thus at least one of \(\mathscr{S} _i\) has \(p^{s+1} \nmid \left\vert \mathscr{S} _i \right\vert \). WLOG, suppose \(p^{s+1} \nmid \vert \mathscr{S} _1 \vert \). Let \(S_1 \in \mathscr{S} _1\). Note that \(\mathscr{S} _1 = \left\{ g \cdot S_1 \mid g \in G \right\} \). Now define \(H = \left\{ h \in G \mid h \cdot S_1 = S_1 \right\} \). Then, \(H < G\). We will show \(\vert H \vert = p^r \): 
    \begin{itemize}
        \item As \(G\) acts on \(\mathscr{S} _1\) transitively, 
        \[
            G / H \to \mathscr{S} _1, \quad gH \mapsto g \cdot S_1
        \] is bijective. 
        Thus, \(\vert \mathscr{S} _1 \vert = \frac{\vert G \vert }{\vert H \vert } \). Hence, \(\vert H \vert = \frac{\vert G \vert }{\vert \mathscr{S}  \vert } \), and since \(\vert G \vert = p^r M = p^rp^s m \), and \(p^{s+1} \nmid \vert \mathscr{S}  \vert \), so \(\vert \mathscr{S} _1 \vert \mid M \). Hence, \(\vert H \vert \) is a multiple of \(p^r\), which means \(\vert H \vert \ge p^r \). 
        \item Next, fix \(x \in S_1\), then 
    \[
        \varphi : H \to S_1, \quad h \mapsto h \cdot x
    \] is injective. Thus, \(\vert H \vert \le \vert S_1 \vert = p^r  \).  
    \end{itemize}      
    Thus, \(\vert H \vert = p^r \). 
              
\end{proof}
\begin{remark}
    Our goal is to find \(H < G\) s.t. \(\vert H \vert = p^r \).  
\end{remark}

Now we show the Sylow's theorem:
\begin{proof}[proof of (1)]
    By previous proposition, it is true.
\end{proof}

\begin{proof}[proof of (2)]
    Let \(P \in \mathrm{Syl}_p(G) \), and 
    \[
        A_p = \left\{ gPg^{-1} \mid g \in G \right\} \subseteq \mathscr{S}. 
    \]
    Let \(N_G(P) \coloneqq \left\{ g \in G \mid gPg^{-1} = P \right\} < G \). Note: \(P \triangleleft N_G(P)\). Hence, 
    \[
        \vert A_p \vert = \frac{\vert G \vert }{\vert N_G(P) \vert } = \left[ G: N_G(P) \right].  
    \] This means 
    \[
        \vert A_p \vert = \frac{\left( \frac{\vert G \vert }{\vert P \vert } \right) }{\left( \frac{\vert N_G(P) \vert }{\vert P \vert } \right) } \implies \vert A_p \vert \mid \frac{\vert G \vert }{\vert P \vert } = \frac{p^e m}{p^e} = m.   
    \] 
    Hence, \(p \nmid \vert A_p \vert \). Next, consider the group action of \(H\) on \(A_p\) by 
    \[
        H \times A_p \to A_p, \quad (h, Q) \mapsto hQh^{-1},
    \] and let \(A_p = \bigcup_{i=1} A_p^{(i)} \) be the decomposition into the orbits with \(A_p^{(1)} = \left\{ hPh^{-1} \mid h \in H \right\} \). let \(P_i\) be a representative of \(A_p^{(i)}\) i.e. 
    \[
        A_p^{(i)} = \left\{ h P_i h^{-1} \mid h \in H \right\}, 
    \] and we know 
    \[
        \left\vert A_p^{(i)} \right\vert = \frac{\vert H \vert }{\vert N_H(P_i) \vert } = \frac{\vert H \vert }{\left\vert H \cap N_G(P_i) \right\vert } 
    \] is a power of \(p\). By the previous argument, we know \(p \nmid \vert A_p \vert \). Thus, there exists \(j\) s.t. \(p \nmid \left\vert A_p^{(j)} \right\vert \), which means \(\left\vert A_p^{(j)} \right\vert = 1\). Thus, \(\vert H \vert = \left\vert H \cap N_G(P_j) \right\vert  \), so \(H \subseteq N_G(P_j)\), which means \(H < N_G(P_j)\). Now recall the second isomorphism theorem: 
    \begin{theorem}[Second Isomorphism Theorem]
        Suppose \(H < G\) and \(N \triangleleft G\), then 
        \begin{itemize}
            \item \(HN < G\)
            \item \(N \triangleleft HN\)
            \item \(H \cap N \triangleleft H\)
            \item \(HN / N \simeq H / (H \cap N)\).    
        \end{itemize}  
    \end{theorem}
    Since we know \(H < N_G(P_j)\) and \(P_j \triangleleft N_G(P_j)\), so 
    \[
        \frac{\left\vert H P_j \right\vert }{\left\vert P_j \right\vert } = \frac{\vert H \vert }{\left\vert H \cap P_j \right\vert },
    \]  
    Thus, we have 
    \begin{align*}
        \text{L.H.S.} &\mid \frac{\vert G \vert }{\left\vert HP_j \right\vert } \cdot \frac{\left\vert H P_j \right\vert }{\left\vert P_j \right\vert } = \frac{\vert G \vert }{\vert P_j \vert } = \frac{p^e m}{p^e} = m \\
        \text{R.H.S.} &\mid \vert H \vert \text{, which is the power of } p,   
    \end{align*}
    so we know L.H.S. and R.H.S. are equal to \(1\). Thus, \(H = H \cap P_j\), and thus \(H \subseteq P_j\), so \(H < P_j\), where \(P_j \in A_p \subseteq \mathrm{Syl}_p(G) \).     
\end{proof}

\begin{proof}[proof of (3)]
    Let \(P, H \in \mathrm{Syl}_p(G) \), then by (2) we know \(H \subseteq P_j \in A_p\) for some \(j\). Since \(\vert H \vert = \vert P_j \vert = p^e  \), so \(H = P_j \in A_p\): conjugation of \(P\) in \(G\). So (3) is true.      
\end{proof}

\begin{proof}[proof of (4)]
    Let \(P \in \mathrm{Syl}_p(G) \). By changing \(H\) as \(P\) in (2), we know \(A_p^{(1)} = \left\{ P \right\} \), whereas \(\left\vert A_p^{(i)} \right\vert > 1\) if \(i \ge 2\). (If \(\left\{ P_i \right\} = \vert A_p^{(i)} \vert = 1\), then \(P = H \subseteq P_i\), and thus \(P_i = P\), which means \(i = 1\).) Therefore, 
    \[
        \left\vert \mathrm{Syl}_p(G)  \right\vert = \vert A_p \vert = \sum_{i} \left\vert A_p^{(i)} \right\vert = \left\vert A_p^{(1)} \right\vert + \sum_{i \ge 2} \left\vert A_p^{(i)} \right\vert = 1 + \sum_{i} p^{l_i} \equiv 1 \mod{p}. 
    \]    
    Also, 
    \[
        \left\vert \mathrm{Syl}_p(G)  \right\vert = \left\vert A_p \right\vert = \frac{\vert G \vert }{\vert N_G(p) \vert } = [G: N_G(p)]
    \] is a divisor of \(\vert G \vert \). 
\end{proof}