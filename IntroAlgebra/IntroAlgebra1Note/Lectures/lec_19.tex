\lecture{19}{28 Nov. 13:20}{}
\begin{remark}
    \(I \subseteq R\) is called an ideal if \(I\) is a submodule of \(R\) (viewed as an \(R\)-submodule)   
\end{remark}

\begin{remark}
    \(\left\{ 0 \right\}, R \) are \(R\)-modules "trivial", so \(I\) is called a non-trivial ideal of \(R\) if \(\left\{ 0 \right\} \subsetneq I \subsetneq R \).     
\end{remark}

\begin{remark}
    \(R\) is a field iff \(R\) has no non-trivial ideals.  
\end{remark}

\begin{remark}
    If \(I \subseteq R\) is a non-trivial ideal, then \(a \in I\) is not invertible.  
\end{remark}

\begin{eg}[Examples of ideals]
    \vphantom{text}
    \begin{itemize}
        \item \(R = \mathbb{Q} , \mathbb{R} , \mathbb{C}, \dots  \) (fields), then only ideals are \(\left\{ 0 \right\} \) and \(R\) itself. 
        \item \(R = \mathbb{Z} \), then since \(m \mathbb{Z} \) are the only subgroups of \(\mathbb{Z} \), and they are all ideals.      
        \item For a finite subset \(S\) of \(R\), 
        \[
            I(S) \coloneqq \left\{ r_1 s_1 + r_2 s_2 + \dots + r_m s_m \mid r_1, \dots , r_m \in R \right\} 
        \]
        and \(S = \left\{ s_1, s_2, \dots, s_m \right\} \) form an ideal of \(R\) called the ideal generated by \(S\). 
        \begin{remark}
            \(I(S)\) is the minimal ideal of \(R\) containing \(S\).   
        \end{remark}  
        Also, \(I(S)\) is  more commonly denoted by \((s_1, s_2, \dots , s_m)\).
    \end{itemize}
\end{eg}

\begin{question}
    Is the case of \(\mathbb{Z} \), does \(I(S)\) give you anthing now?  
\end{question}
\begin{answer}
    Consider 
    \[
        (3, 5) = \left\{ 3m + 5n \mid m, n \in \mathbb{Z}  \right\}, 
    \]
    then since \(1 \in (3, 5)\), so \((1) = \mathbb{Z} \).  
\end{answer}

\begin{remark}
    If \(k\) is a field, then \(k[t]\) (the polynomial ring over \(k\)) has ideals of the form \((p(t))\), which can be shown by Euclidean algorithm.   
\end{remark}

\begin{question}
    If \(R = k[x, y]\), then is \((x, y)\) generated by one element?  
\end{question}
\begin{answer}
    It is not possible. Since 
    \[
        (x, y) = \left\{ p(x, y) x + q(x, y) y \mid p,q \in k[x, y] \right\} = \left\{ \sum_{(i, j) \neq (0,0)} a_{ij} x^i y^j \mid \left\{ a_{ij} \right\} \subseteq k   \right\}. 
    \]
    We show that there is no \(r(x, y) \in k[x, y]\) s.t. \((x, y) = (r(x, y))\). Since 
    \[
        r(x, y) \mid x \text{ and } r(x, y) \mid y, 
    \]  
    so 
    \[
        r(x, y) \mid \gcd(x, y) = 1.
    \]
    Hence, \((x, y) = (r(x,y)) = (1) = k[x, y]\), which is a contradiction. 
\end{answer}

\subsubsection{Numbers to Ideals}
Note that \(d \mid n\) for \(n, d \in \mathbb{Z} \) if and only if \(n = dm\) for some \(m \in \mathbb{Z} \) if and only if \(n \in (d)\) if and only if \((n) \subseteq (d)\). Also, \(p\) is prime means 
\[
    p \mid ab \implies p \mid a \text{ or } p \mid b. 
\]       
This is the number perspective. In the ideal perspective, 
\[
    ab \in (p) \implies a \in (p) \text{ or } b \in (p). 
\]
\begin{definition}
    An ideal \(\mathcal{P} \) of \(R\) is called a prime ideal if 
    \[
        ab \in \mathcal{P} \implies a \in \mathcal{P} \text{ or } b \in \mathcal{P}. 
    \]  
\end{definition}

\begin{definition}
    \(m \subseteq R\) is called a maximal ideal if there is no ideals between \(m\) and \(R\), i.e. 
    \[
        m \subseteq I \subseteq R \implies I = m \text{ or } I = R. 
    \]   
\end{definition}

\begin{remark}
    Biggest prime ideals are maximal ideals.
\end{remark}

\begin{eg}
    \((x-a, y-b)\) are maximal ideals of \(k[x, y]\) for \(a, b \in k\).   
\end{eg}

\begin{eg}
    \(p\) with \(p\) prime are maximal ideals of \(\mathbb{Z} \).   
\end{eg}

\begin{theorem}
    Let \(\phi :A \to B\) be a ring homomorphism: 
    \begin{itemize}
        \item \(\Im \phi \) is a subring of \(B\). 
        \item \(\ker \phi = \phi ^{-1} (\left\{ 0 \right\} )\) is an ideal of \(A\). 
        \item For any ideal \(J \subseteq B\), \(\phi ^{-1}(J)\) is an ideal of \(A\).      
    \end{itemize} 
\end{theorem}
\begin{proof}
    Let's show the third one. If \(a \in A\) and \(x \in \phi ^{-1} (J)\), then \(a \cdot x \in \phi ^{-1}(J)\).   
\end{proof}

\begin{eg}
    \(\phi :\mathbb{Z} \to 3\mathbb{Z} \) defined by modulo \(3\)  where \(\ker \phi = 4\mathbb{Z} \) is not a subring but an ideal. 
\end{eg}

Recall that \(H \triangleleft G\) gives \(G / H\) a group, and 
\[
    \phi :G \to G / H
\]  
has \(\ker \phi = H\). If \(I \subseteq R\) is an ideal, then \(R / I\) forms a ring by 
\[
    (x + I) \cdot (y + I) = (xy + I).
\]   And \(\phi : R \to R / I\) has the \(\ker \) of \(I\).  