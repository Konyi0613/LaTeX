\lecture{22}{12 Dec. 13:20}{}
\begin{eg}
    Now suppose \(A = \mathbb{Z} [x]\) and \(J = \left( x^2 + 1, 2 + x \right) \) and \(I = \left( x^2 + 1 \right) \), then since 
    \[
        A / I \simeq \mathbb{Z} [\sqrt{-1} ],
    \]  
    so 
    \[
        J / I \simeq \left( 2 + \sqrt{-1}  \right) \mathbb{Z} [\sqrt{-1} ]. 
    \] 
\end{eg}

\begin{remark}
    \(A / I \simeq \mathbb{Z} [\sqrt{-1} ]\) since every equivalence class in \(A / I\) is \([ax + b]\) for some \(a, b \in \mathbb{Z} \), and \(ax + b \mapsto ai + b\) is a bijection, and \(\mathbb{Z} [\sqrt{-1} ] = \left\{ a + bi: a,b \in \mathbb{Z}  \right\} \).      
\end{remark}

\begin{eg}
    If \(A = \mathbb{Z} [x]\) and \(J = \left( x^2 + 1, 2 + x \right) \) and \(I^{\prime} = (2 + x)\), then 
    \[
        (A / I^{\prime} ) / (J / I^{\prime} ) \simeq A / J.
    \]   
    Also, 
    \[
        A / I^{\prime} \simeq \mathbb{Z},
    \]
    and 
    \[
        J = \left\{ \left( x^2 + 1 \right) f(x) + (2 + x) g(x)  \right\}, 
    \]
    so
    \[
        J / I^{\prime} = \left\{ \left( (-2)^2 + i \right) f(-2) + 0 \cdot g(-2)  \right\} = 5\mathbb{Z} . 
    \] 
    Thus, 
    \[
        J / I^{\prime} \simeq 5\mathbb{Z} .
    \]
    Hence,
    \[
        \mathbb{Z} [x] / \left( x^2 + 1, 2 + x \right) \simeq \mathbb{Z} / 5\mathbb{Z} . 
    \]
\end{eg}

\begin{eg}
    Suppose \(A = \mathbb{Z} [\sqrt{-1} ] = \left\{ a + bi \mid a, b \in \mathbb{Z}  \right\} \) and \(m = (2 + \sqrt{-1} ) = (2 + \sqrt{-1} )\mathbb{Z} [\sqrt{-1} ] \), then 
    \[
        A / m \simeq F_5 = \mathbb{Z} / 5 \mathbb{Z} 
    \] 
    since 
    \[
    \mathbb{Z} [\sqrt{-1} ] / (2 + \sqrt{-1} ) \simeq \mathbb{Z} [x] / \left( x^2 + 1, 2 + x \right) \simeq \mathbb{Z} / (5) = \mathbb{Z} / 5 \mathbb{Z}. 
    \]
\end{eg}

\begin{eg}
    How about \(\mathbb{Z} [i] / (3)\)? Suppose \(A = \mathbb{Z} [x]\) and \(J = \left( x^2 + 1, 3 \right) \) and \(I = \left( x^2 + 1 \right) \) and \(I^{\prime} = (3)\), then 
    \[
        (A / I) / (J / I) \simeq A / J.
    \]
    Note that 
    \[
        (A / I ) / (J / I) \simeq \mathbb{Z} [i] / (3).
    \]     
    Also, 
    \[
        (A / I^{\prime} ) / (J / I^{\prime} ) \simeq A / J,
    \]
    and 
    \[
        A / I^{\prime} \simeq \mathbb{F} _3 [x] \quad J / I^{\prime} \simeq \left( x^2 + 1 \right) \mathbb{F} _3[x]. 
    \]
    Thus, 
    \[
        \mathbb{Z} [i] / (3) \simeq \mathbb{F} _3 [x] / \left( x^2 + 1 \right). 
    \]
\end{eg}

Note that \(\left( x^2 + 1 \right) \) is prime if and only if \(x^2 + 1\) is irreducible. If \(x^2 + 1\) is reducible, then 
\[
    x^2 + 1 = (x - a)(x - b) \text{ with } a,b \in \mathbb{F} _3. 
\]   
However, 
\[
    0^2 + 1 = 1 \neq 0, \quad 1^1 + 1 = 2 \neq 0, 2^2 + 1 = 2 \neq 0,
\]
so neither \(0, 1, 2\) is the root of \((x - a)(x - b)\), i.e. \(x^2 + 1\) is irreducible. Since \(\left( x^2 + 1 \right) \) is maximal iff \(x^2 + 1\) is prime iff \(x^2 + 1\) is irreducible, so we know \(\left( x^2 + 1 \right) \) is maximal. 

Note that we extend \(3\) to any prime \(p \in \mathbb{Z} \). We have seen \(\mathbb{F} _p = \mathbb{Z} / p \mathbb{Z} \) for prime \(p\). Note that we have 
\[
\mathbb{F} _p [x] / \left( x^2 + 1 \right) \simeq \mathbb{Z}[i] / (p),  
\]  
and 
\begin{align*}
    \text{The primality of } p \text{ in } \mathbb{Z} [i] &\iff \text{Irreducibility of } x^2 + 1 \text{ in } \mathbb{F} _p[x] \\
    &\iff x^2 \equiv -1 \mod{p} \text{ has no solutions in } \mathbb{F} _p \\
    &\implies p \equiv 3 \mod{4}.    
\end{align*}

\begin{remark}
    \(p\) is prime in \(\mathbb{Z} [i]\) if \(p \equiv 3 \mod{4}\) and \(p\) is not prime if \(p \equiv 1, 2 \mod{4}\).     
\end{remark}

\begin{remark}
    We define 
    \[
        \left( \frac{D}{p} \right) = \begin{dcases}
            1, &\text{ if } x^2 \equiv D \mod{p} \text{ for some } x \in \mathbb{F} _p ;\\
            -1, &\text{ otherwise} .
        \end{dcases} 
    \]
    We have 
    \[
        \left( \frac{D_1}{p} \right) \left( \frac{D_2}{p} \right) = \left( \frac{D_1 D_2}{p} \right).   
    \]
\end{remark}

\begin{proposition}
    If \(K\) is a field with \(\vert K \vert < \infty  \), then \(\vert K \vert = p^m\) with \(p\) prime and \(m \ge 1\).    
\end{proposition}
\begin{proof}
    If \(n \coloneqq \underbrace{1 + 1 + \dots + 1}_{n}\) are all different, then \(\vert K \vert = \infty  \). Thus, suppose \(n = m\) with \(n > m\), then \((n - m)_K = 0_K\). Let \(\ell \) be the smallest positive integer s.t. \(\ell _k = 0_k\). If \(\ell \) is a composite number, say \(\ell = \ell _1 \times \ell _2\) for \(\ell _1, \ell _2 > 1\), then \(K\) has non-zero zero divisors since \(\ell _1, \ell _2 \neq 0\) and \(\ell _1 \ell _2 = 0\). Thus, \(\ell \) must be a prime number since all fields are integral domains, say \(\ell = p\), where \(p\) is prime. Now since \(\mathbb{F} _p = \left\{ 0, 1, 2, \dots , p-1 \right\} \subseteq K \) and form a subfield, so \(K\) is an \(\mathbb{F} _p\) vector space. Let \(v_1, v_2, \dots , v_m\) ve the basis of \(K\), then 
    \[
        K = \left\{ \sum_{i=1}^m a_i v_i \mid a_i \in \mathbb{F} _p  \right\},  
    \]                      
    so \(\vert K \vert = p^m \).  
\end{proof}

\begin{theorem}
    Let \(K\) be a field of \(q = p^m\) elements, then \(K^{\times }\) is cyclic.   
\end{theorem}
Before proof, we need the following: 
\begin{theorem}
    Let \(G\) be a finite abelian group, then 
    \[
        G \simeq \prod _{i=1}^r \mathbb{Z} / m_i \mathbb{Z} 
    \] where \(m_1 \mid m_2 \mid \dots \mid m_r\). 
\end{theorem}

\begin{theorem}
    Let \(K\) be a field of \(q = p^m\) elements, then \(K^{\times }\) is cyclic.   
\end{theorem}
\begin{proof}
    There are two useful facts: 
    \begin{itemize}
        \item [(1)] \(\vert S_d \vert = \left\vert \left\{ x \in K \mid x^d = 1 \right\} \right\vert  \le d \) for \(d \ge 1\) . 
        \item [(2)] \(G\): abelian group of \(\vert G \vert = n \) and \(\# \left\{ x \in G \mid x^d = 1 \right\} \le d \) for \(d \mid n\) implies \(G\) cyclic.      
    \end{itemize}
    We prove the first one. Note that \(a \in S_d\) iff \((x - a) \mid x^d - 1\), so \(\vert S_d \vert \le d \). Now we prove the second one. Suppose 
    \[
        G \simeq \mathbb{Z} / m_1 \mathbb{Z} \times \mathbb{Z} / m_2 \mathbb{Z} \times \dots \times \mathbb{Z} / m_{\ell }  \mathbb{Z}    
    \]    
    and \(\ell \ge 2\), i.e. not cyclic, then consider the element of 
    \[
        \left\{ x^{m_1} = 1 \right\}, 
    \] 
    then 
    \[
        \left\vert \left\{ x \in G \mid x^{m_1} = 1  \right\}  \right\vert = m_1^{\ell } \le m_1, 
    \]
    so \(\ell = 1\), which shows \(G\) cyclic. Thus, \(\ell = 1\), and thus \(K^{\times }\) is cyclic and of order \(q - 1\). Thus, 
    \[
        K^{\times } = \left\{ x \in K \mid x^{q - 1} = 1 \right\}.  
    \]     
    This shows 
    \[
        K^{\times } = \left\{ x \in K \mid x^{q-1} = 1 \right\},
    \]
    so 
    \[
        K = \left\{ x \in K \mid x^q = x \right\}. 
    \]
\end{proof}