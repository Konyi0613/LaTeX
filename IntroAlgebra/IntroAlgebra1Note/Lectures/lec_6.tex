\lecture{6}{26 Sep. 13:20}{}
\begin{definition}[Quotient Set] \label{def: quotient map}
    Let \(S\) be a set, and let \(\sim \) be an equivalence relation on \(S\). Then, the quotient set is defined to be
    \[
        S / \sim \coloneqq  \left\{ \text{equivalence classes} \right\} 
    \]    
\end{definition}

\begin{eg}
    Consider the set \(\left\{ 1,2, \dots , 10 \right\} \) and the relation is \(\equiv \mod{2}\), then
    \[
        \left\{ 1,2,\dots ,10 \right\} / (\equiv \mod{2}) = \left\{ \left\{ 1,3,5,7,9 \right\}, \left\{ 2,4,6,8,10 \right\}   \right\}. 
    \]  
\end{eg}

\begin{eg}
    \[
        \mathbb{Z} / N \mathbb{Z} = \left\{ \text{Congruence classes to } N\mathbb{Z} \text{ under the operation} \mod{N}  \right\}  
    \]
\end{eg}

\begin{definition}[Quotient map] \label{def: quotient map}
    We say \(\pi : S \to S / n\) is a "quotient map" if \(\pi (x) = \overline{x} \).  
\end{definition}

\begin{eg}
    \(\pi : \mathbb{Z} \to \mathbb{Z} / n \mathbb{Z} \). 
\end{eg}

\begin{definition}[Representative elements]
    Representative element is whatever element of an equivalence class.
\end{definition}

\begin{definition}[Complete system of representative (CSR)]
    \(R \subseteq S\) is called complete system of representative if \(R\) contains all elements that represent the quotient set without redundancy.   
\end{definition}
\begin{eg}
    \(\left\{ 0, 1, \dots , N - 1 \right\} \subseteq \mathbb{Z} \) is a CSR, while \(\left\{ 1,2,\dots , N \right\} \subseteq \mathbb{Z}  \) is another CSR. Also \(\left\{ 2N, -5N+ 1, \dots , 10N - 1 \right\}  \) is another CSR.    
\end{eg}

\begin{eg}
    \(\left\{ 0,1,2, \dots ,N \right\} \) is NOT a CSR because \(0\) and \(N\) are two representatives of the same class. Also, \(\left\{ 0, 2,3, \dots , N \right\} \) is NOT a CSR because there no representative for \(1 + N \mathbb{Z} \).      
\end{eg}

Now we talk about the quotient of group by an equivalence relation defined by its subgroup. 

\begin{definition}
    For a group \(G\) and its subgroup \(H\), we define its left coset as 
    \[
        G / H \coloneqq  G / \sim 
    \] where \(g_1 \sim g_2\) if \(\exists h \in H\) s.t. \(g_1 = g_2 h\). In the same way, the right coset is defined as 
    \[
        H \setminus G \coloneqq G / \sim 
    \]  where \(g_1 \sim g_2\) if \(\exists h \in H\) s.t. \(g_1 = h g_2\).      
\end{definition}

We first need to check \(\sim \) is an equivalence relation on \(G\). 
\begin{itemize}
    \item Reflexive: \(g = g \cdot 1_G\) 
    \item Symmetry: \(g_1 \sim g_2\) iff \(\exists h \in H\) s.t. \(g_1 = g_2 h\) and this holds if and only if \(\exists h^{\prime} \in H\) s.t. \( g_2 = g_1 h^{\prime} \). Here \(h^{\prime} = h^{-1} \) which exists because \(H\) is a subgroup. 
    \item Transitivity: If \(g_1 \sim g_2\) and \(g_2 \sim g_3\), then \(g_1 = g_2 h_1 \) and \(g_2 = g_3 h_2\) for some \(h_1, h_2 \in H\), then 
    \[
        g_1 = (g_3 h_2)h_1 = g_3 (h_2 h_1),
    \]  which shows \(g_1 \sim g_3\).         
\end{itemize}  
Thus, we verifies the well-definedness of the quotient \(G / H\), and similarly we can show \(H \setminus G\) is well-defined. 

\begin{notation}
    The element of \(G / H \) is commonly denoted as \(gH\), and the right coset is denoted by \(Hg\).    
\end{notation}

\begin{note}
    If \(H\) is clear from the context, then \(gH\) may be denoted more simply as \(\overline{g} \). 
\end{note}  

\begin{eg}
    If we have \(G = (\mathbb{Z} , +)\) and \(H = (N \mathbb{Z} , +)\), then 
    \[
        G / H = \left\{ 0 + N \mathbb{Z} , 1 + N \mathbb{Z} , \dots , (N - 1) + N \mathbb{Z}  \right\}. 
    \]
\end{eg}

\begin{remark}
    For a finite set \(S\), we denote by \(\vert S \vert = \# \text{ of elements of } S\).  
\end{remark}

\begin{theorem}
    \vphantom{text}
    \begin{itemize}
        \item \(\left\vert G / H \right\vert = \left\vert H / G \right\vert  \). 
        \item \(\left\vert gH \right\vert = \left\vert Hg \right\vert  \).  
    \end{itemize}
    given that the numbers are finite.
\end{theorem}

\begin{notation}
    \[
        \left\vert G / H \right\vert = \left\vert H \setminus G \right\vert  
    \] is called the index of \(H \subseteq G\), and denoted as \((G:H)\).   
\end{notation}

\begin{theorem}
    \[
        \vert G \vert = (G:H) \cdot \vert H \vert.
    \]
\end{theorem}

\begin{corollary}[Lagrange's theorem] \label{cl: Lagrange theorem}
    For any subgroup \(H\) of \(G\), \(H\) divides \(\vert G \vert \).
\end{corollary}

\begin{eg}
    For a prime \(p\), 
    \[
        \left( \mathbb{Z} / p \mathbb{Z}  \right) \setminus \left\{ \overline{0} \right\} = \left\{ \overline{1}, \overline{2}, \dots , \overline{p-1}   \right\}   
    \]forms a (commutative) group by "\(\cdot\)"(multiplicaiton), where we called it \(\left( \mathbb{Z} / p \mathbb{Z}  \right)^{\times } \). In this case, if we have a subgroup \(H \subseteq (\mathbb{Z} / p \mathbb{Z} )^{\times }\), then we have
    \[
        \vert H \vert \mid \left\vert \mathbb{Z} / p \mathbb{Z}  \right\vert  = p - 1.  
    \] In particular, consider the subset
    \[
        H = \left\{ \overline{1}, \overline{2}, \overline{2^2}, \dots    \right\},
    \] then it forms a subgroup. Also, if \(r\) is the smallest positive integer s.t. \(\overline{2^r} = \overline{1} \), then we know \(\vert H \vert \) is the period of \(2^n \mod{p}\), and thus this period divides \(p - 1\). 
\end{eg}