\lecture{21}{5 Dec. 13:20}{}
\begin{theorem}[First isomorphism theorem]
    Let \(\phi : R \to R^{\prime} \) be ring homomorphism, then there exists unique ring isomorphism 
    \[
        \psi : R / \ker \phi \to \Im \phi 
    \]
    s.t. \(\phi = \psi \circ \pi \) where \(\pi : R \to R / \ker \phi \) by \(r \mapsto r \ker \phi \), which is a group homomorphism. Such a \(\psi \) exists as a group homomorphism \((R / \ker \phi , +) \to \left( R^{\prime} , + \right) \). As for the multiplication, we can view 
    \[
        \psi (x y) = \psi \left( \pi (x^{\prime} )  \pi (y^{\prime} ) \right) 
    \]   
    for some \(x^{\prime} , y^{\prime} \) since \(\pi \) is surjective and then use the fact that \(\phi \) is a ring homomorphism, so 
    \[
        \psi (x y) = \psi \left( \pi (x^{\prime} ) \pi (y^{\prime} ) \right) = \psi \left( \pi (x^{\prime} y^{\prime} ) \right) = \psi \pi (x^{\prime} y^{\prime} ) = \phi (x^{\prime} y^{\prime} ) = \phi (x^{\prime} ) \phi (y^{\prime} ) = \psi (x) \psi (y).
    \]   
\end{theorem}

\begin{eg}
    \(R = K[t]\) where \(K\) is a field, then \(R\) is a PID, i.e. all ideals are of the form \((p(t))\) for some \(p(t) \in K[t]\). Hence, prime ideals are \((p(t))\) with \(p(t)\) irreducible or \(0\). Thus, \((p(t))\) is maximal iff \(p(t)\) is irreducible, and \((p(t))\) maximal means \(K[t] / (p(t))\) is a field. 
    \begin{eg}
        For \(K = \mathbb{C} \), we know \(K\) is algebraically closed, so the only irreducible polynomial are \(a(t - b)\) for \(a \neq 0\) and \(b \in \mathbb{C} \), so 
        \[
            \mathbb{C} [t] / (a(t - b)) = \mathbb{C} [t] / (t - b) \to \mathbb{C},
        \]     
        where \(\mathbb{C} [t] / (t - b) = C + ((t - b))\). Now consider 
        \[
            \phi : \mathbb{C} [t] \to  \mathbb{C}, \quad f(t) \mapsto f(b),
        \] 
        then this is a surjective ring homomorphism, which is trivial. Thus. by first isomorphism theorem, 
        \[
            \mathbb{C} [t] / \ker \phi \simeq \mathbb{C} ,
        \]
        where \(\ker \phi = \left\{ f(t) \mid f(b) = 0 \right\} = \left\{ f(t) \mid (t - b) \mid f(t) \right\}  \), so 
        \[
            \mathbb{C} [t] / (t - b) \simeq \mathbb{C} .
        \] 
    \end{eg}

    \begin{eg}
        For \(K = \mathbb{R} \), then the irreducible polynomials are 
        \[
            f_1(t) = a(t - b) \text{ for } a\neq 0, b \in \mathbb{R} , f_2(t) = a(t^2 - bt + c) \text{ for } a\neq 0, b^2 - 4ac < 0.  
        \] 
        Hence, 
        \[
            \mathbb{R} [t] / (f_1(t)) \simeq \mathbb{R} , \quad \mathbb{R} [t] / (f_2(t)) \simeq \mathbb{R} [u] / (u^2 + 1).
        \]
        Note that \(\mathbb{C} = \mathbb{R} [\sqrt{-1} ]\), so 
        \[
            \phi : \mathbb{R} [u] \to \mathbb{R} [\sqrt{-1} ], \quad u \mapsto \sqrt{-1} 
        \] 
        is a surjective ring homomorphism, and thus 
        \[
            \mathbb{R} [u] / \ker (\phi ) \simeq \mathbb{R} [\sqrt{-1} ] = \mathbb{C} ,
        \]
        and \(\ker \phi = (u^2 + 1)\). Hence, \(\mathbb{R} [u] / (u^2 + 1) \simeq \mathbb{C} \).
    \end{eg}
\end{eg}

\begin{theorem}[Second isomorphism theorem]
    Suppose \(B\) is a ring and \(A \subseteq B\) is a subring and \(I \subseteq B\) is an ideal of \(B\), and \(\pi : B \to B / I\) is the natural ring homomorphism, then \(I \cap A\) is an ideal of \(A\) and 
    \[
        A / (I \cap A) \simeq \pi (A).
    \]    
\end{theorem}
\begin{proof}
    Let \(x \in I \cap A\) and \(a \in A\), then 
    \[
        \begin{dcases}
            ax \in I \text{ since } I \subseteq B \text{ is an ideal} \\
            ax \in A \text{ since } a, x \in A,   
        \end{dcases}
    \]  
    so \(ax \in I \cap A\) and thus \(I \cap A \subseteq A\) is an ideal. Now consider \(\pi \vert _A\), so we know 
    \[
        A / \ker \pi \vert_A \simeq \pi (A).
    \]   
    Note that \(\ker \pi\vert_A = I \cap A\).  
\end{proof}

\begin{eg}
    
\end{eg}

\begin{theorem}
    Let \(A\) be a ring and \(I \subseteq A\) be an ideal and \(\pi : A \to A / I\) is the natural homomorphism, and 
    \[
        X = \left\{ \text{ideals of } A / I   \right\}, \quad Y = \left\{ \text{ideals of } A \text{ containing } I   \right\}.  
    \]   
    Then there is a bijection between \(X\) and \(Y\).   
\end{theorem}
\begin{proof}
    Consider 
    \[
        \phi : X \to Y, \quad J \mapsto \pi ^{-1} (J) = \left\{ a \in A \mid \pi (a) \in J \right\}, 
    \]
    and 
    \[
        \psi : Y \to X, \quad J \mapsto \pi (J),
    \]
    then this gives a bijection. We first show the well-definedness of \(\phi \). 
    \begin{itemize}
        \item \(\pi ^{-1}(J)\) is an ideal containing \(I\) since 
        \[
            \pi \left( A \pi ^{-1}(J) \right) = \pi (A) \cdot \pi ^{-1} (J) \subseteq J 
        \]
        since \(\pi (A) = A / I\) and \(\pi (\pi ^{-1} (J)) = J\). Hence, we know 
        \[
            A \pi ^{-1}(J) \subseteq \pi ^{-1} (J).
        \]
        Now since 
        \[
            I = \pi ^{-1} \left( 0_{A / I} \right) \subseteq \pi ^{-1} (J), 
        \]
        so we're done.
        \item well-definedness of \(\psi \): For \(J \subseteq A\) s.t. \(I \subseteq J\), we want to show \(\pi (J) \subseteq A / I \) is an ideal. For \(a + I \in A / I\) and \(x + I \in \pi (J)\),
        \[
            (a + I)(x + I) = ax + xI + aI + I^2 \subseteq ax + I \subseteq J + I = J.
        \]
    \end{itemize} 
\end{proof}

\begin{theorem}[3rd Isomorphism theorem]
    \(A\) is a ring and \(I \subseteq J \subseteq A\) are ideals, then 
    \begin{itemize}
        \item [(1)] \(\phi : A / I \to A / J\) defined by \(x + I \mapsto x + J\) is a well-defined surjective ring homomorphism with \(\ker \phi = J / I \subseteq A / I\). 
        \item [(2)] \((A / I) / (J / I) \simeq A / J\).   
    \end{itemize}  
\end{theorem}

