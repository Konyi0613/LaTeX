\lecture{18}{26 Nov. 13:20}{}
\begin{definition}
    We say a ring \(A\) is an integral domain if \(A\) is not the trivial ring and for all \(a, b \neq 0\) we have \(ab \neq 0\), i.e. there is no zero divisors.    
\end{definition}

\begin{definition}
    We say a ring \(A\) is a field if \(A\) is not the trivial ring and for all \(a \in A \setminus \left\{ 0 \right\} \), \(\exists b \in A\) s.t. \(ab = ba = 1_A\), i.e. 
    \[
        A^{\times } = A \setminus \left\{ 0 \right\}. 
    \]     
\end{definition}

\begin{remark}
    In fact, we can generalize integral domains to field of fractions. Thr prototype is 
    \[
        \mathbb{Z} \to \mathbb{Q} = \left\{ \frac{p}{q} \mid p \in \mathbb{Z} , q \in \mathbb{Z} \setminus \left\{ 0 \right\}  \right\}. 
    \]
    Here the meaning of \(\frac{p}{q}\) is 
    \[
        (p, q) \sim \left( p^{\prime} , q^{\prime}  \right) \iff \exists r, r^{\prime}  \text{ s.t. } r \cdot p = r^{\prime} \cdot p^{\prime} \text{ and } r \cdot q = r^{\prime} \cdot q^{\prime}.   
    \] 
    And we can define 
    \begin{itemize}
        \item 
        \[
            \frac{p}{q} + \frac{p^{\prime} }{q^{\prime} } = \frac{pq^{\prime} +p^{\prime} q}{qq^{\prime} }, \text{ i.e. } (p, q) + \left( p^{\prime} , q^{\prime}  \right) = \left( pq^{\prime} + p^{\prime} q, qq^{\prime}  \right).     
        \]
        \item 
        \[
            \frac{p}{q} \cdot \frac{p^{\prime} }{q^{\prime} } = \frac{pp^{\prime} }{qq^{\prime} }, \text{ i.e. } (p, q) \cdot \left( p^{\prime} , q^{\prime}  \right) = \left( pp^{\prime} , qq^{\prime}  \right).   
        \]
    \end{itemize}
    Also, there is an injective homomorphism:
    \[
        \mathbb{Z} \to \mathbb{Q}, \quad m \mapsto \frac{m}{1}.
    \]

    The same construction is possible for 
    \[
        \mathbb{C}[t] \to \mathbb{C}(t) = \left\{ \frac{p(t)}{q(t)} \mid p(t) \in \mathbb{C} [t], q(t) \in \mathbb{C} [t] \setminus \left\{ 0 \right\}  \right\}.  
    \]
    In general, given an integral domain \(A\), 
    \[
        A \sim Q(A) = \left\{ (p, q) \in A \times A\setminus \left\{ 0 \right\}  \right\} / \sim  
    \] 
    where \((p, q) \sim \left( p^{\prime} , q^{\prime}  \right) \iff pq^{\prime} = p^{\prime} q\). 
\end{remark}

\begin{proposition}
    \(\varphi : A \to Q(A)\) where \(\varphi (a) = (a, 1)\) is an injective ring. 
\end{proposition}
\begin{proof}
    Check that \(\ker \varphi = \left\{ 0 \right\} \) and 
    \begin{align*}
        \varphi (a + b) &= \varphi (a) + \varphi (b) \\
        \varphi (ab) &= \varphi (a) \varphi (b) \\
        \varphi (1_A) &= 1_{Q(A)}. 
    \end{align*} 
\end{proof}

\begin{corollary}
    \(Q(A)\) is a field.
\end{corollary}
\begin{proof}
    Note that 
    \[
        (p, q)^{-1} = (q, p).
    \]
\end{proof}

Modules are the generalization of vector spaces
\begin{definition}
    A vector space \(V / k\) (or \(V\) over \(k\)) where \(k\) is a field means 
    \begin{itemize}
        \item [(1)] \(V\) is an abelian group (vector can be added)
        \item [(2)] \(\forall r \in k\), \(\forall v \in V\), \(rv \in V\). 
        \item [(3)] \(r(v + v^{\prime} ) = rv + rv^{\prime} \) and \(\left( r + r^{\prime}  \right)v = rv + r^{\prime} v \) for \(r, r^{\prime} \in k\) and \(v, v^{\prime} \in V \).        
    \end{itemize} 
\end{definition}

\begin{definition}
    Modules \(M / R\) has same definition as vector spaces if replacing \(k\) (a field) with \(R\) (a ring), and \(V\) (vector spaces) by \(M\) (modules). In this case, \(M\) is called an \(R\)-module.       
\end{definition}

\subsubsection{Comparision between vector spaces and modules}
If \(V\) is a vector space over \(\mathbb{R} \) (of finite dimension), then \(V \simeq \mathbb{R} ^n\) with some \(n\). However, there are huge varieties of \(\mathbb{Z} \)-modules. 
For example, \(M = \mathbb{Z} ^n\) is a \(\mathbb{Z} \)-module, and \(M = \mathbb{Z} / 5 \mathbb{Z} \) is also a \(\mathbb{Z} \)-module. In fact, any module under any ring is a \(\mathbb{Z} \)-module. Hence, for an \(R\)-module, \(M\), we can always view \(M\) as a \(\mathbb{Z} \)-module. 

\begin{remark}
    We can view every module as a \(\mathbb{Z} \) module since for all \(m \in \mathbb{Z} \) and \(x \in M\) , we can define 
    \[
      m \cdot x = x + x + \dots + x \in M,  
    \]  
    and thus we know all rules in the definition of modules hold.
\end{remark}

\begin{definition}
    We call \(M^{\prime} \) a sub \(R\)-module if \(M\) is a \(R\)-module and \(M^{\prime} \subseteq M\) forms an \(R\)-module.      
\end{definition}

\begin{definition}
    An ideal \(I\) of a ring \(R\) is a sub \(R\)-module of \(R\), i.e. 
    \begin{itemize}
        \item \(I\) is a subgroup of \((R, +)\). 
        \item For \(r \in R\), \(x \in I\), \(r \cdot x \in I\). 
        \item Distributive.   
    \end{itemize}  
\end{definition}

\begin{eg}
    If \(R = \mathbb{Z} \), then all ideals are of the form \(m \cdot \mathbb{Z} \). 
\end{eg}