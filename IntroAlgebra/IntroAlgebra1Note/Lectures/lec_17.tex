\lecture{17}{19 Nov. 13:20}{}
If \(A\) is a ring and \(0_A = 1_A\), then for \(x \in A\), 
\[
    x = x \cdot 1_A = x \cdot 0_A = 0_A,
\] so \(A = \left\{ 0 \right\} \), which is called a singleton or zero ring.

\subsubsection{Division}
We define \(z = \frac{x}{y}\) if \(x = y \cdot z\) for some \(z\) existing uniquely. Thus, division by \(0\) is not defined. Thus, division isn't defined for every element of a ring. 

\begin{definition}
    A ring \(A\) is called a division ring if \(x \in A \setminus \left\{ 0 \right\} \) has an inverse.   
\end{definition}

\begin{remark}
    Zero ring is excluded usually.
\end{remark}

\begin{remark}
    A division ring \(A\) is called a field if \(A\) is commutative. 
\end{remark}

There are so many rings other than the one we usually deal with.
\begin{definition}[Zero divisors]
    If \(a, b \in A \setminus \left\{ 0 \right\} \) satisfies \(ab = 0\), then \(a, b\) are zero divisors. 
\end{definition}

\begin{eg}
    In \(\mathbb{R} \times \mathbb{R} \), if we define 
    \[
        \begin{dcases}
            (a, b) + (c, d) = (a+c, b+d) \\
            (a, b) \cdot (c, d) = (ac, bd),
        \end{dcases}
    \] 
    then \(\mathbb{R} \times \mathbb{R} \) has zero divisors:
    \[
        (1, 0) \cdots (0, 3) = (0, 0).
    \] 
\end{eg}

\begin{definition}
    The ring without zero-divisors are called integral domains.
\end{definition}

\begin{definition}[Subrings]
    For a ring \(R\), if a subset \(S \subseteq R\) forms a ring with the same ring structure as \(R\), then \(S\) is called a subring.    
\end{definition}

\begin{eg}
    If \(R = \mathbb{Z} \), then is there any subring of \(\mathbb{Z} \)? 
\end{eg}
\begin{explanation}
    First, subgroups of \(\mathbb{Z} \) are of the forms \(n \cdot \mathbb{Z} \), but \(1 \notin n\mathbb{Z} \) if \(n \neq 1\), so \(\mathbb{Z} \) doesn't have any nontrivial subring.     
\end{explanation}

\begin{eg}
    If \(R = \mathbb{Q} \), then is there any subring? 
\end{eg}
\begin{explanation}
    Consider
    \[
        \mathbb{Z} \left[ \frac{1}{2} \right] = \left\{ \frac{n}{2^{\ell } } \mid \ell \ge 0, n \in \mathbb{Z}  \right\}  
    \]
    and 
    \[
        \mathbb{Z} _{(2)} = \left\{ \frac{n}{m} \mid n \in \mathbb{Z} , m \text{ is odd}  \right\},
    \] they are both subrings of \(R\). 
\end{explanation}

Suppose \(R, S\) are rings, then if 
\begin{align*}
    &\phi :R \to S \\
    &\phi(x + y) = \phi (x) + \phi (y) \\
    &\phi (x \cdot y) = \phi (x) \cdot \phi(y) \\
    &\phi (1_R) = 1_S \\
    &\phi (0_R) = 0_S,
\end{align*} 
then \(\phi \) is a ring homomorphism and \(\ker \phi \) forms an ideal of \(R\) (in group homomorphism it is a normal subgroup).    