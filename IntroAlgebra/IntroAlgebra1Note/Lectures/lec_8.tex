\lecture{8}{3 Oct. 13:20}{}
\begin{prev}
    We want to solve a question: For what \(H < G\), does \(G / H\) form a group by 
    \[
        (g_1 H)(g_2 H) = (g_1 g_2)H.
    \]  
    \begin{note}
        \(g^{-1} H g = H \) for all \(g \in G\) iff \(\forall g \in G\) and \(h \in H\), \(g^{-1} h g \in H\).      
    \end{note}
    We have the answer is \autoref{thm: G/H group iff H nomral}.
\end{prev}

\begin{eg}
    If \(G\) is abelian, then every subgroup is normal. 
\end{eg}
\begin{explanation}
    Let \(H < G\) and \(h \in H\), \(g \in G\), then \(g^{-1} h g = g^{-1} g h = h \in H\), so \(H \trianglelefteq G\).     
\end{explanation}

\begin{eg}
    If \(G = S_3\), show that \(V_3 = \left\{ (1), (123), (132) \right\} \) form a normal subgroup, where 
    \[
        \left\{ (1), (12) \right\}, \ \left\{ (1), (13) \right\}, \ \left\{ (1), (23) \right\}   
    \] are not normal subgroups. 
\end{eg}

\begin{eg}
    If \(G = \mathrm{GL}_n(\mathbb{R}) = \left\{ \text{invertible } n\times n \text{ real matrices} \right\} \), then
    \[
        \mathrm{SL}_n(\mathbb{R} ) = \left\{ g \in \mathrm{GL}_n(\mathbb{R} ) \mid \det g = 1 \right\}  
    \]
    forms a normal subgroup of \(G\).    
\end{eg}
\begin{explanation}
    It is enough to show 
    \[
        \forall g \in G, h \in H \implies g^{-1} h g \in H.
    \]Since \(h \in SL_n(\mathbb{R} )\) and \(\det h = 1\), then 
    \[
        \det \left( g^{-1} h g  \right) = \det \left( g^{-1} \right) \det (h) \det (g) = \det \left( g^{-1} g  \right) \det (h) = 1 \cdot 1 = 1. 
    \] Thus, \(g^{-1} h g \in H \), and thus \(H \trianglelefteq G\).  
\end{explanation}

\begin{eg}[First isomorphism theorem]
    Let \(\phi : G \to H\) be a group homomorphism, then 
    \begin{itemize}
        \item [(1)] \(\Im \phi < H\). 
        \item [(2)] \(\ker \phi \trianglelefteq G\). 
        \item [(3)] \(G / \ker \phi \simeq \Im \phi \).   
    \end{itemize} 
\end{eg}
\begin{explanation}
    \vphantom{text}
    \begin{itemize}
        \item [(1)] Enough to show 
        \begin{itemize}
            \item [(i)] For \(h_1, h_2 \in \Im \phi \), \(h_1 \cdot h_2 \in \Im \phi \).  
            \item [(ii)] \(\forall h \in \Im \phi \), \(h^{-1} \in \Im \phi  \).   
        \end{itemize}
        For (i), \(\exists g_1, g_2 \in G\) s.t. \(h_1 = \phi \left( g_1 \right) \) and \(h_2 = \phi \left( g_2 \right) \), then \(h_1 h_2 = \phi (g_1) \phi (g_2) = \phi (g_1 g_2)\), so \(h_1 h_2 \in \Im \phi \). For (ii), for \(h \in H\), \(\exists g \in G\) s.t. \(h = \phi (g)\), so 
        \[
            h^{-1} = \phi (g)^{-1} = \phi (g^{-1}) \in \Im \phi .  
        \]       
        \item [(2)] Enough to show 
        \begin{itemize}
            \item [(i)] \(\ker \phi < G\)
            \item [(ii)] \(g \in G, h \in \ker \phi \), \(g^{-1} h g \in \ker \phi  \).   
        \end{itemize}
        We first show (i). Let \(g_1, g_2 \in \ker \phi \), then \(\phi (g_1) = \phi (g_2) = 1\). Thus, \(\phi (g_1 g_2) = \phi (g_1) \phi (g_2) = 1\), and thus \(g_1 g_2 \in \ker \phi \). Now for \(g \in \ker \phi \), we have \(\phi (g) = 1\). Thus, \(\phi \left( g^{-1}  \right) = \phi (g)^{-1} = e_H^{-1} = e_H  \), so \(g^{-1} \in \ker \phi \). Now we show (ii). Let \(g \in G\) and \(h \in \ker \phi \), then \(\phi (h) = 1\). Now since
        \[
            \phi  \left( g^{-1} h g \right) = \phi \left( g^{-1} \right) \phi (h) \phi (g) = \phi \left( g g^{-1} \right) \phi (h) = 1 * 1 = 1,   
        \] so \(g^{-1} h g \in \ker \phi \).
        \item [(3)] Let \(N = \ker \phi \), and note that the map we want is something like \(g \mapsto \phi(g)\). We can think of decomposing \(\phi \) to  
        \begin{align*}
            &\underbrace{G \to G / \ker (\phi )}_{\text{surj}} \to \underbrace{\Im \phi \to H}_{\text{inj}}. \\
            &g \mapsto \overline{g} \mapsto \phi (g) \mapsto \phi (g), 
        \end{align*}
        where the \(G / \ker(\phi ) \to \Im (\phi )\) part is an isomorphism, and we call it \(\widetilde{\phi }: G / \ker \phi \to \Im \phi  \). We have to show that the map is well-defined first, suppose 
        \[
            \overline{g} = \left\{ g_1, g_2, g_3, \dots  \right\},  
        \]then  we want to show \(\phi (g_1) = \phi (g_2) = \phi (g_3)\). More precisedly, we have to check that if \(g_1 N = g_2 N\), then \(\phi (g_1) = \phi (g_2)\). Since \(g_1 N = g_2 N\), so \(g_2 = g_1 n\) for some \(n \in N\). Thus,
        \[
            \phi (g_2) = \phi (g_1 n) = \phi (g_1) \phi (n) = \phi (g_1).
        \] Thus, the map is well-defined. Then, we have to show that the \(\overline{g} \mapsto \phi (g) \) part is bijective and it is an homomorphism. For surjectivity. Let \(h \in \Im \phi \), then \(\exists g \in G\) s.t. \(h = \phi (g)\). By well-definedness of \(\widetilde{\phi } \), we know \(h = \widetilde{\phi }(gN) \in \Im \widetilde{\phi }  \). Next we show the injectivity. Assuming the homomorphy of \(\widetilde{\phi } \), it is enough to show \(\ker \widetilde{\phi } = \left\{ \overline{1}  \right\} = \overline{N} \in G / N  \). Hence, we want to show that if \(gN \in \ker \widetilde{\phi } \), then \(gN=N\). Suppose \(gN \in \ker \widetilde{\phi } \), then \(\phi (g) = \widetilde{\phi }(gN) = 1 \). Thus, \(g \in \ker \phi = N\). Hence, \(gN = N\). (Since \(g^{-1} \in \ker \phi  \)) Next, we show the homomorphy: 
        \[
            \widetilde{\phi }(g_1 N * g_2 N) = \widetilde{\phi }((g_1 * g_2)N)  = \phi (g_1 * g_2) = \phi (g_1) \phi (g_2) = \widetilde{\phi }(g_1 N) \widetilde{\phi }(g_2 N)    
        \] since \(N\) is normal,  so \(\widetilde{\phi } \) is an homomorphism. 
        
        Combining the well-definedness, surjectivity, injectivity, and group homomorphism, we know \(\widetilde{\phi } \) is an isomorphism.
    \end{itemize}
\end{explanation}

\begin{eg}
    Consider
    \[
        \det : \mathrm{GL}_n(\mathbb{R} ) \to \mathbb{R} ^{\times } \left( = \left( \mathbb{R} \setminus \left\{ 0 \right\}  \right), \cdot  \right), 
    \] then \(\Im \phi = \mathbb{R} ^{\times }\), and \(\ker \phi = \left\{ g \in \mathrm{GL}_n(\mathbb{R} ) \mid \det (g) = 1  \right\} \). Hence, 
    \[
        G / \ker \phi = \mathrm{GL}_n(\mathbb{R} ) / \mathrm{SL}_n(\mathbb{R} ) = \left\{ g \cdot \mathrm{SL}_n(\mathbb{R} ) \mid g \in \mathrm{GL}_n(\mathbb{R} )   \right\},   
    \]  which means each equivalence class contains matrices with same determinant, and it is isomorphic to \(\mathbb{R} ^{\times }\).  
\end{eg}

\section{Direct Product (= Cartesian Product)} 
\begin{proposition} \label{prop: normal and cap is zero means abelian}
    Let \(G\) be a group and \(H, K \trianglelefteq G\) s.t. \(H \cap K = \left\{ 1 \right\} \), then for \(h \in H\) and \(k \in K\), \(hk = kh\).      
\end{proposition}
\begin{proof}
    The goal is \(hk = kh\), which means \(h^{-1} k^{-1} hk =1\). Note that \(h^{-1} k^{-1} h \in K\) and \(k \in K\), so \(h^{-1} k^{-1} h k \in K\). Also, \(h^{-1}\in H\) and \(k^{-1} h k \in H\), so \(h^{-1} k^{-1} h k \in H\). Hence, \(h^{-1} k^{-1} h k \in H \cap K = \left\{ 1 \right\} \).          
\end{proof}

\begin{proposition}
    Suppose \(H, K \trianglelefteq G\) satisfy 
    \[
        \begin{dcases}
            H \cap K = \left\{ 1 \right\} \\
            H \cdot K = \left\{ h \cdot k \mid h \in H, k \in K \right\} = G, 
        \end{dcases}
    \] then 
    \begin{align*}
        \phi : H \times K &\to G \\
        (h, k) &\mapsto hk
    \end{align*}
    is an isomorphism. Note that in \(H \times K\), for \((h_1, k_1), (h_2, k_2) \in H \times K\), we have
    \[
        (h_1, k_1) \cdot (h_2, k_2) = (h_1 h_2, k_1 k_2).
    \]  
\end{proposition}
\begin{proof}
    \vphantom{text}
    \begin{itemize}
        \item [(1)] Homomorphy: Let \((h_1, k_1), (h_2, k_2) \in H \times K\), then 
        \[
            \phi \left( (h_1, k_1) \cdot (h_2, k_2) \right) = \phi ((h_1 h_2, k_1 k_2)) = h_1 h_2 k_1 k_2 = h_1 k_1 h_2 k_2 = \phi (h_1 k_1) \phi (h_2 k_2) 
        \] by \autoref{prop: normal and cap is zero means abelian}. 
        \item [(2)] Surjectivity: Trivial. 
        \item [(3)] Injectivity: Need to show \(\ker \phi = \left\{ 1 \right\} \). Let \((h , k)  \in \ker \phi \), then \(hk = 1\). Thus, \(h = k^{-1} \in K\), and \(h \in H\), so \(h \in H \cap K = \left\{ 1 \right\} \), so \(h = k = 1\).     
    \end{itemize}
    By (1), (2), (3), we know \(\phi \) is an isomorphism. 
\end{proof}

\begin{theorem}
    If \((m, n) = 1\), then 
    \[
        \mathbb{Z} / (mn) \mathbb{Z} \simeq \mathbb{Z} / n \mathbb{Z} \times \mathbb{Z} / m \mathbb{Z}.
    \] 
\end{theorem}