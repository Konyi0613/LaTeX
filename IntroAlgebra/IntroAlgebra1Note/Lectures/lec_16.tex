\chapter{Ring theory}
\lecture{16}{14 Nov. 13:30}{}
\begin{definition}[Ring]
    A set \(A\) is called a ring if it has two binary operations \(+\) and \(\cdot\) satisfying the following conditions:
    \begin{itemize}
        \item \((A, +)\) is an abelian group.
        \item \((A, \cdot)\) is a monoid. (Only has associativity and identity). 
        \item \(+\) and \(\cdot\) are coherent in the following way:
        \[
            \begin{dcases}
                (a+b) \cdot c = a \cdot c + b \cdot c \\
                a \cdot (b+c) = a \cdot b + a \cdot c.
            \end{dcases}
        \]
    \end{itemize}   
\end{definition}

\begin{eg}
\(\mathbb{Z} , \mathbb{Q} , \mathbb{R} , \mathbb{C} \) are rings. 
\end{eg}

\begin{note}
    If \(\cdot\) is commutative, then \(A\) is called a commutative ring.  
\end{note}

\begin{eg}
    \(\mathbb{Z} , \mathbb{Q} , \mathbb{R} , \mathbb{C} \) are all commutative rings, while 
    \[
        M_n(\mathbb{Z} ) = \left\{ n \times n \text{ matrices with integers entries}  \right\} 
    \] is non-commutative if \(n \ge 2\).
\end{eg}

Given a ring \(A\), we know \((A, +)\) is an abelian group, and 
\[
    A^{\times } = \left\{ a \in A \mid \exists b \text{ s.t. } ab=ba=1_A \right\} 
\] forms a group called multiplication group of \(A\) sisnce for \(a, a^{\prime} \in A^{\times }\) we know \(a a^{\prime} \in A^{\times }\). 

\begin{eg}
    Let \(G\) be a finite group, \(A\) commutative ring. The group ring is the ring denoted by \(A[G]\), defined as: 
    \begin{itemize}
        \item underlying set: 
        \[
            \left\{ \sum_{g \in G} a_g \cdot g \mid a_g \in A  \right\}. 
        \]
        \item Addition \(+\): If 
        \[
            \begin{dcases}
                a = \sum a_g \cdot g \\
                b = \sum b_g \cdot g, 
            \end{dcases}
        \]then \(a + b =\sum \left( a_g + b_g \right) \cdot g  \).
        \item Multiplication: 
        \[
            ab = \sum_{g, h \in G} a_g b_h (g \cdot h) = \sum_{\sigma \in G} \left( \sum_{gh = \sigma } a_g b_h  \right) \sigma.   
        \]
    \end{itemize}   

    For example, \(G = \mathbb{Z} / 3 \mathbb{Z} = \left\{ \overline{0}, \overline{1}, \overline{2}    \right\} \), then the group ring is 
    \[
        \left\{ a_0 \cdot \overline{0} + a_1 \cdot \overline{1} + a_2 \cdot \overline{2} \mid a_0, a_1, a_2 \in A \right\},
    \] where for \(a = a_0 \overline{0} + a_1 \overline{1} + a_2 \overline{2}   \) and \(b = b_0 \overline{0} + b_1 \overline{1} + b_2 \overline{2}   \) we know 
    \begin{itemize}
        \item \(a + b = (a_0 + b_0) \overline{0} + (a_1 + b_1) \overline{1} + (a_2 + b_2) \overline{2}   \). 
        \item 
        \[
            a \cdot b = (a_0 b_0) (\overline{0} \overline{0}  ) + \dots + a_2 b_2 (\overline{2} \overline{2}  ).
        \]
    \end{itemize}  
\end{eg}

\begin{note}
    In representation theorey of \(G\), group ring is very important (Group cohomology). 
\end{note}

\begin{eg}
    Let \(D\) be a square-free integer, then 
    \[
        \mathbb{Z} [\sqrt{D} ] = \left\{ a + b \sqrt{D} \mid a, b \in \mathbb{Z}   \right\}.
    \] 
    For \(\alpha = a + b \sqrt{D} \) and \(\beta = c + d \sqrt{D} \), then \(\alpha + \beta = (a + c) + (b + d) \sqrt{D} \) and 
    \[
        \alpha \beta = (a + b\sqrt{D} ) (c + d\sqrt{D}) = (ac + bd D) + (ad + bc)\sqrt{D}, 
    \] so \(\mathbb{Z} [\sqrt{D} ]\) forms a ring. On the other hand, 
    \[
        \left\{ a + b\sqrt{2} + c\sqrt{3} \mid a, b, c \in \mathbb{Z}    \right\} 
    \] doesn't form a ring, while 
    \[
        \left\{ a + b\sqrt{2} + c\sqrt{3} + d\sqrt{6} \mid a,b,c,d \in \mathbb{Z} \right\} 
    \] forms a ring.
\end{eg}

\begin{eg}[Quaterums]
\[
    \mathbb{H} = \left\{ a + bi + cj + dk \mid a,b,c,d \in \mathbb{R}  \right\}
\] where \(i, j, k\) are imaginary units 
\[
    i^2 = j^2 = k^2 = -1, \quad ij = k, jk = i, ki = j, \quad ji = -k, kj = -i, ik = -j.
\]
Hence, \(H\) is a non-commutable ring.  
\end{eg}

Hence, we know 
\[
    \underbrace{\mathbb{R} \subseteq \mathbb{C} \subseteq \mathbb{H}}_{\text{rings in our def}}  \subseteq \underbrace{\mathbb{O}}_{\text{NOT associative}}, 
\]
Note that 
\begin{align*}
    \left\{ x \in \mathbb{R} \mid x^2 = -1 \right\} &= \varnothing \\
    \left\{ x \in \mathbb{C} \mid x^2 = -1 \right\} &= \left\{ \pm i \right\} \\
    \left\{ x \in \mathbb{H} \mid x^2 = -1 \right\} &= \left\{ ai + bj + ck \mid a^2 + b^2 + c^2 = 1 \right\} \simeq S^2 \text{ (2 dimensional ball)}  \\
    \left\{ x \in \mathbb{O} \mid x^2 = -1 \right\} &= \left\{ a_1 i_1 + a_2 i_2 + \dots + a_7 i_7 \mid a_1^2 + \dots + a_7^2 = 1 \right\} \simeq S^6 \text{ (6 dimensional sphere)}.      
\end{align*}

\begin{eg}
    \[
        C^{\infty }(\mathbb{R} ) = \left\{ \text{real functions differentiable infinitely times}  \right\}. 
    \]
    Given a set \(X\) (with some geometry), 
    \[
        C(X) = \left\{ \text{functions with conditions}  \right\} \subseteq \mathrm{Map}(X, A),  
    \] where \(A\) is a ring, so \(f \cdot g \in C(X)\) gives \(fg \in C(X)\) by \((f \cdot g)(x) = f(x) \cdot g(x)\).    
\end{eg}

In general, given a space, considering certain class of functions on that space is a very important idea of investigating the space, and in this way. One may study the ring (or modules) of functions.

\begin{note}
    If \(0_A = 1_A\), \(A = \left\{ 0_A \right\} \), and in many statements, we need to exclude this case.  
\end{note}

Next, we consider the maps between rings. 
\begin{definition}[Ring homomorphism/isomorphism]
    Let \(A, B\) be rings, and 
    \[
        f:A \to B 
    \] is called a ring homomorphism if it respects the ring statements, i.e. 
    \begin{itemize}
        \item \(f(x+y) = f(x) + f(y)\) for all \(x,y \in A\). 
        \item \(f(xy) = f(x) f(y)\) for all \(x, y \in A\). 
        \item \(f(1_A) = 1_B\).  
    \end{itemize}
    If \(f:A \to B\) has an inverse, then \(f\) is said to be an isomorphism, denoted as \(A \simeq B\).  
\end{definition}

\begin{proposition}
    If \(f:A \to B\) and \(g:B \to C\) are ring homomorphisms, then \(g \circ f : A \to C\) is a ring homomorphism.   
\end{proposition}

\begin{note}
    Thus, we may define
    \[
        \mathrm{Aut}^{\mathrm{alg}}(A) = \left\{ \text{All ring automorphisms of } A  \right\} (=\text{isomorphism from } A \text{ to itself}),
    \] and this forms a group.
\end{note}