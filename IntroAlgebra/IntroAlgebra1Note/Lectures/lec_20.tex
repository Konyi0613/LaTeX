\lecture{20}{3 Dec. 13:20}{}
Recall that we have \(R / \ker \phi \simeq \Im \phi \) if \(R\) is a group. Now let \(R\) be a ring, and \(I \subseteq R\) be an ideal, then 
\[
    \phi : R \to R / I, \quad r \mapsto r + I = \left\{ r + a \mid a \in I \right\}. 
\]    
is a group homomorphism (since we assume \(R\) is a ring and thus abelian in addition and thus \(I\) is normal in \(R\) and thus \(R / I\) is a group). Thus, we have 
\[
    (x + I) \cdot (y + I) = xy + xI + yI + I^2 = xy + I.
\]
Hence,
\[
    R / \ker \phi \simeq \Im \phi 
\]
is naturally viewed as a ring isomorphism. 

\begin{proposition}
    \(P\) is a prime ideal of \(R\) iff \(R / P\) is an integral domain.   
\end{proposition}
\begin{proof}
    Since \(x, y \notin P\) implies \(xy \notin P\), so if \(x \neq \overline{0} \) and \(y \neq \overline{0} \), then \(xy \neq \overline{0} \). Note that \(a = \overline{0} \) iff \(a \in P\).     
\end{proof}

\begin{proposition}
    \(m\) is a maximal ideal of \(R\) iff \(R / m\) is a field.   
\end{proposition}
\begin{proof}
    Need to show \(x \neq \overline{0} \in R / m \) is invertible. Suppose \(x^{\prime} \in R\) s.t. \(\overline{x^{\prime} } = x\). By assumption, \(x^{\prime} \notin m\). Consider 
    \[
        (x^{\prime} ) + m = \left\{ a \cdot x^{\prime}  + p: a \in R, p \in m \right\}, 
    \]    
    then \(m \subsetneq m + (x^{\prime} ) \subseteq R\) since \(x^{\prime} \in (x^{\prime} ) + m\) but \(x^{\prime} \notin m\). Hence, \(m + (x^{\prime} ) = R\). Thus, there exists \(\alpha \in m\) and \(r \in R\) s.t. 
    \[
        1 = \alpha + r x^{\prime},
    \]      
    and by taking modulo \(m\), we know 
    \[
        \overline{1} = \overline{r} \cdot x,  
    \] 
    which shows \(x\) is invertible. 
\end{proof}

\begin{eg}
    If \(R = K\) is a field, then \(a \in K \setminus \left\{ 0 \right\} \) generates \(K\), so the only ideals of \(K\) are \(K\) and \(\left\{ 0 \right\} \).      
\end{eg}

\begin{eg}
    If \(R = \mathbb{Z} \), then \(n\mathbb{Z} \) are the only ideals and \(p\mathbb{Z} \) for prime \(p\) are the only prime ideals. Also, \(p\mathbb{Z} \) for prime \(p\) are the only maximal ideals. Hence, \(\mathbb{Z} / p \mathbb{Z} \) and \(\mathbb{Z} / 0 \mathbb{Z} \) are integral domains and \(\mathbb{Z} / p\mathbb{Z} \) is a fields.      
\end{eg}

\begin{eg}
    If \(R = \mathbb{Q} [t]\), and every ideal is principle, i.e. of the form \((p(t))\) by Euclidean algorithm. As for the prime ideal, if \(I = (p(t))\) and \(p(t)\) is irreducible, then \(I\) is prime. Also, \((0)\) is prime.       
\end{eg}