\lecture{2}{12 Sep. 13:20}{}
Set is a collection of elements. 
\begin{eg}
    Different sets:
\begin{align*}
    \mathbb{N} &= \left\{ 1,2,3,\dots  \right\} \\
    \mathbb{Z} &= \left\{ 0, \pm 1, \pm 2, \dots  \right\} \\
    \mathbb{R} &= \left\{ \text{real numbers} \right\} \\
    M_2(\mathbb{R} ) &= \left\{ \begin{pmatrix}
        a & b  \\
        c & d  \\
    \end{pmatrix} \mid a,b,c,d \in \mathbb{R}  \right\}  \\
    \mathrm{GL} _2(\mathbb{R}) &= \left\{ \begin{pmatrix}
        a & b  \\
        c & d  \\
    \end{pmatrix} \mid a,b,c,d \in \mathbb{R} , ad - bc \neq 0\right\} \\
    \text{The set of integers modulo } 5 &= \left\{ \overline{0}, \overline{1}, \overline{2}, \overline{3}, \overline{4} \right\} \text{, where } \overline{i} = \left\{ 5k + i \mid k \in \mathbb{N} \cup \left\{ 0 \right\}   \right\}.
\end{align*}
\end{eg}

\begin{notation}
    For a set \(X\), \(x \in X\) means that \(x\) is a member of \(X\). For sets \(X, Y\), a map \(f\) from \(X\) to \(Y\) means that \(f\) is a rule that assigns a member of \(Y\) to every member of \(X\). It is commonly denoted as \(f: X \to Y\). The assigned element of \(Y\) to \(x \in X\) is denoted as \(f(x)\).     \(X\) is said to be a subset of \(Y\) if all numbers of \(X\) are members of \(Y\). It is denoted by \(X \subseteq Y\). Sets are often denoted as 
    \[
        \left\{ x \mid \text{conditions on } x \right\} \text{ or } \left\{ x \in X \mid \text{extra conditions on } x \right\} 
    \]                
\end{notation}

\begin{eg}
    \((\mathbb{N} , +)\) is a semigroup, and \((\mathbb{N} \cup \left\{ 0 \right\}  , +)\) is a monoid with identity \(0\), and \((\mathbb{N} , \times )\) is a monoid with identity \(1\).  
\end{eg}

\begin{eg}
    \((X, +)\) with \(X = \mathbb{Z} , \mathbb{Q} , \mathbb{R} \) are abelian groups. \((X, \cdot)\) with \(X=\mathbb{Q} \setminus \left\{ 0 \right\} , \mathbb{R} \setminus \left\{ 0 \right\} \) are abelian groups. Also, \((\overline{0}, \overline{1}, \overline{2}, \overline{3} , \overline{4} , +)\) is an abelian group.      
\end{eg}

\begin{eg}
    \(\mathcal{S}_n = \left\{ \text{Permutations on } n \text{ letters} \right\} \) is a group, and non-abelian if \(n \ge 3\) and abelian if \(n = 1, 2\).   
\end{eg}

\begin{eg}
    Suppose \(\mathrm{GL}_n(\mathbb{R}) = \left\{ \text{real invertible } n \times n \text{matrices} \right\}\), then \((\mathrm{GL}(\mathbb{R} ), \cdot) \) is a non-abelian group for \(n \ge 2\), and abelian for \(n = 1\).   
\end{eg}

\section{Basis Properties of Groups}
\begin{theorem}
    Suppose \(G = (G, *)\) is a group, then 
    \begin{itemize}
        \item [1.] Identity element is unique.
        \item [2.] For \(g \in G\), \(g^{-1} \) is unique. 
        \item [3.] For \(g, h \in G\), then \((g * h)^{-1} = h^{-1} * g^{-1}\).
        \item [4.] For \(g \in G\), \(\left( g^{-1}  \right)^{-1} = g\).    
    \end{itemize} 
\end{theorem}
\begin{proof}
    \vphantom{text}
    \begin{itemize}
        \item [1.] Suppose \(e, e^{\prime} \) are identites, i.e. 
        \begin{align*}
            e * g &= g = g * e \\
            e^{\prime} * g &= g = g * e^{\prime}, 
        \end{align*}
        then \(e = e * e^{\prime} = e^{\prime} \). 
        \item [2.] Suppose \(h, h^{\prime} \) such that
        \begin{align*}
            g * h &= h * g = e \\
            h^{\prime}  * g &= g * h^{\prime} = e.
        \end{align*}
        Then, 
        \[
            h^{\prime} = e * h^{\prime} = h * g * h^{\prime} = he = h.
        \]
        \item [3.] Since the inverse is unique, it sufficies to show that \(h^{-1} g^{-1}  \) is the inverse of \(gh\), so \(h^{-1} g^{-1} = (gh)^{-1}   \).   
        \item [4.] Trivial.
    \end{itemize}
\end{proof}