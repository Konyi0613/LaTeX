\lecture{7}{1 Oct. 13:20}{}
\begin{prev}
    \[
        G / \sim = \left\{ g H: g \in G \right\}. 
    \]
    Note that if \(g \in G\) belongs to a coset, then \(gh\) must belong to the same coset.   
\end{prev}

Note that 
\[
    \left\vert G / H \right\vert = \left\vert H \setminus G \right\vert 
\] since \(gH \leftrightarrow Hg^{-1}\) is a well-defined bijective map between these two sets. (since \(gh \leftrightarrow h^{-1} g\) is a bijective map). 

\begin{theorem}
    Suppose \(G\) is finite, then 
    \[
        \left\vert G \right\vert = [G : H] \cdot \left\vert H \right\vert,  
    \] where \([G:H] = \vert G / H \vert \). 
\end{theorem}
\begin{proof}
    Consider the map \(H \to gH\) by \(h \mapsto gh\), we say this map is \(\psi \), then \(\psi \) is obviously surjective, and injectivity can be checked as follows: If \(\psi (h_1) = \psi (h_2)\), then \(g h_1 = g h_2\), and thus \(h_1 = h_2\), which shows \(\psi \) is injective. Thus, \(\psi \) is bijective. Hence, \(\vert H \vert = \vert gH \vert  \). Now we know the number of cosets is \([G:H]\), and since we can partition \(G\) by the equivalence relation given by \(G / H\), and thus we know \(\vert G \vert = [G:H] \cdot \vert H \vert  \).               
\end{proof}

\begin{proposition}
    If \(\vert G \vert \) is a prime \(p\), then \(G \simeq \mathbb{Z} / p\mathbb{Z} \) (cyclic subgroup of order \(p\)).   
\end{proposition}
\begin{proof}
    Since \(\vert H \vert \) divides \(\vert G \vert \), so \(H = \left\{ 1 \right\} \) or \(G\). Suppose \(G\) is not cyclic, then for \(g \in G\), consider the subgroup generated by \(g\) i.e. 
    \[
        \langle g \rangle = \left\{ \dots , g^{-1}, 1, g, g^2, \dots  \right\}. 
    \]
    Since \(\langle g \rangle \subseteq G\) and \(\vert G \vert < \infty  \), so \(\langle g \rangle \) is also finite, so there eixsts \(i > j \in \mathbb{Z} \) s.t. \(g^i = g^j\), so \(g^{j - i} = 1\). Thus, there exists \(N \in \mathbb{Z} _{> 0}\) s.t. \(g^N = 1\), pick the smallest such \(N\), then
    \[
        \langle g \rangle = \left\{ 1, g, \dots , g^{N-1} \right\} \simeq \mathbb{Z} / N \mathbb{Z}, 
    \] which is a cyclic group. However, it is a subgroup of \(G\), so \(\langle g \rangle = \left\{ 1 \right\}   \) or \(G\). If \(\langle g \rangle = \left\{ 1 \right\}  \), then \(o(g) = 1\), which means \(g = 1\). If \(g \neq 1\), then \(\langle g \rangle = G\), but it shows \(G\) is cyclic, which gives a contradiction. Hence, \(g = 1\) is the only element of \(G\), but \(\vert G \vert \) is prime, so \(\vert G \vert > 1 \), and thus it is impossible.     
\end{proof}

\section{Normal subgroups}
\begin{question}
    When does \(G / H\) admit a group structure (inherited from \(G\)).
\end{question}

\begin{eg}
    \(G = (\mathbb{Z} , +)\) and \(H = (n\mathbb{Z} , +)\), then 
    \[
        G / H = \left\{ n\mathbb{Z} , 1 + n\mathbb{Z} , \dots , (n - 1) + n\mathbb{Z}  \right\}. 
    \]  
    In this case, \(G / H\) with addition naturally forms a group. 
\end{eg}

Hence, if we have \(g_1 H\) and \(g_2 H\), then we want that \((g_1 g_2) H\) is the result of operating \(g_1 H\) and \(g_2 H\). That is, for \(h_1, h_2 \in H\), we want 
\[
    g_1 h_1 * g_2 h_2 = (g_1 g_2) h_3
\] for some \(h_3 \in H\). Fix \(g_1, g_2\), then for any \(h_1, h_2 \in H\) there must be  \(h_3 \in H\) s.t. the equation holds. Note that 
\[
    g_1 h_1 g_2 h_2 = g_1 g_2 h_3 \iff  h_1 g_2 h_2 = g_2 h_3 \iff  g_2^{-1} h_1 g_2 h_2 = h_3 \iff g_2^{-1} h_1 g_2 = h_3 h_2^{-1} \in H.
\] Thus, the requirement is that \(g^{-1} H g \subseteq H\) for all \(g \in G\) , which means \(H \subseteq gHg^{-1} \) for all \(g \in G\). This gives \(H \subseteq g^{-1} H g\) by replacing \(g^{-1} \) with \(g\). This gives \(g^{-1} H g = H\). 

\begin{definition}
    Suppose \(H \subseteq G\), \(H\) is called a normal subgroup if 
    \[
        g^{-1} H g = H \quad \forall g \in G.
    \]
\end{definition}

\begin{theorem} \label{thm: G/H group iff H nomral}
    The quotient \(G / H\) inherits the group structure of \(G\) if and only if \(H\) is a normal subgroup.   
\end{theorem}