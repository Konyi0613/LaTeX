\lecture{15}{12 Nov. 13:20}{}
\section{Semidirect Product}
Suppose \(N, N^{\prime} \) are groups, then 
\[
    N \times N^{\prime} = \left\{ \left( n, n^{\prime}  \right) \mid n \in N, n^{\prime} \in N^{\prime}   \right\} 
\] has a componentwise multiplication 
\[
    \left( n_1, n_1^{\prime}  \right) \cdot \left( n_2, n_2^{\prime}  \right) = \left( n_1 n_2, n_1^{\prime} n_2^{\prime}   \right),   
\] and we call it the outer product. 

Besides, starting from the product group \(G\), suppose \(N \triangleleft G\) and \(N^{\prime}  \triangleleft G\) s.t. \(N N^{\prime} = G\) and \(N \cap N^{\prime} = \left\{ 1 \right\} \), then \(g = n n^{\prime} \) is unique. Since if \(n_1 n_1^{\prime} = n_2 n_2^{\prime}  \), then \(n_2^{-1} n_1 = n_2^{\prime} n_1^{\prime -1} \in N \cap N^{\prime} = \left\{ 1 \right\} \). 

Thus, 
\[
    N \times N^{\prime} \simeq NN^{\prime} , \quad \left( n, n^{\prime}  \right) \mapsto n n^{\prime}.  
\]
Now let's generalize this. Let \(G\) be a group and suppose \(N \triangleleft G\) and \(H < G\) s.t. \(NH=G\) and \(N \cap H = \left\{ 1 \right\} \), then we can similarly deduce that \(g = nh\) is unique. Besides, 
for \(nh, n^{\prime} h^{\prime} \in NH = G\), we know 
\[
    \left( nh \right) \cdot \left( n^{\prime} h^{\prime}  \right) = nhn^{\prime} h^{-1} h h^{\prime} = n \left( h n^{\prime} h^{-1} \right) hh^{\prime} \in N H   
\]       
since \(n, hn^{\prime} h^{-1} \in N\) and \(h h^{\prime} \in H\). So in terms of the multiplication on the set \(N \times H\), the multiplication in \(G\)
\[
    \iff (n, h) \cdot \left( n^{\prime} , h^{\prime}  \right) = \left( n \varphi _h \left( n^{\prime}  \right), hh^{\prime}   \right)  
\] with \(\varphi _h(n) \coloneqq hnh^{-1} \) for \(h \in H\) and \(n \in N\). 

\begin{note}
    The inner viewpoint requires the multiplication of \(G\) in \(\varphi _h\).  
\end{note}

\begin{question}
    How can we reconstruct such groups?
\end{question}

Observe that \(\varphi_h : N \to N\) is a group automorphism, so there is a group action of \(H\) on \(N\): 
\[(h, n) \mapsto \varphi _h(n).\]
Now we can equivalently define
\[
    \varphi : H \to \mathrm{Aut}(N), \quad h \mapsto \varphi _h. 
\] 
In fact,
\begin{align*}
    \varphi _g \circ \varphi _h (n) \coloneqq \varphi_g \left( \varphi_h(n)  \right) = ghnh^{-1}g^{-1} = \varphi _{gh} (n).     
\end{align*}
This shows 
\[
    \varphi : H \to \mathrm{Aut}(N) 
\] is a group homomorphism. 

\begin{question}
    What if we start with a general group action of \(H\) on \(N\): 
    \[
        H \to \mathrm{Aut}(N), \quad h \mapsto \varphi_h, 
    \]  
    and define a multiplication on the set \(N \times H\) as 
    \[
        (*) \quad (n, h) \cdot \left( n^{\prime} , h^{\prime}  \right) \coloneqq \left( n \varphi _h \left( n^{\prime}  \right), hh^{\prime}   \right)?  
    \] 
\end{question}

\begin{theorem}
    The binary operation \((*)\) satisfies all group laws, so it defines a group structure on \(N \times H\) (the product set). 
\end{theorem}

\begin{notation}
    The resulting group is defined as 
    \[
        N \rtimes_{\varphi }  H,
    \] 
    where \(\varphi \) may be ommited if it is clear. 
\end{notation}

Here,
\[
    \begin{dcases}
        N \times \left\{ 1 \right\} \triangleleft N \rtimes_{\varphi } H \\
        \left\{ 1 \right\} \times H < N \rtimes_{\varphi } H    
    \end{dcases}
\] using these subgroups.

\begin{proof}[(Check the group axioms)]
    \vphantom{text}
\begin{itemize}
    \item Associativity:\begin{align*}
        &(n_1, h_1) \cdot \left( (n_2, h_2) \cdot (n_3, h_3) \right) \\
        &= (n_1, h_1) \cdot \left( n_2 \varphi _{h_2}(n_3), h_2 h_3 \right) \\
        &= \left( n_1 \varphi _{h_1} \left( n_2 \varphi _{h_2} \left( n_3 \right)  \right), h_1 h_2 h_3  \right) \\
        &= \left( n_1 \varphi _{h_1} \left( n_2 \right) \varphi _{h_1} \left( \varphi _{h_2} \left( n_3 \right)  \right), h_1 h_2 h_3   \right) \\
        &= \left( n_1 \varphi _{h_1} \left( n_2 \right) \varphi _{h_1 h_2} \left( n_3  \right), h_1 h_2 h_3   \right).    
    \end{align*}
    Also, we know 
    \begin{align*}
        \left( (n_1, h_1) \cdot (n_2, h_2) \right) \cdot \left( n_3, h_3 \right) = \left( n_1 \varphi _{h_1}(n_2) \varphi _{h_1 h_2}(n_3), h_1 h_2 h_3 \right),   
    \end{align*}
    so the associativity holds. 
    \item Inverse: 
    \[
        (n, h)^{-1} = \left( \varphi _{h^{-1}}\left( n^{-1} \right), h^{-1}  \right). 
    \]
\end{itemize}
\end{proof}

\begin{eg}
    \vphantom{text}
    \begin{itemize}
        \item If \(\varphi : H \to \mathrm{Aut}(N) \) is trivial i.e. \(\varphi (H) = \left\{ 1 \right\} \), then 
        \[
            N \rtimes_{\varphi } H = N \times H.
        \]
        \item Suppose \(N = \mathbb{Z} / m\mathbb{Z} = C_m\), which is cyclic, then since \(\mathbb{Z} / m\mathbb{Z} = \langle 1 \rangle \), so \(\varphi \in \mathrm{Aut}\left( \mathbb{Z} / m\mathbb{Z}  \right)   \) is determined by \(\varphi (1)\) since \(\varphi(1^n) = \left( \varphi (1) \right)^n  \), so we need \(\varphi (1)\) coprime to \(m\). That is, \(\varphi (1) \in \left( \mathbb{Z} / m \mathbb{Z}  \right)^{\times } \).   
    \end{itemize}
\end{eg}