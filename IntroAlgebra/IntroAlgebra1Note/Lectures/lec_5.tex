\lecture{5}{24 Sep. 13:20}{}
\begin{prev}
    Group homomorphism means there exists \(\varphi : (G, *) \to (H, \circ )\) with
    \[
        \varphi (g_1 * g_2) = \varphi (g_1) \circ \varphi (g_2).
    \]
    Thus, we have 
    \[
        \begin{dcases}
            \varphi (1_G) = 1_H \\
            \varphi \left( g^{-1} \right) = \varphi (g)^{-1}
        \end{dcases}.
    \]  
        Group isomorphism means \(\varphi: G \to H\) is an homomorphism and there exists another group homomorphism \(\psi : H \to G\) s.t. 
        \[
            \begin{dcases}
                \psi \circ \varphi : G \to G \\
                \varphi \circ \psi : H \to H
            \end{dcases}
        \] are identity groups. Note that 
        \begin{itemize}
            \item \(\varphi \) is surjective if \(\varphi (G) = H\). 
            \item \(\varphi \) is injective if \(\forall g_1 \neq g_2 \in G\), \(\varphi (g_1) \neq \varphi (g_2)\).      
        \end{itemize}
        Also, we know 
        \begin{itemize}
            \item surjective \(\iff \Im \varphi = H\)
            \item injective \(\iff \ker \varphi = \left\{ 1 \right\} \).   
        \end{itemize}
\end{prev}

\begin{proof}[why \(\ker \varphi = \left\{ 1 \right\} \) means injective?]
    Suppose \(\varphi (g_1) = \varphi (g_2)\), then
    \[
        1_H = \varphi (g_1)^{-1}\varphi (g_1) = \varphi (g_1)^{-1} \varphi (g_2) = \varphi \left( g_1^{-1} \right) \varphi (g_2) = \varphi \left( g_1^{-1} g_2 \right).  
    \] 
    Hence, we have \(g_1^{-1} g_2 = 1_G\), and thus \(g_2 = g_2\).  
\end{proof}

\begin{theorem}
    Let \(\varphi :G \to H\) be a group homomorphism, then \(\varphi \) is an isomorphism iff \(\ker \varphi = \left\{ 1 \right\} \) and \(\Im \varphi = H\).     
\end{theorem}
\section{Equivalenec relation}
\begin{definition}[relation]
    Let \(S\) be a set. A subset \(R \subseteq S \times S \) is called a relation.   
\end{definition}

\begin{eg}
    Suppose \(S = \left\{ 1,2,3,4 \right\} \), then
    \[
        R = \left\{ (1,2), (1,3), (1,4), (2,3), (2,4), (3,4) \right\} 
    \] is the relation \(<\). 
\end{eg}

\begin{notation}
    \((a, b) \in R\) is commonly denoted as \(a \cdot b\) with some symbol \(\cdot\).    
\end{notation}

\begin{definition}[Equivalence relation] \label{def: equivalence relation}
    Let \(S\) be a set and \(\sim\) is a relation on \(S\), then \(\sim \) is called an equivalence relation if it satisfies:
    \begin{itemize}
        \item Reflexive: \(x\sim x\)
        \item Symmetric: If \(x\sim y\), then \(y\sim x\). 
        \item Transitive: If \(x\sim y\) and \(y\sim z\), then \(x\sim z\).      
    \end{itemize}   
\end{definition}

\begin{definition}[Equivalence class] \label{def: equivalence class}
    Suppose \(S\) is a set and \(\sim \) is an equivalence relation on \(S\). We define 
    \[
        C(x) = \left\{ y \in S \mid x\sim y \right\}.
    \]   
\end{definition}

\begin{eg}
    Suppose \(S = \left\{ 1,2,3,4,5,6 \right\} \), and \(x \sim y\) if \(x - y \in 3\mathbb{Z} \), then \(\sim \) is an equivalence relation. List all the equivalence classes.    
\end{eg}
\begin{explanation}
    
    \begin{align*}
        C(1) &= C(4) = \left\{ 1, 4 \right\} \\
        C(2) &= C(5) = \left\{ 2, 5 \right\} \\
        C(3) &= C(6) = \left\{ 3, 6 \right\}.   
    \end{align*}
\end{explanation}


\begin{theorem}
    \vphantom{text}
    \begin{itemize}
        \item If \(y, z \in C(x)\), then \(y \sim z\). 
        \item If \(y \in C(x)\), then \(C(x) = C(y)\). 
        \item If \(C(x) \cap C(y) \neq \varnothing \), then \(C(x) = C(y)\).     
    \end{itemize}
\end{theorem}