\lecture{11}{17 Oct. 13:20}{}
\begin{prev}
    A group \(G\) acts on a set \(X\) means for each \(g \in G\), it gives a map sends \(x\) to \(g(x)\) where \(g(x) \in X\) and the maps satisfy \((gh)(x) = g(h(x))\). \(\iff \)  Formally, it is \(G \times X \to X\) with \((g, x) \mapsto g(x)\) s.t. \((gh)(x) = g(h(x))\). \(\iff \) There is a group homomorphism s.t. \(G \to \mathrm{Aut}(X) \).    
    \begin{remark}
        Last equivalence is because we can let 
        \[
            \Phi :G \to \mathrm{Aut}(X), \quad \Phi (g) = \phi _g, \quad \text{where } \phi _g(x) = g(x).  
        \]
    \end{remark}
    Conjugation is a group action on the group itself defined by 
    \[
        G \times G \to G, \quad (g, x) \mapsto g x g^{-1},
    \] and the conjugating class is a \(G\)-orbit, which means 
    \[
        C(x) = \left\{ g x g^{-1} \mid g \in G \right\} \text{ for all } g \in G. 
    \] 
    \begin{note}
        \(G\) is abelian iff \(C(x) = \left\{ x \right\} \) for all \(x 
        \in G\).   
    \end{note}

    Symmetric group has cycle representation, and conjugation class of \(S_n\) is the set of all permutations of same cycle types.  
\end{prev}

\begin{theorem}
    Conjugation classes of \(S_n\) are cycle types \((n_1, n_2, \dots , n_k)\) with \(n_1 \le n_2 \le \dots \le n_k\) and \(k \ge 1\) s.t. \(n_1 + n_2 + \dots + n_k = n\), and the corresponding class consists of all elements having that cycle type.     
\end{theorem}

Note that for \(H \triangleleft G\), we know \(g H g^{-1} = H\). Hence, a normal subgroup is a union of conjugating classes:
\[
    H = \bigcup_{x \in H} C(x). 
\]  Vice versa, if a subgroup \(H < G\) is a union of conjugating classes, then \(H \triangleleft G\). 

\begin{note}
    For \(G\) finite, one can look at conjugating classes to classify normal subgroups. 
\end{note}

\begin{theorem}[Class equation]
    Suppose \(C\) represents the congugacy classes, then 
    \[
        \vert G \vert = \sum_{C} \vert C \vert,   
    \] and 
    \begin{itemize}
        \item [(1)] \(\# \left\{ C \mid \vert C \vert = 1  \right\} \) divides \(\vert G \vert \). 
        \item [(2)] \(\vert C \vert \) divides \(\vert G \vert \).    
    \end{itemize}
\end{theorem}
\begin{proof}
    Since we can define an equivalence relation s.t. \(x \sim y\) iff \(x = g y g^{-1}\) for some \(g \in G\), and the equivalence classes corresponding to this relation are the conjugacy classes, so 
    \[
        \left\vert G \right\vert = \sum_{C} \vert C \vert.   
    \] 
    \begin{itemize}
        \item [(1)] If \(\vert C \vert = 1\), then there exists \(x \in G\) s.t. \(C(x) = \left\{ x \right\} \). Hence, we know \(g x g^{-1} = x\) for all \(g \in G\), which means \(gx = xg\) for all \(g \in G\). Define 
        \[
            Z(G) = \left\{ x \in G \mid gx = xg \right\}, 
        \] which is the center of \(G\), then this forms a subgroup of \(G\). (This is easy to check). Now since \(\bigcup_{\vert C \vert = 1 } C = Z(G) \), and \(Z(G) \triangleleft G \), so we have
        \[
            \# \left\{ C \mid \vert C \vert = 1  \right\} = \vert Z(G) \vert,  
        \] and by Lagrange's theorem, we know \(\vert Z(G) \vert \mid \vert G \vert  \), so we're done. 
        \item [(2)] Let \(Z_G(x) = \left\{ g \in G \mid gx = xg \right\} \). Then \(Z_G(x) \) is a subgroup of \(G\). (This is easy to check). Now consider \(G / Z_G(x)\), we know it is the collection of equivalence classes, and for all conjugacy classes \(C\), there is a one-to-one correspondence mapping \(C\) to \(\left\{ gxg^{-1} \mid g \in G \right\} = \left\{ h x h^{-1} \mid h \in G / Z_G(x) \right\} \), so 
        \[
            \vert C(x) \vert = \vert G / Z_G(x) \vert = \frac{\vert G \vert }{\vert Z_G(x) \vert },  
        \] and we're done.
    \end{itemize}
\end{proof}

Here we go back to \(S_n\). If \(C = (n_1, \dots , n_k)\) with \(n_1 + \dots + n_k = n\), then what is \(\vert C \vert \)? We can easily show that the answer is 
\[
    \left\vert C \left( 1^{v_1} 2^{v_2} 3^{v_3} \dots r^{v_r} \right)  \right\vert = \frac{n!}{1^{v_1} (v_1 !) 2^{v_2} (v_2 !) 3^{v_3} (v_3!) \dots }, 
\] and we can find that 
\[
   \left\vert C \left( 1^{v_1} 2^{v_2} 3^{v_3} \dots r^{v_r} \right) \right\vert = \frac{\vert S_n \vert }{\vert Z_{S_n}(x) \vert }, \text{ where } x \in \left( 1^{v_1} 2^{v_2} \dots  \right).  
\] by orbit-stabilizer theorem.