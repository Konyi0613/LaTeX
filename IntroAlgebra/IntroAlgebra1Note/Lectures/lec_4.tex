\lecture{4}{19 Sep. 13:20}{}
\begin{prev}
    \vphantom{text}
    \begin{itemize}
        \item \(\mathbb{Z} = (\mathbb{Z} , +)\) is a infinite cyclic group s.t. its subgroup is \(d \mathbb{Z} \) with all \(d = 0, 1, 2, \dots \). 
        \item \(C_n = (\mathbb{Z} / n \mathbb{Z} , +)\) is a cyclic group of order \(n\).
        \begin{align*}
            C_1 &= \left\{ 1 \right\} \\
            C_2 &= \left\{ 1, \sigma  \right\} \text{ with } \sigma ^2 = 1 \\
            C_3 &= \left\{ 1, \sigma , \sigma ^2 \right\} \text{ with } \sigma ^3 = 1.  \\
            C_4 &= \left\{ 1, \sigma , \sigma ^2, \sigma ^3 \right\} \text{ with } \sigma ^4 = 1. \\
            C_5 &= \left\{ 1, \sigma , \sigma ^2, \sigma ^3, \sigma ^4 \right\}  \text{ with } \sigma ^5 = 1. \\
            C_6 &= \left\{ 1, \sigma , \sigma ^2, \sigma ^3, \sigma ^4, \sigma ^5 \right\}  \text{ with } \sigma ^6 = 1. 
        \end{align*}
            Observe that the subgroups of \(C_n\) are of the form \(C_d\) with \(d \mid n\) (\(+\) unique for each \(d\)).
    \begin{exercise}
    Prove it.
    \end{exercise}   
    \item \(S_n\): the symmetric group of degree \(n\). \(S_3 = \left\{ 1, \sigma , \sigma ^2, \tau , \tau \sigma , \theta \sigma ^2 \right\} \).  
    \item \(g \in O_n(\mathbb{R} ) \iff \langle gv, gw \rangle = \langle v, w \rangle  \), where \(\langle v,w \rangle = v_1 w_1 + v_2 w_2 + \dots + v_n w_n \). Also,
    \[
        \langle gv ,gw \rangle = \langle v,w \rangle \iff \lVert gv \rVert = \lVert v \rVert.  
    \]
    Note that 
\[
    \mathrm{SO}_n(\mathbb{R}) = \left\{ g \in O_n(\mathbb{R} ) \mid \det g = 1 \right\},  
\] and 
\[
    \mathrm{O} _n(\mathbb{R} ) = \mathrm{SO}_n(\mathbb{R} ) \cup \varepsilon  \mathrm{SO}_n(\mathbb{R} )  
\]where \(\varepsilon \in \mathrm{O}_n(\mathbb{R} ) \) s.t. \(\det \varepsilon = -1\).  
    \item Suppose \(G, H\) are groups and 
    \[
        G \times H = \left\{ (g, h) \mid g \in G, h \in H \right\}, 
    \] then \(G \times H\) is a group since we can define 
    \[
        (g_1, h_1) * (g_2, h_2) = (g_1 g_2, h_1 h_2).
    \]

    \end{itemize}
\end{prev}

\begin{eg}
    Suppose 
    \begin{align*}
        C_2 &= \left\{ 1, \tau  \right\} \text{ with } \tau ^2 = 1 \\
        C_3 &= \left\{ 1, \sigma , \sigma ^2 \right\} \text{ with } \sigma ^3 = 1.  
    \end{align*}
    Then, 
    \[
        C_2 \times C_3 = \left\{ (1, 1), (1, \sigma ), (1, \sigma ^2), (\tau ,1), (\tau , \sigma ), (\tau, \sigma^2) \right\}. 
    \]
    Note that \(C_2 \times C_3\) is not \(S_3\) because \(S_3\) is not commutative and \(C_2 \times C_3\) is. What is the subgroups?   
\end{eg}
\begin{explanation}
    \begin{align*}
        (\tau , \sigma )^2 &= (1, \sigma ^2) \\
        (\tau , \sigma )^3 &= (\tau ,1) \\
        (\tau , \sigma )^4 &= (1, \sigma ) \\
        (\tau , \sigma )^5 &= (\tau , \sigma ^2) \\
        (\tau , \sigma )^6 &= (1, 1)
    \end{align*}
    Letting \(\mu = (\tau , \sigma )\), then we know that 
    \[
        C_2 \times C_3 = \left\{ 1, \mu , \mu ^2, \mu ^3, \mu ^4, \mu ^5 \right\} \simeq  C_6. 
    \] 
\end{explanation}
As groups, 
\begin{align*}
    S_3 &\simeq \left( \left\{ f_1, f_2, f_3, f_4, f_5, f_6 \right\}, \circ  \right) \text{ where } f_1(x) = x, f_2(x) = 1-x, f_3(x) = \frac{1}{x} \dots \\
    &\simeq  \text{symmetry of triangle} \\
    &\simeq C_6
\end{align*}
\section{Group homomorphisms/isomorphisms}
The idea of isomorphisms is: Suppose \(G, H\) are groups and \(\phi :G \to  H\) is defined by \(g \mapsto \phi (g)\). Now if \(g_1, g_2 \in G\), we want that \(g_1 g_2\) corresponds to \(\phi (g_1) \phi (g_2)\). Hence, if we have \(\phi (g_1 g_2)= \phi (g_1) \phi (g_2)\), then it would be a great property, and it seems that \(G,H\) have same structure. But, consider the map 
\[
    \phi : G \to \left\{ 1 \right\}, 
\] then this map satisfies \(\phi (g_1 g_2) = \phi (g_1) \phi (g_2)\), but obviously \(G\) and \(\left\{ 1 \right\} \) do not have same structure, so we have to give further restriction. Hence, we should restrict that 
\begin{itemize}
    \item Any two elements of \(G\) should not be mapped to the same element. 
\end{itemize}   
Hence, if we have a map from \(G\) to \(G \times H\) with  
\[
    g \mapsto (g, 1),
\] then it also satisfies \(\phi (g_1 g_2) = \phi (g_1) \phi (g_2)\). However, it is not enough, we need the surjection so that we can say any two isomorphic things have same structure. 
\begin{itemize}
    \item The image of \(\phi \) should cover \(H\).  
\end{itemize}

\subsubsection{Summary}
\begin{itemize}
    \item The first restriction \(\iff \forall g_1 \neq g_2 \in G\), we must have \(\phi (g_1) \neq \phi (g_2)\). 
    \item The second restriction \(\iff \forall h \in H, \ \exists g \in G\) s.t. \(h = \phi (g)\).    
\end{itemize}
\begin{definition}
    A map \(\phi :G \to H\) is said to be a homomorphism if 
    \[
        \phi (g_1 g_2) = \phi (g_1) \phi (g_2)
    \] for all \(g_1, g_2 \in G\). 
\end{definition}

\begin{definition}
    A homomorphism \(\phi : G \to H\) is said to be an isomorphism if \(\phi \) is said to be an isomorphism if it is injective and surjective.  
\end{definition}

\begin{definition}[Another definition of Isomorphism]
A map $\phi: G \to H$ is an \textbf{isomorphism} if it is a group homomorphism that is also a bijection. An equivalent, and often more formal, definition is:
Two groups $G$ and $H$ are said to be \textbf{isomorphic} ($G \cong H$) if there exist two group homomorphisms, $\phi: G \to H$ and $\psi: H \to G$, such that they are mutual inverses:
\[
    \begin{dcases}
        \phi (g_1 g_2) = \phi (g_1) \phi (g_2) \quad \text{for } g_1, g_2 \in G \\
        \psi (h_1 h_2) = \psi (h_1) \psi (h_2) \quad \text{for } h_1, h_2 \in H
    \end{dcases}
\]
AND
\[
    \begin{dcases}
        \psi \circ \phi (g) = g \quad \text{for all } g \in G \\
        \phi \circ \psi (h) = h \quad \text{for all } h \in H.
    \end{dcases}
\]
\end{definition}

\begin{exercise}
    Check that two definitions agree.
\end{exercise}

Note that \((\mathbb{Z} / 3 \mathbb{Z}, +) \simeq C_3\), and \((\mathbb{Z} / 3 \mathbb{Z} )^{\times } \simeq C_2 \simeq (\mathbb{Z} / 2 \mathbb{Z} , +)\).  Also, \((\mathbb{Z} / 5 \mathbb{Z} )^{\times } \simeq C_4 \simeq (\mathbb{Z} / 4 \mathbb{Z} , +)\). Thus, more generally, we can see that 
\[
    (\mathbb{Z} / p \mathbb{Z} )^{\times } \simeq C_{p-1} \simeq (\mathbb{Z}  / (p-1) \mathbb{Z} , +)
\] for all prime \(p\). 

\begin{eg}
\(\exp : \mathbb{R} \to \mathbb{R} _{>0}.\). Note that it satisfies \(\exp (x+y)=\exp (x)\exp (y)\). In terms of the group structure, \(\exp \) gives a group homomorphism
\[
    (\mathbb{R} , +) \to (\mathbb{R} _{>0}, \cdot)
\]   
\end{eg}

\section{Properties of homomorphism}
\begin{definition}
    Let \(\phi : G \to H\) to be a group homomorphism. 
    \begin{itemize}
        \item \(\ker \phi = \left\{ g \in G \mid \phi (g) = 1 \right\} \), which can be used to measure how far it is from being injective. 
        \item \(\Im \phi = \left\{\phi (g) \mid g \in G  \right\} \), which can be used to measure how far it is from being surjective. 
    \end{itemize} 
\end{definition}

\subsubsection{Summary}
\[
    \begin{dcases}
        \ker \phi = \left\{ 1 \right\} \iff \phi \text{ is injective} \\
        \Im \phi = H \iff \phi \text{ is surjective}. 
    \end{dcases}
\]