\chapter{Introduction}
\lecture{1}{10 Sep. 13:20}{}
\section{Why study groups?}
Since groups appear everywhere, so we have to study them.
\begin{itemize}
    \item Galois Theory: permutations of roots of polynomials. 
    \item Number Theory: Ideal Class Group, Unit Group (unique factorization). 
    \item Topology: 
    \begin{figure}[H]
        \centering
        \incfig{DonutCup}
        \caption{Fundamental Groups}
        \label{fig:DonutCup}
    \end{figure}
    \item Physics/Chemistry: crystal symmetries and Gauge theory.
\end{itemize}

\begin{definition}[mod] \label{def: mod}
    For two integers \(a,b\) we define \(a \equiv b \mod{N}\) if and only if \(a - b \mid n\).  
\end{definition}

Consider the sequence \(1,2,4,8,16,32, \dots \), and observe the remainders after mod \(p\) for different prime \(p\), then 
\begin{itemize}
    \item \(p=5\): \(\overbrace{1,2,4,3}, \overbrace{1,2,4,3}, \dots \)
    \item \(p=7\): \(\overbrace{1,2,4}, \overbrace{1,2,4}, \dots \) 
\end{itemize}   

\begin{theorem}[Fermat's little theorem] \label{thm: Fermat's little theorem}
    The period divides \(p-1\). 
\end{theorem}
\begin{note}
    This is the special case of Lagrange's theorem.
\end{note}

Consider the symmetry of a triangle. 
\begin{figure}[H]
    \centering
    \incfig{triangle}
    \caption{Triangle}
    \label{fig:triangle}
\end{figure}
Consider the rotation:
\begin{figure}[H]
    \centering
    \incfig{TraingleRotate}
    \caption{title}
    \label{fig:TraingleRotate}
\end{figure}
and reflection
\begin{figure}[H]
    \centering
    \incfig{TriangleReflection}
    \caption{title}
    \label{fig:TriangleReflection}
\end{figure}
Hence, symmetrices are defined by permutations of the vertices \(\left\{ 1,2,3 \right\} \), and thus there are \(6\) operations \(id, \sigma , \sigma ^2, \tau_1, \tau_2, \tau_3\). It is trivial that there are \(3 \times 2 \times 1\) permutations of \(\left\{ 1, 2, 3 \right\} \). Next, consider the six functions 
\begin{align*}
    \varphi _1(x) &= x \\
    \varphi _2(x) &= 1-x \\
    \varphi _3(x) &= \frac{1}{x} \\
    \varphi _4(x) &= \frac{x-1}{x} \\
    \varphi _5(x) &= \frac{1}{1-x} \\
    \varphi _6(x) &= \frac{x}{x-1}
\end{align*}  

Observe that 
\begin{align*}
    \varphi _2 \left( \varphi _3(x) \right) &= 1 - \frac{1}{x} = \frac{x-1}{x} \\
    \varphi _4\left( \varphi _4(x) \right) &= \frac{1}{1-x} = \varphi _5(x) \\
    \varphi _4 \left( \varphi _4 \left( \varphi _4(x) \right)  \right) = x = \varphi_1 (x)  
\end{align*}

\begin{theorem}
    \(\varphi _1, \varphi _2, \dots , \varphi _6\) are closed under composition.
\end{theorem}

\begin{note}
    There's a fact that:
    \begin{align*}
        &\text{operations preserving symmetry of triangle} \\
        &\iff \text{permutations on } \left\{ 1, 2, 3 \right\}    \\
        &\iff \text{compositions of } \varphi _1, \dots , \varphi _6
    \end{align*}
\end{note}
Actually, below things are somewhere similar,
\begin{itemize}
    \item Addition of integers, 
    \item Addition of classes of integers \(\mod{p}\),
    \item Operations on geometric shape,
    \item Permutation on letters,
    \item Composition of functions. 
\end{itemize}
 Since they are all binary operations. 
 \begin{definition}[Binary operations] \label{def: binary operation}
    Suppose \(X\) is a set. Binary operation \(\star\) is a rule that allocates an element of \(X\) to a pair of elements of \(X\).   
 \end{definition}

 \begin{eg}
    \vphantom{text}
    \begin{itemize}
        \item Addition on \(\mathbb{N} , \mathbb{Z} , \mathbb{Q} , \mathbb{R} , \mathbb{C} \) or vector spaces.
        \item Subtractions on \(\mathbb{Z} , \mathbb{Q} , \mathbb{R} , \mathbb{C}\) or vector spaces.  
        \item A map \(X \to X\) (self map) with composition \(\left( \varphi _1 \star \varphi _2 \right) (x)  = \varphi_1 \left( \varphi _2 (x) \right) \). 
        \item Set of subsets of \(\mathbb{R} \). We can define 
        \begin{itemize}
            \item \((A, B) \mapsto A \cup B\) 
            \item \((A, B) \mapsto A \cap B\)
            \item \((A, B) \mapsto A \setminus B\).    
        \end{itemize}
        \item \(n \times n\) real square matrices
        \[
            (A, B) \mapsto A \cdot B.
        \]
    \end{itemize}
 \end{eg}

\begin{definition*}[Special relations]
    Suppose \(X\) is a set and \(*\) is a binary operation on \(X\). 
    \begin{definition}[Associativity]
        \((a * b) * c = a * (b *c)\). 
    \end{definition}  
    \begin{definition}[Identity]
        \(\exists e \in X\) s.t. \(a * e = e * a = a\) for all \(a \in X\).   
    \end{definition} 
    \begin{definition}[Inverse]
        \(\forall a \in X\), \(\exists a^{-1} \in X\) s.t. \(a * a^{-1} = a^{-1} * a = e\).   
    \end{definition}
    \begin{definition}[Commutativity]
        \(a * b = b * a\). 
    \end{definition}
\end{definition*}

\begin{definition}
    Some names:
    \begin{definition}[Semigroup]
        Only has Associativity.
    \end{definition}
    \begin{definition}[Monoid]
        Only has Associativity and Identity.
    \end{definition}
    \begin{definition}[Group]
        Only has Associativity and Identity and Inverse.
    \end{definition}
    \begin{definition}[Abedian Group]
       Has all the \(4\) properties.  
    \end{definition}
\end{definition}