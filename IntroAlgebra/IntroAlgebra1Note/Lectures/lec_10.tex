\section{Group action}
\lecture{10}{15 Oct. 13:20}{}
\begin{definition}[Group Action]
    If \(G\) is a group and \(X\) is a set, then we say \(G\) acts on \(X\) if there exists a map 
    \[
        G \times X \to X, \quad (g, x) \mapsto g \cdot x
    \] satisfying \(g \left( hx \right) = (gh) \cdot x \), and we call this map a group action. 
\end{definition}

\begin{eg}
    \(X = G\) and \(g \cdot x = gx\).  
\end{eg}

\begin{eg}
    \(X=G\) and \(g \cdot x = gxg^{-1}\). We call this a conjugation.  
\end{eg}

\begin{definition}
    We say
    \[
        Gx = \left\{ g \cdot x \mid g \in G \right\} \text{ for some } x \in X
    \] is an orbit of a group action.
\end{definition}

\begin{eg}
    \(Gx=G\) for all \(x \in G\).  
\end{eg}

\begin{eg}
    \[
        Gx = \left\{ gxg^{-1} \mid g \in G \right\} = \left\{ h^{-1} x h \mid h \in G \right\}.  
    \]
\end{eg}

\begin{definition}[Conjuagcy classes]
    We call the \(G\)-orbits under the congugation actions the conjugacy classes. It is an equivalence class defined by 
    \[
        x \sim g^{-1} x g, 
    \] so we have 
    \[
        \vert G \vert = \sum_{C \in \mathrm{Conj}(G) }  \vert C \vert .,  
    \]where \(\mathrm{Conj}(G) \) is the set of all conjugation classes of \(G\).  
\end{definition}
\begin{note}
    The definition of the equivalence relation in the conjugation classes is 
    \[
        x \sim y \text{ iff } \exists g \in G \text{ s.t. } x = g^{-1} y g.
    \]
\end{note}
\begin{proposition}
    \[
        \vert C(x) \vert = \frac{\vert G \vert}{\left\vert Z_G(x) \right\vert }, 
    \] where 
    \[
        Z_G(x) = \left\{ g \in G \mid g^{-1} x g = x \right\}. 
    \]
\end{proposition}
\begin{remark}
    See orbit-stabalizer theorem. (HW5)
\end{remark}

\section{Symmetric groups}
\begin{definition}
    \[
        S_n = \left\{ \text{permutations on } n\text{ letters}  \right\}. 
    \]
\end{definition}

\begin{question}
    What is the conjugation classes of \(S_n\)? 
\end{question}

Consider 
\[
    \tau = \begin{pmatrix}
        1 & 2 & \cdots & n  \\
        i_1 & i_2 & \cdots & i_n  \\
    \end{pmatrix},
\]
then what is \(\sigma ^{-1} \tau \sigma \)?

\begin{note}
    Here we first operate \(\sigma ^{-1}\) rather than \(\sigma \), it is from left to right. 
\end{note}

Thus, we have 
\[
    \sigma ^{-1} \tau \sigma = \begin{pmatrix}
        \sigma (1) & \sigma (2) & \cdots & \sigma (n)  \\
        \sigma (i_1) & \sigma (i_2) & \cdots & \sigma (i_n)  \\
    \end{pmatrix}.
\]

\begin{eg}
    If 
    \[
        \tau = \begin{pmatrix}
            1 & 2 & 3  \\
            3 & 2 & 1  \\
        \end{pmatrix} = (13)(2),
    \] then 
    \[
        \sigma ^{-1} \tau \sigma = \begin{pmatrix}
            \sigma (1) & \sigma (2) & \sigma (3)  \\
            \sigma (3) & \sigma (2) & \sigma (1)  \\
        \end{pmatrix}.
    \]
    Note that \(\sigma ^{-1} \tau \sigma \) can be either:
    \[
        (13)(2), \quad (12)(3), \quad (23)(1).
    \] Thus, the cycle type is preserved. Vice versa, if two permutation have the same cycle type, then they are conjugate to each other.
\end{eg}

\begin{theorem}
    Conjugacy classes of \(S_n\) is described by the partition of \(n\).  
\end{theorem}

For example, \(7 = 1 + 2 + 4\), then it represents the conjugacy class of type 
\[
    (a)(bc)(defg).
\] 
\begin{eg}
    For \(S_3\), the conjugation classes are 
    \begin{align*}
        3 &\leftrightarrow (123), (132) \\
        1 + 2 &\leftrightarrow (1)(23), (2)(13), (3)(12) \\
        1+1+1 &\leftrightarrow (1)(2)(3).
    \end{align*} 
\end{eg}
