\lecture{13}{5 Nov. 13:20}{}
\section{Sylow's theorem}
\begin{definition}[Sylow \(p\)-group]
    Let \(G\) be a finite group with \(\left\vert G \right\vert = p^m a \) where \(p \nmid a\) and \(p\) is prime. A subgroup \(H < G\) with \(\left\vert H \right\vert = p^m \) is called Sylow \(p\)-group.
\end{definition}       

\begin{theorem}[Sylow's theorem]
    \vphantom{text}
    \begin{itemize}
        \item [(1)] Sylow \(p\)-subgroup exists. 
        \item [(2)] If \(K < G\) has the order \(\left\vert K \right\vert = p^l \) with \(l \le m\), then there exists Sylow \(p\)-subgroup containing \(K\).
        \item [(3)] Sylow \(p\)-subgroup are conjugate to each other i.e. if \(P_1, P_2\) are Sylow \(p\)-subgroup, then there exists \(g \in G\) s.t. \(P_2 = g P_1 g^{-1}\). 
        \item [(4)] Let \(n_p \coloneqq \# \left\{ \text{Sylow } p \text{-subgroups}   \right\} \), then \(n_p \equiv 1 \mod{p}\).         
    \end{itemize}
\end{theorem}

\subsubsection{Application of Sylow's theorem}
\begin{proposition}
    Let \(G\) be a group of order \(pq\) with \(p,q\) distinct (\(p < q\)) and both prime s.t. \(q \not\equiv 1 \mod{p}\), then 
    \[
        G \simeq \mathbb{Z} / pq \mathbb{Z} .
    \] i.e. The group of order \(pq\) is unique.  
\end{proposition}
\begin{proof}
    Since \(\vert G \vert = pq \), we know \(n_q \equiv 1 \mod{q}\). Also, since we can define a group actions of \(G\) on \(\mathrm{Syl}_q(G) = \left\{ \text{Sylow } q \text{-subgroup}   \right\} \) by
    \[
        \varphi : (G, \mathrm{Syl}_q(G)) \to \mathrm{Syl}_q(G), \quad g \cdot P = g P g^{-1}, 
    \] and this action is well-defined by (3) of Sylow's theorem. Thus, we know \(\mathrm{Syl}_q(G) = \mathrm{Orb}(Q)  \) for some \(Q \in \mathrm{Syl}_q(G) \) since (1) of Sylow's theorem gyarantee the existence. Thus, by orbit-stabalizer theorem we know 
    \[
        \mathrm{Orb}(Q) \cdot \mathrm{Stab}(Q) = \vert G \vert \implies  \mathrm{Syl}_q(G) = \mathrm{Orb}(Q) \mid \vert G \vert = pq,    
    \] and since \(n_q \equiv 1 \mod{q}\), so we have \(n_q \mid p\), so \(n_q = 1, p\). If \(n_q = p\), then \(p \equiv 1 \mod{q}\), which means \(q \mid p - 1\), but
    \[
       p - 1 < q - 1 < q, 
    \] so this is impossible.
    Now we know \(n_q = 1\). Thus, we know Sylow \(q\)-subgroup is a unique \(Q\), and it is normal by plugging \(P_1, P_2\) both to be \(Q\) in (3) of Sylow's theorem. Similarly we can show \(n_p = 1\) and thus Sylow \(p\)-subgroup is a normal \(P\). Hence, \(\vert P \vert = p \) and \(\left\vert Q \right\vert = q \), and since \(P \cap Q\) is a subgroup of \(P\) and \(Q\), so \(\left\vert P \cap Q \right\vert \mid p \) and \(\left\vert P \cap Q \right\vert \mid q \), so we have \(P \cap Q = \left\{ 1 \right\}  \), which means 
    \[
        P \times Q \simeq PQ=G
    \]
    since 
    \[
        \left\vert P Q \right\vert = \frac{\left\vert P \right\vert \vert Q \vert  }{\left\vert P \cap Q \right\vert } = \vert P \vert \vert Q \vert.
    \]
    This proves \(G \simeq \mathbb{Z} / p \mathbb{Z} \times \mathbb{Z} / q\mathbb{Z} \) and since \(p, q\) are distinct prime (implies \(P, Q\) are cyclic), so 
    \[
        \mathbb{Z} / p \mathbb{Z} \times \mathbb{Z} / q \mathbb{Z} \simeq \mathbb{Z} / pq \mathbb{Z} .
    \]                
\end{proof}

\begin{eg}
If \(\vert G \vert = 15 \), then \(G \simeq \mathbb{Z} / 15 \mathbb{Z} \), but if \(\left\vert G \right\vert = 21 \), then \(G\) may be non-abelian since \(7 \equiv 1 \mod{3}\).     
\end{eg}

\begin{proposition}
    If \(\left\vert G \right\vert = pq \) with \(p, q\) distinct primes s.t. \(q \equiv 1 \mod{p}\), then there are two possibilities:
    \begin{itemize}
        \item \(G \simeq \mathbb{Z} / p \mathbb{Z} \times \mathbb{Z} / q \mathbb{Z} \). 
        \item \(G \simeq \mathbb{Z} / q \mathbb{Z} \rtimes \mathbb{Z} / p\mathbb{Z} \), where \(\rtimes\) is the semi-direct product. 
    \end{itemize}
\end{proposition}

\begin{definition}[Semi-direct product]
    Let \(G\) be a group, then \(G = N \rtimes H\) means \(N \triangleleft G\) and \(H < G\) and \(N \cap H = \left\{ 1 \right\} \), and there exists \(\varphi : H \to \mathrm{Aut}(N) \) s.t. 
    \[
        \varphi (h)(n) = hnh^{-1}.
    \]     
    Then, we can define a product structure on \(N \times H\) as
    \[
        (n, h) \cdot (n^{\prime}, h^{\prime} ) = (nhn^{-1} n^{\prime} , hh^{\prime} )
    \] since for
    \[
        g = nh (n \in N, h \in H) \quad g^{\prime} = n^{\prime} h^{\prime} \left( n^{\prime} \in N, h^{\prime} \in H \right), 
    \] and 
    \[
        gg^{\prime} = nh n^{\prime} h^{\prime} = nhn^{\prime} h^{-1} h h^{\prime} \in N \cdot H (\text{Note that } n \in N, hn^{\prime} h^{-1} \in N, hh^{\prime} \in H ). 
    \]
\end{definition}

The upshot is suppose \(G\) is a group and \(N \triangleleft G\) and there exists \(H<G\) s.t. \(H \simeq G / N\) with \(h \mapsto hN\). Then, \(G\) can be reconstructed by the information of \(H, N\) and \(\varphi \), which is a group action of \(H\) acts on \(N\).           