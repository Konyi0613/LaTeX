\documentclass[a4paper,12pt]{article}
\usepackage{bbding,combelow,textcomp,amsfonts,amsthm,amsmath,amssymb,graphicx,color,hyperref,etoolbox}
\usepackage[utf8x]{inputenc}
\usepackage[lined,boxed,commentsnumbered]{algorithm2e}
\usepackage{enumitem}
\usepackage{xeCJK}
\setCJKmainfont{PingFang TC}
\usepackage{pdfpages}
\newtoggle{story}

%%%%%%%%%%%%%%%%%%%%%%%%%%%%%%%%%%%%%%%%%%
%% End of customisation options
%%%%%%%%%%%%%%%%%%%%%%%%%%%%%%%%%%%%%%%%%%

\pdfpagewidth 8.5in
\pdfpageheight 11in
\topmargin -1in
\headheight 0in
\headsep 0in
\textheight 8.5in
\textwidth 6.5in
\oddsidemargin 0in
\evensidemargin 0in
\headheight 77pt
\headsep 0in
\footskip .75in

\makeatletter
\newcommand{\buildtitle}[4]{
\begin{flushleft}
{\large
#1
\hfill{}
#2
\par
#3
}
\end{flushleft}
\vskip 4pt
\begin{center}
{\large\bfseries#4\par}
\end{center}
\bigskip
}
\makeatother

\renewcommand{\thesection}{\normalsize\arabic{section}}

\renewcommand{\deg}{\textrm{deg}}
\newcommand{\eps}{\varepsilon}
\newcommand{\ceil}[1]{\left \lceil #1 \right \rceil}

\newcommand{\hint}{\begin{flushright} [Hint at \url{\hintURL}.] \end{flushright}}

\newcommand{\card}[1]{\left| #1 \right|}
\newcommand{\mb}[1]{\mathbb{#1}}
\newcommand{\mc}[1]{\mathcal{#1}}


\newcommand{\story}[1]{\iftoggle{story}{\footnote{#1}}{}}
\newcommand{\storymark}[1][42]{\iftoggle{story}{\footnotemark[#1]}{}}
\newcommand{\storytext}[2][42]{\iftoggle{story}{\footnotetext[#1]{#2}}{}}

\newcommand{\bonus}[2]{\paragraph{Bonus (#1 pt\ifstrequal{#1}{1}{}{s})} #2}


\begin{document}

\begin{center}
{\LARGE Abstract Algebra I}\\[6pt]
{\Large Homework 7}\\[6pt]
\textbf{Due: 19th November 2025}
\end{center}

\vspace{1em}

For a finite group $G$ and a prime $p$ dividing $|G|$, let $n_p(G)$ denote the number of Sylow $p$-subgroups of $G$.

\paragraph{Exercise 1} Let $G$ be a group of order $24$, and suppose $n_2(G)>1$.

\begin{enumerate}[label=(\roman*)]
\item Prove that $G$ has a normal subgroup of order $4$.

\item Is it possible that $n_3(G)>1$?
\end{enumerate}
\paragraph{Exercise 2} Let $m$ be an odd integer and $G$ be a group of order $2m$.  
Consider the action of $G$ on itself via left multiplication; this induces a group homomorphism 
\[
\pi : G \to \mathrm{Perm}(G).
\]
Recall that we have a group homomorphism 
\[
\mathrm{sgn} : \mathrm{Perm}(G) \to \{\pm 1\}
\]
that sends each permutation to its sign.

\begin{enumerate}[label=(\roman*)]
\item Show that the composition $\mathrm{sgn}\circ\pi : G \to \{\pm 1\}$ is surjective.  
\textit{(Hint: Let $h$ be a generator of a Sylow $2$-subgroup of $G$, and decompose $G$ into right cosets 
$\{e,h\}g_1,\dots,\{e,h\}g_m$.  Now consider $\pi(h)$.)}

\item Deduce that $G$ has a normal subgroup of order $m$.
\end{enumerate}

For the next two questions, the following fact will be helpful (try to prove it on your own): If $N$ is a normal subgroup of a group $G$ and a Sylow $p$-subgroup $P$ of $N$ is normal in $N$, then $P$ is normal in $G$.
\paragraph{Exercise 3} et $G$ be a group of order $105$.

\begin{enumerate}[label=(\roman*)]
\item Show that $G$ has a normal subgroup $H$ of order $35$.

\item Show that $H$ is cyclic.

\item Prove that $n_5(G)=n_7(G)=1$.
\end{enumerate}
\paragraph{Exercise 4} More generally, let $G$ be a group of order $pqr$, where $p,q,r$ are distinct primes and $p<q<r$.  
We want to show that $n_r(G)=1$.

\begin{enumerate}[label=(\roman*)]
\item Suppose $n_r(G)>1$. Show that $n_q(G)=1$.  
Thus we deduce already that $G$ is not simple.

\item We now suppose $n_q(G)=1$, and let $Q$ be the unique Sylow $q$-subgroup of $G$.  
Show that $G/Q$ has a normal subgroup of order $r$.

\item Deduce that $G$ has a normal subgroup of order $qr$ and conclude that $n_r(G)=1$.
\end{enumerate}
\paragraph{Exercise 5} Let $R$ be a commutative ring.  
We say an element $x\in R$ is a \emph{zero divisor} if there is some nonzero element $y\in R$ such that $xy=0$.  
If a ring has no nonzero zero divisors, we say the ring is an \emph{integral domain}, or sometimes just \emph{domain} for short.

\begin{enumerate}[label=(\roman*)]
\item Classify all zero divisors of the following rings:  
\[
\mathbb{Z},\ \mathbb{Z}/6\mathbb{Z},\ \mathbb{Q},\ \mathbb{C}[x].
\]
Which of them are integral domains?  
(You don’t need to check they are commutative rings; addition and multiplication are carried out as usual.)

\item Show that any element in $R$ cannot be both invertible \emph{and} a zero divisor at the same time.

\item However, an element may be neither invertible nor a zero divisor.  
Find an example.

\item Show that the invertible elements of the polynomial ring $R[x]$ coincide with the invertible elements of $R$.
\end{enumerate}

\end{document}