\documentclass[a4paper,12pt]{article}
\usepackage{bbding,combelow,textcomp,amsfonts,amsthm,amsmath,amssymb,graphicx,color,hyperref,etoolbox}
\usepackage{booktabs}
\usepackage[utf8x]{inputenc}
\usepackage[lined,boxed,commentsnumbered]{algorithm2e}
\usepackage{enumitem}
\usepackage{float}
\usepackage{xeCJK}
\setCJKmainfont{PingFang TC}
\usepackage{pdfpages}
\newtoggle{story}

%%%%%%%%%%%%%%%%%%%%%%%%%%%%%%%%%%%%%%%%%%
%% End of customisation options
%%%%%%%%%%%%%%%%%%%%%%%%%%%%%%%%%%%%%%%%%%

\pdfpagewidth 8.5in
\pdfpageheight 11in
\topmargin -1in
\headheight 0in
\headsep 0in
\textheight 8.5in
\textwidth 6.5in
\oddsidemargin 0in
\evensidemargin 0in
\headheight 77pt
\headsep 0in
\footskip .75in

\makeatletter
\newcommand{\buildtitle}[4]{
\begin{flushleft}
{\large
#1
\hfill{}
#2
\par
#3
}
\end{flushleft}
\vskip 4pt
\begin{center}
{\large\bfseries#4\par}
\end{center}
\bigskip
}
\makeatother

\renewcommand{\thesection}{\normalsize\arabic{section}}
\renewcommand{\Im}{\mathrm{Im}}

\renewcommand{\deg}{\textrm{deg}}
\newcommand{\eps}{\varepsilon}
\newcommand{\ceil}[1]{\left \lceil #1 \right \rceil}
\DeclareMathOperator{\sgn}{sgn}


\newcommand{\hint}{\begin{flushright} [Hint at \url{\hintURL}.] \end{flushright}}

\newcommand{\card}[1]{\left| #1 \right|}
\newcommand{\mb}[1]{\mathbb{#1}}
\newcommand{\mc}[1]{\mathcal{#1}}


\newcommand{\story}[1]{\iftoggle{story}{\footnote{#1}}{}}
\newcommand{\storymark}[1][42]{\iftoggle{story}{\footnotemark[#1]}{}}
\newcommand{\storytext}[2][42]{\iftoggle{story}{\footnotetext[#1]{#2}}{}}

\newcommand{\bonus}[2]{\paragraph{Bonus (#1 pt\ifstrequal{#1}{1}{}{s})} #2}


\begin{document}

\begin{center}
{\LARGE Abstract Algebra I}\\[6pt]
{\Large Homework 7}\\[6pt]
B13902024 張沂魁 \\
\textbf{Due: 19th November 2025}
\end{center}

\vspace{1em}

For a finite group $G$ and a prime $p$ dividing $|G|$, let $n_p(G)$ denote the number of Sylow $p$-subgroups of $G$.

\paragraph{Exercise 1} Let $G$ be a group of order $24$, and suppose $n_2(G)>1$.

\begin{enumerate}[label=(\roman*)]
\item Prove that $G$ has a normal subgroup of order $4$.

\item Is it possible that $n_3(G)>1$?
\end{enumerate}
\paragraph{Solution: }
\begin{itemize}
    \item [(i)] Note that \(24 = 2^3 \cdot 3\), and we know \(n_2(G) \equiv 1 \mod{2}\), so there are at least \(3\) Sylow \(2\)-subgroup of \(G\). Suppose they are \(P_1, P_2, P_3\), and let \(P = \left\{ P_1, P_2, P_3 \right\} \), then consider
    \[
        \varphi : G \to S_3, \quad g \mapsto \varphi _g, \quad \text{where } \varphi _g (P_i) = g P_i g^{-1} \quad \forall i = 1,2,3. 
    \]
    Note that \(\varphi \) is a homomorphism, and 
    \[
        \ker \varphi = \left\{ g \in G : \varphi _g = 1 \right\} = \left\{ g \in G : g P_i g^{-1} = P_i \quad \forall i = 1,2,3 \right\} = \bigcap_{i=1}^{3} N_G(P_i).   
    \]
    Now if we define a group action of \(G\) on \(P\) by \(g \cdot P_i = g P_i g^{-1}\), then 
    \begin{align*}
        \mathrm{Stab}(P_i) &= \left\{ g \in G : g \cdot P_i = P_i \right\} = \left\{ g \in G: gP_i g^{-1} = P_i \right\} = N_G(P_i) \\
        \mathrm{Orb}(P_i) &= \left\{ g \cdot P_i : g \in G \right\} = \left\{ g P_i g^{-1} : g \in G \right\} = P.      
    \end{align*}
    Hence, by orbit-stabilizer theorem, we know 
    \[
        \left\vert N_G(P_i) \right\vert = \left\vert \mathrm{Stab}(P_i)  \right\vert = \frac{\vert G \vert }{\vert P \vert } = 8,
    \] and note that 
    \[
        P_i \subseteq N_G(P_i), \quad \text{and }  \vert P_i \vert = \vert N_G(P_i) \vert,  
    \]so we know \(P_i = N_G(P_i)\). Hence, \(\ker \varphi = \bigcap_{i=1}^{3} N_G(P_i) = \bigcap_{i=1}^{3} P_i \). Note that \(\mathrm{Im} \varphi < S_3\), so by Lagrange's theorem: 
    \begin{itemize}
        \item Case 1: \(\vert \mathrm{Im}  \varphi  \vert = 1, 2, 3\), then by first isomorphism theorem, 
        \[
            \vert \ker \varphi  \vert = \frac{\vert G \vert }{\vert \mathrm{Im}  \varphi  \vert } \ge 8, 
        \] but \(\ker \varphi = \bigcap_{i=1}^{3} P_i \), and \(\vert P_i \vert = 8\) for all \(i = 1,2,3\). Hence, \(P_1 = P_2 = P_3\), otherwise \(\vert \ker \varphi  \vert < 8\). However, \(P_1, P_2, P_3\) are pairwise distinct, so this case is impossible.      
        \item Case 2: \(\vert \Im \varphi \vert = 6 \), then \(\vert \ker \varphi  \vert = \frac{24}{6} = 4 \), and thus \(\ker \varphi \) is a normal subgroup of \(G\) of order \(4\), and this is the only possible case, so we're done.   
    \end{itemize}   
    \item [(ii)] Consider \(S_4\), then we know \(\vert S_4 \vert = 24 \), and note that 
    \begin{table}[H]
        \centering
        \begin{tabular}{c|c}
            \toprule
                cycle types & \(S_4\)   \\
            \midrule
                \((1)^4\)  & \(1\)   \\
                \((2)(1)^2\)  & \(6\)   \\
                \((3)(1)\)  & \(8\)   \\
                \((4)\)  & \(6\)   \\
                \((2)(2)\)  &  \(3\)  \\
            \bottomrule
        \end{tabular}
        \caption{The number of permutations of \([4]\) of different cycle types}
    \end{table}
    If \(n_2(S_4) = 1\), then there exists a normal subgroup of \(S_4\) of order \(8\), but we can see from the above table that it is impossible such normal subgroup of \(S_4\) exists since normal subgroups of \(S_4\) are union of permutations of same cycle types. Hence, \(n_2(S_4) > 1\). However, we know 
    \[
        \left\{ (1)(2)(3)(4), (123), (132) \right\} \text{ and } \left\{ (1)(2)(3)(4), (124), (142) \right\}   
    \] are both Sylow \(3\)-subgroups of \(S_4\), so it is possible that \(n_3(G) > 1\).   
\end{itemize}


\paragraph{Exercise 2} Let $m$ be an odd integer and $G$ be a group of order $2m$.  
Consider the action of $G$ on itself via left multiplication; this induces a group homomorphism 
\[
\pi : G \to \mathrm{Perm}(G).
\]
Recall that we have a group homomorphism 
\[
\mathrm{sgn} : \mathrm{Perm}(G) \to \{\pm 1\}
\]
that sends each permutation to its sign.

\begin{enumerate}[label=(\roman*)]
\item Show that the composition $\mathrm{sgn}\circ\pi : G \to \{\pm 1\}$ is surjective.  
\textit{(Hint: Let $h$ be a generator of a Sylow $2$-subgroup of $G$, and decompose $G$ into right cosets 
$\{e,h\}g_1,\dots,\{e,h\}g_m$.  Now consider $\pi(h)$.)}

\item Deduce that $G$ has a normal subgroup of order $m$.
\end{enumerate}
\paragraph{Solution:} 
\begin{itemize}
    \item [(i)] Let \(P \in \mathrm{Syl}_2(G) \), and suppose \(P = \left\{ e, h \right\} \), then since we know \(\sgn \circ \pi (e) = +1\), so we just need to show that there exists some \(x \in G\) s.t. \(\sgn \circ \pi (e) = -1\). Now we claim that \(\sgn \circ \pi (h) = -1\), and we're done. Note that 
    \[
        G = P g_0 \cup P g_1 \cup \dots \cup P g_{m-1},
    \] where \(g_0, g_1, \dots , g_{m-1} \in G\) and we let \(g_0 = e\). This is because \([G:P] = \frac{\vert G \vert }{\vert P \vert } = \frac{2m}{2} = m\), so we can write \(G\) into union of \(m\) right cosets of \(P\). Hence, we know
    \[
        G = \bigcup_{i=0}^{m-1} \left\{ e g_i, h g_i \right\},  
    \] and if we define \(\pi (h) = \pi _h\), then 
    \[
        \pi _h (g_i) = h g_i, \quad \text{and } \pi _h(h g_i) = h^2 g_i = g_i. 
    \]       
    Thus, \(\pi _h\) swaps \(g_i\) and \(h g_i\) for all \(i\), i.e. 
    \[
        \pi (h) = (g_0 \quad hg_0) (g_1 \quad hg_1) \dots (g_{m-1} \quad h g_{m-1}).
    \]
    Note that \(m\) is odd, so we know \(\sgn \circ \pi(h) = -1\), and we're done.      
    \item [(ii)] Let \(\varphi = \sgn \circ \pi \), then \(\varphi \) is a homomorphism and \(\vert \Im \varphi  \vert = 2 \) since \(\varphi \) is surjective, and by first isomorphism theorem we know 
    \[
        \vert \ker \phi  \vert = \frac{\vert G \vert }{\vert \Im \varphi  \vert } = \frac{2m}{2} = m, 
    \] and \(\ker \varphi \trianglelefteq G\), so we're done. 
\end{itemize}


For the next two questions, the following fact will be helpful (try to prove it on your own): If $N$ is a normal subgroup of a group $G$ and a Sylow $p$-subgroup $P$ of $N$ is normal in $N$, then $P$ is normal in $G$.
\paragraph{Proof of this fact:} For all \(g \in G\), we have \(\left\vert g P g^{-1} \right\vert = \vert G \vert  \), and \(g P g^{-1} \subseteq g N g^{-1} = N\) since \(N \trianglelefteq G\), so \(g P g^{-1} \in \mathrm{Syl}_p(N) \). Now since \(P \trianglelefteq N\), so we know the Sylow \(p\)-subgroup of \(N\) is unique, and thus 
\[
    gPg^{-1} = P \quad \forall g \in G \implies P \trianglelefteq G.
\]        
\paragraph{Exercise 3} et $G$ be a group of order $105$.

\begin{enumerate}[label=(\roman*)]
\item Show that $G$ has a normal subgroup $H$ of order $35$.

\item Show that $H$ is cyclic.

\item Prove that $n_5(G)=n_7(G)=1$.
\end{enumerate}

\paragraph{Solution:}
\begin{itemize}
    \item [(i)] Consider \(n_5(G), n_7(G)\), then since \(n_5(G) \equiv 1 \mod{5}\) and thus \(n_5(G) \mid \frac{\vert G \vert }{5}\), so we know \(n_5(G) \in \left\{ 1, 21 \right\} \), and similarly we can derive \(n_7(G) \in \left\{ 1, 15 \right\} \). Note that for distinct Sylow \(5\)-subgroup of \(G\), say \(P, Q\), then \(P \cap Q = \left\{ 1 \right\} \) since \(\left\vert P \cap Q \right\vert \mid 5 \), and this statement holds also for Sylow \(7\)-subgroups. Now if \(n_7(G) = 15\) and \(n_5(G) = 21\), then there are at least \((6 - 1)15 + (5-1)21 > 105\) elements are in \(G\), so either \(n_5(G)\) or \(n_7(G)\) is equal to \(1\). WLOG, suppose \(n_5(G) = 1\) (if \(n_7(G) = 1\), then it can be proved similarly.), and suppose \(P \in \mathrm{Syl}_5(G) \) and \(Q \in \mathrm{Syl}_7(G) \). Then, \(P \trianglelefteq G\) and \(Q < G\) and \(P \cap Q = \left\{ e \right\}  \). Note that \(PQ < G\) and 
    \[
        \vert PQ \vert = \frac{\vert P \vert \cdot \vert Q \vert }{\vert P \cap Q \vert } = 5 \cdot 7 = 35.
    \]
    Consider \(G / P\), then \(\vert G / P \vert = \frac{105}{5} = 21 \). Note that \(n_7 (G / P) \equiv 1 \mod{7}\) and this gives \(n_7(G / P) \mid \frac{21}{7} = 3\), so we know \(n_7(G) = 1\). Say \(S \in \mathrm{Syl}_7(G / P) \), then \(S \trianglelefteq G / P\). Suppose \(\pi : G \to G / P\) is the map that \(\pi (g) = gP\), then suppose \(P^{\prime} = \pi ^{-1}(S)\) is the preimage of \(S\) under \(\pi \). Then, we have 
    \[
        \left\vert P^{\prime}  \right\vert = \vert S \vert \vert P \vert = 7 \cdot 5 = 35 
    \] since \(g_1 P = gP\) if and only if \(g_1 = gp\) for some \(p \in P\), and thus we have \(\vert P \vert \) choices for \(g_1\). Now we show that \(P^{\prime} \trianglelefteq G\) and then since \(\left\vert P^{\prime}  \right\vert = 35\), so we're done. We first show that \(P^{\prime} < G\): If \(x, y \in P^{\prime} \), then \(\pi (xy) = (xy)P = (xP)(yP) = \pi (x) \pi (y) \in S\) since \(P \trianglelefteq G\). Also, if \(x \in P^{\prime} \), then \(\pi \left( x^{-1} \right) = x^{-1} P = (xP)^{-1} \in S \). Thus, \(P^{\prime} < G\). Now we show that \(P^{\prime} \trianglelefteq G\): Suppose \(g \in G\) and \(p^{\prime} \in P^{\prime} \), then 
    \[
        \pi \left( g p^{\prime} g^{-1} \right) = \pi (g) \pi \left( p^{\prime}  \right) \pi \left( g^{-1} \right) \in S   
    \] since \(\pi \left( p^{\prime}  \right) \in S \) and \(S \trianglelefteq G / P\). Hence, \(g p^{\prime} g^{-1} \in P^{\prime} \), and thus \(g P^{\prime} g^{-1} = P^{\prime} \) for all \(g \in G\), and we're done.                               
    \item [(ii)] Suppose \(H \trianglelefteq G\) and \(\vert H \vert = 35 \). By (i), we know such \(H\) exists. Thus, we know \(n_5(H) = n_7(H) = 1\) by Sylow's theorem.  Hence, suppose \(P_5 \in \mathrm{Syl}_5(H) \) and \(P_7 \in \mathrm{Syl}_7(H) \), then \(P_5 \trianglelefteq H\) and \(P_7 \trianglelefteq H\). Also, \(P_5 \cap P_7 = \left\{ e \right\} \). Hence, 
    \[
        \left\vert P_5 P_7 \right\vert = \frac{\vert P_5 \vert \cdot \vert P_7 \vert  }{\vert P_5 \cap P_7 \vert } = 5 \cdot 7 = 35. 
    \]
    However, \(P_5 P_7 \subseteq H\) and \(\vert H \vert = 35 \). Hence, 
    \[
        H = P_5 P_7 \simeq P_5 \times P_7 \simeq C_5 \times C_7 \simeq C_{35} 
    \]
    since \(P_5, P_7 \trianglelefteq H\) and \(5, 7\) are prime (so \(P_5, P_7\) are cyclic) and \(\mathrm{gcd}(5, 7) = 1 \) (so \(C_5 \times C_7 \simeq C_{35} \)). Hence, \(H\) is cyclic.    
    \item [(iii)] Continuing (ii). Since \(P_5 \trianglelefteq H\) and \(P_7 \trianglelefteq H\), and \(H \trianglelefteq G\), so by the fact mentioned before this problem, we know \(P_5 \trianglelefteq G\) and \(P_7 \trianglelefteq G\). Also, \(P_5, P_7\) are the Sylow \(5\)-subgroup and the Sylow \(7\)-subgroup of \(G\), respectively. Hence, \(n_5(G) = n_7(G) = 1\) since \(P_5, P_7 \trianglelefteq G\).           
\end{itemize}


\paragraph{Exercise 4} More generally, let $G$ be a group of order $pqr$, where $p,q,r$ are distinct primes and $p<q<r$.  
We want to show that $n_r(G)=1$.

\begin{enumerate}[label=(\roman*)]
\item Suppose $n_r(G)>1$. Show that $n_q(G)=1$.  
Thus we deduce already that $G$ is not simple.

\item We now suppose $n_q(G)=1$, and let $Q$ be the unique Sylow $q$-subgroup of $G$.  
Show that $G/Q$ has a normal subgroup of order $r$.

\item Deduce that $G$ has a normal subgroup of order $qr$ and conclude that $n_r(G)=1$.
\end{enumerate}
\paragraph{Solution:} 
\begin{itemize}
    \item [(i)] Since \(n_r(G) \equiv 1 \mod{r}\) and \(n_r(G) \mid \frac{pqr}{r} = pq\), so \(n_r(G) \in \left\{ 1, p, q, pq \right\} \). Since \(n_r(G) > 1\), and \(n_r(G) \equiv 1 \mod{r}\), and \(p, q < r\), so the only possibility is \(n_r(G) = pq\). Hence, \(\mathrm{Syl}_r(G) \) contributes \(pq(r-1)\) elements of order \(>1\). Now if \(n_q(G) > 1\), then since \(n_q(G) \equiv 1\mod{q}\) and \(n_q(G) \mid \frac{pqr}{q} = pr\), so \(n_q(G) \in \left\{ p, r, pr \right\} \). Since \(q > p\), so \(p \not\equiv 1 \mod{q}\), so \(n_q(G) \ge r\). Hence, \(\mathrm{Syl}_q(G) \) contributes at least \(r(q-1) = rq - r\) elements of order \(>1\). Also, \(n_p(G) \ge 1\), so \(\mathrm{Syl}_p(G) \) contributes at least \(p-1\) elements of order \(>1\). Hence, \(G\) has at least 
    \[
        pq(r-1) + r(q-1) + p-1 + 1 = pqr -pq + rq - r + p = pqr + (q-1)(r - p) > pqr
    \]
    elements, which is impossible. Hence, \(n_q(G) = 1\). 
    \item [(ii)] Note that \(\left\vert G / Q \right\vert = pr \), then consider \(n_r(G / Q)\), we know \(n_r(G / Q) \equiv 1 \mod{r}\) and \(n_r(G / Q) \mid \frac{pr}{r} = p\), so \(n_r(G / Q) = 1\) since \(p \not\equiv 1 \mod{r}\). Hence, suppose \(R \in \mathrm{Syl}_r(G / Q) \), then \(R \trianglelefteq G / Q\) since it is the unique Sylow \(r\)-subgroup of \(G / Q\), and note that \(\vert R \vert = r \), so we're done.          
    \item [(iii)] Consider the preimage of \(R\) under the map \(\pi : G \to G / Q\) where \(\pi (g) = gQ\). Hence, 
    \[
        \left\vert \pi ^{-1}(R) \right\vert = \vert R \vert \cdot \vert Q \vert = r \cdot q = qr
    \]   
    since \(g_1 Q = gQ\) iff \(g_1 = gq\) for some \(q \in Q\). Now we show that \(\pi ^{-1} (R) \trianglelefteq G\), and thus \(\pi ^{-1}(R)\) is a subgroup of \(G\) of order \(qr\). We first show that \(\pi ^{-1}(R) < G\): If \(x, y \in \pi ^{-1}(R)\), then \(\pi (xy) = xyQ = (xQ)(yQ) = \pi (x) \pi (y) \in R\) since \(Q \trianglelefteq G\). Also, if \(x \in \pi ^{-1}(R)\), then \(\pi \left( x^{-1} \right) = x^{-1}Q = (xQ)^{-1} \in R\), so we can conclude that \(\pi ^{-1}(R) < G\). Now we show that \(\pi ^{-1}(R) \trianglelefteq G\): If \(x \in \pi ^{-1}(R)\), then for all \(g \in G\), \(\pi \left( g^{-1} x g \right) = \left( g^{-1} x g \right)Q = \pi \left( g^{-1} \right) \pi (x) \pi (g) \in R \) since \(\pi (x) \in R\) and \(R \trianglelefteq G / Q\). Hence, \(g^{-1} \pi ^{-1}(R) g \subseteq \pi ^{-1}(R)\) for all \(g \in G\), and thus \(\pi ^{-1}(R) \trianglelefteq G\). Now we conclude that \(n_r(G) = 1\). Since \(n_r \left( \pi ^{-1}(R) \right) \equiv 1 \mod{r} \) and \(n_r \left( \pi ^{-1}(R) \right) \mid \frac{qr}{r} = q \), and \(q \not\equiv 1\mod{r}\), so \(n_r \left( \pi ^{-1}(R) \right) = 1 \). Now suppose \(P_r\) is the unique Sylow \(r\)-subgroup of \(\pi ^{-1}(R)\), then since \(\pi ^{-1}(R) \trianglelefteq G\), so \(P_r \trianglelefteq G\) by the fact before Exercise 3. Note that \(P_r\) is also a Sylow \(r\)-subgroup of \(G\), so \(n_r(G) = 1\). Hence, if \(n_r(G) > 1\), then by (i) (ii) (iii), we will show \(n_r(G) = 1\), which is a contradiction, so we must have \(n_r(G) = 1\).                                       
\end{itemize}


\paragraph{Exercise 5} Let $R$ be a commutative ring.  
We say an element $x\in R$ is a \emph{zero divisor} if there is some nonzero element $y\in R$ such that $xy=0$.  
If a ring has no nonzero zero divisors, we say the ring is an \emph{integral domain}, or sometimes just \emph{domain} for short.

\begin{enumerate}[label=(\roman*)]
\item Classify all zero divisors of the following rings:  
\[
\mathbb{Z},\ \mathbb{Z}/6\mathbb{Z},\ \mathbb{Q},\ \mathbb{C}[x].
\]
Which of them are integral domains?  
(You don’t need to check they are commutative rings; addition and multiplication are carried out as usual.)

\item Show that any element in $R$ cannot be both invertible \emph{and} a zero divisor at the same time.

\item However, an element may be neither invertible nor a zero divisor.  
Find an example.

\item Show that the invertible elements of the polynomial ring $R[x]$ coincide with the invertible elements of $R$.
\end{enumerate}

\paragraph{Solution:}
\begin{itemize}
    \item [(i)] 
    \begin{itemize}
        \item \(\mathbb{Z} \): If \(xy = 0\) for \(x, y \in \mathbb{Z} \) and \(y \neq 0\), then \(x = 0\), so the only zero divisor is \(0\), which means \(\mathbb{Z} \) is an integral domain. 
        \item \(\mathbb{Z} / 6 \mathbb{Z} \): Note that the zero divisors are \([2], [3], [4], [0]\) since
        \[
            [0] = [2] \cdot [3] = [3] \cdot [2] = [4] \cdot [3] = [0] \cdot [1].
        \]
        Hence, \(\mathbb{Z} / 6\mathbb{Z} \) is not an integral domain. 
    \end{itemize}
    \item [(ii)] If \(x \in R\) is invertible, then there exists \(x^{-1}\) s.t. \(xx^{-1} = 1\). Now if \(x\) is a zero divisor, then \(xy = 0\) for some \(y \neq 0\), and thus 
    \[
        0 = x^{-1} \left( xy \right) = \left( x^{-1} x \right) y = y,  
    \] which is a contradiction. Hence, \(x\) can not be a zero divisor if \(x\) is invertible. 
    \item [(iii)] In \(\mathbb{Z} \), \(48763\) is neither invertible nor a zero divisor.  
    \item [(iv)] If \(f(x)\) is invertible in \(R[x]\), then \(\exists g(x) \in R[x]\) s.t. \(f(x) g(x) = 1\). This implies \(f, g\) must be constant polynomials. Hence, \(f \in R\) and \(f\) is invertible in \(R\). Now if \(p\) is invertible in \(R\), then there exists \(q \in R\) s.t. \(pq = 1\). Then, since \(f, g \in R[x]\), so \(f\) is invertible in \(R[x]\). Hence, we can conclude that invertible elements of \(R[x]\) concide with those of \(R\).                 
\end{itemize}

\end{document}