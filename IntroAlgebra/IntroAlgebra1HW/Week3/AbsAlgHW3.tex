\documentclass[12pt]{article}
\usepackage[a4paper, margin=1in]{geometry}
\usepackage{amsmath, amssymb, amsthm}
% basics
\usepackage{xeCJK}
\setCJKmainfont{PingFang TC} % 新細明體

\usepackage[utf8]{inputenc}
\usepackage[T1]{fontenc}
\usepackage{textcomp}
\usepackage[hyphens]{url}
\usepackage[style=alphabetic,maxcitenames=1]{biblatex}
\usepackage[colorlinks=true,linkcolor=cyan,urlcolor=magenta,citecolor=violet]{hyperref}
\usepackage{graphicx}
\usepackage{float}
\usepackage{booktabs}
\usepackage{emptypage}
\usepackage{subcaption}
\usepackage[usenames,dvipsnames]{xcolor}


\usepackage{amsmath, amsfonts, mathtools, amsthm, amssymb, mathrsfs}
\usepackage{geometry}
\usepackage{cancel}
\usepackage{systeme}

\usepackage[inline, shortlabels]{enumitem}
\usepackage{multicol}
\setlength\multicolsep{0pt}

\usepackage{caption}
\captionsetup{belowskip=0pt}
\geometry{a4paper,left=2.54cm,right=2.54cm,top=2.54cm,bottom=2.54cm}

% for the big braces
\usepackage{bigdelim}

% correct
\definecolor{correct}{HTML}{009900}
\newcommand\correct[2]{\ensuremath{\:}{\color{red}{#1}}\ensuremath{\to }{\color{correct}{#2}}\ensuremath{\:}}
\newcommand\green[1]{{\color{correct}{#1}}}

% hide parts
\newcommand\hide[1]{}

% si unitx
\usepackage{siunitx}
\sisetup{locale = FR}
% \renewcommand\vec[1]{\mathbf{#1}}
\newcommand\mat[1]{\mathbf{#1}}


% tikz
\usepackage{tikz}
\usetikzlibrary{intersections, angles, quotes, positioning}
\usetikzlibrary{arrows.meta}
\usepackage{pgfplots}
\pgfplotsset{compat=1.13}

\begin{document}

\begin{center}
    \Large \textbf{Abstract algebra I} \\
    \normalsize Homework 3 \\
    B13902024 張沂魁 \\
    Due: 1st October 2025
\end{center}

\begin{enumerate}
    \item For the following pairs of groups, determine whether they are isomorphic. If so, construct an isomorphism, otherwise, explain why they are not.
    \begin{enumerate}
        \item $(\mathbb{Z}/15\mathbb{Z})^{\times}$ and $\mathbb{Z}/8\mathbb{Z}.$
        \item $\mathbb{Z}/4\mathbb{Z}$ and $\{z\in\mathbb{C}\setminus\{0\}:z^{4}=1\}$
        \item $\mathbb{Z}$ and $3\mathbb{Z}=\{3n:n\in\mathbb{Z}\}$
        \item $\mathbb{Z}$ and $\{z\in\mathbb{C}\setminus\{0\}:z^{n}=1 \text{ for some } n\ge1\}$. (The second group is the group of all roots of unity. You may want to verify that this collection is indeed a subgroup of $\mathbb{C}\setminus\{0\}$.)
        \item $S_{3}$ and $D_{3}=\langle r,s:r^{3}=s^{2}=1,srs=r^{-1}\rangle$. (The second group is described as follows: it is generated by two elements $r, s$, and they satisfy the given relations. You may easily check that there are a total of six distinct elements. Such a description is called a group presentation.)
        \item $S_{4}$ and $D_{4}=\langle r,s:r^{4}=s^{2}=1,srs=r^{-1}\rangle$
        \item $Q=\langle i,j,k:i^{2}=j^{2}=k^{2}=ijk=-1\rangle$ and $T = \langle a,b: a^{6} = 1,b^{2} = a^{3}, ba = a^{5}b\rangle$.
        \item An infinite cyclic group $G$ and one of its non-trivial proper subgroup $H$, i.e., $H\ne\{e\}$.
    \end{enumerate}
    \textbf{Solution:} We first give a claim, which will be used in some of the later problems. \\
    \textbf{Claim:} If \(G, H\) are groups and they are isomorphic, then 
    \[
        \left\{ p: p\text{ is the order of some } g \in G \right\} = \left\{ q: q\text{ is the order of some } h \in H \right\}.
    \]
    \textit{proof.} Suppose \(\phi : G \to H\) is an isomorphism, then for every \(g \in G\), if \(o\left( g \right)  = m\), then \(o\left( \phi (g) \right) = m\) since \(\left( \phi (g) \right)^m = \phi \left( g^m \right) = \phi (e_G) = e_H  \). Also, if there exists some \(n < m\) s.t. \(\left( \phi (g) \right)^n = e_H \), then \(e_H = \left( \phi (g) \right)^n = \phi \left( g^n \right)  \), but since \(\phi \) is injective, so \(g^n = e_G\), which is a contradiction since \(n < m = o(g)\). Now since we know for all \(g \in G\), \(o(g) = o\left( \phi (g) \right) \), and since \(\phi \) is bijective, so our claim is true.     
    \begin{itemize}
        \item [(a)] By the claim, since in \(\mathbb{Z} / 8 \mathbb{Z} \), we know \(o(1) = 8 \), but \(\left( \mathbb{Z} / 15 \mathbb{Z}  \right)^{\times } \) is not cyclic, and \(\left\vert \left( \mathbb{Z} / 15 \mathbb{Z}  \right)^{\times }  \right\vert = 8\), so \(\mathbb{Z} / 8\mathbb{Z} \) and \(\left( \mathbb{Z} / 15 \mathbb{Z}  \right)^{\times } \) are not isomorphic.   
        \item [(b)] Note that 
        \[
            \left\{ z \in \mathbb{C} \setminus \left\{ 0 \right\} : z^4 = 1 \right\} = \left\{ e^{i \cdot \frac{2 \pi k}{4}}: k = 0, 1, 2, 3 \right\}.  
        \]
        Hence, we can define \(\phi : \mathbb{Z} / 4 \mathbb{Z} \to \left\{ z \in \mathbb{C} \setminus \left\{ 0 \right\}: z^4 = 1  \right\} \) as 
        \[
            \phi \left( \overline{i} \right)  = e^{i \cdot \frac{2 \pi i}{4}}.
        \]
        Note that this is an isomorphism since it is bijective and is a homomorphism, so \(\mathbb{Z} / 4\mathbb{Z} \) and \(\left\{ z \in \mathbb{C} \setminus \left\{ 0 \right\}: z^4 = 1 \right\} \) are isomorphic. 
        \item [(c)] We can construct an isomorphism between them as 
        \[
            \phi : \mathbb{Z} \to \mathbb{Z} / 3 \mathbb{Z},  \quad \phi (n) = 3n.
        \]
        \item [(d)] Note that \(\mathbb{Z} \) has no element which has order \(2\) but in 
        \[
            \left\{ z \in \mathbb{C} \setminus \left\{ 0 \right\} : z^n = 1 \text{ for some } n \ge 1 \right\}
        \] we know \(o(-1) = 2\), so by the claim above, they are not isomorphic. 
        \item [(e)] Note that 
        \(
            S_3 = \left\{ (1), (12), (23), (13), (123), (132) \right\} 
        \) and 
        \(
            D_3 = \left\{ 1, s, r, r^2, sr, sr^2 \right\} 
        \), so we can build an isomorphism \(\phi:S_3 \to D_3 \) as:
        \begin{align*}
            \phi ((1)) &= 1 \\
            \phi ((12)) &= s \\
            \phi ((23)) &= sr \\
            \phi ((13)) &= sr^2 \\
            \phi ((123)) &= r \\
            \phi ((132)) &= r^2.
        \end{align*}
        \item [(f)] Note that \(\left\vert S_4 \right\vert = 24 \) and \(\left\vert D_4 \right\vert = 8 \), so there does not exist bijective fundtion \(\phi : S_4 \to D_4\), which means \(S_4\) and \(D_4\) are not bijective.   
        \item [(g)] Since \(\left\vert Q \right\vert = 8\) and \(\left\vert T \right\vert = 12\), so \(Q\) and \(T\) are not isomorphic.     
        \item [(h)] Suppose \(G = \langle g \rangle \), then since every subgroup of cyclic subgroup is cyclic, so \(H = \langle g^k \rangle \) for some \(k \ge 1\). Now we can build an isomorphism \(\phi :G \to H\) by \(\phi \left( g^i \right) = \left( g^k \right)^i  \) for all \(g^i \in G\). Note that this is an isomorphism.  
    \end{itemize}

    \item Define the additive group
    \[
    G=\left\{\begin{pmatrix}a&-b\\ b&a\end{pmatrix}:a,b\in\mathbb{R}\right\}
    \]
    Show that the map $\varphi:G\rightarrow\mathbb{C}$ given by $\begin{pmatrix}a&-b\\ b&a\end{pmatrix}\mapsto a+bi$ is a group isomorphism. You don't have to check for well-definedness. \\
    \textbf{Solution:} We first show that \(\varphi \) is a group homomorphism. Note that 
    \[
        \varphi \left( \begin{pmatrix}
            a & -b  \\
            b &  a \\
        \end{pmatrix} \begin{pmatrix}
            c & -d  \\
            d &  c \\
        \end{pmatrix} \right) = \varphi \left( \begin{pmatrix}
            ac - bd & -ad - bc  \\
            bc + ad & -bd + ac  \\
        \end{pmatrix} \right) = (ac - bd) + (bc + ad)i.
    \]
    \[
        \varphi \left( \begin{pmatrix}
            a & -b  \\
            b &  a \\
        \end{pmatrix} \right) \varphi \left( \begin{pmatrix}
            c & -d  \\
            d &  c \\
        \end{pmatrix} \right) = (a + bi)(c + di) = (ac - bd) + (ad + bc)i. 
    \]
    Hence, we know \(\varphi \) is a group homomorphism. Now we show that \(\varphi \) is bijective. If 
    \[
        \varphi \left( \begin{pmatrix}
            a & -b  \\
            b & a  \\
        \end{pmatrix} \right) = \varphi \left( \begin{pmatrix}
            c & -d  \\
            d & c  \\
        \end{pmatrix} \right),   
    \] then \(a + bi = c + di\), which gives \(a = c\) and \(b = d\). Hence, \(\varphi \) is injective. Now we show that it is surjective. For all \(c \in \mathbb{C} \), we know \(c = a + bi\) for unique \(a, b \in \mathbb{R} \), and 
    \[
        \varphi \left( \begin{pmatrix}
            a & -b  \\
            b & a  \\
        \end{pmatrix} \right) = a+bi, 
    \] so we know \(\varphi \) is surjective. Hence, we know \(\varphi \) is bijective and thus an isomorphism.

    \item Let $H$ be a normal subgroup of a group $G$, and $N$ a subgroup of $H$.
    \begin{enumerate}
        \item If $H$ is cyclic, prove that $N$ is a normal subgroup of $G$.
        \item If $N$ is a normal subgroup of $H,$ show that $N$ is not necessarily a normal subgroup of $G$. In other words, give a counterexample.
    \end{enumerate}
    \textbf{Solution:} 
    \begin{itemize}
        \item [(a)] If \(H\) is cyclic, then suppose \(H = \langle h \rangle \), and we want to show that \(g N g^{-1} = N \) for all \(g \in G\). Since we know \(H\) is normal, so \(g H g^{-1} = H \) for all \(g \in G\). Now fix some \(g \in G\), we know \(ghg^{-1} = h^k \) for some \(k \ge 0\). Now given any \(h^m \in N\), we have 
        \[
            g h^m g^{-1} = \left( ghg^{-1}  \right)^m = h^{km} = \left( h^m \right)^k \in N  
        \] since \(h^m \in N\). Thus, \(g N g^{-1} \subseteq N \). Now we show that \(N \subseteq gNg^{-1} \). Given \(g^m \in N\), by repeating the above argument but replacing \(g\) with \(g^{-1} \), we know there exists some \(h^j \in N\) s.t. \(g^{-1} h^m g = h^j \in N\). Hence, we have \(h^m = g h^j g^{-1} \in g N g^{-1}  \), which shows \(N \subseteq g N g^{-1} \). Hence, we know \(g N g^{-1} = N \), and since this \(g\) can be any element in \(G\), so we know \(N\) is a normal subgroup of \(G\).          
        \item [(b)] Consider \(G = S_4\), \(H = V_4 = \left\{ e, (12)(34), (13)(24), (14)(23) \right\} \), and \(N = \left\{ e, (12)(34) \right\} \), then \(H\) is a normal subgroup of \(G\) and \(N\) is a normal subgroup of \(H\), but \(N\) is not a normal subgroup of \(S_4\) since \((31)(12)(34)(13) = (14)(23) \notin N\).         
    \end{itemize}

    \item Let $f:G\rightarrow H$ be a group homomorphism with $H$ abelian and let $N$ be a subgroup of $G$ containing $\ker f$.
    Prove that $N$ is normal in $G$. \\
    \textbf{Solution:} We want to show that for all \(g \in G\), we have \(gNg^{-1} = N \). Now fix some \(g \in G\) and suppose \(n_1 \in N\), then  
    \begin{align*}
        f \left( g n_1 g^{-1} n_1^{-1} \right) &= f(g) f \left( n_1  \right) f \left( g^{-1} \right) f \left( n_1^{-1} \right) \\
        &= f(g) f \left( g^{-1} \right) f(n_1) f \left( n_1^{-1} \right) \\
        &= f(g)f(g)^{-1} f(n_1) f(n_1)^{-1} = e_H
    \end{align*}
    since \(H\) is abelian. Hence, \(g n_1 g^{-1} n_1^{-1} \in \ker f \subseteq N  \), and thus 
    \[
        g n_1 g^{-1} n_1^{-1} = n_2 \in N  
    \] for some \(n_2 \in N\) and thus \(g n_1 g^{-1} = n_2 n_1 \in N \). Since \(g\) can be arbitrary element of \(G\), so \(gNg^{-1} \subseteq N \) for all \(g \in G\). Now given \(n_2 \in N\), and repeat the same argument but replacing \(g\) with \(g^{-1} \), we will get
    \[
        g^{-1} n_2 g = n_3 \in N
    \] for some \(n_3 \in N\), and thus \(n_2 = g n_3 g^{-1} \), which means \(N \subseteq g N g^{-1} \) for all \(g \in G\). Hence, we have \(N = g N g^{-1} \) for all \(g \in G\).   
\end{enumerate}

\end{document}