\documentclass[12pt]{article}
\usepackage[a4paper, margin=1in]{geometry}
\usepackage{amsmath, amssymb, amsthm}
% basics
\usepackage{xeCJK}
\setCJKmainfont{PingFang TC} % 新細明體

\usepackage[utf8]{inputenc}
\usepackage[T1]{fontenc}
\usepackage{textcomp}
\usepackage[hyphens]{url}
\usepackage[style=alphabetic,maxcitenames=1]{biblatex}
\usepackage[colorlinks=true,linkcolor=cyan,urlcolor=magenta,citecolor=violet]{hyperref}
\usepackage{graphicx}
\usepackage{float}
\usepackage{booktabs}
\usepackage{emptypage}
\usepackage{subcaption}
\usepackage[usenames,dvipsnames]{xcolor}


\usepackage{amsmath, amsfonts, mathtools, amsthm, amssymb, mathrsfs}
\usepackage{geometry}
\usepackage{cancel}
\usepackage{systeme}

\usepackage[inline, shortlabels]{enumitem}
\usepackage{multicol}
\setlength\multicolsep{0pt}

\usepackage{caption}
\captionsetup{belowskip=0pt}
\geometry{a4paper,left=2.54cm,right=2.54cm,top=2.54cm,bottom=2.54cm}

% for the big braces
\usepackage{bigdelim}

% correct
\definecolor{correct}{HTML}{009900}
\newcommand\correct[2]{\ensuremath{\:}{\color{red}{#1}}\ensuremath{\to }{\color{correct}{#2}}\ensuremath{\:}}
\newcommand\green[1]{{\color{correct}{#1}}}

% hide parts
\newcommand\hide[1]{}

% si unitx
\usepackage{siunitx}
\sisetup{locale = FR}
% \renewcommand\vec[1]{\mathbf{#1}}
\newcommand\mat[1]{\mathbf{#1}}


% tikz
\usepackage{tikz}
\usetikzlibrary{intersections, angles, quotes, positioning}
\usetikzlibrary{arrows.meta}
\usepackage{pgfplots}
\pgfplotsset{compat=1.13}

\begin{document}

\begin{center}
    \Large \textbf{Abstract algebra I} \\
    \normalsize Homework 2 \\
    B13902024 張沂魁 \\
    Due: 24th September 2025
\end{center}

\bigskip

\begin{enumerate}
    \item Let $G = \mathbb{Z}/30\mathbb{Z}$, which is an additive group as we have known. Among the following
    subsets of $G$, determine whether each of them is a subgroup. Prove or disprove them.
    \begin{enumerate}
        \item $G_1 = \{0, 1, 2, 3, \dots , 14\}$
        \item $G_2 = \{0, 2, 4, 6, \dots , 28\}$
        \item $G_3 = \{1, 7, 11, 13, 17, 19, 23, 29\}$
    \end{enumerate}
    \textbf{Solution:} \begin{itemize}
        \item [(a)] No. Since \(14 + 3 = 17 \notin G_1\), so \(G_1\) is not closed. 
        \item [(b)] Note that \(G_2 = \langle 2 \rangle \), so it is a subgroup of \(G\). 
        \item [(c)] Note that \(0 \notin G_3\), so there does not exist an identity element in \(G_3\) and thus \(G_3\) can not be a subgroup.    
    \end{itemize}

    \item Recall that a subgroup $H$ of a group $G$ is a subset $H \subset G$ that itself is a group, with
    the induced operation. We now introduce the criterion of a subset and we aim to
    prove this handful criterion and give an application.
    \begin{enumerate}
        \item[(i)] \textbf{(Subgroup criterion)} Suppose $(G, \ast)$ is a group and $H \subset G$. Suppose that
        \begin{itemize}
            \item $H$ is not empty,
            \item (closedness of $\ast$) for any $a, b \in H$, we have $a \ast b \in H$, and
            \item (closedness of inverse) for all $a \in H$, its inverse $a^{-1} \in H$.
        \end{itemize}
        Prove that $H$ is a subgroup of $G$.

        \item[(ii)] Using the above criterion, check that the special linear group $SL_n(\mathbb{R})$ is a subgroup
        of the general linear group $GL_n(\mathbb{R})$. (Recall that $GL_n(\mathbb{R})$ is the set of all $n \times n$
        matrices with real coefficients (entries) so that their determinants are not zero,
        while $SL_n(\mathbb{R})$ is the subset of $GL_n(\mathbb{R})$ containing those matrices with determinant $1$.)
    \end{enumerate}
    \textbf{Solution:} 
    \begin{itemize}
        \item [(a)] For all \(a, b, c \in H\), since \(a, b, c \in G\), so \((a * b) * c = a * (b * c)\), and this can be inherited to \(H\). Also, since for all \(a \in H\) we have \(a^{-1} \in H \), and \(*\) is closed under \(H\), so \(e = a * a^{-1} \in H\), so the identity element is in \(H\). Also, we already know the closedness of inverse, so we know \(H\) is a subgroup of \(G\).              
        \item [(b)] Note that \(\mathrm{SL}_n(\mathbb{R} ) \) is not empty since \(I_n \in SL_n(\mathbb{R} )\), where \(I_n\) is the matrix with all diagonal entries \(1\) and all the other entries \(0\). Also, if \(a, b \in \mathrm{SL}_n(\mathbb{R} )\), then \(\det(a) = \det (b) = 1\), so we know \(\det (ab) = \det (a) \det (b) = 1\), and thus \(ab \in \mathrm{SL}_n(\mathbb{R} ) \). Besides, for any \(a \in \mathrm{SL}_n(\mathbb{R} ) \), since \(\det (a) = 1 > 0\), so it is invertible, and thus there exists \(a^{-1}\) such that \(a a^{-1} = a^{-1} a = I_n\), which implies \(1 = \det(a) \det\left( a^{-1}  \right) = 1 \cdot \det \left( a^{-1} \right)  \), and thus \(a^{-1} \in \mathrm{SL}_n(\mathbb{R} ) \).          
    \end{itemize}

    \item We define the order of a finite group $G$ to be $|G|$, i.e., the number of elements (or the
    cardinality) of $G$. Also, for any $g \in G$, the order of $g$, denoted by $|g|$ (or sometimes
    $o(g)$), is the smallest positive integer $m$ such that $g^m = e$.
    \begin{enumerate}
        \item[(a)] For $n \geq 1$, show that the set of bijections $\{1, 2, \dots, n\} \to \{1, 2, \dots, n\}$, denoted by
        $S_n$, is a group of order $n!$.

        \item[(b)] Show that the subset given by $\{\sigma \in S_4 : \sigma(1) = 1\}$, i.e., the collection of bijections
        fixing $1$, is a subgroup of $S_4$. Find its order.

        \item[(c)] Consider the multiplicative non-abelian group
        \[
        \left\{
        \begin{pmatrix}
        a & b \\
        c & d
        \end{pmatrix}
        : a, b, c, d \in \mathbb{Z}, \ ad - bc = \pm 1
        \right\}.
        \]
        Find the orders of
        \[
        a = \begin{pmatrix}
        0 & -1 \\
        1 & 0
        \end{pmatrix}
        \quad \text{and} \quad
        b = \begin{pmatrix}
        0 & 1 \\
        -1 & -1
        \end{pmatrix}.
        \]
    \end{enumerate}
    \textbf{Solution:}
    \begin{itemize}
        \item [(a)] If \(\pi : [n] \to [n]\) is a bijection, where \([n] = \left\{ 1,2, \dots ,n \right\} \), then \(\pi (1)\) has \(n\) choices, \(\pi (2)\) has \(n-1\) choices, and so on, so there are \(n!\) possibilities for a bijection from \([n]\) to \([n]\), which means \(S_n\) has \(n!\) elements. Now we show that \(S_n\) is a group. Since for \(\pi , \psi \in S_n\), we know \(\pi \circ \psi \) is still a bijection from \([n]\) to \([n]\) (the composition of two bijection is a bijeciton), so \(\pi \circ \psi \in S_n\). Also, since every \(\pi  \in S_n \) is a bijeciton from \([n]\) to \([n]\), so the inverse of \(\pi \) exists, say it is \(\pi ^{-1}\), and notice that \(\pi ^{-1}\) is still a bijection from \([n]\) to \([n]\), and thus \(\pi ^{-1} \in S_n\) and \(\pi \circ \pi ^{-1} = \pi ^{-1} \circ \pi = e \in S_n\), where \(e\) is the identity map, which is defined by \(e(i) = i\) for all \(i \in [n]\). Hence, we know \(e \in S_n\), and \(e\) is the identity element, so \(S_n\) is a group of order \(n!\).
        \item [(b)] Suppose 
        \[
            H_4 = \left\{ \sigma \in S_4 : \sigma (1) = 1 \right\}, 
        \]
        then for \(a, b \in H_4\), we know \(a(1) = b(1) = 1\), so \((a \circ b )(1) = a(1) = 1\), which means \(a \circ b \in H_4\) since \(S_4\) is a group and thus \(a \circ b \in S_4\). Also, notice that the identity map \(e \in H_4\) since \(e(1) = 1\) and \(e \in S_4\). Besides, for any \(a \in H_4\), we can define \(a^{-1}\) as \(a^{-1}(1) = 1\) and \(a^{-1}(a(i)) = i\) for all \(i \in \left\{ 2,3,4 \right\} \). Note that \(a^{-1}\) is well-defined since \(a\) is a bijection and \(a(1) = 1\) , so \(a^{-1}\) is also a bijection. Note that \(a \circ a^{-1} = a^{-1} \circ a = e\), so \(a^{-1}\) is the inverse of \(a\) and \(a^{-1} \in H_4\), so we know \(H_4\) is a subgroup of \(S_4\). Note that for any \(\pi  \in H_4\), since \(\pi (1) = 1\), so just need to decide the value of \(\pi (2), \pi (3), \pi (4)\) and make sure \(\pi \) is a bijection, so we have \(3\) options for \(\pi (2)\), \(2\) for \(\pi (3)\), and \(1\) for \(\pi (4)\), which means there are \(6\) elements in \(H_4\), so the order of \(H_4\) is \(6\).                            
        \item [(c)] Since we know 
        \begin{align*}
            a = \begin{pmatrix}
                0 & -1  \\
                1 &  0 \\
            \end{pmatrix}  \quad
            a^2 = \begin{pmatrix}
                -1 &  0 \\
                0 &  -1 \\
            \end{pmatrix} \quad
            a^3 = \begin{pmatrix}
                0 & 1  \\
                -1 & 0  \\
            \end{pmatrix}  \quad
            a^4 = \begin{pmatrix}
                1 & 0  \\
                0 & 1  \\
            \end{pmatrix} = I_2,
        \end{align*}
        so the order of \(a\) is \(4\). Also, we know 
        \[
            b = \begin{pmatrix}
                0 & 1  \\
                -1 & -1  \\
            \end{pmatrix} \quad b^2 = \begin{pmatrix}
                -1 & -1  \\
                1 & 0  \\
            \end{pmatrix} \quad b^3 = 
            \begin{pmatrix}
                 1& 0  \\
                 0& 1  \\
            \end{pmatrix},
        \]
         so the order of \(b\) is \(3\).  
    \end{itemize}           

    \item For some positive integer $n$, let $G$ be a finite cyclic group of order $n$. (Recall that a
    cyclic group is a group generated by one of its elements, i.e., there exists some $g \in G$
    of order $n$.)
    \begin{enumerate}
        \item[(a)] Prove that if $g$ is a generator of $G$, then $g^k$ is a generator of $G$ if and only if
        $\gcd(k, n) = 1$.
        \item[(b)] For any $m \mid n$, prove that $G$ has exactly one subgroup of order $m$.
        \item[(c)] Show that $S_3$, the symmetric group on 3 letters, is not cyclic.
    \end{enumerate}
    \textbf{Solution:} \vphantom{text}
    \begin{itemize}
        \item [(a)] \vphantom{text}
        \begin{itemize}
            \item [\((\implies)\)] If \(g\) is a generator of \(G\) and \(g^k\) is a generator of \(G\). If \(\gcd(k, n) = d > 1\), then 
            \[
                d \mid ak - bn \quad \forall a, b \in \mathbb{Z} .
            \] That is, if \(c < d\) and \(c \in \mathbb{N} \), then \(g^c \notin \langle g^k \rangle = G \), which is a contradiction.  
            \item [\((\impliedby)\)] If \(\gcd(k, n) = 1\), we claim that \(\langle g^k \rangle = \langle g \rangle \). If not, then there exists \(c \in \mathbb{N} \) s.t. \(g^c \notin \langle g^k \rangle \), so there exists \(i \neq j\) with \(1 \le i < j \le n\) s.t. \(g^{ik} = g^{jk}\) by Pigeonhole principle. However, this means \(g^{(j - i)k} = e\), which means \(n \mid (j - i)k\), but \(j - i < n\) and \(\gcd(k, n) = 1\), so this is impossible. Thus, we know \(\langle g^k \rangle = \langle g \rangle \).            
        \end{itemize}
        \item [(b)] We first prove the existence, that is, there exists a subgroup of \(G\) of order \(m\). Suppose \(n = md\), then 
        \[
            \langle g^d \rangle = \left\{ g^d, g^{2d}, \dots , g^{md} \right\} 
        \] is a subgroup of order \(m\) since \(g^{md} = g^n = e\), and for every \(g^{cd} \in \langle g^d \rangle \), we know \(g^{md - cd} \in G\), so the inverse of every element of \(\langle g^d \rangle \) is also in \(\langle g^d \rangle \). Now we show that the uniqueness. We claim that the subgroup of a cyclic group must also be a cyclic group. If \(H\) is a subgroup of \(G\) and \(G = \langle g \rangle \) is cyclic, then let \(d = \min \left\{ k : g^k \in H \right\} \), then we know \(\langle g^d \rangle \subseteq H\). Now we show that \(H \subseteq \langle g^d \rangle \). Suppose \(g^k \in H\) and \(k = sd + r\) where \(0 \le r < d\) and \(k, r \in \mathbb{N} \cup \left\{ 0 \right\} \). If \(r > 0\), then we know \(g^{sd} * g^r = g^k \in H\), and \(g^r = g^{-sd} * g^{k} \in H \) since \(g^{-sd} \in \langle g^d \rangle \subseteq H\) and \(g^k \in H\). However, \(r < d = \min \left\{ k: g^k \in H \right\} \), so this is impossible. Now we know \(r = 0\), so \(H \subseteq \langle g^d \rangle \). Thus, the claim is true. Now go back to the original problem, we know \(G\) is cyclic, and thus if \(H\) is a subgroup of \(G\), then \(H\) is cyclic. Suppose \(H = \langle g^k \rangle \), where \(o\left( g^k \right) = m\), then we have 
        \[
            md = n \mid km ,
        \] and thus \(d \mid k\), which means \(k = dr\) for some \(r \in \mathbb{N}\). Since \(g^k\) is a generator of \(H\). If \(\gcd(r, m) = d^{\prime}\), then \(o\left( g^{rd} \right) = \frac{m}{d^{\prime} } \).  Hence, we know \(d^{\prime} = 1\) as a result of \(o\left( g^{rd} \right) = m\). Now we know \(g^d\) is a generator of \(\langle g^d \rangle \) and \(\gcd(r, m) = 1\), so \(\left( g^d \right)^r \) is also a generator of \(\langle g^d \rangle \) by (a). Hence, \(H = \langle g^{d} \rangle \) since \(g^{dr}\) is a generator of \(H\) and \(\langle g^d \rangle \).      
        \item [(c)] Note that every cyclic group \(G\) is Abelian since if \(g\) is a generator of \(G\), then \(\forall a, b \in G\), we can write \(a = g^p\) and \(b = g^q\) for some \(p, q \in \mathbb{N} \), and thus 
        \[
            a * b = g^p * g^q = g^{p + q} = g^q * g^p = b * a.
        \]
        However, notice that \(S_3\) is not Abelian since \((12)(13)=(132)\) but \((13)(12)=(123)\). 
    \end{itemize}

    \item For any subgroup $N$ of a group $(G, \ast)$, and any $g \in G$, we define
    \[
    gN = \{g \ast n : n \in N\}, \quad Ng = \{n \ast g : n \in N\}.
    \]
    We say that $N$ is a normal subgroup of $G$, denoted $N \trianglelefteq G$, if $gN = Ng$ for all $g \in G$.
    \begin{enumerate}
        \item[(a)] Show that if $N \trianglelefteq G$, then $(G/N, \cdot)$ is a group under the operation given by
        \[
        g_1N \cdot g_2N = (g_1 \ast g_2)N.
        \]
        \item[(b)] Consider the subgroup $H = \{(1), (12)\} \subset S_3$. Show that $H$ is not a normal
        subgroup of $S_3$.
        \item[(c)] Show that every subgroup of an abelian group $G$ is normal.
        \item[(d)] The converse is false; there exist non-abelian groups with all subgroups normal.
        Search for one, no justification required.
    \end{enumerate}
    \textbf{Solution:} 
    \begin{itemize}
        \item [(a)]
        \begin{itemize}
            \item Well-defined: We need to show the operation \(\cdot\) is well-defined. That is, if \(g_1 N = g_1 ^{\prime} N\) and \(g_2 N = g_2^{\prime} N\), then \((g_1 * g_2)N = (g_1^{\prime} * g_2^{\prime})N\). Since \(N\) is a subgroup of \(G\), so \(e \in G\), and thus \(g_1^{\prime} = g_1^{\prime}  * e \in g_1^{\prime}  N = g_1 N\), so \(g_1^{\prime} = g_1 * n_1\) for some \(n_1 \in N\). Similarly, we know \(g_2^{\prime} = g_2 * n_2\)  for some \(n_2 \in N\). Note that for any \(n \in N\), we have  
            \[
                g_1^{\prime} * g_2^{\prime} * n = (g_1 * n_1) * (g_2 * n_2) * n = g_1 * (n_1 * g_2) * n_2 * n.
            \] Also, since \(N\) is a normal subgroup of \(G\), so
            \[
                n_1 * g_2 \in N g_2 = g_2 N,
            \]so we can write \(n_1 * g_2 = g_2 * n_1^{\prime} \) for some \(n_1^{\prime} \in G\). Hence, we have 
            \[
                g_1 * (n_1 * g_2) * n_2 * n = g_1 * (g_2 * n_1^{\prime} ) * n_2 * n \in (g_1 * g_2)N.
            \]  Hence, \((g_1^{\prime} * g_2^{\prime} ) N \subseteq (g_1 * g_2)N\). Similarly, we can get \((g_1 * g_2)N \subseteq (g_1^{\prime} * g_2^{\prime} )N\). Hence, we have 
            \((g_1 * g_2)N = (g_1^{\prime} * g_2^{\prime} )N\).  
            \item Associativity: For any \(g_1 N, g_2 N, g_3 N \in G/N\), we have 
            \[
                (g_1 N \cdot g_2 N) \cdot g_3 N = (g_1 * g_2)N \cdot g_3 N = (g_1 * g_2 * g_3)N,
            \] and 
            \[
                g_1 N \cdot (g_2 N \cdot g_3 N) = g_1 N \cdot (g_2 * g_3)N = (g_1 * g_2 * g_3)N.
            \]
            Hence, we know \((g_1 N \cdot g_2 N) \cdot g_3 N = g_1 N \cdot (g_2 N \cdot g_3 N)\). 
            \item Identity: \(\forall g N \in G/N\), we know 
            \[
                gN \cdot eN = (g * e)N = gN \quad (e * g)N = eN * gN.
            \]
            \item Inverse: \(\forall gN \in G/N\), we know
            \[
                gN \cdot g^{-1} N = (g * g^{-1}) N = eN \quad g^{-1}N \cdot gN = (g^{-1} * g)N = eN.
            \] 
        \end{itemize}
        \item [(b)] Consider \((23) \in S_3\), then 
        \[
            (23)H = \left\{ (23), (132) \right\} \quad H(23)=\left\{ (23), (123) \right\}, 
        \] so \((23)H \neq H(23)\), which means \(H\) is not a normal subgroup of \(G\).   
        \item [(c)] Suppose \(H\) is a subgroup of an abelian group \(G\), then for all \(g * h \in gH\) with \(g \in G\) and \(h \in H\), we have \(g * h = h * g \in Hg\), so \(gH \subseteq Hg\), and similarly we can show \(Hg \subseteq gH\), so \(gH = Hg\) for all \(g \in G\), and thus \(H\) is normal.        
        \item [(d)] Consider \(Q_8 = \left\{ 1, -1, i, -i, j, -j, k, -k \right\} \), where the definition of multiplication operation of \(Q_8\) is 
        \[
            \begin{array}{c|cccccccc}
            * & 1 & -1 & i & -i & j & -j & k & -k \\
            \hline
            1 & 1 & -1 & i & -i & j & -j & k & -k \\
            -1 & -1 & 1 & -i & i & -j & j & -k & k \\
            i & i & -i & -1 & 1 & k & -k & -j & j \\
            -i & -i & i & 1 & -1 & -k & k & j & -j \\
            j & j & -j & -k & k & -1 & 1 & i & -i \\
            -j & -j & j & k & -k & 1 & -1 & -i & i \\
            k & k & -k & j & -j & -i & i & -1 & 1 \\
            -k & -k & k & -j & j & i & -i & 1 & -1
            \end{array}
        \]

    \end{itemize}
\end{enumerate}

\end{document}
